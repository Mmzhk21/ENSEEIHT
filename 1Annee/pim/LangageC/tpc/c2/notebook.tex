
% Default to the notebook output style

    


% Inherit from the specified cell style.




    
\documentclass[11pt]{article}

    
    
    \usepackage[T1]{fontenc}
    % Nicer default font (+ math font) than Computer Modern for most use cases
    \usepackage{mathpazo}

    % Basic figure setup, for now with no caption control since it's done
    % automatically by Pandoc (which extracts ![](path) syntax from Markdown).
    \usepackage{graphicx}
    % We will generate all images so they have a width \maxwidth. This means
    % that they will get their normal width if they fit onto the page, but
    % are scaled down if they would overflow the margins.
    \makeatletter
    \def\maxwidth{\ifdim\Gin@nat@width>\linewidth\linewidth
    \else\Gin@nat@width\fi}
    \makeatother
    \let\Oldincludegraphics\includegraphics
    % Set max figure width to be 80% of text width, for now hardcoded.
    \renewcommand{\includegraphics}[1]{\Oldincludegraphics[width=.8\maxwidth]{#1}}
    % Ensure that by default, figures have no caption (until we provide a
    % proper Figure object with a Caption API and a way to capture that
    % in the conversion process - todo).
    \usepackage{caption}
    \DeclareCaptionLabelFormat{nolabel}{}
    \captionsetup{labelformat=nolabel}

    \usepackage{adjustbox} % Used to constrain images to a maximum size 
    \usepackage{xcolor} % Allow colors to be defined
    \usepackage{enumerate} % Needed for markdown enumerations to work
    \usepackage{geometry} % Used to adjust the document margins
    \usepackage{amsmath} % Equations
    \usepackage{amssymb} % Equations
    \usepackage{textcomp} % defines textquotesingle
    % Hack from http://tex.stackexchange.com/a/47451/13684:
    \AtBeginDocument{%
        \def\PYZsq{\textquotesingle}% Upright quotes in Pygmentized code
    }
    \usepackage{upquote} % Upright quotes for verbatim code
    \usepackage{eurosym} % defines \euro
    \usepackage[mathletters]{ucs} % Extended unicode (utf-8) support
    \usepackage[utf8x]{inputenc} % Allow utf-8 characters in the tex document
    \usepackage{fancyvrb} % verbatim replacement that allows latex
    \usepackage{grffile} % extends the file name processing of package graphics 
                         % to support a larger range 
    % The hyperref package gives us a pdf with properly built
    % internal navigation ('pdf bookmarks' for the table of contents,
    % internal cross-reference links, web links for URLs, etc.)
    \usepackage{hyperref}
    \usepackage{longtable} % longtable support required by pandoc >1.10
    \usepackage{booktabs}  % table support for pandoc > 1.12.2
    \usepackage[inline]{enumitem} % IRkernel/repr support (it uses the enumerate* environment)
    \usepackage[normalem]{ulem} % ulem is needed to support strikethroughs (\sout)
                                % normalem makes italics be italics, not underlines
    

    
    
    % Colors for the hyperref package
    \definecolor{urlcolor}{rgb}{0,.145,.698}
    \definecolor{linkcolor}{rgb}{.71,0.21,0.01}
    \definecolor{citecolor}{rgb}{.12,.54,.11}

    % ANSI colors
    \definecolor{ansi-black}{HTML}{3E424D}
    \definecolor{ansi-black-intense}{HTML}{282C36}
    \definecolor{ansi-red}{HTML}{E75C58}
    \definecolor{ansi-red-intense}{HTML}{B22B31}
    \definecolor{ansi-green}{HTML}{00A250}
    \definecolor{ansi-green-intense}{HTML}{007427}
    \definecolor{ansi-yellow}{HTML}{DDB62B}
    \definecolor{ansi-yellow-intense}{HTML}{B27D12}
    \definecolor{ansi-blue}{HTML}{208FFB}
    \definecolor{ansi-blue-intense}{HTML}{0065CA}
    \definecolor{ansi-magenta}{HTML}{D160C4}
    \definecolor{ansi-magenta-intense}{HTML}{A03196}
    \definecolor{ansi-cyan}{HTML}{60C6C8}
    \definecolor{ansi-cyan-intense}{HTML}{258F8F}
    \definecolor{ansi-white}{HTML}{C5C1B4}
    \definecolor{ansi-white-intense}{HTML}{A1A6B2}

    % commands and environments needed by pandoc snippets
    % extracted from the output of `pandoc -s`
    \providecommand{\tightlist}{%
      \setlength{\itemsep}{0pt}\setlength{\parskip}{0pt}}
    \DefineVerbatimEnvironment{Highlighting}{Verbatim}{commandchars=\\\{\}}
    % Add ',fontsize=\small' for more characters per line
    \newenvironment{Shaded}{}{}
    \newcommand{\KeywordTok}[1]{\textcolor[rgb]{0.00,0.44,0.13}{\textbf{{#1}}}}
    \newcommand{\DataTypeTok}[1]{\textcolor[rgb]{0.56,0.13,0.00}{{#1}}}
    \newcommand{\DecValTok}[1]{\textcolor[rgb]{0.25,0.63,0.44}{{#1}}}
    \newcommand{\BaseNTok}[1]{\textcolor[rgb]{0.25,0.63,0.44}{{#1}}}
    \newcommand{\FloatTok}[1]{\textcolor[rgb]{0.25,0.63,0.44}{{#1}}}
    \newcommand{\CharTok}[1]{\textcolor[rgb]{0.25,0.44,0.63}{{#1}}}
    \newcommand{\StringTok}[1]{\textcolor[rgb]{0.25,0.44,0.63}{{#1}}}
    \newcommand{\CommentTok}[1]{\textcolor[rgb]{0.38,0.63,0.69}{\textit{{#1}}}}
    \newcommand{\OtherTok}[1]{\textcolor[rgb]{0.00,0.44,0.13}{{#1}}}
    \newcommand{\AlertTok}[1]{\textcolor[rgb]{1.00,0.00,0.00}{\textbf{{#1}}}}
    \newcommand{\FunctionTok}[1]{\textcolor[rgb]{0.02,0.16,0.49}{{#1}}}
    \newcommand{\RegionMarkerTok}[1]{{#1}}
    \newcommand{\ErrorTok}[1]{\textcolor[rgb]{1.00,0.00,0.00}{\textbf{{#1}}}}
    \newcommand{\NormalTok}[1]{{#1}}
    
    % Additional commands for more recent versions of Pandoc
    \newcommand{\ConstantTok}[1]{\textcolor[rgb]{0.53,0.00,0.00}{{#1}}}
    \newcommand{\SpecialCharTok}[1]{\textcolor[rgb]{0.25,0.44,0.63}{{#1}}}
    \newcommand{\VerbatimStringTok}[1]{\textcolor[rgb]{0.25,0.44,0.63}{{#1}}}
    \newcommand{\SpecialStringTok}[1]{\textcolor[rgb]{0.73,0.40,0.53}{{#1}}}
    \newcommand{\ImportTok}[1]{{#1}}
    \newcommand{\DocumentationTok}[1]{\textcolor[rgb]{0.73,0.13,0.13}{\textit{{#1}}}}
    \newcommand{\AnnotationTok}[1]{\textcolor[rgb]{0.38,0.63,0.69}{\textbf{\textit{{#1}}}}}
    \newcommand{\CommentVarTok}[1]{\textcolor[rgb]{0.38,0.63,0.69}{\textbf{\textit{{#1}}}}}
    \newcommand{\VariableTok}[1]{\textcolor[rgb]{0.10,0.09,0.49}{{#1}}}
    \newcommand{\ControlFlowTok}[1]{\textcolor[rgb]{0.00,0.44,0.13}{\textbf{{#1}}}}
    \newcommand{\OperatorTok}[1]{\textcolor[rgb]{0.40,0.40,0.40}{{#1}}}
    \newcommand{\BuiltInTok}[1]{{#1}}
    \newcommand{\ExtensionTok}[1]{{#1}}
    \newcommand{\PreprocessorTok}[1]{\textcolor[rgb]{0.74,0.48,0.00}{{#1}}}
    \newcommand{\AttributeTok}[1]{\textcolor[rgb]{0.49,0.56,0.16}{{#1}}}
    \newcommand{\InformationTok}[1]{\textcolor[rgb]{0.38,0.63,0.69}{\textbf{\textit{{#1}}}}}
    \newcommand{\WarningTok}[1]{\textcolor[rgb]{0.38,0.63,0.69}{\textbf{\textit{{#1}}}}}
    
    
    % Define a nice break command that doesn't care if a line doesn't already
    % exist.
    \def\br{\hspace*{\fill} \\* }
    % Math Jax compatability definitions
    \def\gt{>}
    \def\lt{<}
    % Document parameters
    \title{1SN\_LangageC\_C2}
    
    
    

    % Pygments definitions
    
\makeatletter
\def\PY@reset{\let\PY@it=\relax \let\PY@bf=\relax%
    \let\PY@ul=\relax \let\PY@tc=\relax%
    \let\PY@bc=\relax \let\PY@ff=\relax}
\def\PY@tok#1{\csname PY@tok@#1\endcsname}
\def\PY@toks#1+{\ifx\relax#1\empty\else%
    \PY@tok{#1}\expandafter\PY@toks\fi}
\def\PY@do#1{\PY@bc{\PY@tc{\PY@ul{%
    \PY@it{\PY@bf{\PY@ff{#1}}}}}}}
\def\PY#1#2{\PY@reset\PY@toks#1+\relax+\PY@do{#2}}

\expandafter\def\csname PY@tok@w\endcsname{\def\PY@tc##1{\textcolor[rgb]{0.73,0.73,0.73}{##1}}}
\expandafter\def\csname PY@tok@c\endcsname{\let\PY@it=\textit\def\PY@tc##1{\textcolor[rgb]{0.25,0.50,0.50}{##1}}}
\expandafter\def\csname PY@tok@cp\endcsname{\def\PY@tc##1{\textcolor[rgb]{0.74,0.48,0.00}{##1}}}
\expandafter\def\csname PY@tok@k\endcsname{\let\PY@bf=\textbf\def\PY@tc##1{\textcolor[rgb]{0.00,0.50,0.00}{##1}}}
\expandafter\def\csname PY@tok@kp\endcsname{\def\PY@tc##1{\textcolor[rgb]{0.00,0.50,0.00}{##1}}}
\expandafter\def\csname PY@tok@kt\endcsname{\def\PY@tc##1{\textcolor[rgb]{0.69,0.00,0.25}{##1}}}
\expandafter\def\csname PY@tok@o\endcsname{\def\PY@tc##1{\textcolor[rgb]{0.40,0.40,0.40}{##1}}}
\expandafter\def\csname PY@tok@ow\endcsname{\let\PY@bf=\textbf\def\PY@tc##1{\textcolor[rgb]{0.67,0.13,1.00}{##1}}}
\expandafter\def\csname PY@tok@nb\endcsname{\def\PY@tc##1{\textcolor[rgb]{0.00,0.50,0.00}{##1}}}
\expandafter\def\csname PY@tok@nf\endcsname{\def\PY@tc##1{\textcolor[rgb]{0.00,0.00,1.00}{##1}}}
\expandafter\def\csname PY@tok@nc\endcsname{\let\PY@bf=\textbf\def\PY@tc##1{\textcolor[rgb]{0.00,0.00,1.00}{##1}}}
\expandafter\def\csname PY@tok@nn\endcsname{\let\PY@bf=\textbf\def\PY@tc##1{\textcolor[rgb]{0.00,0.00,1.00}{##1}}}
\expandafter\def\csname PY@tok@ne\endcsname{\let\PY@bf=\textbf\def\PY@tc##1{\textcolor[rgb]{0.82,0.25,0.23}{##1}}}
\expandafter\def\csname PY@tok@nv\endcsname{\def\PY@tc##1{\textcolor[rgb]{0.10,0.09,0.49}{##1}}}
\expandafter\def\csname PY@tok@no\endcsname{\def\PY@tc##1{\textcolor[rgb]{0.53,0.00,0.00}{##1}}}
\expandafter\def\csname PY@tok@nl\endcsname{\def\PY@tc##1{\textcolor[rgb]{0.63,0.63,0.00}{##1}}}
\expandafter\def\csname PY@tok@ni\endcsname{\let\PY@bf=\textbf\def\PY@tc##1{\textcolor[rgb]{0.60,0.60,0.60}{##1}}}
\expandafter\def\csname PY@tok@na\endcsname{\def\PY@tc##1{\textcolor[rgb]{0.49,0.56,0.16}{##1}}}
\expandafter\def\csname PY@tok@nt\endcsname{\let\PY@bf=\textbf\def\PY@tc##1{\textcolor[rgb]{0.00,0.50,0.00}{##1}}}
\expandafter\def\csname PY@tok@nd\endcsname{\def\PY@tc##1{\textcolor[rgb]{0.67,0.13,1.00}{##1}}}
\expandafter\def\csname PY@tok@s\endcsname{\def\PY@tc##1{\textcolor[rgb]{0.73,0.13,0.13}{##1}}}
\expandafter\def\csname PY@tok@sd\endcsname{\let\PY@it=\textit\def\PY@tc##1{\textcolor[rgb]{0.73,0.13,0.13}{##1}}}
\expandafter\def\csname PY@tok@si\endcsname{\let\PY@bf=\textbf\def\PY@tc##1{\textcolor[rgb]{0.73,0.40,0.53}{##1}}}
\expandafter\def\csname PY@tok@se\endcsname{\let\PY@bf=\textbf\def\PY@tc##1{\textcolor[rgb]{0.73,0.40,0.13}{##1}}}
\expandafter\def\csname PY@tok@sr\endcsname{\def\PY@tc##1{\textcolor[rgb]{0.73,0.40,0.53}{##1}}}
\expandafter\def\csname PY@tok@ss\endcsname{\def\PY@tc##1{\textcolor[rgb]{0.10,0.09,0.49}{##1}}}
\expandafter\def\csname PY@tok@sx\endcsname{\def\PY@tc##1{\textcolor[rgb]{0.00,0.50,0.00}{##1}}}
\expandafter\def\csname PY@tok@m\endcsname{\def\PY@tc##1{\textcolor[rgb]{0.40,0.40,0.40}{##1}}}
\expandafter\def\csname PY@tok@gh\endcsname{\let\PY@bf=\textbf\def\PY@tc##1{\textcolor[rgb]{0.00,0.00,0.50}{##1}}}
\expandafter\def\csname PY@tok@gu\endcsname{\let\PY@bf=\textbf\def\PY@tc##1{\textcolor[rgb]{0.50,0.00,0.50}{##1}}}
\expandafter\def\csname PY@tok@gd\endcsname{\def\PY@tc##1{\textcolor[rgb]{0.63,0.00,0.00}{##1}}}
\expandafter\def\csname PY@tok@gi\endcsname{\def\PY@tc##1{\textcolor[rgb]{0.00,0.63,0.00}{##1}}}
\expandafter\def\csname PY@tok@gr\endcsname{\def\PY@tc##1{\textcolor[rgb]{1.00,0.00,0.00}{##1}}}
\expandafter\def\csname PY@tok@ge\endcsname{\let\PY@it=\textit}
\expandafter\def\csname PY@tok@gs\endcsname{\let\PY@bf=\textbf}
\expandafter\def\csname PY@tok@gp\endcsname{\let\PY@bf=\textbf\def\PY@tc##1{\textcolor[rgb]{0.00,0.00,0.50}{##1}}}
\expandafter\def\csname PY@tok@go\endcsname{\def\PY@tc##1{\textcolor[rgb]{0.53,0.53,0.53}{##1}}}
\expandafter\def\csname PY@tok@gt\endcsname{\def\PY@tc##1{\textcolor[rgb]{0.00,0.27,0.87}{##1}}}
\expandafter\def\csname PY@tok@err\endcsname{\def\PY@bc##1{\setlength{\fboxsep}{0pt}\fcolorbox[rgb]{1.00,0.00,0.00}{1,1,1}{\strut ##1}}}
\expandafter\def\csname PY@tok@kc\endcsname{\let\PY@bf=\textbf\def\PY@tc##1{\textcolor[rgb]{0.00,0.50,0.00}{##1}}}
\expandafter\def\csname PY@tok@kd\endcsname{\let\PY@bf=\textbf\def\PY@tc##1{\textcolor[rgb]{0.00,0.50,0.00}{##1}}}
\expandafter\def\csname PY@tok@kn\endcsname{\let\PY@bf=\textbf\def\PY@tc##1{\textcolor[rgb]{0.00,0.50,0.00}{##1}}}
\expandafter\def\csname PY@tok@kr\endcsname{\let\PY@bf=\textbf\def\PY@tc##1{\textcolor[rgb]{0.00,0.50,0.00}{##1}}}
\expandafter\def\csname PY@tok@bp\endcsname{\def\PY@tc##1{\textcolor[rgb]{0.00,0.50,0.00}{##1}}}
\expandafter\def\csname PY@tok@fm\endcsname{\def\PY@tc##1{\textcolor[rgb]{0.00,0.00,1.00}{##1}}}
\expandafter\def\csname PY@tok@vc\endcsname{\def\PY@tc##1{\textcolor[rgb]{0.10,0.09,0.49}{##1}}}
\expandafter\def\csname PY@tok@vg\endcsname{\def\PY@tc##1{\textcolor[rgb]{0.10,0.09,0.49}{##1}}}
\expandafter\def\csname PY@tok@vi\endcsname{\def\PY@tc##1{\textcolor[rgb]{0.10,0.09,0.49}{##1}}}
\expandafter\def\csname PY@tok@vm\endcsname{\def\PY@tc##1{\textcolor[rgb]{0.10,0.09,0.49}{##1}}}
\expandafter\def\csname PY@tok@sa\endcsname{\def\PY@tc##1{\textcolor[rgb]{0.73,0.13,0.13}{##1}}}
\expandafter\def\csname PY@tok@sb\endcsname{\def\PY@tc##1{\textcolor[rgb]{0.73,0.13,0.13}{##1}}}
\expandafter\def\csname PY@tok@sc\endcsname{\def\PY@tc##1{\textcolor[rgb]{0.73,0.13,0.13}{##1}}}
\expandafter\def\csname PY@tok@dl\endcsname{\def\PY@tc##1{\textcolor[rgb]{0.73,0.13,0.13}{##1}}}
\expandafter\def\csname PY@tok@s2\endcsname{\def\PY@tc##1{\textcolor[rgb]{0.73,0.13,0.13}{##1}}}
\expandafter\def\csname PY@tok@sh\endcsname{\def\PY@tc##1{\textcolor[rgb]{0.73,0.13,0.13}{##1}}}
\expandafter\def\csname PY@tok@s1\endcsname{\def\PY@tc##1{\textcolor[rgb]{0.73,0.13,0.13}{##1}}}
\expandafter\def\csname PY@tok@mb\endcsname{\def\PY@tc##1{\textcolor[rgb]{0.40,0.40,0.40}{##1}}}
\expandafter\def\csname PY@tok@mf\endcsname{\def\PY@tc##1{\textcolor[rgb]{0.40,0.40,0.40}{##1}}}
\expandafter\def\csname PY@tok@mh\endcsname{\def\PY@tc##1{\textcolor[rgb]{0.40,0.40,0.40}{##1}}}
\expandafter\def\csname PY@tok@mi\endcsname{\def\PY@tc##1{\textcolor[rgb]{0.40,0.40,0.40}{##1}}}
\expandafter\def\csname PY@tok@il\endcsname{\def\PY@tc##1{\textcolor[rgb]{0.40,0.40,0.40}{##1}}}
\expandafter\def\csname PY@tok@mo\endcsname{\def\PY@tc##1{\textcolor[rgb]{0.40,0.40,0.40}{##1}}}
\expandafter\def\csname PY@tok@ch\endcsname{\let\PY@it=\textit\def\PY@tc##1{\textcolor[rgb]{0.25,0.50,0.50}{##1}}}
\expandafter\def\csname PY@tok@cm\endcsname{\let\PY@it=\textit\def\PY@tc##1{\textcolor[rgb]{0.25,0.50,0.50}{##1}}}
\expandafter\def\csname PY@tok@cpf\endcsname{\let\PY@it=\textit\def\PY@tc##1{\textcolor[rgb]{0.25,0.50,0.50}{##1}}}
\expandafter\def\csname PY@tok@c1\endcsname{\let\PY@it=\textit\def\PY@tc##1{\textcolor[rgb]{0.25,0.50,0.50}{##1}}}
\expandafter\def\csname PY@tok@cs\endcsname{\let\PY@it=\textit\def\PY@tc##1{\textcolor[rgb]{0.25,0.50,0.50}{##1}}}

\def\PYZbs{\char`\\}
\def\PYZus{\char`\_}
\def\PYZob{\char`\{}
\def\PYZcb{\char`\}}
\def\PYZca{\char`\^}
\def\PYZam{\char`\&}
\def\PYZlt{\char`\<}
\def\PYZgt{\char`\>}
\def\PYZsh{\char`\#}
\def\PYZpc{\char`\%}
\def\PYZdl{\char`\$}
\def\PYZhy{\char`\-}
\def\PYZsq{\char`\'}
\def\PYZdq{\char`\"}
\def\PYZti{\char`\~}
% for compatibility with earlier versions
\def\PYZat{@}
\def\PYZlb{[}
\def\PYZrb{]}
\makeatother


    % Exact colors from NB
    \definecolor{incolor}{rgb}{0.0, 0.0, 0.5}
    \definecolor{outcolor}{rgb}{0.545, 0.0, 0.0}



    
    % Prevent overflowing lines due to hard-to-break entities
    \sloppy 
    % Setup hyperref package
    \hypersetup{
      breaklinks=true,  % so long urls are correctly broken across lines
      colorlinks=true,
      urlcolor=urlcolor,
      linkcolor=linkcolor,
      citecolor=citecolor,
      }
    % Slightly bigger margins than the latex defaults
    
    \geometry{verbose,tmargin=1in,bmargin=1in,lmargin=1in,rmargin=1in}
    
    

    \begin{document}
    
    
    \maketitle
    
    

    
    \section{Langage C - Notebook C2}\label{langage-c---notebook-c2}

\subsection{Allocation dynamique}\label{allocation-dynamique}

\paragraph{Katia Jaffrès-Runser, Xavier
Crégut}\label{katia-jaffruxe8s-runser-xavier-cruxe9gut}

Toulouse INP - ENSEEIHT,

1ère année, Dept. Sciences du Numérique, 2020-2021.

    \subsection{\#\# 1. Déroulement du cours}\label{duxe9roulement-du-cours}

Ce cours se déroule sur 6 séances de TP.

\begin{itemize}
\tightlist
\item
  Lors des trois premières séances, vous avez suivi le sujet C1 sous la
  forme d'un notebook Jupyter.
\item
  Lors des trois dernières séances, vous suivrez deux autres notebook
  Jupyter, C2 et C3, à votre rythme.
\end{itemize}

Chaque sujet, C1, C2 et C3, se termine par un exercice Bilan à rendre
via votre dépot SVN. Les échéances sont indiquées sur Moodle. Les 3
exercices bilans sont notés, et leur moyenne fournit une note
d'exercices.

Vous aurez, en fin de cours, un QCM d'une heure. La note finale est une
moyenne des deux notes (QCM et exercices rendus).

    \subsection{2. Objectifs}\label{objectifs}

Ce cours, sous la forme de notebooks Jupyter et d'un ensemble
d'exercices à réaliser en TP, a pour objectif de vous présenter les
spécificités de la programmation en langage C. Il se base sur vos acquis
du cours de Programmation Impérative en algorithmique et vous détaille
les éléments du langage C nécessaires à la production d'un programme en
C.

Un support de cours PDF vous est également fournit sur Moodle :
\href{http://moodle-n7.inp-toulouse.fr/pluginfile.php/49240/mod_resource/content/5/LangageC_poly.pdf}{Cours
C}.

    \subsection{\#\# 3. Plan du sujet C2.}\label{plan-du-sujet-c2.}

Ce sujet se focalise sur l'allocation dynamique de mémoire en C. Il vous
présente :

\begin{itemize}
\tightlist
\item
  Les principaux allocateurs de mémoire, et leur utilisation.
\item
  La libération de mémoire, et son utilisation.
\item
  La distinction entre manipulation d'un tableau statique et dynamique
  en C
\item
  Le sous-programme de réallocation de mémoire
\end{itemize}

    \begin{center}\rule{0.5\linewidth}{\linethickness}\end{center}

    \subsection{Rappel : Jupyter notebook}\label{rappel-jupyter-notebook}

Le support de cours que vous lisez est un notebook Jupyter. Pour
visualiser le notebook, lancer l'editeur web avec la commande\\
\textgreater{} \texttt{jupyter-notebook}

et rechercher le fichier dans l'arborescence. Le fichier est édité dans
votre navigateur Web par défaut. L'enregistrement est automatique
(\texttt{CTRL\ S} pour le forcer).

Pour fermer votre fichier, il faut fermer le navigateur et terminer le
processus serveur qui s'exécute dans le terminal (\texttt{CTRL\ C}, puis
\texttt{y}).

\begin{quote}
\textbf{Important} : - Pour faire fonctionner le kernel C de jupyter
notebook, il faut, avant une \textbf{première utilisation} de Notebook,
lancer la commande suivante dans un \texttt{Terminal} :
\end{quote}

\begin{quote}
\texttt{install\_c\_kernel\ -\/-user}
\end{quote}

Ce notebook se compose de cellules présentant soit : - Des éléments de
cours, au format
\href{https://fr.wikipedia.org/wiki/Markdown}{Markdown}. Ce langage est
traduit en HTML pour un affichage aisé quand on clique sur la flèche
\texttt{Exécuter\ (run)} et que la cellule est active. - Du code en
Langage C (ou Python, ou autre..). Pour compiler et exécuter le code
écrit dans la cellule active, on clique sur la flèche
\texttt{Exécuter\ (run)}. Si la compilation se déroule sans erreur ni
avertissement, le programme est exécuté et les sorties sont affichées en
bas de la cellule. Si ce n'est pas le cas, les avertissements et
warnings sont affichés en bas de la cellule.

En double-cliquant sur une cellule, on peut éditer son contenu. Vous
pouvez ainsi : - Editer une cellule markdown pour y intégrer vos propres
notes. - Modifier les programmes pour répondre aux questions et
exercices proposés.

Il est possible d'exporter votre travail en PDF, HTML, etc. Il est aussi
possible d'afficher les numéros de ligne dans le menu
\textbf{Affichage}.

Le programme dans la cellule suivante s'exécute sans erreur. Vous pouvez
- le tester en l'exécutant. - y introduire une erreur (suppression d'un
point-virgule par exemple) pour observer la sortie du compilateur.

    \begin{Verbatim}[commandchars=\\\{\}]
{\color{incolor}In [{\color{incolor}1}]:} \PY{c+cp}{\PYZsh{}}\PY{c+cp}{include} \PY{c+cpf}{\PYZlt{}stdlib.h\PYZgt{}}\PY{c+c1}{ }
        \PY{c+cp}{\PYZsh{}}\PY{c+cp}{include} \PY{c+cpf}{\PYZlt{}stdio.h\PYZgt{}}
        \PY{k+kt}{int} \PY{n+nf}{main}\PY{p}{(}\PY{p}{)}\PY{p}{\PYZob{}}
            \PY{n}{printf}\PY{p}{(}\PY{l+s}{\PYZdq{}}\PY{l+s}{******************************}\PY{l+s+se}{\PYZbs{}n}\PY{l+s}{\PYZdq{}}\PY{p}{)}\PY{p}{;}
            \PY{n}{printf}\PY{p}{(}\PY{l+s}{\PYZdq{}}\PY{l+s}{******** Langage C ***********}\PY{l+s+se}{\PYZbs{}n}\PY{l+s}{\PYZdq{}}\PY{p}{)}\PY{p}{;}
            \PY{n}{printf}\PY{p}{(}\PY{l+s}{\PYZdq{}}\PY{l+s}{******************************}\PY{l+s+se}{\PYZbs{}n}\PY{l+s}{\PYZdq{}}\PY{p}{)}\PY{p}{;}
            \PY{k}{return} \PY{n}{EXIT\PYZus{}SUCCESS}\PY{p}{;}
        \PY{p}{\PYZcb{}}
\end{Verbatim}


    \begin{Verbatim}[commandchars=\\\{\}]
******************************
******** Langage C ***********
******************************

    \end{Verbatim}

    \begin{center}\rule{0.5\linewidth}{\linethickness}\end{center}

    \subsection{4. Allocation dynamique de
mémoire}\label{allocation-dynamique-de-muxe9moire}

    \subsubsection{4.1 Structure de la
mémoire}\label{structure-de-la-muxe9moire}

La mémoire vive de votre ordinateur est structurée en différentes
parties : - \textbf{Le mémoire statique} Zone de la mémoire où sont
stockées les données qui ont la même durée de vie que le programme
(variables globales). - \textbf{La mémoire automatique} Zone de la
mémoire appelée \textbf{pile d'exécution} où sont stockés les blocs
d'activation, paramètres et variables locales des sous-programmes. Cette
mémoire est gérée automatiquement par le compilateur (réservation et
libération). La mémoire est contigüe (sans trous). - \textbf{La mémoire
dynamique} Zone de la mémoire aussi appelée \textbf{tas} dans laquelle
le programmeur peut explicitement réserver (allouer) de la place. Il
devra la libérer explicitement. Cette zone est fragmentée (trous).

    \paragraph{Utilisation de la mémoire
dynamique}\label{utilisation-de-la-muxe9moire-dynamique}

L'enregistrement d'une donnée dans une zone de la mémoire dynamique
nécessite une \textbf{demande d'allocation explicite} de ladite zone.
Cette zone mémoire est référencée à traver \textbf{un pointeur}. Ainsi,
l'écriture et la lecture de la donnée se fait exclusivement par ce
pointeur.

Quand la zone mémoire n'est plus utile, il faut \textbf{demander
explicitement la libération} de l'espace mémoire en C. Attention, quand
on réalise cette opération, il faut s'assurer qu'aucun pointeur ne
référence plus cette zone mémoire un fois désallouée.

    \subsubsection{4.2 Allocation de
mémoire}\label{allocation-de-muxe9moire}

La bibliothèque \texttt{stdlib.h} (ou \texttt{malloc.h}) offre 3
procédures d'allocation de mémoire : \texttt{malloc}, \texttt{calloc} et
\texttt{realloc}. Elle définit aussi une procédure de libération de
mémoire \texttt{free}.

\paragraph{\texorpdfstring{L'allocateur
\texttt{malloc}}{L'allocateur malloc}}\label{lallocateur-malloc}

C'est l'allocateur utilisé le plus couramment : \textgreater{}
\texttt{void*\ malloc(size\_t\ taille);}

Ici on trouve : - Le type de retour \texttt{void\ *} qui représente un
type pointeur générique sur une zone mémoire. Le pointeur retourné vaut
\texttt{NULL} si l'allocation échoue (manque d'espace mémoire contigüe).
- Le type \texttt{size\_t} qui est un alias de \texttt{unsigned\ int} et
représente la \textbf{taille en octets} de la zone mémoire réservée.

\paragraph{\texorpdfstring{L'opérateur
\texttt{sizeof}}{L'opérateur sizeof}}\label{lopuxe9rateur-sizeof}

\begin{itemize}
\tightlist
\item
  Pour obtenir la taille en octets d'une variable ou d'un type, on
  utilise la fonction \texttt{sizeof()} : \textgreater{}
  \texttt{sizeof(ma\_variable)} ou \texttt{sizeof(type)}
\end{itemize}

    \subparagraph{Exemples d'allocations :}\label{exemples-dallocations}

\begin{verbatim}
    char* un_char = malloc(sizeof(char));
    char* autre_char = malloc(sizeof(*autre_char)); //taille du type pointé
    
    enum genre = {H, F, NC};
    enum genre * il = malloc(sizeof(enum genre));
\end{verbatim}

    \paragraph{\texorpdfstring{L'allocateur
\texttt{calloc}}{L'allocateur calloc}}\label{lallocateur-calloc}

C'est une variante de malloc : \textgreater{}
\texttt{void*\ calloc(size\_t\ nombre,\ size\_t\ taille\_element);}

Ici, la taille de la zone mémoire allouée est décrite avec deux
paramètres : - La taille d'un élément avec \texttt{taille\_element} - Et
le nombre d'éléments de cette taille avec \texttt{nombre}. Ainsi, on
alloue \texttt{nombre\ *\ taille\_element} octets.

\begin{quote}
\textbf{Note :} À la différence de malloc, tous les bits de la zone
allouée sont positionnés à zéro.
\end{quote}

    \paragraph{\texorpdfstring{Libérer la mémoire avec \texttt{free}
:}{Libérer la mémoire avec free :}}\label{libuxe9rer-la-muxe9moire-avec-free}

Désallouer se fait avec la fonction : \textgreater{}
\texttt{void\ free(void*\ pointeur);}

L'unique paramètre est le pointeur qui désigne l'adresse de la mémoire à
désallouer. \textgreater{} \textbf{Attention} la libération ne modifie
pas l'adresse enregistrée dans le pointeur. Il faut explicitement
oublier l'adresse non-valide en initialisant le pointeur à \texttt{NULL}
:

\begin{Shaded}
\begin{Highlighting}[]
\NormalTok{    free(ptr_int);}
\NormalTok{    ptr_int = NULL;}
\end{Highlighting}
\end{Shaded}

    Voici un exemple d'allocation, utilisation et libération de la mémoire
(cf. fichier \textbf{\texttt{Exemple4\_2.c}}) :

    \begin{Verbatim}[commandchars=\\\{\}]
{\color{incolor}In [{\color{incolor}2}]:} \PY{c+cp}{\PYZsh{}}\PY{c+cp}{include} \PY{c+cpf}{\PYZlt{}stdio.h\PYZgt{}}
        \PY{c+cp}{\PYZsh{}}\PY{c+cp}{include} \PY{c+cpf}{\PYZlt{}stdlib.h\PYZgt{}}
        \PY{c+cp}{\PYZsh{}}\PY{c+cp}{include} \PY{c+cpf}{\PYZlt{}assert.h\PYZgt{}}
        
        \PY{k+kt}{void} \PY{n+nf}{exemple\PYZus{}dynamique}\PY{p}{(}\PY{p}{)}\PY{p}{\PYZob{}}
            \PY{c+c1}{//Allouer dynamiquement un entier}
            \PY{k+kt}{unsigned} \PY{k+kt}{int} \PY{n}{taille} \PY{o}{=} \PY{k}{sizeof}\PY{p}{(}\PY{k+kt}{int}\PY{p}{)}\PY{p}{;}
            \PY{k+kt}{int} \PY{o}{*}\PY{n}{mon\PYZus{}entier} \PY{o}{=} \PY{n}{malloc}\PY{p}{(}\PY{n}{taille}\PY{p}{)}\PY{p}{;}
            \PY{c+c1}{//Vérifier le succès de la demande d\PYZsq{}allocation}
            \PY{n}{assert}\PY{p}{(}\PY{n}{mon\PYZus{}entier} \PY{o}{!}\PY{o}{=} \PY{n+nb}{NULL}\PY{p}{)}\PY{p}{;}
            
            \PY{c+c1}{//Initialiser la donnée à travers le pointeur mon\PYZus{}entier}
            \PY{o}{*}\PY{n}{mon\PYZus{}entier} \PY{o}{=} \PY{l+m+mi}{10}\PY{p}{;}
            \PY{c+c1}{//Accéder à la donnée}
            \PY{n}{printf}\PY{p}{(}\PY{l+s}{\PYZdq{}}\PY{l+s}{Donnée enregistrée : \PYZpc{}d}\PY{l+s+se}{\PYZbs{}n}\PY{l+s}{\PYZdq{}}\PY{p}{,} \PY{o}{*}\PY{n}{mon\PYZus{}entier}\PY{p}{)}\PY{p}{;}
            
            \PY{c+c1}{//Libérer la mémoire dynamique}
            \PY{n}{free}\PY{p}{(}\PY{n}{mon\PYZus{}entier}\PY{p}{)}\PY{p}{;}
            \PY{c+c1}{//Oublier l\PYZsq{}adresse mémoire}
            \PY{n}{mon\PYZus{}entier} \PY{o}{=} \PY{n+nb}{NULL}\PY{p}{;}
        \PY{p}{\PYZcb{}}
        
        \PY{k+kt}{int} \PY{n+nf}{main}\PY{p}{(}\PY{p}{)} \PY{p}{\PYZob{}}
            \PY{n}{exemple\PYZus{}dynamique}\PY{p}{(}\PY{p}{)}\PY{p}{;}
            \PY{k}{return} \PY{n}{EXIT\PYZus{}SUCCESS}\PY{p}{;}
        \PY{p}{\PYZcb{}}
\end{Verbatim}


    \begin{Verbatim}[commandchars=\\\{\}]
Donnée enregistrée : 10

    \end{Verbatim}

    On observe les éléments suivants :

\begin{itemize}
\item
  L'allocation est ici réalisée avec la fonction \texttt{malloc(taille)}
  qui retourne un pointeur. Elle demande l'allocation de \texttt{taille}
  octets. Si l'allocation est réalisée avec succès, elle retourne
  l'adresse de la zone mémoire via le pointeur. Sinon, elle retourne
  \texttt{NULL}.
\item
  L'accès à la donnée est réalisé via le pointeur \texttt{mon\_entier}
\item
  La libération de la mémoire utilise la fonction
  \texttt{free(mon\_entier)}.
\end{itemize}

\begin{quote}
\textbf{Attention} la libération ne modifie pas l'adresse enregistrée
dans le pointeur. Il faut explicitement oublier l'adresse non-valide en
initialisant le pointeur à \texttt{NULL}.
\end{quote}

    \subsubsection{\texorpdfstring{4.3 Allocation d'un tableau avec
\texttt{malloc}}{4.3 Allocation d'un tableau avec malloc}}\label{allocation-dun-tableau-avec-malloc}

Pour allouer un tableau de N éléments dynamiquement, il suffit de
demander l'espace mémoire pour contenir les N éléments :

\begin{verbatim}
    // Allocation d'un tableau de 10 entiers
    int* mon_tableau = malloc(10*sizeof(int));
\end{verbatim}

On peut alors utiliser la notation \texttt{mon\_tableau{[}..{]}} pour
accéder aux éléments du tableau. En effet, les 10 entiers sont
enregistrés dans une portion de mémoire dynamique contigüe.
\texttt{mon\_tableau} est un pointeur qui comporte l'adresse de la
première case du tableau, et l'accès à la 3ème case se fait par simple
décalage d'indice avec l'opérateur \texttt{{[}{]}}.

\begin{verbatim}
    mon_tableau[2] = 20;
\end{verbatim}

\begin{quote}
\textbf{Attention} : \texttt{sizeof(mon\_tableau)} retourne uniquement
la taille du pointeur \texttt{mon\_tableau} alloué dynamiquement ! Par
contre, si un tableau est alloué statiquement, \texttt{sizeof} retourne
la taille totale du tableau.
\end{quote}

    \subparagraph{Exemple}\label{exemple}

Voici un exemple d'allocation de tableaux dynamiques et statiques, et de
la valeur retournée par \texttt{sizeof}. Vous pouvez manipuler le
fichier \textbf{\texttt{Exemple4\_3.c}}.

    \begin{Verbatim}[commandchars=\\\{\}]
{\color{incolor}In [{\color{incolor}3}]:} \PY{c+cp}{\PYZsh{}}\PY{c+cp}{include} \PY{c+cpf}{\PYZlt{}assert.h\PYZgt{}}
        \PY{c+cp}{\PYZsh{}}\PY{c+cp}{include} \PY{c+cpf}{\PYZlt{}stdlib.h\PYZgt{}}
        \PY{c+cp}{\PYZsh{}}\PY{c+cp}{include} \PY{c+cpf}{\PYZlt{}stdio.h\PYZgt{}}
        
        \PY{k+kt}{int} \PY{n+nf}{main}\PY{p}{(}\PY{p}{)}\PY{p}{\PYZob{}}
            \PY{k+kt}{int} \PY{n}{taille\PYZus{}entier} \PY{o}{=} \PY{k}{sizeof}\PY{p}{(}\PY{k+kt}{int}\PY{p}{)}\PY{p}{;}
            \PY{n}{printf}\PY{p}{(}\PY{l+s}{\PYZdq{}}\PY{l+s}{Taille d\PYZsq{}un entier : \PYZpc{}d}\PY{l+s+se}{\PYZbs{}n}\PY{l+s}{\PYZdq{}}\PY{p}{,} \PY{n}{taille\PYZus{}entier}\PY{p}{)}\PY{p}{;}
            \PY{k+kt}{int} \PY{n}{taille\PYZus{}pointeur} \PY{o}{=} \PY{k}{sizeof}\PY{p}{(}\PY{k+kt}{int}\PY{o}{*}\PY{p}{)}\PY{p}{;}
            \PY{n}{printf}\PY{p}{(}\PY{l+s}{\PYZdq{}}\PY{l+s}{Taille d\PYZsq{}un pointeur : \PYZpc{}d}\PY{l+s+se}{\PYZbs{}n}\PY{l+s}{\PYZdq{}}\PY{p}{,} \PY{n}{taille\PYZus{}pointeur}\PY{p}{)}\PY{p}{;}
            
            \PY{c+c1}{// Allouer un tableau de 10 entiers}
            \PY{k+kt}{int}\PY{o}{*} \PY{n}{mon\PYZus{}tableau} \PY{o}{=} \PY{n}{malloc}\PY{p}{(}\PY{l+m+mi}{10}\PY{o}{*}\PY{n}{taille\PYZus{}entier}\PY{p}{)}\PY{p}{;}
            \PY{k+kt}{int} \PY{n}{taille\PYZus{}dynamique} \PY{o}{=} \PY{k}{sizeof}\PY{p}{(}\PY{n}{mon\PYZus{}tableau}\PY{p}{)}\PY{p}{;}
            \PY{n}{assert}\PY{p}{(}\PY{n}{taille\PYZus{}dynamique} \PY{o}{=}\PY{o}{=} \PY{n}{taille\PYZus{}pointeur}\PY{p}{)}\PY{p}{;}
            
            \PY{c+c1}{// Declarer un tableau statique de 10 entiers }
            \PY{k+kt}{int} \PY{n}{mon\PYZus{}tab}\PY{p}{[}\PY{l+m+mi}{10}\PY{p}{]}\PY{p}{;}
            \PY{k+kt}{int} \PY{n}{taille\PYZus{}statique} \PY{o}{=} \PY{k}{sizeof}\PY{p}{(}\PY{n}{mon\PYZus{}tab}\PY{p}{)}\PY{p}{;}
            \PY{n}{assert}\PY{p}{(}\PY{n}{taille\PYZus{}statique} \PY{o}{=}\PY{o}{=} \PY{l+m+mi}{10}\PY{o}{*}\PY{n}{taille\PYZus{}entier}\PY{p}{)}\PY{p}{;}
            
            \PY{n}{printf}\PY{p}{(}\PY{l+s}{\PYZdq{}}\PY{l+s}{\PYZpc{}s}\PY{l+s}{\PYZdq{}}\PY{p}{,} \PY{l+s}{\PYZdq{}}\PY{l+s}{Bravo ! Tous les tests passent.}\PY{l+s+se}{\PYZbs{}n}\PY{l+s}{\PYZdq{}}\PY{p}{)}\PY{p}{;}
            \PY{k}{return} \PY{n}{EXIT\PYZus{}SUCCESS}\PY{p}{;}
        \PY{p}{\PYZcb{}}
\end{Verbatim}


    \begin{Verbatim}[commandchars=\\\{\}]
Taille d'un entier : 4
Taille d'un pointeur : 8
Bravo ! Tous les tests passent.

    \end{Verbatim}

    \paragraph{Conséquence sur la définition d'un type tableau
dynamique}\label{consuxe9quence-sur-la-duxe9finition-dun-type-tableau-dynamique}

Il n'est pas possible d'utiliser \texttt{sizeof} pour connaitre la
taille d'un tableau dynamique.

\begin{quote}
\textbf{Bonne pratique :} Il convient donc d'enregistrer dans une
variable la capacité actuelle du tableau à l'aide d'une enregistrement
\end{quote}

\begin{verbatim}
    struct tab {
        int* tableau; //Le tableau, alloué dynamiquement à l'initialisation
        int capacite; //La capacité
    };
    typedef struct tab tab;
\end{verbatim}

    \begin{center}\rule{0.5\linewidth}{\linethickness}\end{center}

    \subsubsection{Exercice 1a : Manipuler les
allocateurs}\label{exercice-1a-manipuler-les-allocateurs}

Cet exercice a pour but de vous faire manipuler l'allocateur
\texttt{malloc} et de libérer la mémoire avec \texttt{free}. Vous pouvez
compléter le fichier \textbf{\texttt{Exercice1a.c}} si besoin.

    \begin{Verbatim}[commandchars=\\\{\}]
{\color{incolor}In [{\color{incolor}13}]:} \PY{c+cp}{\PYZsh{}}\PY{c+cp}{include} \PY{c+cpf}{\PYZlt{}assert.h\PYZgt{}}
         \PY{c+cp}{\PYZsh{}}\PY{c+cp}{include} \PY{c+cpf}{\PYZlt{}stdlib.h\PYZgt{}}
         \PY{c+cp}{\PYZsh{}}\PY{c+cp}{include} \PY{c+cpf}{\PYZlt{}stdio.h\PYZgt{}}
         
         \PY{c+c1}{// Consignes pour une obtenir une exécution sans erreur : }
         \PY{c+c1}{//     \PYZhy{} compléter les instruction **** TODO **** }
         \PY{c+c1}{// Attention : toutes les variables sont ici allouées et libérées dynamiquent}
         
         \PY{k+kt}{int} \PY{n+nf}{main}\PY{p}{(}\PY{p}{)}\PY{p}{\PYZob{}}
         
              \PY{c+c1}{//un entier en mémoire dynamique }
             \PY{k+kt}{unsigned} \PY{k+kt}{int} \PY{n}{taille} \PY{o}{=} \PY{k}{sizeof}\PY{p}{(}\PY{k+kt}{int}\PY{p}{)}\PY{p}{;}
             \PY{k+kt}{int} \PY{o}{*}\PY{n}{ptr\PYZus{}int} \PY{o}{=} \PY{n}{malloc}\PY{p}{(}\PY{n}{taille}\PY{p}{)}\PY{p}{;}
             \PY{o}{*}\PY{n}{ptr\PYZus{}int} \PY{o}{=} \PY{l+m+mi}{100}\PY{p}{;}
             \PY{c+c1}{// Allocation et initialisation à la valeur 100;}
         
         
             \PY{n}{assert}\PY{p}{(}\PY{o}{*}\PY{n}{ptr\PYZus{}int} \PY{o}{=}\PY{o}{=} \PY{l+m+mi}{100}\PY{p}{)}\PY{p}{;}
         
                 
             \PY{n}{free} \PY{p}{(}\PY{n}{ptr\PYZus{}int}\PY{p}{)}\PY{p}{;}
             \PY{n}{ptr\PYZus{}int} \PY{o}{=} \PY{n+nb}{NULL}\PY{p}{;}
             \PY{c+c1}{//Libérer toute la mémoire dynamique}
             
             \PY{n}{assert}\PY{p}{(}\PY{o}{!}\PY{n}{ptr\PYZus{}int}\PY{p}{)}\PY{p}{;}
         
             \PY{n}{printf}\PY{p}{(}\PY{l+s}{\PYZdq{}}\PY{l+s}{\PYZpc{}s}\PY{l+s}{\PYZdq{}}\PY{p}{,} \PY{l+s}{\PYZdq{}}\PY{l+s}{Bravo ! Tous les tests passent.}\PY{l+s+se}{\PYZbs{}n}\PY{l+s}{\PYZdq{}}\PY{p}{)}\PY{p}{;}
             \PY{k}{return} \PY{n}{EXIT\PYZus{}SUCCESS}\PY{p}{;}
         \PY{p}{\PYZcb{}}
\end{Verbatim}


    \begin{Verbatim}[commandchars=\\\{\}]
Bravo ! Tous les tests passent.

    \end{Verbatim}

    \subsubsection{Exercice 1b : Manipuler les
allocateurs}\label{exercice-1b-manipuler-les-allocateurs}

Cet exercice a pour but de vous faire manipuler l'allocateur
\texttt{calloc} et de libérer la mémoire avec \texttt{free}. Vous pouvez
compléter le fichier \textbf{\texttt{Exercice1b.c}} si besoin.

    \begin{Verbatim}[commandchars=\\\{\}]
{\color{incolor}In [{\color{incolor}15}]:} \PY{c+cp}{\PYZsh{}}\PY{c+cp}{include} \PY{c+cpf}{\PYZlt{}assert.h\PYZgt{}}
         \PY{c+cp}{\PYZsh{}}\PY{c+cp}{include} \PY{c+cpf}{\PYZlt{}stdlib.h\PYZgt{}}
         \PY{c+cp}{\PYZsh{}}\PY{c+cp}{include} \PY{c+cpf}{\PYZlt{}stdio.h\PYZgt{}}
         
         \PY{c+c1}{// Consignes pour une obtenir une exécution sans erreur : }
         \PY{c+c1}{//     \PYZhy{} compléter les instruction **** TODO **** }
         \PY{c+c1}{// Attention : toutes les variables sont ici allouées et libérées dynamiquent}
         
         \PY{k+kt}{int} \PY{n+nf}{main}\PY{p}{(}\PY{p}{)}\PY{p}{\PYZob{}}
         
         
              \PY{c+c1}{//un réel en mémoire dynamique }
             \PY{c+c1}{// **** TODO ****}
             \PY{k+kt}{unsigned} \PY{k+kt}{int} \PY{n}{taille\PYZus{}element} \PY{o}{=} \PY{k}{sizeof}\PY{p}{(}\PY{k+kt}{float}\PY{p}{)}\PY{p}{;}
             \PY{k+kt}{float}\PY{o}{*} \PY{n}{ptr\PYZus{}float} \PY{o}{=} \PY{n}{calloc}\PY{p}{(}\PY{l+m+mi}{10}\PY{p}{,}\PY{n}{taille\PYZus{}element}\PY{p}{)}\PY{p}{;}
             \PY{c+c1}{// Allocation du réel avec CALLOC;}
         
             \PY{n}{assert}\PY{p}{(}\PY{o}{*}\PY{n}{ptr\PYZus{}float} \PY{o}{=}\PY{o}{=} \PY{l+m+mf}{0.0}\PY{p}{)}\PY{p}{;}
         
             \PY{c+c1}{//**** TODO **** }
             \PY{n}{free}\PY{p}{(}\PY{n}{ptr\PYZus{}float}\PY{p}{)}\PY{p}{;}
             \PY{n}{ptr\PYZus{}float} \PY{o}{=} \PY{n+nb}{NULL}\PY{p}{;}
             \PY{c+c1}{//Libérer toute la mémoire dynamique}
             \PY{n}{assert}\PY{p}{(}\PY{o}{!}\PY{n}{ptr\PYZus{}float}\PY{p}{)}\PY{p}{;}
                 
             \PY{n}{printf}\PY{p}{(}\PY{l+s}{\PYZdq{}}\PY{l+s}{\PYZpc{}s}\PY{l+s}{\PYZdq{}}\PY{p}{,} \PY{l+s}{\PYZdq{}}\PY{l+s}{Bravo ! Tous les tests passent.}\PY{l+s+se}{\PYZbs{}n}\PY{l+s}{\PYZdq{}}\PY{p}{)}\PY{p}{;}
             \PY{k}{return} \PY{n}{EXIT\PYZus{}SUCCESS}\PY{p}{;}
         \PY{p}{\PYZcb{}}
\end{Verbatim}


    \begin{Verbatim}[commandchars=\\\{\}]
Bravo ! Tous les tests passent.

    \end{Verbatim}

    \subsubsection{Exercice 1c : Manipuler les
allocateurs}\label{exercice-1c-manipuler-les-allocateurs}

Cet exercice a pour but de vous faire manipuler l'allocateur
\texttt{calloc} et de libérer la mémoire avec \texttt{free}. Vous pouvez
compléter le fichier \textbf{\texttt{Exercice1c.c}} si besoin.

    \begin{Verbatim}[commandchars=\\\{\}]
{\color{incolor}In [{\color{incolor}24}]:} \PY{c+cp}{\PYZsh{}}\PY{c+cp}{define XXX 1}
         
         \PY{c+cp}{\PYZsh{}}\PY{c+cp}{include} \PY{c+cpf}{\PYZlt{}assert.h\PYZgt{}}
         \PY{c+cp}{\PYZsh{}}\PY{c+cp}{include} \PY{c+cpf}{\PYZlt{}stdlib.h\PYZgt{}}
         \PY{c+cp}{\PYZsh{}}\PY{c+cp}{include} \PY{c+cpf}{\PYZlt{}stdio.h\PYZgt{}}
         
         \PY{c+c1}{// Consignes pour une obtenir une exécution sans erreur : }
         \PY{c+c1}{//     \PYZhy{} Remplacer XXX par le bon résultat dans la suite.}
         \PY{c+c1}{// Attention : toutes les variables sont ici allouées et libérées dynamiquent}
         
         \PY{k+kt}{int} \PY{n+nf}{main}\PY{p}{(}\PY{p}{)}\PY{p}{\PYZob{}}
         
             \PY{k}{enum} \PY{n}{chat} \PY{p}{\PYZob{}}\PY{n}{SIAMOIS}\PY{p}{,} \PY{n}{CALICO}\PY{p}{,} \PY{n}{PERSAN}\PY{p}{,} \PY{n}{TABBY}\PY{p}{\PYZcb{}}\PY{p}{;}
             \PY{k}{enum} \PY{n}{chat} \PY{o}{*} \PY{n}{my\PYZus{}cat}\PY{p}{;}
             \PY{n}{my\PYZus{}cat} \PY{o}{=} \PY{n}{calloc}\PY{p}{(}\PY{l+m+mi}{1}\PY{p}{,} \PY{k}{sizeof}\PY{p}{(}\PY{k}{enum} \PY{n}{chat}\PY{p}{)}\PY{p}{)}\PY{p}{;}
             \PY{n}{assert}\PY{p}{(}\PY{o}{*}\PY{n}{my\PYZus{}cat} \PY{o}{=}\PY{o}{=} \PY{n}{SIAMOIS}\PY{p}{)}\PY{p}{;}
         
             \PY{c+c1}{//**** TODO **** }
             \PY{n}{free}\PY{p}{(}\PY{n}{my\PYZus{}cat}\PY{p}{)}\PY{p}{;}
             \PY{n}{my\PYZus{}cat}\PY{o}{=} \PY{n+nb}{NULL}\PY{p}{;}
             \PY{c+c1}{//Libérer toute la mémoire dynamique}
             
             \PY{n}{assert}\PY{p}{(}\PY{o}{!}\PY{n}{my\PYZus{}cat}\PY{p}{)}\PY{p}{;}
         
             \PY{n}{printf}\PY{p}{(}\PY{l+s}{\PYZdq{}}\PY{l+s}{\PYZpc{}s}\PY{l+s}{\PYZdq{}}\PY{p}{,} \PY{l+s}{\PYZdq{}}\PY{l+s}{Bravo ! Tous les tests passent.}\PY{l+s+se}{\PYZbs{}n}\PY{l+s}{\PYZdq{}}\PY{p}{)}\PY{p}{;}
             \PY{k}{return} \PY{n}{EXIT\PYZus{}SUCCESS}\PY{p}{;}
         \PY{p}{\PYZcb{}}
\end{Verbatim}


    \begin{Verbatim}[commandchars=\\\{\}]
Bravo ! Tous les tests passent.

    \end{Verbatim}

    \subsubsection{Exercice 1.d : Manipuler les
allocateurs}\label{exercice-1.d-manipuler-les-allocateurs}

Cet exercice a pour but de vous faire allouer de la mémoire pour
enregistrer un tableau de caractères. Vous pouvez completer le fichier
\textbf{\texttt{Exercice1d.c}} si besoin.

Pour rappel, en C, une chaine de caratères est un tableau de caractères
qui termine par le caractère \texttt{\textbackslash{}0}. Cet exercice
utilise la bibliothèque
\href{https://fr.wikipedia.org/wiki/String.h}{\texttt{string.h}} qui
offre des sous-programmes permettant de manipuler des chaines de
caractère.

    \begin{Verbatim}[commandchars=\\\{\}]
{\color{incolor}In [{\color{incolor}28}]:} \PY{c+cp}{\PYZsh{}}\PY{c+cp}{include} \PY{c+cpf}{\PYZlt{}assert.h\PYZgt{}}
         \PY{c+cp}{\PYZsh{}}\PY{c+cp}{include} \PY{c+cpf}{\PYZlt{}stdlib.h\PYZgt{}}
         \PY{c+cp}{\PYZsh{}}\PY{c+cp}{include} \PY{c+cpf}{\PYZlt{}stdio.h\PYZgt{}}
         \PY{c+cp}{\PYZsh{}}\PY{c+cp}{include} \PY{c+cpf}{\PYZlt{}string.h\PYZgt{}}
         
         \PY{c+c1}{// Consignes pour une obtenir une exécution sans erreur : }
         \PY{c+c1}{//     \PYZhy{} compléter les instruction **** TODO ****}
         \PY{c+c1}{// Attention : toutes les variables sont ici allouées et libérées dynamiquent}
         
         \PY{k+kt}{int} \PY{n+nf}{main}\PY{p}{(}\PY{p}{)}\PY{p}{\PYZob{}}
              \PY{c+c1}{//une chaine de caractère dynamique}
             \PY{c+c1}{// **** TODO ****}
             \PY{k+kt}{char}\PY{o}{*} \PY{n}{chaine} \PY{o}{=} \PY{n}{malloc}\PY{p}{(}\PY{l+m+mi}{10}\PY{o}{*}\PY{k}{sizeof}\PY{p}{(}\PY{k+kt}{char}\PY{p}{)}\PY{p}{)}\PY{p}{;}
             \PY{c+c1}{// Allocation pour pouvoir y copier la chaine constante \PYZdq{}LANGAGE\PYZus{}C\PYZdq{}}
             
             \PY{c+c1}{// à l\PYZsq{}aide de la procédure strcpy() de string.h}
         
         
             \PY{n}{strcpy}\PY{p}{(}\PY{n}{chaine}\PY{p}{,} \PY{l+s}{\PYZdq{}}\PY{l+s}{LANGAGE\PYZus{}C}\PY{l+s}{\PYZdq{}}\PY{p}{)}\PY{p}{;}
             \PY{n}{assert}\PY{p}{(}\PY{n}{strcmp}\PY{p}{(}\PY{n}{chaine}\PY{p}{,} \PY{l+s}{\PYZdq{}}\PY{l+s}{LANGAGE\PYZus{}C}\PY{l+s}{\PYZdq{}}\PY{p}{)}\PY{o}{=}\PY{o}{=}\PY{l+m+mi}{0}\PY{p}{)}\PY{p}{;}
             \PY{n}{assert}\PY{p}{(}\PY{n}{chaine}\PY{p}{[}\PY{l+m+mi}{0}\PY{p}{]} \PY{o}{=}\PY{o}{=} \PY{l+s+sc}{\PYZsq{}}\PY{l+s+sc}{L}\PY{l+s+sc}{\PYZsq{}}\PY{p}{)}\PY{p}{;}
             \PY{n}{assert}\PY{p}{(}\PY{n}{chaine}\PY{p}{[}\PY{l+m+mi}{9}\PY{p}{]} \PY{o}{=}\PY{o}{=} \PY{l+s+sc}{\PYZsq{}}\PY{l+s+sc}{\PYZbs{}0}\PY{l+s+sc}{\PYZsq{}}\PY{p}{)}\PY{p}{;}
             
             \PY{c+c1}{//**** TODO **** }
             \PY{c+c1}{//Libérer toute la mémoire dynamique}
             \PY{n}{free}\PY{p}{(}\PY{n}{chaine}\PY{p}{)}\PY{p}{;}
             \PY{n}{chaine} \PY{o}{=} \PY{n+nb}{NULL}\PY{p}{;}
             
             \PY{n}{assert}\PY{p}{(}\PY{o}{!}\PY{n}{chaine}\PY{p}{)}\PY{p}{;}
                 
             \PY{n}{printf}\PY{p}{(}\PY{l+s}{\PYZdq{}}\PY{l+s}{\PYZpc{}s}\PY{l+s}{\PYZdq{}}\PY{p}{,} \PY{l+s}{\PYZdq{}}\PY{l+s}{Bravo ! Tous les tests passent.}\PY{l+s+se}{\PYZbs{}n}\PY{l+s}{\PYZdq{}}\PY{p}{)}\PY{p}{;}
             \PY{k}{return} \PY{n}{EXIT\PYZus{}SUCCESS}\PY{p}{;}
         \PY{p}{\PYZcb{}}
\end{Verbatim}


    \begin{Verbatim}[commandchars=\\\{\}]
Bravo ! Tous les tests passent.

    \end{Verbatim}

    \begin{center}\rule{0.5\linewidth}{\linethickness}\end{center}

    \subsubsection{Exercice 2 : Allocation dynamique et statique d'un
tableau.}\label{exercice-2-allocation-dynamique-et-statique-dun-tableau.}

Dans l'exercice suivant, il faut compléter des sous-programmes
permettant d'initialiser et manipuler les structures de données
nécessaires à la réalisation d'une version simplifiée du jeu de UNO.

Dans ce jeu, il y a 10 cartes de 4 couleurs différentes (jaune, rouge,
vert et bleu), numérotées entre 0 et 9. Une main de 7 cartes est
distribuée à 2 joueurs. Le premier joueur à avoir posé toutes ses cartes
est le vainqueur. Une carte ne peut être jouée que si elle présente le
même numéro OU la même couleur que la précédente. Si un joueur ne peut
poser une carte, il doit piocher une carte dans le tas de cartes
restantes.

Cet exercice a pour but de vous faire pratiquer la manipulation des
tableaux dynamiques et statiques. Les consignes précises sont décrites
dans le fichier ci-après. Vous pouvez completer le fichier
\textbf{\texttt{Exercice2.c}} si besoin.

L'objectif de l'exercice est de réaliser une exécution sans erreur du
programme de test proposé (\texttt{test\_preparer\_jeu\_UNO}). Ce
programme de test permet de vérifier la bonne préparation du jeu, et
donc des étapes suivantes : - la création du jeu de 4*10 cartes, - la
création de la main des deux joueurs. Chaque main comporte 7 cartes. -
la création de la derniere carte posée pour démarrer le jeu.

    \begin{Verbatim}[commandchars=\\\{\}]
{\color{incolor}In [{\color{incolor}54}]:} \PY{c+cp}{\PYZsh{}}\PY{c+cp}{define XXX 1}
         
         \PY{c+c1}{// Consignes : }
         \PY{c+c1}{//  1. Remplacer XXX par le bon résultat dans la suite.}
         \PY{c+c1}{//  2. Compléter avec les instructions nécessaires en lieu et place de }
         \PY{c+c1}{//     **** TODO ****}
         
         \PY{c+cp}{\PYZsh{}}\PY{c+cp}{include} \PY{c+cpf}{\PYZlt{}assert.h\PYZgt{}}
         \PY{c+cp}{\PYZsh{}}\PY{c+cp}{include} \PY{c+cpf}{\PYZlt{}stdlib.h\PYZgt{}}
         \PY{c+cp}{\PYZsh{}}\PY{c+cp}{include} \PY{c+cpf}{\PYZlt{}stdio.h\PYZgt{}}
         \PY{c+cp}{\PYZsh{}}\PY{c+cp}{include} \PY{c+cpf}{\PYZlt{}stdbool.h\PYZgt{}}
         \PY{c+cp}{\PYZsh{}}\PY{c+cp}{include} \PY{c+cpf}{\PYZlt{}time.h\PYZgt{}}
         
         \PY{c+cp}{\PYZsh{}}\PY{c+cp}{define NB\PYZus{}VALEURS 4}
         \PY{c+cp}{\PYZsh{}}\PY{c+cp}{define NB\PYZus{}CARTES 4*NB\PYZus{}VALEURS}
         
         \PY{c+c1}{//Définition du type enseigne}
         \PY{k}{enum} \PY{n}{couleur} \PY{p}{\PYZob{}}\PY{n}{JAUNE}\PY{p}{,} \PY{n}{ROUGE}\PY{p}{,} \PY{n}{VERT}\PY{p}{,} \PY{n}{BLEU}\PY{p}{\PYZcb{}}\PY{p}{;}
         \PY{k}{typedef} \PY{k}{enum} \PY{n}{couleur} \PY{n}{couleur}\PY{p}{;}
         
         \PY{c+c1}{//Tableau de caractères représentant les couleurs}
         \PY{k+kt}{char} \PY{n}{C}\PY{p}{[}\PY{l+m+mi}{4}\PY{p}{]} \PY{o}{=} \PY{p}{\PYZob{}}\PY{l+s+sc}{\PYZsq{}}\PY{l+s+sc}{J}\PY{l+s+sc}{\PYZsq{}}\PY{p}{,} \PY{l+s+sc}{\PYZsq{}}\PY{l+s+sc}{R}\PY{l+s+sc}{\PYZsq{}}\PY{p}{,} \PY{l+s+sc}{\PYZsq{}}\PY{l+s+sc}{V}\PY{l+s+sc}{\PYZsq{}}\PY{p}{,} \PY{l+s+sc}{\PYZsq{}}\PY{l+s+sc}{B}\PY{l+s+sc}{\PYZsq{}}\PY{p}{\PYZcb{}}\PY{p}{;}
         
         \PY{c+c1}{//Définition du type carte}
         \PY{k}{struct} \PY{n+nc}{carte} \PY{p}{\PYZob{}}
             \PY{n}{couleur} \PY{n}{couleur}\PY{p}{;}
             \PY{k+kt}{int} \PY{n}{valeur}\PY{p}{;} \PY{c+c1}{//valeur \PYZgt{}= 0 \PYZam{}\PYZam{} valeur \PYZlt{} NB\PYZus{}VALEURS}
             \PY{k+kt}{bool} \PY{n}{presente}\PY{p}{;} \PY{c+c1}{// la carte est\PYZhy{}elle presente dans le jeu ?}
         \PY{p}{\PYZcb{}}\PY{p}{;}
         \PY{k}{typedef} \PY{k}{struct} \PY{n+nc}{carte} \PY{n}{carte}\PY{p}{;}
         
         \PY{c+c1}{//Définition du type jeu complet pour enregistrer NB\PYZus{}CARTES cartes.}
         \PY{k}{typedef} \PY{n}{carte} \PY{n}{jeu}\PY{p}{[}\PY{n}{NB\PYZus{}CARTES}\PY{p}{]}\PY{p}{;}
         
         \PY{c+c1}{//Définition du type t\PYZus{}main, capable d\PYZsq{}enregistrer un nombre variable de cartes.}
         \PY{k}{struct} \PY{n+nc}{main} \PY{p}{\PYZob{}}
             \PY{n}{carte} \PY{o}{*} \PY{n}{main}\PY{p}{;} \PY{c+c1}{//tableau des cartes dans la main. }
             \PY{k+kt}{int} \PY{n}{nb}\PY{p}{;} \PY{c+c1}{//monbre de cartes}
         \PY{p}{\PYZcb{}}\PY{p}{;}
         \PY{k}{typedef} \PY{k}{struct} \PY{n+nc}{main} \PY{n}{t\PYZus{}main}\PY{p}{;}
         
         
         \PY{c+cm}{/**}
         \PY{c+cm}{ * \PYZbs{}brief Initialiser une carte avec une couleur et une valeur. }
         \PY{c+cm}{ * \PYZbs{}param[in] c couleur de la carte}
         \PY{c+cm}{ * \PYZbs{}param[in] v valeur de la carte}
         \PY{c+cm}{ * \PYZbs{}param[in] ej booléen presente}
         \PY{c+cm}{ * \PYZbs{}param[out] la\PYZus{}carte }
         \PY{c+cm}{ */}
         \PY{k+kt}{void} \PY{n+nf}{init\PYZus{}carte}\PY{p}{(}\PY{n}{carte}\PY{o}{*} \PY{n}{la\PYZus{}carte}\PY{p}{,} \PY{n}{couleur} \PY{n}{c}\PY{p}{,} \PY{k+kt}{int} \PY{n}{v}\PY{p}{,} \PY{k+kt}{bool} \PY{n}{pr}\PY{p}{)}\PY{p}{\PYZob{}}
             \PY{n}{la\PYZus{}carte}\PY{o}{\PYZhy{}}\PY{o}{\PYZgt{}}\PY{n}{couleur} \PY{o}{=} \PY{n}{c}\PY{p}{;}
             \PY{n}{la\PYZus{}carte}\PY{o}{\PYZhy{}}\PY{o}{\PYZgt{}}\PY{n}{valeur} \PY{o}{=} \PY{n}{v}\PY{p}{;}
             \PY{n}{la\PYZus{}carte}\PY{o}{\PYZhy{}}\PY{o}{\PYZgt{}}\PY{n}{presente} \PY{o}{=} \PY{n}{pr}\PY{p}{;}
         \PY{p}{\PYZcb{}}
         
         \PY{c+cm}{/**}
         \PY{c+cm}{ * \PYZbs{}brief Vérifie si la valeur de la carte est conforme à l\PYZsq{}invariant.}
         \PY{c+cm}{ * \PYZbs{}param[in] c la carte}
         \PY{c+cm}{ * \PYZbs{}return bool vrai si la valeur est conforme, faux sinon.}
         \PY{c+cm}{ */}
          \PY{k+kt}{bool} \PY{n+nf}{est\PYZus{}conforme}\PY{p}{(}\PY{n}{carte} \PY{n}{c}\PY{p}{)}\PY{p}{\PYZob{}}
             \PY{k}{return} \PY{p}{(}\PY{n}{c}\PY{p}{.}\PY{n}{valeur}\PY{o}{\PYZgt{}}\PY{o}{=}\PY{l+m+mi}{0} \PY{o}{\PYZam{}}\PY{o}{\PYZam{}} \PY{n}{c}\PY{p}{.}\PY{n}{valeur}\PY{o}{\PYZlt{}}\PY{n}{NB\PYZus{}VALEURS}\PY{p}{)}\PY{p}{;}
         \PY{p}{\PYZcb{}}
         
         \PY{c+cm}{/**}
         \PY{c+cm}{ * \PYZbs{}brief Initialiser une main.}
         \PY{c+cm}{ * \PYZbs{}param[in] N nombre de cartes composant la main.  Précondition : N \PYZlt{}= (NB\PYZus{}CARTES \PYZhy{} 1) div 2}
         \PY{c+cm}{ * \PYZbs{}param[out] la\PYZus{}main créée}
         \PY{c+cm}{ * \PYZbs{}return true si l\PYZsq{}initialisation a échouée.}
         \PY{c+cm}{ */}
         \PY{k+kt}{bool} \PY{n+nf}{init\PYZus{}main}\PY{p}{(}\PY{n}{t\PYZus{}main}\PY{o}{*} \PY{n}{la\PYZus{}main}\PY{p}{,} \PY{k+kt}{int} \PY{n}{N}\PY{p}{)}\PY{p}{\PYZob{}}
             \PY{n}{assert}\PY{p}{(}\PY{n}{N} \PY{o}{\PYZlt{}}\PY{o}{=} \PY{p}{(}\PY{n}{NB\PYZus{}CARTES}\PY{l+m+mi}{\PYZhy{}1}\PY{p}{)}\PY{o}{/}\PY{l+m+mi}{2}\PY{p}{)}\PY{p}{;}
             \PY{c+c1}{// ***** TODO ***** }
             \PY{c+c1}{// Corriger l\PYZsq{}initialisation du tableau main}
             \PY{n}{la\PYZus{}main}\PY{o}{\PYZhy{}}\PY{o}{\PYZgt{}}\PY{n}{main} \PY{o}{=} \PY{n}{malloc}\PY{p}{(}\PY{k}{sizeof}\PY{p}{(}\PY{n}{la\PYZus{}main}\PY{o}{\PYZhy{}}\PY{o}{\PYZgt{}}\PY{n}{main}\PY{p}{)}\PY{p}{)}\PY{p}{;}
             \PY{n}{la\PYZus{}main}\PY{o}{\PYZhy{}}\PY{o}{\PYZgt{}}\PY{n}{nb} \PY{o}{=} \PY{n}{N}\PY{p}{;}
             \PY{k}{return} \PY{p}{(}\PY{n}{la\PYZus{}main}\PY{o}{=}\PY{o}{=}\PY{n+nb}{NULL}\PY{p}{)}\PY{p}{;} \PY{c+c1}{//allocation réussie ?}
         \PY{p}{\PYZcb{}}
         
         \PY{c+cm}{/**}
         \PY{c+cm}{ * \PYZbs{}brief Initialiser le jeu en ajoutant toutes les cartes possibles au jeu. }
         \PY{c+cm}{ * \PYZbs{}brief Chaque carte est alors présente dans le jeu.}
         \PY{c+cm}{ * \PYZbs{}param[out] le\PYZus{}jeu tableau de cartes avec les 4 couleurs et NB\PYZus{}VALEURS valeurs possibles}
         \PY{c+cm}{ */}
         \PY{k+kt}{void} \PY{n+nf}{init\PYZus{}jeu}\PY{p}{(}\PY{n}{jeu} \PY{n}{le\PYZus{}jeu}\PY{p}{)}\PY{p}{\PYZob{}}
             \PY{k+kt}{int} \PY{n}{k}\PY{o}{=}\PY{l+m+mi}{0}\PY{p}{;}
             \PY{k}{for} \PY{p}{(}\PY{k+kt}{int} \PY{n}{i}\PY{o}{=}\PY{l+m+mi}{0} \PY{p}{;} \PY{n}{i}\PY{o}{\PYZlt{}}\PY{l+m+mi}{4} \PY{p}{;} \PY{n}{i}\PY{o}{+}\PY{o}{+}\PY{p}{)}\PY{p}{\PYZob{}}
                 \PY{k}{for} \PY{p}{(}\PY{k+kt}{int} \PY{n}{j}\PY{o}{=}\PY{l+m+mi}{0} \PY{p}{;} \PY{n}{j}\PY{o}{\PYZlt{}}\PY{n}{NB\PYZus{}VALEURS} \PY{p}{;} \PY{n}{j}\PY{o}{+}\PY{o}{+}\PY{p}{)}\PY{p}{\PYZob{}}
                     \PY{n}{init\PYZus{}carte}\PY{p}{(}\PY{o}{\PYZam{}}\PY{p}{(}\PY{n}{le\PYZus{}jeu}\PY{p}{[}\PY{n}{k}\PY{p}{]}\PY{p}{)}\PY{p}{,} \PY{n}{i}\PY{p}{,} \PY{n}{j}\PY{p}{,} \PY{n+nb}{true}\PY{p}{)}\PY{p}{;}
                     \PY{n}{k}\PY{o}{+}\PY{o}{+}\PY{p}{;}
                 \PY{p}{\PYZcb{}}
             \PY{p}{\PYZcb{}}
         \PY{p}{\PYZcb{}}
         
         
         \PY{c+cm}{/**}
         \PY{c+cm}{ * \PYZbs{}brief Copie les valeurs de la carte src dans la carte dest.}
         \PY{c+cm}{ * \PYZbs{}param[in] src carte à copier}
         \PY{c+cm}{ * \PYZbs{}param[out] dest carte destination de la copie }
         \PY{c+cm}{ */}
         \PY{k+kt}{void} \PY{n+nf}{copier\PYZus{}carte}\PY{p}{(}\PY{n}{carte}\PY{o}{*} \PY{n}{dest}\PY{p}{,} \PY{n}{carte} \PY{n}{src}\PY{p}{)}\PY{p}{\PYZob{}}
             \PY{n}{dest}\PY{o}{\PYZhy{}}\PY{o}{\PYZgt{}}\PY{n}{couleur} \PY{o}{=} \PY{n}{src}\PY{p}{.}\PY{n}{couleur}\PY{p}{;}
             \PY{n}{dest}\PY{o}{\PYZhy{}}\PY{o}{\PYZgt{}}\PY{n}{valeur} \PY{o}{=} \PY{n}{src}\PY{p}{.}\PY{n}{valeur}\PY{p}{;}
             \PY{n}{dest}\PY{o}{\PYZhy{}}\PY{o}{\PYZgt{}}\PY{n}{presente} \PY{o}{=} \PY{n}{src}\PY{p}{.}\PY{n}{presente}\PY{p}{;}
         \PY{p}{\PYZcb{}}
         
         \PY{c+cm}{/**}
         \PY{c+cm}{ * \PYZbs{}brief Afficher une carte.}
         \PY{c+cm}{ * \PYZbs{}param[in] cte carte à afficher}
         \PY{c+cm}{ */}
         \PY{k+kt}{void} \PY{n+nf}{afficher\PYZus{}carte}\PY{p}{(}\PY{n}{carte} \PY{n}{cte}\PY{p}{)}\PY{p}{\PYZob{}}
             \PY{n}{printf}\PY{p}{(}\PY{l+s}{\PYZdq{}}\PY{l+s}{(\PYZpc{}c;\PYZpc{}d;\PYZpc{}d)}\PY{l+s+se}{\PYZbs{}t}\PY{l+s}{\PYZdq{}}\PY{p}{,} \PY{n}{C}\PY{p}{[}\PY{n}{cte}\PY{p}{.}\PY{n}{couleur}\PY{p}{]}\PY{p}{,} \PY{n}{cte}\PY{p}{.}\PY{n}{valeur}\PY{p}{,} \PY{n}{cte}\PY{p}{.}\PY{n}{presente}\PY{p}{)}\PY{p}{;}
         \PY{p}{\PYZcb{}}
         
         
         \PY{c+cm}{/**}
         \PY{c+cm}{ * \PYZbs{}brief Afficher le jeu.}
         \PY{c+cm}{ * \PYZbs{}param[in] le\PYZus{}jeu complet}
         \PY{c+cm}{ */}
         \PY{k+kt}{void} \PY{n+nf}{afficher\PYZus{}jeu}\PY{p}{(}\PY{n}{jeu} \PY{n}{le\PYZus{}jeu}\PY{p}{)}\PY{p}{\PYZob{}}
             \PY{c+c1}{// ***** TODO ***** }
             \PY{c+c1}{// Afficher le jeu complet. Les carte sont listées sur une même ligne, }
             \PY{c+c1}{// et séparées par une tabulation \PYZbs{}t}
             \PY{k}{for} \PY{p}{(}\PY{k+kt}{int} \PY{n}{i}\PY{o}{=}\PY{l+m+mi}{0}\PY{p}{;}\PY{n}{i}\PY{o}{\PYZlt{}}\PY{n}{NB\PYZus{}VALEURS}\PY{p}{;}\PY{n}{i}\PY{o}{+}\PY{o}{+}\PY{p}{)}\PY{p}{\PYZob{}}
                 \PY{n}{afficher\PYZus{}carte}\PY{p}{(}\PY{n}{le\PYZus{}jeu}\PY{p}{[}\PY{n}{i}\PY{p}{]}\PY{p}{)}\PY{p}{;}
                 \PY{n}{printf}\PY{p}{(}\PY{l+s}{\PYZdq{}}\PY{l+s+se}{\PYZbs{}t}\PY{l+s}{\PYZdq{}}\PY{p}{)}\PY{p}{;}
             \PY{p}{\PYZcb{}}
             \PY{n}{printf}\PY{p}{(}\PY{l+s}{\PYZdq{}}\PY{l+s+se}{\PYZbs{}n}\PY{l+s}{\PYZdq{}}\PY{p}{)}\PY{p}{;}
         \PY{p}{\PYZcb{}}
         
         \PY{c+cm}{/**}
         \PY{c+cm}{ * \PYZbs{}brief Afficher une main.}
         \PY{c+cm}{ * \PYZbs{}param[in] la\PYZus{}main la main a afficher}
         \PY{c+cm}{ */}
         \PY{k+kt}{void} \PY{n+nf}{afficher\PYZus{}main}\PY{p}{(}\PY{n}{t\PYZus{}main} \PY{n}{la\PYZus{}main}\PY{p}{)}\PY{p}{\PYZob{}}
             \PY{c+c1}{// ***** TODO ***** }
             \PY{c+c1}{// Afficher le jeu complet. Les carte sont listées sur une même ligne, }
             \PY{c+c1}{// et séparées par une tabulation \PYZbs{}t}
             \PY{k}{for} \PY{p}{(}\PY{k+kt}{int} \PY{n}{i} \PY{o}{=}\PY{l+m+mi}{0}\PY{p}{;}\PY{n}{i}\PY{o}{\PYZlt{}}\PY{n}{la\PYZus{}main}\PY{p}{.}\PY{n}{nb}\PY{p}{;}\PY{n}{i}\PY{o}{+}\PY{o}{+}\PY{p}{)}\PY{p}{\PYZob{}}
                 \PY{n}{afficher\PYZus{}carte}\PY{p}{(}\PY{n}{la\PYZus{}main}\PY{p}{.}\PY{n}{main}\PY{p}{[}\PY{n}{i}\PY{p}{]}\PY{p}{)}\PY{p}{;}
                 \PY{n}{printf}\PY{p}{(}\PY{l+s}{\PYZdq{}}\PY{l+s+se}{\PYZbs{}t}\PY{l+s}{\PYZdq{}}\PY{p}{)}\PY{p}{;}
             \PY{p}{\PYZcb{}}
             
             \PY{n}{printf}\PY{p}{(}\PY{l+s}{\PYZdq{}}\PY{l+s+se}{\PYZbs{}n}\PY{l+s}{\PYZdq{}}\PY{p}{)}\PY{p}{;}
         \PY{p}{\PYZcb{}}
         
         \PY{c+cm}{/**}
         \PY{c+cm}{ * \PYZbs{}brief mélange le jeu.}
         \PY{c+cm}{ * \PYZbs{}param[in out] le\PYZus{}jeu complet mélangé}
         \PY{c+cm}{ */}
         \PY{k+kt}{void} \PY{n+nf}{melanger\PYZus{}jeu}\PY{p}{(}\PY{n}{jeu} \PY{n}{le\PYZus{}jeu}\PY{p}{)}\PY{p}{\PYZob{}}
             \PY{k}{for} \PY{p}{(}\PY{k+kt}{int} \PY{n}{k}\PY{o}{=}\PY{l+m+mi}{0}\PY{p}{;} \PY{n}{k}\PY{o}{\PYZlt{}}\PY{l+m+mi}{1000}\PY{p}{;} \PY{n}{k}\PY{o}{+}\PY{o}{+}\PY{p}{)}\PY{p}{\PYZob{}}
                 \PY{c+c1}{// Choisir deux cartes aléatoirement}
                 \PY{k+kt}{int} \PY{n}{i} \PY{o}{=} \PY{n}{rand}\PY{p}{(}\PY{p}{)}\PY{o}{\PYZpc{}}\PY{n}{NB\PYZus{}CARTES}\PY{p}{;}
                 \PY{k+kt}{int} \PY{n}{j} \PY{o}{=} \PY{n}{rand}\PY{p}{(}\PY{p}{)}\PY{o}{\PYZpc{}}\PY{n}{NB\PYZus{}CARTES}\PY{p}{;}        
                 \PY{c+c1}{// Les échanger}
                 \PY{c+c1}{// ***** TODO **** }
                \PY{n}{carte}\PY{o}{*} \PY{n}{dest} \PY{o}{=} \PY{n}{malloc}\PY{p}{(}\PY{k}{sizeof}\PY{p}{(}\PY{n}{carte}\PY{p}{)}\PY{p}{)}\PY{p}{;}
                \PY{n}{init\PYZus{}carte}\PY{p}{(}\PY{n}{dest}\PY{p}{,}\PY{n}{le\PYZus{}jeu}\PY{p}{[}\PY{n}{i}\PY{p}{]}\PY{p}{.}\PY{n}{couleur}\PY{p}{,}\PY{n}{le\PYZus{}jeu}\PY{p}{[}\PY{n}{i}\PY{p}{]}\PY{p}{.}\PY{n}{valeur}\PY{p}{,}\PY{n}{le\PYZus{}jeu}\PY{p}{[}\PY{n}{i}\PY{p}{]}\PY{p}{.}\PY{n}{presente}\PY{p}{)}\PY{p}{;}
                 \PY{n}{copier\PYZus{}carte}\PY{p}{(}\PY{o}{\PYZam{}}\PY{p}{(}\PY{n}{le\PYZus{}jeu}\PY{p}{[}\PY{n}{i}\PY{p}{]}\PY{p}{)}\PY{p}{,}\PY{n}{le\PYZus{}jeu}\PY{p}{[}\PY{n}{j}\PY{p}{]}\PY{p}{)}\PY{p}{;}
                 \PY{n}{le\PYZus{}jeu}\PY{p}{[}\PY{n}{i}\PY{p}{]}\PY{p}{.}\PY{n}{presente} \PY{o}{=} \PY{n+nb}{false}\PY{p}{;}
                 \PY{n}{copier\PYZus{}carte}\PY{p}{(}\PY{n}{dest}\PY{p}{,}\PY{n}{le\PYZus{}jeu}\PY{p}{[}\PY{n}{i}\PY{p}{]}\PY{p}{)}\PY{p}{;}
             \PY{p}{\PYZcb{}}
         \PY{p}{\PYZcb{}}
         
         \PY{c+cm}{/**}
         \PY{c+cm}{ \PYZbs{}brief Distribuer N cartes à chacun des deux joueurs, en alternant les joueurs.}
         \PY{c+cm}{ * \PYZbs{}param[in out] le\PYZus{}jeu complet.}
         \PY{c+cm}{ *       Si la carte c est distribuée dans une main, c.presente devient faux.}
         \PY{c+cm}{ * \PYZbs{}param[in] N nombre de cartes distribuées à chaque joueur.  Précondition : N \PYZlt{}= (NB\PYZus{}CARTES \PYZhy{} 1) div 2}
         \PY{c+cm}{ * \PYZbs{}param[out] m1 main du joueur 1.}
         \PY{c+cm}{ * \PYZbs{}param[out] m2 main du joueur 2.}
         \PY{c+cm}{ */}
         \PY{k+kt}{void} \PY{n+nf}{distribuer\PYZus{}mains}\PY{p}{(}\PY{n}{jeu} \PY{n}{le\PYZus{}jeu}\PY{p}{,} \PY{k+kt}{int} \PY{n}{N}\PY{p}{,} \PY{n}{t\PYZus{}main}\PY{o}{*} \PY{n}{m1}\PY{p}{,} \PY{n}{t\PYZus{}main}\PY{o}{*} \PY{n}{m2}\PY{p}{)}\PY{p}{\PYZob{}}
             \PY{n}{assert}\PY{p}{(}\PY{n}{N} \PY{o}{\PYZlt{}}\PY{o}{=} \PY{p}{(}\PY{n}{NB\PYZus{}CARTES}\PY{l+m+mi}{\PYZhy{}1}\PY{p}{)}\PY{o}{/}\PY{l+m+mi}{2}\PY{p}{)}\PY{p}{;}
         
             \PY{c+c1}{//Initialiser les mains de N cartes}
             \PY{k+kt}{bool} \PY{n}{errA} \PY{o}{=} \PY{n}{init\PYZus{}main}\PY{p}{(}\PY{n}{m1}\PY{p}{,} \PY{n}{N}\PY{p}{)}\PY{p}{;}
             \PY{k+kt}{bool} \PY{n}{errB} \PY{o}{=} \PY{n}{init\PYZus{}main}\PY{p}{(}\PY{n}{m2}\PY{p}{,} \PY{n}{N}\PY{p}{)}\PY{p}{;}
             \PY{n}{assert}\PY{p}{(}\PY{o}{!}\PY{n}{errA} \PY{o}{\PYZam{}}\PY{o}{\PYZam{}} \PY{o}{!}\PY{n}{errB}\PY{p}{)}\PY{p}{;}
             
             \PY{c+c1}{//Distribuer les cartes}
             \PY{k}{for} \PY{p}{(}\PY{k+kt}{int} \PY{n}{i}\PY{o}{=}\PY{l+m+mi}{0}\PY{p}{;} \PY{n}{i}\PY{o}{\PYZlt{}}\PY{n}{N}\PY{p}{;} \PY{n}{i}\PY{o}{+}\PY{o}{+}\PY{p}{)}\PY{p}{\PYZob{}}
                 \PY{c+c1}{// ajout d\PYZsq{}une carte dans la main m1}
                 \PY{n}{copier\PYZus{}carte}\PY{p}{(}\PY{o}{\PYZam{}}\PY{p}{(}\PY{n}{m1}\PY{o}{\PYZhy{}}\PY{o}{\PYZgt{}}\PY{n}{main}\PY{p}{[}\PY{n}{i}\PY{p}{]}\PY{p}{)}\PY{p}{,} \PY{n}{le\PYZus{}jeu}\PY{p}{[}\PY{l+m+mi}{2}\PY{o}{*}\PY{n}{i}\PY{p}{]}\PY{p}{)}\PY{p}{;}
                 \PY{c+c1}{// ajout d\PYZsq{}une carte dans la main m2}
                 \PY{n}{copier\PYZus{}carte}\PY{p}{(}\PY{o}{\PYZam{}}\PY{p}{(}\PY{n}{m2}\PY{o}{\PYZhy{}}\PY{o}{\PYZgt{}}\PY{n}{main}\PY{p}{[}\PY{n}{i}\PY{p}{]}\PY{p}{)}\PY{p}{,} \PY{n}{le\PYZus{}jeu}\PY{p}{[}\PY{l+m+mi}{2}\PY{o}{*}\PY{n}{i}\PY{o}{+}\PY{l+m+mi}{1}\PY{p}{]}\PY{p}{)}\PY{p}{;}
                 \PY{c+c1}{//mise à jour de presente à false dans le\PYZus{}jeu}
                 \PY{n}{le\PYZus{}jeu}\PY{p}{[}\PY{l+m+mi}{2}\PY{o}{*}\PY{n}{i}\PY{p}{]}\PY{p}{.}\PY{n}{presente} \PY{o}{=} \PY{n+nb}{false}\PY{p}{;}
                 \PY{n}{le\PYZus{}jeu}\PY{p}{[}\PY{l+m+mi}{2}\PY{o}{*}\PY{n}{i}\PY{o}{+}\PY{l+m+mi}{1}\PY{p}{]}\PY{p}{.}\PY{n}{presente} \PY{o}{=} \PY{n+nb}{false}\PY{p}{;}
             \PY{p}{\PYZcb{}}
         \PY{p}{\PYZcb{}}
         
         \PY{c+cm}{/**}
         \PY{c+cm}{ * \PYZbs{}brief Initialise le jeu de carte, les mains des joueurs et la carte \PYZsq{}last\PYZsq{}.}
         \PY{c+cm}{ * \PYZbs{}param[out] le\PYZus{}jeu complet avec les 4 couleurs et 10 valeurs possibles.}
         \PY{c+cm}{ *                Ce jeu est mélangé.}
         \PY{c+cm}{ *                Si la carte est inclue dans une main ou est la derniere carte jouée,}
         \PY{c+cm}{ *                Alors carte.presente vaut faux.}
         \PY{c+cm}{ * \PYZbs{}param[in] N nombre de cartes par main.  Precondition : N \PYZlt{}= (NB\PYZus{}CARTES\PYZhy{}1)/2);}
         \PY{c+cm}{ * \PYZbs{}param[out] main\PYZus{}A main du joueur A.}
         \PY{c+cm}{ * \PYZbs{}param[out] main\PYZus{}B main du joueur B.}
         \PY{c+cm}{ * \PYZbs{}param[out] last la derniere carte jouée par un des joueurs.}
         \PY{c+cm}{ */}
         \PY{k+kt}{int} \PY{n+nf}{preparer\PYZus{}jeu\PYZus{}UNO}\PY{p}{(}\PY{n}{jeu} \PY{n}{le\PYZus{}jeu}\PY{p}{,} \PY{k+kt}{int} \PY{n}{N}\PY{p}{,} \PY{n}{t\PYZus{}main}\PY{o}{*} \PY{n}{main\PYZus{}A}\PY{p}{,} \PY{n}{t\PYZus{}main}\PY{o}{*} \PY{n}{main\PYZus{}B}\PY{p}{,} \PY{n}{carte}\PY{o}{*} \PY{n}{last}\PY{p}{)}\PY{p}{\PYZob{}}
             \PY{n}{assert}\PY{p}{(}\PY{n}{N} \PY{o}{\PYZlt{}}\PY{o}{=} \PY{p}{(}\PY{n}{NB\PYZus{}CARTES}\PY{l+m+mi}{\PYZhy{}1}\PY{p}{)}\PY{o}{/}\PY{l+m+mi}{2}\PY{p}{)}\PY{p}{;}
             
             \PY{c+c1}{//Initialiser le générateur de nombres aléatoires}
             \PY{k+kt}{time\PYZus{}t} \PY{n}{t}\PY{p}{;}
             \PY{n}{srand}\PY{p}{(}\PY{p}{(}\PY{k+kt}{unsigned}\PY{p}{)} \PY{n}{time}\PY{p}{(}\PY{o}{\PYZam{}}\PY{n}{t}\PY{p}{)}\PY{p}{)}\PY{p}{;}
          
             \PY{c+c1}{//Initialiser le jeu}
             \PY{n}{init\PYZus{}jeu}\PY{p}{(}\PY{n}{le\PYZus{}jeu}\PY{p}{)}\PY{p}{;}
             
             \PY{c+c1}{//Melanger le jeu}
             \PY{n}{melanger\PYZus{}jeu}\PY{p}{(}\PY{n}{le\PYZus{}jeu}\PY{p}{)}\PY{p}{;}
         
             \PY{c+c1}{//Distribuer N cartes à chaque joueur}
             \PY{n}{distribuer\PYZus{}mains}\PY{p}{(}\PY{n}{le\PYZus{}jeu}\PY{p}{,} \PY{n}{N}\PY{p}{,} \PY{n}{main\PYZus{}A}\PY{p}{,} \PY{n}{main\PYZus{}B}\PY{p}{)}\PY{p}{;}
         
             \PY{c+c1}{//Initialiser last avec la (2N+1)\PYZhy{}ème carte du jeu.}
             \PY{n}{copier\PYZus{}carte}\PY{p}{(}\PY{n}{last}\PY{p}{,} \PY{n}{le\PYZus{}jeu}\PY{p}{[}\PY{l+m+mi}{2}\PY{o}{*}\PY{n}{N}\PY{p}{]}\PY{p}{)}\PY{p}{;}
             \PY{n}{le\PYZus{}jeu}\PY{p}{[}\PY{l+m+mi}{2}\PY{o}{*}\PY{n}{N}\PY{p}{]}\PY{p}{.}\PY{n}{presente} \PY{o}{=} \PY{n+nb}{false}\PY{p}{;} \PY{c+c1}{//carte n\PYZsq{}est plus presente dans le\PYZus{}jeu}
             
             \PY{k}{return} \PY{n}{EXIT\PYZus{}SUCCESS}\PY{p}{;}
         \PY{p}{\PYZcb{}}
         
         \PY{k+kt}{void} \PY{n+nf}{test\PYZus{}preparer\PYZus{}jeu\PYZus{}UNO}\PY{p}{(}\PY{p}{)}\PY{p}{\PYZob{}}
             \PY{c+c1}{//Déclarer un jeu (tableau statique), les deux mains (tableaux dynamiques) et }
             \PY{c+c1}{//la carte last.}
             \PY{n}{jeu} \PY{n}{le\PYZus{}jeu}\PY{p}{;}
             \PY{n}{t\PYZus{}main} \PY{n}{main\PYZus{}A}\PY{p}{,} \PY{n}{main\PYZus{}B}\PY{p}{;}
             \PY{n}{carte} \PY{n}{last}\PY{p}{;}
          
             \PY{c+c1}{//Préparer le jeu, les deux mains de 7 cartes et la carte last}
             \PY{k+kt}{int} \PY{n}{retour} \PY{o}{=} \PY{n}{preparer\PYZus{}jeu\PYZus{}UNO}\PY{p}{(}\PY{n}{le\PYZus{}jeu}\PY{p}{,} \PY{l+m+mi}{7}\PY{p}{,} \PY{o}{\PYZam{}}\PY{n}{main\PYZus{}A}\PY{p}{,} \PY{o}{\PYZam{}}\PY{n}{main\PYZus{}B}\PY{p}{,} \PY{o}{\PYZam{}}\PY{n}{last}\PY{p}{)}\PY{p}{;}
             \PY{n}{printf}\PY{p}{(}\PY{l+s}{\PYZdq{}}\PY{l+s+se}{\PYZbs{}n}\PY{l+s}{ Le jeu mélangé avec les cartes presentes (c ; v ; p) : }\PY{l+s+se}{\PYZbs{}n}\PY{l+s}{\PYZdq{}}\PY{p}{)}\PY{p}{;}
             \PY{n}{afficher\PYZus{}jeu}\PY{p}{(}\PY{n}{le\PYZus{}jeu}\PY{p}{)}\PY{p}{;}
             \PY{n}{printf}\PY{p}{(}\PY{l+s}{\PYZdq{}}\PY{l+s+se}{\PYZbs{}n}\PY{l+s}{ Les deux mains : }\PY{l+s+se}{\PYZbs{}n}\PY{l+s}{\PYZdq{}}\PY{p}{)}\PY{p}{;}
             \PY{n}{afficher\PYZus{}main}\PY{p}{(}\PY{n}{main\PYZus{}A}\PY{p}{)}\PY{p}{;}
             \PY{n}{afficher\PYZus{}main}\PY{p}{(}\PY{n}{main\PYZus{}B}\PY{p}{)}\PY{p}{;}
             \PY{n}{printf}\PY{p}{(}\PY{l+s}{\PYZdq{}}\PY{l+s+se}{\PYZbs{}n}\PY{l+s}{ La carte last : }\PY{l+s}{\PYZdq{}}\PY{p}{)}\PY{p}{;}
             \PY{n}{afficher\PYZus{}carte}\PY{p}{(}\PY{n}{last}\PY{p}{)}\PY{p}{;}
             \PY{n}{printf}\PY{p}{(}\PY{l+s}{\PYZdq{}}\PY{l+s+se}{\PYZbs{}n}\PY{l+s}{\PYZdq{}}\PY{p}{)}\PY{p}{;}
         
             \PY{c+c1}{//Vérifier le jeu et les mains.}
             \PY{n}{assert}\PY{p}{(}\PY{n}{retour} \PY{o}{=}\PY{o}{=} \PY{n}{EXIT\PYZus{}SUCCESS}\PY{p}{)}\PY{p}{;}
             \PY{n}{assert}\PY{p}{(}\PY{n}{main\PYZus{}A}\PY{p}{.}\PY{n}{nb} \PY{o}{=}\PY{o}{=} \PY{l+m+mi}{7} \PY{o}{\PYZam{}}\PY{o}{\PYZam{}} \PY{n}{main\PYZus{}B}\PY{p}{.}\PY{n}{nb} \PY{o}{=}\PY{o}{=} \PY{l+m+mi}{7}\PY{p}{)}\PY{p}{;}
             \PY{n}{assert}\PY{p}{(}\PY{n}{main\PYZus{}A}\PY{p}{.}\PY{n}{main} \PY{o}{!}\PY{o}{=} \PY{n+nb}{NULL} \PY{o}{\PYZam{}}\PY{o}{\PYZam{}} \PY{n}{main\PYZus{}B}\PY{p}{.}\PY{n}{main} \PY{o}{!}\PY{o}{=} \PY{n+nb}{NULL}\PY{p}{)}\PY{p}{;}    
             \PY{n}{assert}\PY{p}{(}\PY{n}{est\PYZus{}conforme}\PY{p}{(}\PY{n}{main\PYZus{}A}\PY{p}{.}\PY{n}{main}\PY{p}{[}\PY{l+m+mi}{0}\PY{p}{]}\PY{p}{)}\PY{p}{)}\PY{p}{;}
             \PY{n}{assert}\PY{p}{(}\PY{n}{est\PYZus{}conforme}\PY{p}{(}\PY{n}{main\PYZus{}B}\PY{p}{.}\PY{n}{main}\PY{p}{[}\PY{l+m+mi}{0}\PY{p}{]}\PY{p}{)}\PY{p}{)}\PY{p}{;}
             \PY{n}{assert}\PY{p}{(}\PY{n}{est\PYZus{}conforme}\PY{p}{(}\PY{n}{last}\PY{p}{)}\PY{p}{)}\PY{p}{;}
                 
             \PY{c+c1}{//Détruire la mémoire allouée dynamiquement}
             \PY{c+c1}{// ***** TODO *****}
             \PY{n}{free}\PY{p}{(}\PY{n}{main\PYZus{}A}\PY{p}{.}\PY{n}{main}\PY{p}{)}\PY{p}{;}
             \PY{n}{main\PYZus{}A}\PY{p}{.}\PY{n}{main} \PY{o}{=} \PY{n+nb}{NULL}\PY{p}{;}
             \PY{n}{free}\PY{p}{(}\PY{n}{main\PYZus{}B}\PY{p}{.}\PY{n}{main}\PY{p}{)}\PY{p}{;}
             \PY{n}{main\PYZus{}B}\PY{p}{.}\PY{n}{main} \PY{o}{=} \PY{n+nb}{NULL}\PY{p}{;}
            
             
             \PY{n}{assert}\PY{p}{(}\PY{n}{main\PYZus{}A}\PY{p}{.}\PY{n}{main} \PY{o}{=}\PY{o}{=} \PY{n+nb}{NULL}\PY{p}{)}\PY{p}{;}
             \PY{n}{assert}\PY{p}{(}\PY{n}{main\PYZus{}B}\PY{p}{.}\PY{n}{main} \PY{o}{=}\PY{o}{=} \PY{n+nb}{NULL}\PY{p}{)}\PY{p}{;}
          
         \PY{p}{\PYZcb{}}
         
         \PY{k+kt}{int} \PY{n+nf}{main}\PY{p}{(}\PY{k+kt}{void}\PY{p}{)} \PY{p}{\PYZob{}}
           
             \PY{n}{test\PYZus{}preparer\PYZus{}jeu\PYZus{}UNO}\PY{p}{(}\PY{p}{)}\PY{p}{;}
             
             \PY{n}{printf}\PY{p}{(}\PY{l+s}{\PYZdq{}}\PY{l+s}{\PYZpc{}s}\PY{l+s}{\PYZdq{}}\PY{p}{,} \PY{l+s}{\PYZdq{}}\PY{l+s+se}{\PYZbs{}n}\PY{l+s}{ Bravo ! Tous les tests passent.}\PY{l+s+se}{\PYZbs{}n}\PY{l+s}{\PYZdq{}}\PY{p}{)}\PY{p}{;}
             \PY{k}{return} \PY{n}{EXIT\PYZus{}SUCCESS}\PY{p}{;}
         \PY{p}{\PYZcb{}}
\end{Verbatim}


    \begin{Verbatim}[commandchars=\\\{\}]

 Le jeu mélangé avec les cartes presentes (c ; v ; p) : 
(J;0;0)		(J;1;0)		(J;2;0)		(J;3;0)		

 Les deux mains : 
(J;0;0)		(J;2;1)		(J;0;0)		(R;2;1)		(J;0;0)		(V;2;1)		(J;0;0)		
(J;1;2)		(R;0;0)	
    \end{Verbatim}

    \begin{Verbatim}[commandchars=\\\{\}]
free(): invalid pointer
[C kernel] Executable exited with code -6
    \end{Verbatim}

    \begin{center}\rule{0.5\linewidth}{\linethickness}\end{center}

    \subsubsection{\texorpdfstring{4.4 La réallocation avec
\texttt{Realloc}}{4.4 La réallocation avec Realloc}}\label{la-ruxe9allocation-avec-realloc}

En C, il est possible de réallouer une variable dynamique avec la
procédure : \textgreater{}
\texttt{void*\ realloc(void*\ ptr\_mem,\ size\_t\ taille)}

Elle prend en paramètres : - le pointeur \texttt{ptr\_mem} sur la zone
mémoire dont on veut modifier la taille, - la nouvelle \texttt{taille}
de la zone mémoire.

Elle retourne : - un pointeur sur la zone mémoire allouée. \textbf{Si
cette réallocation échoue, elle retoune \texttt{NULL}}.

\begin{quote}
\textbf{Note :} La réallocation copie également les données enregistrées
dans la zone mémoire initiale vers la nouvelle zone mémoire.
\end{quote}

On peut se servir de \texttt{realloc} pour : - Augmenter la taille
mémoire allouée à l'origine. - Réduire la taille mémoire allouée à
l'origine. - Libérer la mémoire. Dans ce cas, le paramètre
\texttt{taille} vaut 0. Ce comportement est équivalement à
\texttt{free}. - Allouer une nouvelle zone mémoire. Dans ce cas, le
paramètre \texttt{ptr\_mem} vaut \texttt{NULL}. Ce comportement est
équivalent à \texttt{malloc}.

Voici quelques exemples d'utilisation.

    \subsubsection{\texorpdfstring{Exemple d'une \emph{mauvaise} utilisation
de
\texttt{realloc}}{Exemple d'une mauvaise utilisation de realloc}}\label{exemple-dune-mauvaise-utilisation-de-realloc}

Vous pouvez manipuler le fichier
\textbf{\texttt{Exemple4\_4\_mauvaise.c}} si besoin.

    \begin{Verbatim}[commandchars=\\\{\}]
{\color{incolor}In [{\color{incolor}44}]:} \PY{c+cp}{\PYZsh{}}\PY{c+cp}{include} \PY{c+cpf}{\PYZlt{}assert.h\PYZgt{}}
         \PY{c+cp}{\PYZsh{}}\PY{c+cp}{include} \PY{c+cpf}{\PYZlt{}stdlib.h\PYZgt{}}
         \PY{c+cp}{\PYZsh{}}\PY{c+cp}{include} \PY{c+cpf}{\PYZlt{}stdio.h\PYZgt{}}
         
         \PY{c+cp}{\PYZsh{}}\PY{c+cp}{define TAILLE 10}
         
         \PY{k+kt}{int} \PY{n+nf}{main}\PY{p}{(}\PY{p}{)}\PY{p}{\PYZob{}}
         
             \PY{c+c1}{// Allouer un tableau de TAILLE entiers}
             \PY{k+kt}{int}\PY{o}{*} \PY{n}{tableau} \PY{o}{=} \PY{n}{malloc}\PY{p}{(}\PY{n}{TAILLE}\PY{o}{*}\PY{k}{sizeof}\PY{p}{(}\PY{k+kt}{int}\PY{p}{)}\PY{p}{)}\PY{p}{;}
             \PY{n}{assert}\PY{p}{(}\PY{n}{tableau}\PY{p}{)}\PY{p}{;} \PY{c+c1}{//allocation réussie ?}
             
             \PY{c+c1}{// Initialiser les éléments à 1}
             \PY{k}{for} \PY{p}{(}\PY{k+kt}{int} \PY{n}{i}\PY{o}{=}\PY{l+m+mi}{0}\PY{p}{;} \PY{n}{i}\PY{o}{\PYZlt{}}\PY{n}{TAILLE}\PY{p}{;} \PY{n}{i}\PY{o}{+}\PY{o}{+}\PY{p}{)}\PY{p}{\PYZob{}}
                 \PY{n}{tableau}\PY{p}{[}\PY{n}{i}\PY{p}{]}\PY{o}{=}\PY{l+m+mi}{1}\PY{p}{;}
             \PY{p}{\PYZcb{}}
             
             \PY{c+c1}{// Augmenter la taille du tableau pour enregistrer TAILLE entiers supplémentaires.}
             \PY{n}{tableau} \PY{o}{=} \PY{n}{realloc}\PY{p}{(}\PY{n}{tableau}\PY{p}{,} \PY{p}{(}\PY{n}{TAILLE}\PY{o}{+}\PY{n}{TAILLE}\PY{p}{)}\PY{o}{*}\PY{k}{sizeof}\PY{p}{(}\PY{k+kt}{int}\PY{p}{)}\PY{p}{)}\PY{p}{;}
             \PY{n}{assert}\PY{p}{(}\PY{n}{tableau}\PY{p}{)}\PY{p}{;} \PY{c+c1}{// ré\PYZhy{}allocation réussie ?}
             
             \PY{c+c1}{//test des 5 premiers éléments}
             \PY{n}{assert}\PY{p}{(}\PY{n}{tableau}\PY{p}{[}\PY{l+m+mi}{0}\PY{p}{]}\PY{o}{=}\PY{o}{=}\PY{l+m+mi}{1} \PY{o}{\PYZam{}}\PY{o}{\PYZam{}} \PY{n}{tableau}\PY{p}{[}\PY{l+m+mi}{1}\PY{p}{]}\PY{o}{=}\PY{o}{=}\PY{l+m+mi}{1} \PY{o}{\PYZam{}}\PY{o}{\PYZam{}} \PY{n}{tableau}\PY{p}{[}\PY{l+m+mi}{2}\PY{p}{]}\PY{o}{=}\PY{o}{=}\PY{l+m+mi}{1} \PY{o}{\PYZam{}}\PY{o}{\PYZam{}} \PY{n}{tableau}\PY{p}{[}\PY{l+m+mi}{3}\PY{p}{]}\PY{o}{=}\PY{o}{=}\PY{l+m+mi}{1} \PY{o}{\PYZam{}}\PY{o}{\PYZam{}} \PY{n}{tableau}\PY{p}{[}\PY{l+m+mi}{4}\PY{p}{]}\PY{o}{=}\PY{o}{=}\PY{l+m+mi}{1}\PY{p}{)}\PY{p}{;} 
         
             \PY{k}{for} \PY{p}{(}\PY{k+kt}{int} \PY{n}{i}\PY{o}{=}\PY{n}{TAILLE}\PY{p}{;} \PY{n}{i}\PY{o}{\PYZlt{}}\PY{n}{TAILLE}\PY{o}{+}\PY{n}{TAILLE}\PY{p}{;} \PY{n}{i}\PY{o}{+}\PY{o}{+}\PY{p}{)}\PY{p}{\PYZob{}}
                 \PY{n}{tableau}\PY{p}{[}\PY{n}{i}\PY{p}{]}\PY{o}{=}\PY{l+m+mi}{2}\PY{p}{;}
             \PY{p}{\PYZcb{}}
             
             \PY{c+c1}{//test de 5 nouveaux éléments}
             \PY{n}{assert}\PY{p}{(}\PY{n}{tableau}\PY{p}{[}\PY{n}{TAILLE}\PY{p}{]}\PY{o}{=}\PY{o}{=}\PY{l+m+mi}{2} \PY{o}{\PYZam{}}\PY{o}{\PYZam{}} \PY{n}{tableau}\PY{p}{[}\PY{n}{TAILLE}\PY{o}{+}\PY{l+m+mi}{1}\PY{p}{]}\PY{o}{=}\PY{o}{=}\PY{l+m+mi}{2} \PY{o}{\PYZam{}}\PY{o}{\PYZam{}} \PY{n}{tableau}\PY{p}{[}\PY{n}{TAILLE}\PY{o}{+}\PY{l+m+mi}{2}\PY{p}{]}\PY{o}{=}\PY{o}{=}\PY{l+m+mi}{2} \PY{o}{\PYZam{}}\PY{o}{\PYZam{}} \PY{n}{tableau}\PY{p}{[}\PY{n}{TAILLE}\PY{o}{+}\PY{l+m+mi}{3}\PY{p}{]}\PY{o}{=}\PY{o}{=}\PY{l+m+mi}{2} \PY{o}{\PYZam{}}\PY{o}{\PYZam{}} \PY{n}{tableau}\PY{p}{[}\PY{n}{TAILLE}\PY{o}{+}\PY{l+m+mi}{4}\PY{p}{]}\PY{o}{=}\PY{o}{=}\PY{l+m+mi}{2}\PY{p}{)}\PY{p}{;} 
             
             \PY{n}{printf}\PY{p}{(}\PY{l+s}{\PYZdq{}}\PY{l+s}{\PYZpc{}s}\PY{l+s}{\PYZdq{}}\PY{p}{,} \PY{l+s}{\PYZdq{}}\PY{l+s+se}{\PYZbs{}n}\PY{l+s}{ Bravo ! Tous les tests passent.}\PY{l+s+se}{\PYZbs{}n}\PY{l+s}{\PYZdq{}}\PY{p}{)}\PY{p}{;}
             \PY{k}{return} \PY{n}{EXIT\PYZus{}SUCCESS}\PY{p}{;}
         \PY{p}{\PYZcb{}}
\end{Verbatim}


    \begin{Verbatim}[commandchars=\\\{\}]

 Bravo ! Tous les tests passent.

    \end{Verbatim}

    \textbf{Observation :} Ici, l'allocation et la réallocation se sont
déroulées avec succès et les données présentes avant la réallocation
sont toujours présentes après l'allocation. De plus, l'utilisation de
l'espace supplémentaire se fait normalement.

Dans l'exemple suivant, on vous propose d'augmenter la constante
pré-processeur \texttt{INC} pour observer l'échec de la demande de
réallocation. Vous pouvez manipuler le fichier
\textbf{\texttt{Exemple4\_4\_mauvaise\_bis.c}} si besoin.

    \begin{Verbatim}[commandchars=\\\{\}]
{\color{incolor}In [{\color{incolor}51}]:} \PY{c+cp}{\PYZsh{}}\PY{c+cp}{include} \PY{c+cpf}{\PYZlt{}assert.h\PYZgt{}}
         \PY{c+cp}{\PYZsh{}}\PY{c+cp}{include} \PY{c+cpf}{\PYZlt{}stdlib.h\PYZgt{}}
         \PY{c+cp}{\PYZsh{}}\PY{c+cp}{include} \PY{c+cpf}{\PYZlt{}stdio.h\PYZgt{}}
         
         \PY{c+cp}{\PYZsh{}}\PY{c+cp}{define TAILLE 1000000}
         \PY{c+cp}{\PYZsh{}}\PY{c+cp}{define INC 1e10}
         
         \PY{k+kt}{int} \PY{n+nf}{main}\PY{p}{(}\PY{p}{)}\PY{p}{\PYZob{}}
         
             \PY{c+c1}{// Allouer un tableau de TAILLE entiers.}
             \PY{k+kt}{int}\PY{o}{*} \PY{n}{tableau} \PY{o}{=} \PY{n}{malloc}\PY{p}{(}\PY{n}{TAILLE}\PY{o}{*}\PY{k}{sizeof}\PY{p}{(}\PY{k+kt}{int}\PY{p}{)}\PY{p}{)}\PY{p}{;}
             \PY{n}{assert}\PY{p}{(}\PY{n}{tableau}\PY{p}{)}\PY{p}{;} \PY{c+c1}{//allocation réussie ?}
             
             \PY{c+c1}{// Initialiser les éléments à 1}
             \PY{k}{for} \PY{p}{(}\PY{k+kt}{int} \PY{n}{i}\PY{o}{=}\PY{l+m+mi}{0}\PY{p}{;} \PY{n}{i}\PY{o}{\PYZlt{}}\PY{n}{TAILLE}\PY{p}{;} \PY{n}{i}\PY{o}{+}\PY{o}{+}\PY{p}{)}\PY{p}{\PYZob{}}
                 \PY{n}{tableau}\PY{p}{[}\PY{n}{i}\PY{p}{]}\PY{o}{=}\PY{l+m+mi}{1}\PY{p}{;}
             \PY{p}{\PYZcb{}}
             
             \PY{c+c1}{// Augmenter la taille du tableau pour enregistrer INC entiers supplémentaires.}
             \PY{n}{tableau} \PY{o}{=} \PY{n}{realloc}\PY{p}{(}\PY{n}{tableau}\PY{p}{,} \PY{p}{(}\PY{n}{TAILLE}\PY{o}{+}\PY{n}{INC}\PY{p}{)}\PY{o}{*}\PY{k}{sizeof}\PY{p}{(}\PY{k+kt}{int}\PY{p}{)}\PY{p}{)}\PY{p}{;}
             \PY{n}{assert}\PY{p}{(}\PY{n}{tableau}\PY{p}{)}\PY{p}{;}
             
             \PY{c+c1}{// Initialiser l\PYZsq{}élément d\PYZsq{}indice 0 à 2}
             \PY{n}{tableau}\PY{p}{[}\PY{l+m+mi}{0}\PY{p}{]}\PY{o}{=}\PY{l+m+mi}{2}\PY{p}{;}
             
             \PY{n}{printf}\PY{p}{(}\PY{l+s}{\PYZdq{}}\PY{l+s}{\PYZpc{}s}\PY{l+s}{\PYZdq{}}\PY{p}{,} \PY{l+s}{\PYZdq{}}\PY{l+s+se}{\PYZbs{}n}\PY{l+s}{ Bravo ! Tous les tests passent.}\PY{l+s+se}{\PYZbs{}n}\PY{l+s}{\PYZdq{}}\PY{p}{)}\PY{p}{;}
             \PY{k}{return} \PY{n}{EXIT\PYZus{}SUCCESS}\PY{p}{;}
         \PY{p}{\PYZcb{}}
\end{Verbatim}


    \begin{Verbatim}[commandchars=\\\{\}]
tmp4l14jo67.out: /tmp/tmp8k5uda3v.c:21: main: Assertion `tableau' failed.
[C kernel] Executable exited with code -6
    \end{Verbatim}

    \textbf{Observations :}

N.B. : Pour lire les obsevations, il faut afficher les numéros de ligne
du programme. Pour cela, il faut se rendre dans le menu 'Affichage' et
sélectionner l'option 'Afficher/Masquer les numéros de ligne'.

Voici les principales observations :

\begin{itemize}
\item
  L'assertion de la ligne 21 n'est pas vérifiée quand \texttt{INC}
  devient trop grand : le pointeur tableau est donc \texttt{NULL}.
\item
  Si on commente l'instruction de la ligne 21, et que l'on ré-exécute le
  programme, on obtient une erreur
  \texttt{Executable\ exited\ with\ code\ -11}, ce qui correspond à un
  accès à une adresse mémoire non valide (i.e. un segmentation fault).
  L'adresse non valide est l'adresse \texttt{NULL} car à la ligne 24, on
  accède au premier élément du tableau. Le premier élément n'est plus
  disponible ici car la réallocation a échoué, et le pointeur tableau a
  été mis à \texttt{NULL} à la ligne 20 par \texttt{realloc}.
\end{itemize}

\textbf{******* Ici, on a perdu l'accès aux données présentes dans le
tableau d'origine avant réallocation *******}

\begin{quote}
On observe ici \textbf{une double peine} : - La réallocation a échoué, -
Les données présentes dans le tableau avant l'appel à \texttt{realloc}
sont définitivement perdues.
\end{quote}

    \subsubsection{\texorpdfstring{Exemple d'une \emph{bonne} utilisation de
\texttt{realloc}.}{Exemple d'une bonne utilisation de realloc.}}\label{exemple-dune-bonne-utilisation-de-realloc.}

Vous pouvez manipuler le fichier \textbf{\texttt{Exemple4\_4\_bonne.c}}
si besoin.

    \begin{Verbatim}[commandchars=\\\{\}]
{\color{incolor}In [{\color{incolor}52}]:} \PY{c+cp}{\PYZsh{}}\PY{c+cp}{include} \PY{c+cpf}{\PYZlt{}assert.h\PYZgt{}}
         \PY{c+cp}{\PYZsh{}}\PY{c+cp}{include} \PY{c+cpf}{\PYZlt{}stdlib.h\PYZgt{}}
         \PY{c+cp}{\PYZsh{}}\PY{c+cp}{include} \PY{c+cpf}{\PYZlt{}stdio.h\PYZgt{}}
         
         \PY{c+cp}{\PYZsh{}}\PY{c+cp}{define TAILLE 1e6}
         \PY{c+cp}{\PYZsh{}}\PY{c+cp}{define INC 1e14}
         
         \PY{k+kt}{int} \PY{n+nf}{main}\PY{p}{(}\PY{p}{)}\PY{p}{\PYZob{}}
         
             \PY{c+c1}{// Allouer un tableau de TAILLE entiers initialisés à 1.}
             \PY{k+kt}{int}\PY{o}{*} \PY{n}{tableau} \PY{o}{=} \PY{n}{malloc}\PY{p}{(}\PY{n}{TAILLE}\PY{o}{*}\PY{k}{sizeof}\PY{p}{(}\PY{k+kt}{int}\PY{p}{)}\PY{p}{)}\PY{p}{;}
             \PY{n}{assert}\PY{p}{(}\PY{n}{tableau}\PY{p}{)}\PY{p}{;} \PY{c+c1}{//allocation réussie ?}
             
             \PY{c+c1}{// Initialisation à 1}
             \PY{k}{for} \PY{p}{(}\PY{k+kt}{int} \PY{n}{i}\PY{o}{=}\PY{l+m+mi}{0}\PY{p}{;} \PY{n}{i}\PY{o}{\PYZlt{}}\PY{n}{TAILLE}\PY{p}{;} \PY{n}{i}\PY{o}{+}\PY{o}{+}\PY{p}{)}\PY{p}{\PYZob{}}
                 \PY{n}{tableau}\PY{p}{[}\PY{n}{i}\PY{p}{]}\PY{o}{=}\PY{l+m+mi}{1}\PY{p}{;}
             \PY{p}{\PYZcb{}}
             
             \PY{c+c1}{// Augmentater la taille du tableau pour enregistrer 10 entiers.}
             \PY{k+kt}{int}\PY{o}{*} \PY{n}{nouveau} \PY{o}{=} \PY{n}{realloc}\PY{p}{(}\PY{n}{tableau}\PY{p}{,} \PY{p}{(}\PY{n}{TAILLE}\PY{o}{+}\PY{n}{INC}\PY{p}{)}\PY{o}{*}\PY{k}{sizeof}\PY{p}{(}\PY{k+kt}{int}\PY{p}{)}\PY{p}{)}\PY{p}{;}
             \PY{k}{if} \PY{p}{(}\PY{n}{nouveau}\PY{p}{)} \PY{p}{\PYZob{}}
                 \PY{c+c1}{//recopie de l\PYZsq{}adresse uniquement si succès }
                 \PY{n}{tableau} \PY{o}{=} \PY{n}{nouveau}\PY{p}{;}
             \PY{p}{\PYZcb{}}
             \PY{n}{assert}\PY{p}{(}\PY{n}{tableau}\PY{p}{[}\PY{l+m+mi}{0}\PY{p}{]}\PY{o}{=}\PY{o}{=}\PY{l+m+mi}{1}\PY{p}{)}\PY{p}{;}
             
             \PY{n}{printf}\PY{p}{(}\PY{l+s}{\PYZdq{}}\PY{l+s}{\PYZpc{}s}\PY{l+s}{\PYZdq{}}\PY{p}{,} \PY{l+s}{\PYZdq{}}\PY{l+s+se}{\PYZbs{}n}\PY{l+s}{ Bravo ! Tous les tests passent.}\PY{l+s+se}{\PYZbs{}n}\PY{l+s}{\PYZdq{}}\PY{p}{)}\PY{p}{;}
             \PY{k}{return} \PY{n}{EXIT\PYZus{}SUCCESS}\PY{p}{;}
         \PY{p}{\PYZcb{}}
\end{Verbatim}


    \begin{Verbatim}[commandchars=\\\{\}]

 Bravo ! Tous les tests passent.

    \end{Verbatim}

    \textbf{Observations :}

Un appel convenable à \texttt{realloc} est illustré à la ligne 20 : ici,
l'éventuel échec d'allocation mettra le pointeur \texttt{nouveau} à
\texttt{NULL}, et non le pointeur \texttt{tableau}. Si et seulement si
\texttt{nouveau} n'est pas \texttt{NULL}, on recopie la nouvelle adresse
dans le pointeur \texttt{tableau}. On évite ainsi la double peine.

    \begin{center}\rule{0.5\linewidth}{\linethickness}\end{center}

    \subsubsection{Exercice 3 : Ré-allocation et jeu de
Uno.}\label{exercice-3-ruxe9-allocation-et-jeu-de-uno.}

Toujour dans le jeu de Uno, on cherche à tester la procédure qui permet
de piocher une carte parmi les cartes qui ne sont pas en jeu. Une carte
est en jeu si elle appartient à une des deux mains, ou si elle est la
dernière jouée.

Dans le tableau statique \texttt{le\_jeu} qui contient toutes les cartes
du jeu, on référence une carte comme 'présente' si elle n'est pas en
jeu. Quand on pioche, on cherche la première carte présente dans
\texttt{le\_jeu} et on l'ajoute à la main courante. La carte piochée du
tableau \texttt{le\_jeu} y est indiquée comme non-présente.

L'objet de cet exercice est d'exécuter les tests de la procédure
\texttt{carte\ *\ piocher(jeu\ le\_jeu,\ t\_main*\ main)} qui : -
retourne un pointeur sur la carte piochée dans le jeu - null si elle
aucune carte ne peut être piochée ou si la réallocation de mémoire
échoue.

L'essentiel du travail est à réaliser au niveau de cette procédure
\texttt{piocher}. Vous pouvez manipuler le fichier
\textbf{\texttt{Exercice3.c}} si besoin.

    \begin{Verbatim}[commandchars=\\\{\}]
{\color{incolor}In [{\color{incolor}81}]:} \PY{c+c1}{// Consignes : }
         \PY{c+c1}{//  1. Compléter avec les instructions requises en lieu et place de *** TODO ***}
         
         \PY{c+cp}{\PYZsh{}}\PY{c+cp}{include} \PY{c+cpf}{\PYZlt{}assert.h\PYZgt{}}
         \PY{c+cp}{\PYZsh{}}\PY{c+cp}{include} \PY{c+cpf}{\PYZlt{}stdlib.h\PYZgt{}}
         \PY{c+cp}{\PYZsh{}}\PY{c+cp}{include} \PY{c+cpf}{\PYZlt{}stdio.h\PYZgt{}}
         \PY{c+cp}{\PYZsh{}}\PY{c+cp}{include} \PY{c+cpf}{\PYZlt{}stdbool.h\PYZgt{}}
         \PY{c+cp}{\PYZsh{}}\PY{c+cp}{include} \PY{c+cpf}{\PYZlt{}time.h\PYZgt{}}
         
         \PY{c+cp}{\PYZsh{}}\PY{c+cp}{define NB\PYZus{}VALEURS 10}
         \PY{c+cp}{\PYZsh{}}\PY{c+cp}{define NB\PYZus{}CARTES 4*NB\PYZus{}VALEURS}
         
         \PY{c+c1}{//Définition du type enseigne}
         \PY{k}{enum} \PY{n}{couleur} \PY{p}{\PYZob{}}\PY{n}{JAUNE}\PY{p}{,} \PY{n}{ROUGE}\PY{p}{,} \PY{n}{VERT}\PY{p}{,} \PY{n}{BLEU}\PY{p}{\PYZcb{}}\PY{p}{;}
         \PY{k}{typedef} \PY{k}{enum} \PY{n}{couleur} \PY{n}{couleur}\PY{p}{;}
         
         \PY{c+c1}{//Tableau de caractères représentant les couleurs}
         \PY{k+kt}{char} \PY{n}{C}\PY{p}{[}\PY{l+m+mi}{4}\PY{p}{]} \PY{o}{=} \PY{p}{\PYZob{}}\PY{l+s+sc}{\PYZsq{}}\PY{l+s+sc}{J}\PY{l+s+sc}{\PYZsq{}}\PY{p}{,} \PY{l+s+sc}{\PYZsq{}}\PY{l+s+sc}{R}\PY{l+s+sc}{\PYZsq{}}\PY{p}{,} \PY{l+s+sc}{\PYZsq{}}\PY{l+s+sc}{V}\PY{l+s+sc}{\PYZsq{}}\PY{p}{,} \PY{l+s+sc}{\PYZsq{}}\PY{l+s+sc}{B}\PY{l+s+sc}{\PYZsq{}}\PY{p}{\PYZcb{}}\PY{p}{;}
         
         \PY{c+c1}{//Définition du type carte}
         \PY{k}{struct} \PY{n+nc}{carte} \PY{p}{\PYZob{}}
             \PY{n}{couleur} \PY{n}{couleur}\PY{p}{;}
             \PY{k+kt}{int} \PY{n}{valeur}\PY{p}{;} \PY{c+c1}{// Invariant : valeur \PYZgt{}= 0 \PYZam{}\PYZam{} valeur \PYZlt{} NB\PYZus{}VALEURS}
             \PY{k+kt}{bool} \PY{n}{presente}\PY{p}{;} \PY{c+c1}{// la carte est\PYZhy{}elle presente dans le jeu ?}
         \PY{p}{\PYZcb{}}\PY{p}{;}
         \PY{k}{typedef} \PY{k}{struct} \PY{n+nc}{carte} \PY{n}{carte}\PY{p}{;}
         
         \PY{c+c1}{//Définition du type jeu complet pour enregistrer NB\PYZus{}CARTES cartes.}
         \PY{k}{typedef} \PY{n}{carte} \PY{n}{jeu}\PY{p}{[}\PY{n}{NB\PYZus{}CARTES}\PY{p}{]}\PY{p}{;}
         
         \PY{c+c1}{//Définition du type t\PYZus{}main, capable d\PYZsq{}enregistrer un nombre variable de cartes.}
         \PY{k}{struct} \PY{n+nc}{main} \PY{p}{\PYZob{}}
             \PY{n}{carte} \PY{o}{*} \PY{n}{main}\PY{p}{;} \PY{c+c1}{//tableau des cartes dans la main. }
             \PY{k+kt}{int} \PY{n}{nb}\PY{p}{;} \PY{c+c1}{//monbre de cartes}
         \PY{p}{\PYZcb{}}\PY{p}{;}
         \PY{k}{typedef} \PY{k}{struct} \PY{n+nc}{main} \PY{n}{t\PYZus{}main}\PY{p}{;}
         
         
         \PY{c+cm}{/**}
         \PY{c+cm}{ * \PYZbs{}brief Initialiser une carte avec une couleur et une valeur. }
         \PY{c+cm}{ * \PYZbs{}param[in] c couleur de la carte}
         \PY{c+cm}{ * \PYZbs{}param[in] v valeur de la carte}
         \PY{c+cm}{ * \PYZbs{}param[in] ej booléen presente}
         \PY{c+cm}{ * \PYZbs{}param[out] la\PYZus{}carte }
         \PY{c+cm}{ */}
         \PY{k+kt}{void} \PY{n+nf}{init\PYZus{}carte}\PY{p}{(}\PY{n}{carte}\PY{o}{*} \PY{n}{la\PYZus{}carte}\PY{p}{,} \PY{n}{couleur} \PY{n}{c}\PY{p}{,} \PY{k+kt}{int} \PY{n}{v}\PY{p}{,} \PY{k+kt}{bool} \PY{n}{pr}\PY{p}{)}\PY{p}{\PYZob{}}
             \PY{n}{la\PYZus{}carte}\PY{o}{\PYZhy{}}\PY{o}{\PYZgt{}}\PY{n}{couleur} \PY{o}{=} \PY{n}{c}\PY{p}{;}
             \PY{n}{la\PYZus{}carte}\PY{o}{\PYZhy{}}\PY{o}{\PYZgt{}}\PY{n}{valeur} \PY{o}{=} \PY{n}{v}\PY{p}{;}
             \PY{n}{la\PYZus{}carte}\PY{o}{\PYZhy{}}\PY{o}{\PYZgt{}}\PY{n}{presente} \PY{o}{=} \PY{n}{pr}\PY{p}{;}
         \PY{p}{\PYZcb{}}
         
         \PY{c+cm}{/**}
         \PY{c+cm}{ * \PYZbs{}brief Vérifie si la valeur de la carte est conforme à l\PYZsq{}invariant.}
         \PY{c+cm}{ * \PYZbs{}param[in] c la carte}
         \PY{c+cm}{ * \PYZbs{}return bool vrai si la valeur est conforme, faux sinon.}
         \PY{c+cm}{ */}
          \PY{k+kt}{bool} \PY{n+nf}{est\PYZus{}conforme}\PY{p}{(}\PY{n}{carte} \PY{n}{c}\PY{p}{)}\PY{p}{\PYZob{}}
             \PY{k}{return} \PY{p}{(}\PY{n}{c}\PY{p}{.}\PY{n}{valeur}\PY{o}{\PYZgt{}}\PY{o}{=}\PY{l+m+mi}{0} \PY{o}{\PYZam{}}\PY{o}{\PYZam{}} \PY{n}{c}\PY{p}{.}\PY{n}{valeur}\PY{o}{\PYZlt{}}\PY{n}{NB\PYZus{}VALEURS}\PY{p}{)}\PY{p}{;}
         \PY{p}{\PYZcb{}}
         
         \PY{c+cm}{/**}
         \PY{c+cm}{ * \PYZbs{}brief Copie les valeurs de la carte src dans la carte dest.}
         \PY{c+cm}{ * \PYZbs{}param[in] src carte à copier}
         \PY{c+cm}{ * \PYZbs{}param[out] dest carte destination de la copie }
         \PY{c+cm}{ */}
         \PY{k+kt}{void} \PY{n+nf}{copier\PYZus{}carte}\PY{p}{(}\PY{n}{carte}\PY{o}{*} \PY{n}{dest}\PY{p}{,} \PY{n}{carte} \PY{n}{src}\PY{p}{)}\PY{p}{\PYZob{}}
             \PY{n}{dest}\PY{o}{\PYZhy{}}\PY{o}{\PYZgt{}}\PY{n}{couleur} \PY{o}{=} \PY{n}{src}\PY{p}{.}\PY{n}{couleur}\PY{p}{;}
             \PY{n}{dest}\PY{o}{\PYZhy{}}\PY{o}{\PYZgt{}}\PY{n}{valeur} \PY{o}{=} \PY{n}{src}\PY{p}{.}\PY{n}{valeur}\PY{p}{;}
             \PY{n}{dest}\PY{o}{\PYZhy{}}\PY{o}{\PYZgt{}}\PY{n}{presente} \PY{o}{=} \PY{n}{src}\PY{p}{.}\PY{n}{presente}\PY{p}{;}
         \PY{p}{\PYZcb{}}
         
         
         \PY{c+cm}{/**}
         \PY{c+cm}{ * \PYZbs{}brief Afficher une carte.}
         \PY{c+cm}{ * \PYZbs{}param[in] cte carte à afficher}
         \PY{c+cm}{ */}
         \PY{k+kt}{void} \PY{n+nf}{afficher\PYZus{}carte}\PY{p}{(}\PY{n}{carte} \PY{n}{cte}\PY{p}{)}\PY{p}{\PYZob{}}
             \PY{n}{printf}\PY{p}{(}\PY{l+s}{\PYZdq{}}\PY{l+s}{(\PYZpc{}c;\PYZpc{}d;\PYZpc{}d)}\PY{l+s+se}{\PYZbs{}t}\PY{l+s}{\PYZdq{}}\PY{p}{,} \PY{n}{C}\PY{p}{[}\PY{n}{cte}\PY{p}{.}\PY{n}{couleur}\PY{p}{]}\PY{p}{,}\PY{n}{cte}\PY{p}{.}\PY{n}{valeur}\PY{p}{,} \PY{n}{cte}\PY{p}{.}\PY{n}{presente}\PY{p}{)}\PY{p}{;}
         \PY{p}{\PYZcb{}}
         
         \PY{c+cm}{/**}
         \PY{c+cm}{ * \PYZbs{}brief Initialiser une main.}
         \PY{c+cm}{ * \PYZbs{}param[in] N nombre de cartes composant la main.  Précondition : N \PYZlt{}= (NB\PYZus{}CARTES \PYZhy{} 1) div 2}
         \PY{c+cm}{ * \PYZbs{}param[out] la\PYZus{}main créée}
         \PY{c+cm}{ * \PYZbs{}return true si l\PYZsq{}initialisation a échouée.}
         \PY{c+cm}{ */}
         \PY{k+kt}{bool} \PY{n+nf}{init\PYZus{}main}\PY{p}{(}\PY{n}{t\PYZus{}main}\PY{o}{*} \PY{n}{la\PYZus{}main}\PY{p}{,} \PY{k+kt}{int} \PY{n}{N}\PY{p}{)}\PY{p}{\PYZob{}}
             \PY{n}{assert}\PY{p}{(}\PY{n}{N} \PY{o}{\PYZlt{}}\PY{o}{=} \PY{p}{(}\PY{n}{NB\PYZus{}CARTES}\PY{l+m+mi}{\PYZhy{}1}\PY{p}{)}\PY{o}{/}\PY{l+m+mi}{2}\PY{p}{)}\PY{p}{;}
             \PY{n}{la\PYZus{}main}\PY{o}{\PYZhy{}}\PY{o}{\PYZgt{}}\PY{n}{main} \PY{o}{=} \PY{n}{malloc}\PY{p}{(}\PY{n}{N}\PY{o}{*}\PY{k}{sizeof}\PY{p}{(}\PY{n}{carte}\PY{p}{)}\PY{p}{)}\PY{p}{;}
             \PY{n}{la\PYZus{}main}\PY{o}{\PYZhy{}}\PY{o}{\PYZgt{}}\PY{n}{nb} \PY{o}{=} \PY{n}{N}\PY{p}{;}
             \PY{k}{return} \PY{p}{(}\PY{n}{la\PYZus{}main}\PY{o}{=}\PY{o}{=}\PY{n+nb}{NULL}\PY{p}{)}\PY{p}{;} \PY{c+c1}{//allocation réussie ?}
         \PY{p}{\PYZcb{}}
         
         \PY{c+cm}{/**}
         \PY{c+cm}{ * \PYZbs{}brief Initialiser le jeu en ajoutant toutes les cartes possibles au jeu.}
         \PY{c+cm}{ * \PYZbs{}param[out] le\PYZus{}jeu tableau de cartes avec les 4 couleurs et NB\PYZus{}VALEURS valeurs possibles}
         \PY{c+cm}{ */}
         \PY{k+kt}{void} \PY{n+nf}{init\PYZus{}jeu}\PY{p}{(}\PY{n}{jeu} \PY{n}{le\PYZus{}jeu}\PY{p}{)}\PY{p}{\PYZob{}}
             \PY{k+kt}{int} \PY{n}{k}\PY{o}{=}\PY{l+m+mi}{0}\PY{p}{;}
             \PY{k}{for} \PY{p}{(}\PY{k+kt}{int} \PY{n}{i}\PY{o}{=}\PY{l+m+mi}{0} \PY{p}{;} \PY{n}{i}\PY{o}{\PYZlt{}}\PY{l+m+mi}{4} \PY{p}{;} \PY{n}{i}\PY{o}{+}\PY{o}{+}\PY{p}{)}\PY{p}{\PYZob{}}
                 \PY{k}{for} \PY{p}{(}\PY{k+kt}{int} \PY{n}{j}\PY{o}{=}\PY{l+m+mi}{0} \PY{p}{;} \PY{n}{j}\PY{o}{\PYZlt{}}\PY{n}{NB\PYZus{}VALEURS} \PY{p}{;} \PY{n}{j}\PY{o}{+}\PY{o}{+}\PY{p}{)}\PY{p}{\PYZob{}}
                     \PY{n}{init\PYZus{}carte}\PY{p}{(}\PY{o}{\PYZam{}}\PY{p}{(}\PY{n}{le\PYZus{}jeu}\PY{p}{[}\PY{n}{k}\PY{p}{]}\PY{p}{)}\PY{p}{,} \PY{n}{i}\PY{p}{,} \PY{n}{j}\PY{p}{,} \PY{n+nb}{true}\PY{p}{)}\PY{p}{;}
                     \PY{n}{k}\PY{o}{+}\PY{o}{+}\PY{p}{;}
                 \PY{p}{\PYZcb{}}
             \PY{p}{\PYZcb{}}
         \PY{p}{\PYZcb{}}
         
         \PY{c+cm}{/**}
         \PY{c+cm}{ * \PYZbs{}brief Afficher le jeu.}
         \PY{c+cm}{ * \PYZbs{}param[in] le\PYZus{}jeu complet avec les 4 couleurs et 910valeurs possibles}
         \PY{c+cm}{ */}
         \PY{k+kt}{void} \PY{n+nf}{afficher\PYZus{}jeu}\PY{p}{(}\PY{n}{jeu} \PY{n}{le\PYZus{}jeu}\PY{p}{)}\PY{p}{\PYZob{}}
             \PY{k}{for} \PY{p}{(}\PY{k+kt}{int} \PY{n}{k}\PY{o}{=}\PY{l+m+mi}{0}\PY{p}{;} \PY{n}{k}\PY{o}{\PYZlt{}}\PY{n}{NB\PYZus{}CARTES}\PY{p}{;} \PY{n}{k}\PY{o}{+}\PY{o}{+}\PY{p}{)}\PY{p}{\PYZob{}}
                 \PY{n}{afficher\PYZus{}carte}\PY{p}{(}\PY{n}{le\PYZus{}jeu}\PY{p}{[}\PY{n}{k}\PY{p}{]}\PY{p}{)}\PY{p}{;}
             \PY{p}{\PYZcb{}}
             \PY{n}{printf}\PY{p}{(}\PY{l+s}{\PYZdq{}}\PY{l+s+se}{\PYZbs{}n}\PY{l+s}{\PYZdq{}}\PY{p}{)}\PY{p}{;}
         \PY{p}{\PYZcb{}}
         
         \PY{c+cm}{/**}
         \PY{c+cm}{ * \PYZbs{}brief Afficher une main.}
         \PY{c+cm}{ * \PYZbs{}param[in] la\PYZus{}main la main a afficher}
         \PY{c+cm}{ */}
         \PY{k+kt}{void} \PY{n+nf}{afficher\PYZus{}main}\PY{p}{(}\PY{n}{t\PYZus{}main} \PY{n}{la\PYZus{}main}\PY{p}{)}\PY{p}{\PYZob{}}
             \PY{k}{for} \PY{p}{(}\PY{k+kt}{int} \PY{n}{k}\PY{o}{=}\PY{l+m+mi}{0}\PY{p}{;} \PY{n}{k}\PY{o}{\PYZlt{}}\PY{n}{la\PYZus{}main}\PY{p}{.}\PY{n}{nb}\PY{p}{;} \PY{n}{k}\PY{o}{+}\PY{o}{+}\PY{p}{)}\PY{p}{\PYZob{}}
                 \PY{n}{afficher\PYZus{}carte}\PY{p}{(}\PY{n}{la\PYZus{}main}\PY{p}{.}\PY{n}{main}\PY{p}{[}\PY{n}{k}\PY{p}{]}\PY{p}{)}\PY{p}{;}
             \PY{p}{\PYZcb{}}
             \PY{n}{printf}\PY{p}{(}\PY{l+s}{\PYZdq{}}\PY{l+s+se}{\PYZbs{}n}\PY{l+s}{\PYZdq{}}\PY{p}{)}\PY{p}{;}
         \PY{p}{\PYZcb{}}
         
         \PY{c+cm}{/**}
         \PY{c+cm}{ * \PYZbs{}brief mélange le jeu.}
         \PY{c+cm}{ * \PYZbs{}param[in out] le\PYZus{}jeu complet}
         \PY{c+cm}{ */}
         \PY{k+kt}{void} \PY{n+nf}{melanger\PYZus{}jeu}\PY{p}{(}\PY{n}{jeu} \PY{n}{le\PYZus{}jeu}\PY{p}{)}\PY{p}{\PYZob{}}
             \PY{k}{for} \PY{p}{(}\PY{k+kt}{int} \PY{n}{i}\PY{o}{=}\PY{l+m+mi}{0}\PY{p}{;} \PY{n}{i}\PY{o}{\PYZlt{}}\PY{l+m+mi}{1000}\PY{p}{;} \PY{n}{i}\PY{o}{+}\PY{o}{+}\PY{p}{)}\PY{p}{\PYZob{}}
                 \PY{c+c1}{// Choisir deux cartes aléatoirement}
                 \PY{k+kt}{int} \PY{n}{i} \PY{o}{=} \PY{n}{rand}\PY{p}{(}\PY{p}{)}\PY{o}{\PYZpc{}}\PY{n}{NB\PYZus{}CARTES}\PY{p}{;}
                 \PY{k+kt}{int} \PY{n}{j} \PY{o}{=} \PY{n}{rand}\PY{p}{(}\PY{p}{)}\PY{o}{\PYZpc{}}\PY{n}{NB\PYZus{}CARTES}\PY{p}{;}        
                 \PY{c+c1}{// Les échanger}
                 \PY{n}{carte} \PY{n}{mem}\PY{p}{;}
                 \PY{n}{copier\PYZus{}carte}\PY{p}{(}\PY{o}{\PYZam{}}\PY{n}{mem}\PY{p}{,} \PY{n}{le\PYZus{}jeu}\PY{p}{[}\PY{n}{i}\PY{p}{]}\PY{p}{)}\PY{p}{;} 
                 \PY{n}{copier\PYZus{}carte}\PY{p}{(}\PY{o}{\PYZam{}}\PY{p}{(}\PY{n}{le\PYZus{}jeu}\PY{p}{[}\PY{n}{i}\PY{p}{]}\PY{p}{)}\PY{p}{,} \PY{n}{le\PYZus{}jeu}\PY{p}{[}\PY{n}{j}\PY{p}{]}\PY{p}{)}\PY{p}{;} 
                 \PY{n}{copier\PYZus{}carte}\PY{p}{(}\PY{o}{\PYZam{}}\PY{p}{(}\PY{n}{le\PYZus{}jeu}\PY{p}{[}\PY{n}{j}\PY{p}{]}\PY{p}{)}\PY{p}{,} \PY{n}{mem}\PY{p}{)}\PY{p}{;} 
             \PY{p}{\PYZcb{}}
         \PY{p}{\PYZcb{}}
         
         
         \PY{c+cm}{/**}
         \PY{c+cm}{ \PYZbs{}brief Distribuer N cartes à chacun des deux joueurs, en alternant les joueurs.}
         \PY{c+cm}{ * \PYZbs{}param[in out] le\PYZus{}jeu complet.}
         \PY{c+cm}{ *       Si la carte c est distribuée dans une main, c.presente devient faux.}
         \PY{c+cm}{ * \PYZbs{}param[in] N nombre de cartes distribuées à chaque joueur.  Précondition : N \PYZlt{}= (NB\PYZus{}CARTES \PYZhy{} 1) div 2}
         \PY{c+cm}{ * \PYZbs{}param[out] m1 main du joueur 1.}
         \PY{c+cm}{ * \PYZbs{}param[out] m2 main du joueur 2.}
         \PY{c+cm}{ */}
         \PY{k+kt}{void} \PY{n+nf}{distribuer\PYZus{}mains}\PY{p}{(}\PY{n}{jeu} \PY{n}{le\PYZus{}jeu}\PY{p}{,} \PY{k+kt}{int} \PY{n}{N}\PY{p}{,} \PY{n}{t\PYZus{}main}\PY{o}{*} \PY{n}{m1}\PY{p}{,} \PY{n}{t\PYZus{}main}\PY{o}{*} \PY{n}{m2}\PY{p}{)}\PY{p}{\PYZob{}}
             \PY{n}{assert}\PY{p}{(}\PY{n}{N} \PY{o}{\PYZlt{}}\PY{o}{=} \PY{p}{(}\PY{n}{NB\PYZus{}CARTES}\PY{l+m+mi}{\PYZhy{}1}\PY{p}{)}\PY{o}{/}\PY{l+m+mi}{2}\PY{p}{)}\PY{p}{;}
         
             \PY{c+c1}{//Initialiser les mains de N cartes}
             \PY{k+kt}{bool} \PY{n}{errA} \PY{o}{=} \PY{n}{init\PYZus{}main}\PY{p}{(}\PY{n}{m1}\PY{p}{,} \PY{n}{N}\PY{p}{)}\PY{p}{;}
             \PY{k+kt}{bool} \PY{n}{errB} \PY{o}{=} \PY{n}{init\PYZus{}main}\PY{p}{(}\PY{n}{m2}\PY{p}{,} \PY{n}{N}\PY{p}{)}\PY{p}{;}
             \PY{n}{assert}\PY{p}{(}\PY{o}{!}\PY{n}{errA} \PY{o}{\PYZam{}}\PY{o}{\PYZam{}} \PY{o}{!}\PY{n}{errB}\PY{p}{)}\PY{p}{;}
             
             \PY{c+c1}{//Distribuer les cartes}
             \PY{k}{for} \PY{p}{(}\PY{k+kt}{int} \PY{n}{i}\PY{o}{=}\PY{l+m+mi}{0}\PY{p}{;} \PY{n}{i}\PY{o}{\PYZlt{}}\PY{n}{N}\PY{p}{;} \PY{n}{i}\PY{o}{+}\PY{o}{+}\PY{p}{)}\PY{p}{\PYZob{}}
                 \PY{c+c1}{// ajout d\PYZsq{}une carte dans la main m1}
                 \PY{n}{copier\PYZus{}carte}\PY{p}{(}\PY{o}{\PYZam{}}\PY{p}{(}\PY{n}{m1}\PY{o}{\PYZhy{}}\PY{o}{\PYZgt{}}\PY{n}{main}\PY{p}{[}\PY{n}{i}\PY{p}{]}\PY{p}{)}\PY{p}{,} \PY{n}{le\PYZus{}jeu}\PY{p}{[}\PY{l+m+mi}{2}\PY{o}{*}\PY{n}{i}\PY{p}{]}\PY{p}{)}\PY{p}{;}
                 \PY{c+c1}{// ajout d\PYZsq{}une carte dans la main m2}
                 \PY{n}{copier\PYZus{}carte}\PY{p}{(}\PY{o}{\PYZam{}}\PY{p}{(}\PY{n}{m2}\PY{o}{\PYZhy{}}\PY{o}{\PYZgt{}}\PY{n}{main}\PY{p}{[}\PY{n}{i}\PY{p}{]}\PY{p}{)}\PY{p}{,} \PY{n}{le\PYZus{}jeu}\PY{p}{[}\PY{l+m+mi}{2}\PY{o}{*}\PY{n}{i}\PY{o}{+}\PY{l+m+mi}{1}\PY{p}{]}\PY{p}{)}\PY{p}{;}
                 \PY{c+c1}{//mise à jour de presente à false dans le\PYZus{}jeu}
                 \PY{n}{le\PYZus{}jeu}\PY{p}{[}\PY{l+m+mi}{2}\PY{o}{*}\PY{n}{i}\PY{p}{]}\PY{p}{.}\PY{n}{presente} \PY{o}{=} \PY{n+nb}{false}\PY{p}{;}
                 \PY{n}{le\PYZus{}jeu}\PY{p}{[}\PY{l+m+mi}{2}\PY{o}{*}\PY{n}{i}\PY{o}{+}\PY{l+m+mi}{1}\PY{p}{]}\PY{p}{.}\PY{n}{presente} \PY{o}{=} \PY{n+nb}{false}\PY{p}{;}
             \PY{p}{\PYZcb{}}
         \PY{p}{\PYZcb{}}
         
         \PY{c+cm}{/**}
         \PY{c+cm}{ * \PYZbs{}brief Vérifie si les cartes c1 et c2 on la même couleur et la même valeur.}
         \PY{c+cm}{ * \PYZbs{}param[in] c1 carte}
         \PY{c+cm}{ * \PYZbs{}param[in] c2 carte}
         \PY{c+cm}{ * \PYZbs{}return bool Vrai si les deux cartes ont même valeur et couleur.}
         \PY{c+cm}{*/}
         \PY{k+kt}{bool} \PY{n+nf}{est\PYZus{}egale}\PY{p}{(}\PY{n}{carte} \PY{n}{c1}\PY{p}{,} \PY{n}{carte} \PY{n}{c2}\PY{p}{)}\PY{p}{\PYZob{}}
             \PY{k}{return} \PY{p}{(}\PY{p}{(}\PY{n}{c1}\PY{p}{.}\PY{n}{couleur} \PY{o}{=}\PY{o}{=} \PY{n}{c2}\PY{p}{.}\PY{n}{couleur}\PY{p}{)} \PY{o}{\PYZam{}}\PY{o}{\PYZam{}} \PY{p}{(}\PY{n}{c1}\PY{p}{.}\PY{n}{valeur} \PY{o}{=}\PY{o}{=} \PY{n}{c2}\PY{p}{.}\PY{n}{valeur}\PY{p}{)}\PY{p}{)}\PY{p}{;}
         \PY{p}{\PYZcb{}}
         
         \PY{c+cm}{/**}
         \PY{c+cm}{ * \PYZbs{}brief Vérifie si la carte c est présente dans la main.}
         \PY{c+cm}{ * \PYZbs{}param[in] main main d\PYZsq{}un joueur}
         \PY{c+cm}{ * \PYZbs{}param[in] c carte}
         \PY{c+cm}{ * \PYZbs{}return bool Vrai si la carte est presente dans la main, faux sinon.}
         \PY{c+cm}{*/}
         \PY{k+kt}{bool} \PY{n+nf}{est\PYZus{}presente\PYZus{}main}\PY{p}{(}\PY{n}{t\PYZus{}main} \PY{n}{main}\PY{p}{,} \PY{n}{carte} \PY{n}{c}\PY{p}{)}\PY{p}{\PYZob{}}
             \PY{k+kt}{int} \PY{n}{i} \PY{o}{=} \PY{l+m+mi}{0}\PY{p}{;}
             \PY{k}{while} \PY{p}{(}\PY{n}{i} \PY{o}{\PYZlt{}} \PY{n}{main}\PY{p}{.}\PY{n}{nb} \PY{o}{\PYZam{}}\PY{o}{\PYZam{}} \PY{o}{!}\PY{n}{est\PYZus{}egale}\PY{p}{(}\PY{n}{main}\PY{p}{.}\PY{n}{main}\PY{p}{[}\PY{n}{i}\PY{p}{]}\PY{p}{,} \PY{n}{c}\PY{p}{)}\PY{p}{)} \PY{p}{\PYZob{}}
                 \PY{n}{i}\PY{o}{+}\PY{o}{+}\PY{p}{;}
             \PY{p}{\PYZcb{}}
             \PY{k}{return} \PY{o}{!}\PY{p}{(}\PY{n}{i} \PY{o}{=}\PY{o}{=} \PY{n}{main}\PY{p}{.}\PY{n}{nb}\PY{p}{)}\PY{p}{;}
         \PY{p}{\PYZcb{}}
         
         
         \PY{c+cm}{/**}
         \PY{c+cm}{ * \PYZbs{}brief Piocher une carte dans le jeu et l\PYZsq{}inclure dans la main en paramètre.}
         \PY{c+cm}{ * \PYZbs{}param[in out] le\PYZus{}jeu complet avec les 4 couleurs et 10 valeurs possibles.}
         \PY{c+cm}{ *                Ce jeu est mélangé.}
         \PY{c+cm}{ *                Si la carte est inclue dans une main ou est la derniere carte jouée,}
         \PY{c+cm}{ *                Alors carte.presente vaut faux.}
         \PY{c+cm}{ * \PYZbs{}param[in out] main main d\PYZsq{}un joueur}
         \PY{c+cm}{ * \PYZbs{}return carte * un pointeur sur la carte piochee dans le\PYZus{}jeu en paramètre. }
         \PY{c+cm}{ * Ce pointeur vaut NULL si aucune carte ne peut être piochée ou si l\PYZsq{}allocation de mémoire échoue.}
         \PY{c+cm}{*/}
         \PY{n}{carte} \PY{o}{*} \PY{n+nf}{piocher}\PY{p}{(}\PY{n}{jeu} \PY{n}{le\PYZus{}jeu}\PY{p}{,} \PY{n}{t\PYZus{}main}\PY{o}{*} \PY{n}{main}\PY{p}{)}\PY{p}{\PYZob{}}
             \PY{c+c1}{// Recherche une carte presente dans le jeu.}
             \PY{n}{carte} \PY{o}{*}\PY{n}{carte\PYZus{}piochee} \PY{o}{=} \PY{n}{le\PYZus{}jeu}\PY{p}{;}
             \PY{k+kt}{int} \PY{n}{i} \PY{o}{=} \PY{l+m+mi}{0}\PY{p}{;}
             \PY{k}{while}\PY{p}{(}\PY{n}{i} \PY{o}{\PYZlt{}} \PY{n}{NB\PYZus{}CARTES} \PY{o}{\PYZam{}}\PY{o}{\PYZam{}} \PY{n}{carte\PYZus{}piochee}\PY{o}{\PYZhy{}}\PY{o}{\PYZgt{}}\PY{n}{presente} \PY{o}{=}\PY{o}{=} \PY{n+nb}{false}\PY{p}{)}\PY{p}{\PYZob{}}
                 \PY{n}{carte\PYZus{}piochee} \PY{o}{=} \PY{n}{carte\PYZus{}piochee} \PY{o}{+} \PY{l+m+mi}{1}\PY{p}{;}
                 \PY{n}{i}\PY{o}{+}\PY{o}{+}\PY{p}{;}
             \PY{p}{\PYZcb{}}
             \PY{k}{if} \PY{p}{(}\PY{n}{i} \PY{o}{=}\PY{o}{=} \PY{n}{NB\PYZus{}CARTES}\PY{p}{)} \PY{p}{\PYZob{}}
                 \PY{n}{carte\PYZus{}piochee} \PY{o}{=} \PY{n+nb}{NULL}\PY{p}{;}
             \PY{p}{\PYZcb{}} \PY{k}{else} \PY{p}{\PYZob{}}
                 \PY{c+c1}{// Inserer la carte dans la main       }
                 \PY{c+c1}{//*** TODO *** ;}
                 
                 \PY{c+c1}{// Reallouer la mémoire pour enregistrer une carte de plus dans la main.}
                 \PY{n}{t\PYZus{}main}\PY{o}{*} \PY{n}{nouveau} \PY{o}{=} \PY{n}{realloc}\PY{p}{(}\PY{n}{main}\PY{p}{,}\PY{p}{(}\PY{p}{(}\PY{n}{main}\PY{o}{\PYZhy{}}\PY{o}{\PYZgt{}}\PY{n}{nb}\PY{p}{)}\PY{o}{+}\PY{l+m+mi}{1}\PY{p}{)}\PY{o}{*}\PY{k}{sizeof}\PY{p}{(}\PY{n}{carte}\PY{p}{)}\PY{p}{)}\PY{p}{;}
                 \PY{c+c1}{// Penser à l\PYZsq{}echec de la reallocation}
                 \PY{k}{if} \PY{p}{(}\PY{n}{nouveau}\PY{p}{)} \PY{p}{\PYZob{}}
                         \PY{n}{main} \PY{o}{=} \PY{n}{nouveau}\PY{p}{;}
                 \PY{p}{\PYZcb{}}
                 \PY{c+c1}{// Copier la carte\PYZus{}piochee dans la main}
                   \PY{n}{copier\PYZus{}carte}\PY{p}{(}\PY{n}{main}\PY{o}{\PYZhy{}}\PY{o}{\PYZgt{}}\PY{n}{main}\PY{p}{[}\PY{n}{main}\PY{o}{\PYZhy{}}\PY{o}{\PYZgt{}}\PY{n}{nb}\PY{p}{]}\PY{p}{,}\PY{n}{le\PYZus{}jeu}\PY{p}{[}\PY{n}{i}\PY{p}{]}\PY{p}{)}\PY{p}{;}
                 \PY{c+c1}{// Indiquer que carte\PYZus{}piochee n\PYZsq{}est plus presente dans le\PYZus{}jeu}
                   \PY{n}{le\PYZus{}jeu}\PY{p}{[}\PY{n}{i}\PY{p}{]}\PY{p}{.}\PY{n}{presente} \PY{o}{=} \PY{n+nb}{false}\PY{p}{;}
             \PY{p}{\PYZcb{}}
             \PY{k}{return} \PY{n}{carte\PYZus{}piochee}\PY{p}{;}
         \PY{p}{\PYZcb{}}
         
         \PY{c+cm}{/**}
         \PY{c+cm}{ * \PYZbs{}brief Initialise le jeu de carte, les mains des joueurs et la carte \PYZsq{}last\PYZsq{}.}
         \PY{c+cm}{ * \PYZbs{}param[out] le\PYZus{}jeu complet avec les 4 couleurs et 10 valeurs possibles.}
         \PY{c+cm}{ *                Ce jeu est mélangé.}
         \PY{c+cm}{ *                Si la carte est inclue dans une main ou est la derniere carte jouée,}
         \PY{c+cm}{ *                Alors carte.presente vaut faux.}
         \PY{c+cm}{ * \PYZbs{}param[in] N nombre de cartes par main.  Precondition : N \PYZlt{}= (NB\PYZus{}CARTES\PYZhy{}1)/2);}
         \PY{c+cm}{ * \PYZbs{}param[out] main\PYZus{}A main du joueur A.}
         \PY{c+cm}{ * \PYZbs{}param[out] main\PYZus{}B main du joueur B.}
         \PY{c+cm}{ * \PYZbs{}param[out] last la derniere carte jouée par un des joueurs.}
         \PY{c+cm}{ */}
         \PY{k+kt}{int} \PY{n+nf}{preparer\PYZus{}jeu\PYZus{}UNO}\PY{p}{(}\PY{n}{jeu} \PY{n}{le\PYZus{}jeu}\PY{p}{,} \PY{k+kt}{int} \PY{n}{N}\PY{p}{,} \PY{n}{t\PYZus{}main}\PY{o}{*} \PY{n}{main\PYZus{}A}\PY{p}{,} \PY{n}{t\PYZus{}main}\PY{o}{*} \PY{n}{main\PYZus{}B}\PY{p}{,} \PY{n}{carte}\PY{o}{*} \PY{n}{last}\PY{p}{)}\PY{p}{\PYZob{}}
             \PY{n}{assert}\PY{p}{(}\PY{n}{N} \PY{o}{\PYZlt{}}\PY{o}{=} \PY{p}{(}\PY{n}{NB\PYZus{}CARTES}\PY{l+m+mi}{\PYZhy{}1}\PY{p}{)}\PY{o}{/}\PY{l+m+mi}{2}\PY{p}{)}\PY{p}{;}
         
             \PY{c+c1}{//Initialiser le générateur de nombres aléatoires}
             \PY{k+kt}{time\PYZus{}t} \PY{n}{t}\PY{p}{;}
             \PY{n}{srand}\PY{p}{(}\PY{p}{(}\PY{k+kt}{unsigned}\PY{p}{)} \PY{n}{time}\PY{p}{(}\PY{o}{\PYZam{}}\PY{n}{t}\PY{p}{)}\PY{p}{)}\PY{p}{;}
          
             \PY{c+c1}{//Initialiser le jeu}
             \PY{n}{init\PYZus{}jeu}\PY{p}{(}\PY{n}{le\PYZus{}jeu}\PY{p}{)}\PY{p}{;}
             
             \PY{c+c1}{//Melanger le jeu}
             \PY{n}{melanger\PYZus{}jeu}\PY{p}{(}\PY{n}{le\PYZus{}jeu}\PY{p}{)}\PY{p}{;}
         
             \PY{c+c1}{//Distribuer N cartes à chaque joueur}
             \PY{n}{distribuer\PYZus{}mains}\PY{p}{(}\PY{n}{le\PYZus{}jeu}\PY{p}{,} \PY{n}{N}\PY{p}{,} \PY{n}{main\PYZus{}A}\PY{p}{,} \PY{n}{main\PYZus{}B}\PY{p}{)}\PY{p}{;}
         
             \PY{c+c1}{//Initialiser last avec la (2N+1)\PYZhy{}ème carte du jeu.}
             \PY{n}{copier\PYZus{}carte}\PY{p}{(}\PY{n}{last}\PY{p}{,} \PY{n}{le\PYZus{}jeu}\PY{p}{[}\PY{l+m+mi}{2}\PY{o}{*}\PY{n}{N}\PY{p}{]}\PY{p}{)}\PY{p}{;}
             \PY{n}{le\PYZus{}jeu}\PY{p}{[}\PY{l+m+mi}{2}\PY{o}{*}\PY{n}{N}\PY{p}{]}\PY{p}{.}\PY{n}{presente} \PY{o}{=} \PY{n+nb}{false}\PY{p}{;} \PY{c+c1}{//carte n\PYZsq{}est plus presente dans le\PYZus{}jeu}
         
             \PY{k}{return} \PY{n}{EXIT\PYZus{}SUCCESS}\PY{p}{;}
         \PY{p}{\PYZcb{}}
         
         \PY{k+kt}{void} \PY{n+nf}{test\PYZus{}piocher}\PY{p}{(}\PY{p}{)}\PY{p}{\PYZob{}}
             \PY{n}{jeu} \PY{n}{le\PYZus{}jeu}\PY{p}{;} \PY{c+c1}{// le jeu de cartes}
             \PY{n}{t\PYZus{}main} \PY{n}{main\PYZus{}A}\PY{p}{,} \PY{n}{main\PYZus{}B}\PY{p}{;} \PY{c+c1}{// les deux mains}
             \PY{n}{carte} \PY{n}{last}\PY{p}{;} \PY{c+c1}{// la derniere carte posee}
            
             \PY{c+c1}{//Préparer le jeu, les deux mains de 7 cartes et la carte last}
             \PY{k+kt}{int} \PY{n}{retour} \PY{o}{=} \PY{n}{preparer\PYZus{}jeu\PYZus{}UNO}\PY{p}{(}\PY{n}{le\PYZus{}jeu}\PY{p}{,} \PY{l+m+mi}{7}\PY{p}{,} \PY{o}{\PYZam{}}\PY{n}{main\PYZus{}A}\PY{p}{,} \PY{o}{\PYZam{}}\PY{n}{main\PYZus{}B}\PY{p}{,} \PY{o}{\PYZam{}}\PY{n}{last}\PY{p}{)}\PY{p}{;}
             \PY{n}{printf}\PY{p}{(}\PY{l+s}{\PYZdq{}}\PY{l+s+se}{\PYZbs{}n}\PY{l+s}{ Les deux mains : }\PY{l+s+se}{\PYZbs{}n}\PY{l+s}{\PYZdq{}}\PY{p}{)}\PY{p}{;}
             \PY{n}{afficher\PYZus{}main}\PY{p}{(}\PY{n}{main\PYZus{}A}\PY{p}{)}\PY{p}{;}
             \PY{n}{afficher\PYZus{}main}\PY{p}{(}\PY{n}{main\PYZus{}B}\PY{p}{)}\PY{p}{;}
         
             \PY{k+kt}{int} \PY{n}{mem\PYZus{}taille} \PY{o}{=} \PY{n}{main\PYZus{}A}\PY{p}{.}\PY{n}{nb}\PY{p}{;}
             
             \PY{c+c1}{//Le joueur A pioche une carte dans le\PYZus{}jeu}
             \PY{n}{carte} \PY{o}{*}\PY{n}{c\PYZus{}piochee} \PY{o}{=} \PY{n}{piocher}\PY{p}{(}\PY{n}{le\PYZus{}jeu}\PY{p}{,} \PY{o}{\PYZam{}}\PY{n}{main\PYZus{}A}\PY{p}{)}\PY{p}{;}
             
             \PY{c+c1}{// Une carte a\PYZhy{}t\PYZhy{}elle été piochée ?}
             \PY{n}{assert}\PY{p}{(}\PY{n}{c\PYZus{}piochee}\PY{p}{)}\PY{p}{;}
             \PY{n}{assert}\PY{p}{(}\PY{n}{c\PYZus{}piochee}\PY{o}{\PYZhy{}}\PY{o}{\PYZgt{}}\PY{n}{presente}\PY{o}{=}\PY{o}{=}\PY{n+nb}{false}\PY{p}{)}\PY{p}{;} \PY{c+c1}{// absence du jeu ?}
             \PY{n}{assert}\PY{p}{(}\PY{n}{est\PYZus{}presente\PYZus{}main}\PY{p}{(}\PY{n}{main\PYZus{}A}\PY{p}{,} \PY{o}{*}\PY{n}{c\PYZus{}piochee}\PY{p}{)}\PY{p}{)}\PY{p}{;}
             \PY{n}{assert}\PY{p}{(}\PY{n}{main\PYZus{}A}\PY{p}{.}\PY{n}{nb} \PY{o}{=} \PY{n}{mem\PYZus{}taille} \PY{o}{+} \PY{l+m+mi}{1}\PY{p}{)}\PY{p}{;}
         
             \PY{c+c1}{// Affichage}
             \PY{n}{printf}\PY{p}{(}\PY{l+s}{\PYZdq{}}\PY{l+s+se}{\PYZbs{}n}\PY{l+s+se}{\PYZbs{}n}\PY{l+s}{ ***** APRES la pioche : }\PY{l+s}{\PYZdq{}}\PY{p}{)}\PY{p}{;}
             \PY{n}{printf}\PY{p}{(}\PY{l+s}{\PYZdq{}}\PY{l+s+se}{\PYZbs{}n}\PY{l+s}{ La carte piochee : }\PY{l+s}{\PYZdq{}}\PY{p}{)}\PY{p}{;}
             \PY{n}{afficher\PYZus{}carte}\PY{p}{(}\PY{o}{*}\PY{n}{c\PYZus{}piochee}\PY{p}{)}\PY{p}{;}
             \PY{n}{printf}\PY{p}{(}\PY{l+s}{\PYZdq{}}\PY{l+s+se}{\PYZbs{}n}\PY{l+s}{ La nouvelle main A après pioche : }\PY{l+s+se}{\PYZbs{}n}\PY{l+s}{\PYZdq{}}\PY{p}{)}\PY{p}{;}
             \PY{n}{afficher\PYZus{}main}\PY{p}{(}\PY{n}{main\PYZus{}A}\PY{p}{)}\PY{p}{;}
             \PY{n}{printf}\PY{p}{(}\PY{l+s}{\PYZdq{}}\PY{l+s+se}{\PYZbs{}n}\PY{l+s}{ Le nouveau jeu après pioche : }\PY{l+s+se}{\PYZbs{}n}\PY{l+s}{\PYZdq{}}\PY{p}{)}\PY{p}{;}
             \PY{n}{afficher\PYZus{}jeu}\PY{p}{(}\PY{n}{le\PYZus{}jeu}\PY{p}{)}\PY{p}{;}
         
             \PY{c+c1}{//Détruire la mémoire allouée dynamiquement}
             \PY{n}{free}\PY{p}{(}\PY{n}{main\PYZus{}A}\PY{p}{.}\PY{n}{main}\PY{p}{)}\PY{p}{;}
             \PY{n}{free}\PY{p}{(}\PY{n}{main\PYZus{}B}\PY{p}{.}\PY{n}{main}\PY{p}{)}\PY{p}{;}
             \PY{n}{main\PYZus{}A}\PY{p}{.}\PY{n}{main} \PY{o}{=} \PY{n+nb}{NULL}\PY{p}{;}
             \PY{n}{main\PYZus{}B}\PY{p}{.}\PY{n}{main} \PY{o}{=} \PY{n+nb}{NULL}\PY{p}{;}
         \PY{p}{\PYZcb{}}
         
         \PY{k+kt}{int} \PY{n+nf}{main}\PY{p}{(}\PY{k+kt}{void}\PY{p}{)} \PY{p}{\PYZob{}}
           
             \PY{n}{test\PYZus{}piocher}\PY{p}{(}\PY{p}{)}\PY{p}{;}
             
             \PY{n}{printf}\PY{p}{(}\PY{l+s}{\PYZdq{}}\PY{l+s}{\PYZpc{}s}\PY{l+s}{\PYZdq{}}\PY{p}{,} \PY{l+s}{\PYZdq{}}\PY{l+s+se}{\PYZbs{}n}\PY{l+s}{ Bravo ! Tous les tests passent.}\PY{l+s+se}{\PYZbs{}n}\PY{l+s}{\PYZdq{}}\PY{p}{)}\PY{p}{;}
             \PY{k}{return} \PY{n}{EXIT\PYZus{}SUCCESS}\PY{p}{;}
         \PY{p}{\PYZcb{}}
\end{Verbatim}


    \begin{Verbatim}[commandchars=\\\{\}]
/tmp/tmp5\_daek6p.c: In function ‘piocher’:
/tmp/tmp5\_daek6p.c:233:24: error: incompatible type for argument 1 of ‘copier\_carte’
           copier\_carte(main->main[main->nb],le\_jeu[i]);
                        \^{}\textasciitilde{}\textasciitilde{}\textasciitilde{}
/tmp/tmp5\_daek6p.c:67:6: note: expected ‘carte * \{aka struct carte *\}’ but argument is of type ‘carte \{aka struct carte\}’
 void copier\_carte(carte* dest, carte src)\{
      \^{}\textasciitilde{}\textasciitilde{}\textasciitilde{}\textasciitilde{}\textasciitilde{}\textasciitilde{}\textasciitilde{}\textasciitilde{}\textasciitilde{}\textasciitilde{}\textasciitilde{}
[C kernel] GCC exited with code 1, the executable will not be executed
    \end{Verbatim}

    \subsection{\#\# BILAN sur l'allocation dynamique. (à
rendre)}\label{bilan-sur-lallocation-dynamique.-uxe0-rendre}

\begin{center}\rule{0.5\linewidth}{\linethickness}\end{center}

    \subsubsection{String : Chaines de caractères à la Java /
C++}\label{string-chaines-de-caractuxe8res-uxe0-la-java-c}

L'objectif de ces exercices est de définir un vrai type « chaîne de
caractères » appelé String par analogie avec le type correspondant des
langages Java ou C++. Outre les opérations de haut niveau disponibles
sur ce type, sa caractéristique essentielle est de décharger
l'utilisateur de la gestion des problèmes de capacité de la chaîne. Si
elle est trop petite pour faire une opération d'ajout, elle est agrandie
de manière transparente pour l'utilisateur.

Les opérations disponibles sur une String sont : - \textbf{create} :
initialiser une variable de type String à partir d'une chaîne de
caractères classique (tableau de caractères terminé par le caractère
nul) ; - \textbf{destroy} : détruire un variable de type String. Elle ne
pourra plus être utilisée (sauf à être de nouveau initialisée) ; -
\textbf{length} : obtenir le nombre de caractères de la chaîne ; -
\textbf{get} : obtenir le caractère à la position \texttt{i} de la
chaîne. Le premier caractère a la position 0 ; - \textbf{replace} :
remplacer le caractère à la position \texttt{i} de la chaîne par un
nouveau caractère ; - \textbf{add} : ajouter un nouveau caractère à la
fin de la chaîne. Sa longueur est donc augmentée de 1 ; -
\textbf{append} : ajouter une chaîne de caractères à la fin de d'une
chaîne ; - \textbf{insert} : ajouter un nouveau caractère en position
\texttt{i} de la chaîne \texttt{str}. La longueur de la chaîne est donc
augmentée de 1. La valeur de \texttt{i} doit être comprise entre 0 et
\texttt{length(str)}. Si \texttt{i} vaut \texttt{length(str)}, alors
\texttt{insert} se comporte comme \texttt{add} ; - \textbf{delete} :
supprimer le caractère à la position \texttt{i}. - \textbf{substring} :
retourne une nouvelle String initialisée avec la partie de la chaîne
comprise entre les indices début et fin, début inclu et fin exclu.

Une chaine de caractères sera définie dans ce travail comme un
enregistrement d'un tableau de caractères dynamique et d'une taille.

\textbf{Question}

\begin{quote}
Dans cet exercice bilan, il faut compléter l'implantation des
sous-programmes \texttt{create}, \texttt{add}, \texttt{delete},
\texttt{length} et \texttt{destroy} de façon à ce que les tests
s'exécutent avec succès.
\end{quote}

\textbf{RENDU} Le rendu de cet exercice Bilan est attendu dans le
fichier \textbf{\texttt{1SN\_LangageC\_C2\_Bilan.c}} via SVN.

    \begin{Verbatim}[commandchars=\\\{\}]
{\color{incolor}In [{\color{incolor}89}]:} \PY{o}{\PYZhy{}}
\end{Verbatim}


    \begin{Verbatim}[commandchars=\\\{\}]
[C kernel] Executable exited with code -11
    \end{Verbatim}


    % Add a bibliography block to the postdoc
    
    
    
    \end{document}
