
% Default to the notebook output style

    


% Inherit from the specified cell style.




    
\documentclass[11pt]{article}

    
    
    \usepackage[T1]{fontenc}
    % Nicer default font (+ math font) than Computer Modern for most use cases
    \usepackage{mathpazo}

    % Basic figure setup, for now with no caption control since it's done
    % automatically by Pandoc (which extracts ![](path) syntax from Markdown).
    \usepackage{graphicx}
    % We will generate all images so they have a width \maxwidth. This means
    % that they will get their normal width if they fit onto the page, but
    % are scaled down if they would overflow the margins.
    \makeatletter
    \def\maxwidth{\ifdim\Gin@nat@width>\linewidth\linewidth
    \else\Gin@nat@width\fi}
    \makeatother
    \let\Oldincludegraphics\includegraphics
    % Set max figure width to be 80% of text width, for now hardcoded.
    \renewcommand{\includegraphics}[1]{\Oldincludegraphics[width=.8\maxwidth]{#1}}
    % Ensure that by default, figures have no caption (until we provide a
    % proper Figure object with a Caption API and a way to capture that
    % in the conversion process - todo).
    \usepackage{caption}
    \DeclareCaptionLabelFormat{nolabel}{}
    \captionsetup{labelformat=nolabel}

    \usepackage{adjustbox} % Used to constrain images to a maximum size 
    \usepackage{xcolor} % Allow colors to be defined
    \usepackage{enumerate} % Needed for markdown enumerations to work
    \usepackage{geometry} % Used to adjust the document margins
    \usepackage{amsmath} % Equations
    \usepackage{amssymb} % Equations
    \usepackage{textcomp} % defines textquotesingle
    % Hack from http://tex.stackexchange.com/a/47451/13684:
    \AtBeginDocument{%
        \def\PYZsq{\textquotesingle}% Upright quotes in Pygmentized code
    }
    \usepackage{upquote} % Upright quotes for verbatim code
    \usepackage{eurosym} % defines \euro
    \usepackage[mathletters]{ucs} % Extended unicode (utf-8) support
    \usepackage[utf8x]{inputenc} % Allow utf-8 characters in the tex document
    \usepackage{fancyvrb} % verbatim replacement that allows latex
    \usepackage{grffile} % extends the file name processing of package graphics 
                         % to support a larger range 
    % The hyperref package gives us a pdf with properly built
    % internal navigation ('pdf bookmarks' for the table of contents,
    % internal cross-reference links, web links for URLs, etc.)
    \usepackage{hyperref}
    \usepackage{longtable} % longtable support required by pandoc >1.10
    \usepackage{booktabs}  % table support for pandoc > 1.12.2
    \usepackage[inline]{enumitem} % IRkernel/repr support (it uses the enumerate* environment)
    \usepackage[normalem]{ulem} % ulem is needed to support strikethroughs (\sout)
                                % normalem makes italics be italics, not underlines
    

    
    
    % Colors for the hyperref package
    \definecolor{urlcolor}{rgb}{0,.145,.698}
    \definecolor{linkcolor}{rgb}{.71,0.21,0.01}
    \definecolor{citecolor}{rgb}{.12,.54,.11}

    % ANSI colors
    \definecolor{ansi-black}{HTML}{3E424D}
    \definecolor{ansi-black-intense}{HTML}{282C36}
    \definecolor{ansi-red}{HTML}{E75C58}
    \definecolor{ansi-red-intense}{HTML}{B22B31}
    \definecolor{ansi-green}{HTML}{00A250}
    \definecolor{ansi-green-intense}{HTML}{007427}
    \definecolor{ansi-yellow}{HTML}{DDB62B}
    \definecolor{ansi-yellow-intense}{HTML}{B27D12}
    \definecolor{ansi-blue}{HTML}{208FFB}
    \definecolor{ansi-blue-intense}{HTML}{0065CA}
    \definecolor{ansi-magenta}{HTML}{D160C4}
    \definecolor{ansi-magenta-intense}{HTML}{A03196}
    \definecolor{ansi-cyan}{HTML}{60C6C8}
    \definecolor{ansi-cyan-intense}{HTML}{258F8F}
    \definecolor{ansi-white}{HTML}{C5C1B4}
    \definecolor{ansi-white-intense}{HTML}{A1A6B2}

    % commands and environments needed by pandoc snippets
    % extracted from the output of `pandoc -s`
    \providecommand{\tightlist}{%
      \setlength{\itemsep}{0pt}\setlength{\parskip}{0pt}}
    \DefineVerbatimEnvironment{Highlighting}{Verbatim}{commandchars=\\\{\}}
    % Add ',fontsize=\small' for more characters per line
    \newenvironment{Shaded}{}{}
    \newcommand{\KeywordTok}[1]{\textcolor[rgb]{0.00,0.44,0.13}{\textbf{{#1}}}}
    \newcommand{\DataTypeTok}[1]{\textcolor[rgb]{0.56,0.13,0.00}{{#1}}}
    \newcommand{\DecValTok}[1]{\textcolor[rgb]{0.25,0.63,0.44}{{#1}}}
    \newcommand{\BaseNTok}[1]{\textcolor[rgb]{0.25,0.63,0.44}{{#1}}}
    \newcommand{\FloatTok}[1]{\textcolor[rgb]{0.25,0.63,0.44}{{#1}}}
    \newcommand{\CharTok}[1]{\textcolor[rgb]{0.25,0.44,0.63}{{#1}}}
    \newcommand{\StringTok}[1]{\textcolor[rgb]{0.25,0.44,0.63}{{#1}}}
    \newcommand{\CommentTok}[1]{\textcolor[rgb]{0.38,0.63,0.69}{\textit{{#1}}}}
    \newcommand{\OtherTok}[1]{\textcolor[rgb]{0.00,0.44,0.13}{{#1}}}
    \newcommand{\AlertTok}[1]{\textcolor[rgb]{1.00,0.00,0.00}{\textbf{{#1}}}}
    \newcommand{\FunctionTok}[1]{\textcolor[rgb]{0.02,0.16,0.49}{{#1}}}
    \newcommand{\RegionMarkerTok}[1]{{#1}}
    \newcommand{\ErrorTok}[1]{\textcolor[rgb]{1.00,0.00,0.00}{\textbf{{#1}}}}
    \newcommand{\NormalTok}[1]{{#1}}
    
    % Additional commands for more recent versions of Pandoc
    \newcommand{\ConstantTok}[1]{\textcolor[rgb]{0.53,0.00,0.00}{{#1}}}
    \newcommand{\SpecialCharTok}[1]{\textcolor[rgb]{0.25,0.44,0.63}{{#1}}}
    \newcommand{\VerbatimStringTok}[1]{\textcolor[rgb]{0.25,0.44,0.63}{{#1}}}
    \newcommand{\SpecialStringTok}[1]{\textcolor[rgb]{0.73,0.40,0.53}{{#1}}}
    \newcommand{\ImportTok}[1]{{#1}}
    \newcommand{\DocumentationTok}[1]{\textcolor[rgb]{0.73,0.13,0.13}{\textit{{#1}}}}
    \newcommand{\AnnotationTok}[1]{\textcolor[rgb]{0.38,0.63,0.69}{\textbf{\textit{{#1}}}}}
    \newcommand{\CommentVarTok}[1]{\textcolor[rgb]{0.38,0.63,0.69}{\textbf{\textit{{#1}}}}}
    \newcommand{\VariableTok}[1]{\textcolor[rgb]{0.10,0.09,0.49}{{#1}}}
    \newcommand{\ControlFlowTok}[1]{\textcolor[rgb]{0.00,0.44,0.13}{\textbf{{#1}}}}
    \newcommand{\OperatorTok}[1]{\textcolor[rgb]{0.40,0.40,0.40}{{#1}}}
    \newcommand{\BuiltInTok}[1]{{#1}}
    \newcommand{\ExtensionTok}[1]{{#1}}
    \newcommand{\PreprocessorTok}[1]{\textcolor[rgb]{0.74,0.48,0.00}{{#1}}}
    \newcommand{\AttributeTok}[1]{\textcolor[rgb]{0.49,0.56,0.16}{{#1}}}
    \newcommand{\InformationTok}[1]{\textcolor[rgb]{0.38,0.63,0.69}{\textbf{\textit{{#1}}}}}
    \newcommand{\WarningTok}[1]{\textcolor[rgb]{0.38,0.63,0.69}{\textbf{\textit{{#1}}}}}
    
    
    % Define a nice break command that doesn't care if a line doesn't already
    % exist.
    \def\br{\hspace*{\fill} \\* }
    % Math Jax compatability definitions
    \def\gt{>}
    \def\lt{<}
    % Document parameters
    \title{1SN\_LangageC\_C3}
    
    
    

    % Pygments definitions
    
\makeatletter
\def\PY@reset{\let\PY@it=\relax \let\PY@bf=\relax%
    \let\PY@ul=\relax \let\PY@tc=\relax%
    \let\PY@bc=\relax \let\PY@ff=\relax}
\def\PY@tok#1{\csname PY@tok@#1\endcsname}
\def\PY@toks#1+{\ifx\relax#1\empty\else%
    \PY@tok{#1}\expandafter\PY@toks\fi}
\def\PY@do#1{\PY@bc{\PY@tc{\PY@ul{%
    \PY@it{\PY@bf{\PY@ff{#1}}}}}}}
\def\PY#1#2{\PY@reset\PY@toks#1+\relax+\PY@do{#2}}

\expandafter\def\csname PY@tok@w\endcsname{\def\PY@tc##1{\textcolor[rgb]{0.73,0.73,0.73}{##1}}}
\expandafter\def\csname PY@tok@c\endcsname{\let\PY@it=\textit\def\PY@tc##1{\textcolor[rgb]{0.25,0.50,0.50}{##1}}}
\expandafter\def\csname PY@tok@cp\endcsname{\def\PY@tc##1{\textcolor[rgb]{0.74,0.48,0.00}{##1}}}
\expandafter\def\csname PY@tok@k\endcsname{\let\PY@bf=\textbf\def\PY@tc##1{\textcolor[rgb]{0.00,0.50,0.00}{##1}}}
\expandafter\def\csname PY@tok@kp\endcsname{\def\PY@tc##1{\textcolor[rgb]{0.00,0.50,0.00}{##1}}}
\expandafter\def\csname PY@tok@kt\endcsname{\def\PY@tc##1{\textcolor[rgb]{0.69,0.00,0.25}{##1}}}
\expandafter\def\csname PY@tok@o\endcsname{\def\PY@tc##1{\textcolor[rgb]{0.40,0.40,0.40}{##1}}}
\expandafter\def\csname PY@tok@ow\endcsname{\let\PY@bf=\textbf\def\PY@tc##1{\textcolor[rgb]{0.67,0.13,1.00}{##1}}}
\expandafter\def\csname PY@tok@nb\endcsname{\def\PY@tc##1{\textcolor[rgb]{0.00,0.50,0.00}{##1}}}
\expandafter\def\csname PY@tok@nf\endcsname{\def\PY@tc##1{\textcolor[rgb]{0.00,0.00,1.00}{##1}}}
\expandafter\def\csname PY@tok@nc\endcsname{\let\PY@bf=\textbf\def\PY@tc##1{\textcolor[rgb]{0.00,0.00,1.00}{##1}}}
\expandafter\def\csname PY@tok@nn\endcsname{\let\PY@bf=\textbf\def\PY@tc##1{\textcolor[rgb]{0.00,0.00,1.00}{##1}}}
\expandafter\def\csname PY@tok@ne\endcsname{\let\PY@bf=\textbf\def\PY@tc##1{\textcolor[rgb]{0.82,0.25,0.23}{##1}}}
\expandafter\def\csname PY@tok@nv\endcsname{\def\PY@tc##1{\textcolor[rgb]{0.10,0.09,0.49}{##1}}}
\expandafter\def\csname PY@tok@no\endcsname{\def\PY@tc##1{\textcolor[rgb]{0.53,0.00,0.00}{##1}}}
\expandafter\def\csname PY@tok@nl\endcsname{\def\PY@tc##1{\textcolor[rgb]{0.63,0.63,0.00}{##1}}}
\expandafter\def\csname PY@tok@ni\endcsname{\let\PY@bf=\textbf\def\PY@tc##1{\textcolor[rgb]{0.60,0.60,0.60}{##1}}}
\expandafter\def\csname PY@tok@na\endcsname{\def\PY@tc##1{\textcolor[rgb]{0.49,0.56,0.16}{##1}}}
\expandafter\def\csname PY@tok@nt\endcsname{\let\PY@bf=\textbf\def\PY@tc##1{\textcolor[rgb]{0.00,0.50,0.00}{##1}}}
\expandafter\def\csname PY@tok@nd\endcsname{\def\PY@tc##1{\textcolor[rgb]{0.67,0.13,1.00}{##1}}}
\expandafter\def\csname PY@tok@s\endcsname{\def\PY@tc##1{\textcolor[rgb]{0.73,0.13,0.13}{##1}}}
\expandafter\def\csname PY@tok@sd\endcsname{\let\PY@it=\textit\def\PY@tc##1{\textcolor[rgb]{0.73,0.13,0.13}{##1}}}
\expandafter\def\csname PY@tok@si\endcsname{\let\PY@bf=\textbf\def\PY@tc##1{\textcolor[rgb]{0.73,0.40,0.53}{##1}}}
\expandafter\def\csname PY@tok@se\endcsname{\let\PY@bf=\textbf\def\PY@tc##1{\textcolor[rgb]{0.73,0.40,0.13}{##1}}}
\expandafter\def\csname PY@tok@sr\endcsname{\def\PY@tc##1{\textcolor[rgb]{0.73,0.40,0.53}{##1}}}
\expandafter\def\csname PY@tok@ss\endcsname{\def\PY@tc##1{\textcolor[rgb]{0.10,0.09,0.49}{##1}}}
\expandafter\def\csname PY@tok@sx\endcsname{\def\PY@tc##1{\textcolor[rgb]{0.00,0.50,0.00}{##1}}}
\expandafter\def\csname PY@tok@m\endcsname{\def\PY@tc##1{\textcolor[rgb]{0.40,0.40,0.40}{##1}}}
\expandafter\def\csname PY@tok@gh\endcsname{\let\PY@bf=\textbf\def\PY@tc##1{\textcolor[rgb]{0.00,0.00,0.50}{##1}}}
\expandafter\def\csname PY@tok@gu\endcsname{\let\PY@bf=\textbf\def\PY@tc##1{\textcolor[rgb]{0.50,0.00,0.50}{##1}}}
\expandafter\def\csname PY@tok@gd\endcsname{\def\PY@tc##1{\textcolor[rgb]{0.63,0.00,0.00}{##1}}}
\expandafter\def\csname PY@tok@gi\endcsname{\def\PY@tc##1{\textcolor[rgb]{0.00,0.63,0.00}{##1}}}
\expandafter\def\csname PY@tok@gr\endcsname{\def\PY@tc##1{\textcolor[rgb]{1.00,0.00,0.00}{##1}}}
\expandafter\def\csname PY@tok@ge\endcsname{\let\PY@it=\textit}
\expandafter\def\csname PY@tok@gs\endcsname{\let\PY@bf=\textbf}
\expandafter\def\csname PY@tok@gp\endcsname{\let\PY@bf=\textbf\def\PY@tc##1{\textcolor[rgb]{0.00,0.00,0.50}{##1}}}
\expandafter\def\csname PY@tok@go\endcsname{\def\PY@tc##1{\textcolor[rgb]{0.53,0.53,0.53}{##1}}}
\expandafter\def\csname PY@tok@gt\endcsname{\def\PY@tc##1{\textcolor[rgb]{0.00,0.27,0.87}{##1}}}
\expandafter\def\csname PY@tok@err\endcsname{\def\PY@bc##1{\setlength{\fboxsep}{0pt}\fcolorbox[rgb]{1.00,0.00,0.00}{1,1,1}{\strut ##1}}}
\expandafter\def\csname PY@tok@kc\endcsname{\let\PY@bf=\textbf\def\PY@tc##1{\textcolor[rgb]{0.00,0.50,0.00}{##1}}}
\expandafter\def\csname PY@tok@kd\endcsname{\let\PY@bf=\textbf\def\PY@tc##1{\textcolor[rgb]{0.00,0.50,0.00}{##1}}}
\expandafter\def\csname PY@tok@kn\endcsname{\let\PY@bf=\textbf\def\PY@tc##1{\textcolor[rgb]{0.00,0.50,0.00}{##1}}}
\expandafter\def\csname PY@tok@kr\endcsname{\let\PY@bf=\textbf\def\PY@tc##1{\textcolor[rgb]{0.00,0.50,0.00}{##1}}}
\expandafter\def\csname PY@tok@bp\endcsname{\def\PY@tc##1{\textcolor[rgb]{0.00,0.50,0.00}{##1}}}
\expandafter\def\csname PY@tok@fm\endcsname{\def\PY@tc##1{\textcolor[rgb]{0.00,0.00,1.00}{##1}}}
\expandafter\def\csname PY@tok@vc\endcsname{\def\PY@tc##1{\textcolor[rgb]{0.10,0.09,0.49}{##1}}}
\expandafter\def\csname PY@tok@vg\endcsname{\def\PY@tc##1{\textcolor[rgb]{0.10,0.09,0.49}{##1}}}
\expandafter\def\csname PY@tok@vi\endcsname{\def\PY@tc##1{\textcolor[rgb]{0.10,0.09,0.49}{##1}}}
\expandafter\def\csname PY@tok@vm\endcsname{\def\PY@tc##1{\textcolor[rgb]{0.10,0.09,0.49}{##1}}}
\expandafter\def\csname PY@tok@sa\endcsname{\def\PY@tc##1{\textcolor[rgb]{0.73,0.13,0.13}{##1}}}
\expandafter\def\csname PY@tok@sb\endcsname{\def\PY@tc##1{\textcolor[rgb]{0.73,0.13,0.13}{##1}}}
\expandafter\def\csname PY@tok@sc\endcsname{\def\PY@tc##1{\textcolor[rgb]{0.73,0.13,0.13}{##1}}}
\expandafter\def\csname PY@tok@dl\endcsname{\def\PY@tc##1{\textcolor[rgb]{0.73,0.13,0.13}{##1}}}
\expandafter\def\csname PY@tok@s2\endcsname{\def\PY@tc##1{\textcolor[rgb]{0.73,0.13,0.13}{##1}}}
\expandafter\def\csname PY@tok@sh\endcsname{\def\PY@tc##1{\textcolor[rgb]{0.73,0.13,0.13}{##1}}}
\expandafter\def\csname PY@tok@s1\endcsname{\def\PY@tc##1{\textcolor[rgb]{0.73,0.13,0.13}{##1}}}
\expandafter\def\csname PY@tok@mb\endcsname{\def\PY@tc##1{\textcolor[rgb]{0.40,0.40,0.40}{##1}}}
\expandafter\def\csname PY@tok@mf\endcsname{\def\PY@tc##1{\textcolor[rgb]{0.40,0.40,0.40}{##1}}}
\expandafter\def\csname PY@tok@mh\endcsname{\def\PY@tc##1{\textcolor[rgb]{0.40,0.40,0.40}{##1}}}
\expandafter\def\csname PY@tok@mi\endcsname{\def\PY@tc##1{\textcolor[rgb]{0.40,0.40,0.40}{##1}}}
\expandafter\def\csname PY@tok@il\endcsname{\def\PY@tc##1{\textcolor[rgb]{0.40,0.40,0.40}{##1}}}
\expandafter\def\csname PY@tok@mo\endcsname{\def\PY@tc##1{\textcolor[rgb]{0.40,0.40,0.40}{##1}}}
\expandafter\def\csname PY@tok@ch\endcsname{\let\PY@it=\textit\def\PY@tc##1{\textcolor[rgb]{0.25,0.50,0.50}{##1}}}
\expandafter\def\csname PY@tok@cm\endcsname{\let\PY@it=\textit\def\PY@tc##1{\textcolor[rgb]{0.25,0.50,0.50}{##1}}}
\expandafter\def\csname PY@tok@cpf\endcsname{\let\PY@it=\textit\def\PY@tc##1{\textcolor[rgb]{0.25,0.50,0.50}{##1}}}
\expandafter\def\csname PY@tok@c1\endcsname{\let\PY@it=\textit\def\PY@tc##1{\textcolor[rgb]{0.25,0.50,0.50}{##1}}}
\expandafter\def\csname PY@tok@cs\endcsname{\let\PY@it=\textit\def\PY@tc##1{\textcolor[rgb]{0.25,0.50,0.50}{##1}}}

\def\PYZbs{\char`\\}
\def\PYZus{\char`\_}
\def\PYZob{\char`\{}
\def\PYZcb{\char`\}}
\def\PYZca{\char`\^}
\def\PYZam{\char`\&}
\def\PYZlt{\char`\<}
\def\PYZgt{\char`\>}
\def\PYZsh{\char`\#}
\def\PYZpc{\char`\%}
\def\PYZdl{\char`\$}
\def\PYZhy{\char`\-}
\def\PYZsq{\char`\'}
\def\PYZdq{\char`\"}
\def\PYZti{\char`\~}
% for compatibility with earlier versions
\def\PYZat{@}
\def\PYZlb{[}
\def\PYZrb{]}
\makeatother


    % Exact colors from NB
    \definecolor{incolor}{rgb}{0.0, 0.0, 0.5}
    \definecolor{outcolor}{rgb}{0.545, 0.0, 0.0}



    
    % Prevent overflowing lines due to hard-to-break entities
    \sloppy 
    % Setup hyperref package
    \hypersetup{
      breaklinks=true,  % so long urls are correctly broken across lines
      colorlinks=true,
      urlcolor=urlcolor,
      linkcolor=linkcolor,
      citecolor=citecolor,
      }
    % Slightly bigger margins than the latex defaults
    
    \geometry{verbose,tmargin=1in,bmargin=1in,lmargin=1in,rmargin=1in}
    
    

    \begin{document}
    
    
    \maketitle
    
    

    
    \begin{quote}
\section{Langage C - Notebook C3}\label{langage-c---notebook-c3}

\subsection{\texorpdfstring{Modules en C et compilation automatique avec
\texttt{make}.}{Modules en C et compilation automatique avec make.}}\label{modules-en-c-et-compilation-automatique-avec-make.}

\mbox{}%
\paragraph{Katia Jaffrès-Runser, Xavier
Crégut}\label{katia-jaffruxe8s-runser-xavier-cruxe9gut}

Toulouse INP - ENSEEIHT, 1ère année, Dept. Sciences du Numérique,
2020-2021.
\end{quote}

    \#\# ++++ ATTENTION +++++ \textgreater{} - Les exercices présents dans
ce notebook C3 \textbf{ne seront pas réalisés via le Notebook}. Vous
travaillerez \textbf{uniquement} sur les fichiers C présents dans le
répertoire SVN \texttt{c3}. \textgreater{} - L'exercice bilan, donné à
la fin de ce notebook, est à rendre (cf. échéance sur Moodle).
\textgreater{} - Un exercice supplémentaire, proposé dans le notebook
'1SN\_LangageC\_Extra' est optionnel. Il vous permet d'aller plus loin
si vous le désirez, notamment sur la notion de généricité en C.

    \subsection{\#\# 1. Déroulement du cours
(rappel)}\label{duxe9roulement-du-cours-rappel}

Ce cours se déroule sur 6 séances de TP.

\begin{itemize}
\tightlist
\item
  Lors des trois premières séances, vous avez suivi le sujet C1 sous la
  forme d'un notebook Jupyter.
\item
  Lors des trois dernières séances, vous suivez deux autres notebook
  Jupyter, C2 et C3, à votre rythme.
\end{itemize}

Chaque sujet, C1, C2 et C3, se termine par un exercice Bilan à rendre
via votre dépot SVN. Les échéances sont indiquées sur Moodle. Les 3
exercices bilans sont notés, et leur moyenne fournit une note
d'exercices.

Vous aurez, en fin de cours, un QCM d'une heure. La note finale est une
moyenne des deux notes (QCM et exercices rendus).

    \subsection{2. Objectifs (rappel)}\label{objectifs-rappel}

Ce cours, sous la forme de notebooks Jupyter et d'un ensemble
d'exercices à réaliser en TP, a pour objectif de vous présenter les
spécificités de la programmation en langage C. Il se base sur vos acquis
du cours de Programmation Impérative en algorithmique et vous détaille
les éléments du langage C nécessaires à la production d'un programme en
C.

Un support de cours PDF vous est également fournit sur Moodle :
\href{http://moodle-n7.inp-toulouse.fr/pluginfile.php/49240/mod_resource/content/5/LangageC_poly.pdf}{Cours
C}.

    \subsection{\#\# 3. Plan du sujet C3.}\label{plan-du-sujet-c3.}

Les éléments suivants de la programmation en Langage C sont présentés
dans ce sujet : - Les modules en C - L'automatisation de la compilation
avec l'outil \texttt{make}

Quelques éléments au sujet de la généricité sont présentés dans le
notebook \texttt{Extra}, optionnel.

    \subsection{4. Jupyter notebook (rappel)}\label{jupyter-notebook-rappel}

Le support de cours que vous lisez est un notebook Jupyter. Pour
visualiser le notebook, lancer l'editeur web avec la commande\\
\textgreater{} \texttt{jupyter-notebook}

et rechercher le fichier dans l'arborescence. Le fichier est édité dans
votre navigateur Web par défaut. L'enregistrement est automatique
(\texttt{CTRL\ S} pour le forcer).

Pour fermer votre fichier, il faut fermer le navigateur et terminer le
processus serveur qui s'exécute dans le terminal (\texttt{CTRL\ C}, puis
\texttt{y}).

\begin{quote}
\textbf{Important} : - Pour faire fonctionner le kernel C de jupyter
notebook, il faut, avant une \textbf{première utilisation} de Notebook,
lancer la commande suivante dans un \texttt{Terminal} :
\end{quote}

\begin{quote}
\texttt{install\_c\_kernel\ -\/-user}
\end{quote}

Ce notebook se compose de cellules présentant soit : - Des éléments de
cours, au format
\href{https://fr.wikipedia.org/wiki/Markdown}{Markdown}. Ce langage est
traduit en HTML pour un affichage aisé quand on clique sur la flèche
\texttt{Exécuter\ (run)} et que la cellule est active. - Du code en
Langage C (ou Python, ou autre..). Pour compiler et exécuter le code
écrit dans la cellule active, on clique sur la flèche
\texttt{Exécuter\ (run)}. Si la compilation se déroule sans erreur ni
avertissement, le programme est exécuté et les sorties sont affichées en
bas de la cellule. Si ce n'est pas le cas, les avertissements et
warnings sont affichés en bas de la cellule.

En double-cliquant sur une cellule, on peut éditer son contenu. Vous
pouvez ainsi : - Editer une cellule markdown pour y intégrer vos propres
notes. - Modifier les programmes pour répondre aux questions et
exercices proposés.

Il est possible d'exporter votre travail en PDF, HTML, etc. Il est aussi
possible d'afficher les numéros de ligne dans le menu
\textbf{Affichage}.

Le programme dans la cellule suivante s'exécute sans erreur. Vous pouvez
- le tester en l'exécutant. - y introduire une erreur (suppression d'un
point-virgule par exemple) pour observer la sortie du compilateur.

    \begin{Verbatim}[commandchars=\\\{\}]
{\color{incolor}In [{\color{incolor} }]:} \PY{c+cp}{\PYZsh{}}\PY{c+cp}{include} \PY{c+cpf}{\PYZlt{}stdlib.h\PYZgt{}}\PY{c+c1}{ }
        \PY{c+cp}{\PYZsh{}}\PY{c+cp}{include} \PY{c+cpf}{\PYZlt{}stdio.h\PYZgt{}}
        \PY{k+kt}{int} \PY{n+nf}{main}\PY{p}{(}\PY{p}{)}\PY{p}{\PYZob{}}
            \PY{n}{printf}\PY{p}{(}\PY{l+s}{\PYZdq{}}\PY{l+s}{******************************}\PY{l+s+se}{\PYZbs{}n}\PY{l+s}{\PYZdq{}}\PY{p}{)}\PY{p}{;}
            \PY{n}{printf}\PY{p}{(}\PY{l+s}{\PYZdq{}}\PY{l+s}{******** Langage C ***********}\PY{l+s+se}{\PYZbs{}n}\PY{l+s}{\PYZdq{}}\PY{p}{)}\PY{p}{;}
            \PY{n}{printf}\PY{p}{(}\PY{l+s}{\PYZdq{}}\PY{l+s}{******************************}\PY{l+s+se}{\PYZbs{}n}\PY{l+s}{\PYZdq{}}\PY{p}{)}\PY{p}{;}
            \PY{k}{return} \PY{n}{EXIT\PYZus{}SUCCESS}\PY{p}{;}
        \PY{p}{\PYZcb{}}
\end{Verbatim}


    \begin{center}\rule{0.5\linewidth}{\linethickness}\end{center}

    \subsection{5. Les modules}\label{les-modules}

\begin{quote}
\textbf{Note :} L'ensemble des fichiers liés à cette partie se trouvent
dans le répertoire : \textbf{\texttt{c3/fichiers\_C/modules}}.
\end{quote}

    \subsubsection{5.1 Rappels : les modules en
algorithmique}\label{rappels-les-modules-en-algorithmique}

\paragraph{Définition d'un module}\label{duxe9finition-dun-module}

C'est une partie d'un programme définissant une \textbf{unité
structurelle et fonctionnelle}.\\
Un module regroupe : - Un ensemble de déclarations, de constantes, de
types, d'attributs et de sous-programmes\\
- L'ensemble des implantations (corps) de ces sous-programmes
satisfaisant au principe de séparation.

\paragraph{Structure d'un module}\label{structure-dun-module}

Un module se compose : - d'une \textbf{interface (ou spécification)} qui
permet de déclarer les constantes, types, attributs au module et de
spécifier les sous-programmes.\\
- d'un \textbf{corps (ou définition)} où on regroupe l'implantation des
différents sous-programmes spécifiés dans l'interface. Et éventuellement
d'autres constantes, types, attributs et sous-programmes internes au
module.

    \subsubsection{5.2 Les modules en C}\label{les-modules-en-c}

\begin{quote}
\begin{quote}
=\textgreater{} n'existent pas à proprement parler !!!\\
En effet, le langage C n'offre pas de \emph{support syntaxique} à la
définition des modules.
\end{quote}
\end{quote}

Le principe est le suivant. Un fichier C est : - Soit un programme
principal avec un (et un seul) programme principal \texttt{int\ main()};
- Soit un \texttt{module}

Un module en C, c'est : - Une \textbf{convention de nommage} qui
décompose un module en deux fichiers, l'en-tête \texttt{.h} et le corps
\texttt{.c}.\\
Par exemple, le module \texttt{complexe} se compose : \textgreater{} -
d'un fichier d'interface typiquement nommé \texttt{complexe.h}
\textgreater{} - d'un fichier corps typiquement nommé
\texttt{complexe.c} - Un outillage pour compiler cette structure de
fichiers : \textgreater{} - Comme le compilateur C ne sait travailler
que sur un unique fichier qui regroupe interface et corps, il faut
\textbf{inclure l'interface \texttt{complexe.h} au début du corps
\texttt{complexe.c}} à l'aide de la commande pré-processeur
\texttt{\#include\ "complexe.h"}. \textgreater{} - \textbf{Pour utiliser
un module \texttt{complexe} dans un programme principal}
(\texttt{calculer.c} par exemple), on inclut son interface
\texttt{complexe.h} au début du fichier avec la commande\\
\texttt{\#include\ "complexe.h"}

    \textbf{Note} : la commande \texttt{\#include} indique le nom du module
à inclure de deux façons différentes. - Soit entre
\texttt{\textless{}\ \textgreater{}} comme pour
\texttt{\#include\ \textless{}stdlib.h\textgreater{}}. - Soit entre
guillements, comme pour \texttt{\#include\ "complexe.h"}.

Dans le premier cas, le module est recherché par le pré-processeur dans
des répertoires systèmes pré-définis.\\
Dans le second cas, le module est recherché dans le répertoire courant.
Il est possible d'indiquer le chemin relatif ou absolu du fichier
d'en-tête si on le souhaite :\\
\texttt{\#include\ "libs/complexe.h"} recherche le fichier d'en-tête
dans le sous-répertoire \texttt{libs} du répertoire courant.

    \paragraph{\texorpdfstring{Interface ou fichier d'en-tête
(\texttt{.h})}{Interface ou fichier d'en-tête (.h)}}\label{interface-ou-fichier-den-tuxeate-.h}

Pour expliciter ce que fournit le module, on décrit dans l'interface :\\
- la spécification avec la \textbf{déclaration en avant} des
sous-programmes, - la déclaration des éventuels types, constantes (et
variables globales).

\paragraph{\texorpdfstring{Corps ou définition
(\texttt{.c})}{Corps ou définition (.c)}}\label{corps-ou-duxe9finition-.c}

Le corps du module comporte : - l'inclusion de l'interface
\texttt{\#include\ "module.h"} - la définition des sous-programmes
déclarés en avant dans le fichier d'en-tête, - la déclaration
d'éventuels types, constantes (et variables globales), - la
spécification et définition des sous-programmes \textbf{internes} au
modules.

    \subsubsection{\texorpdfstring{5.3 Exemple : module
\texttt{date}}{5.3 Exemple : module date}}\label{exemple-module-date}

Voici l'exemple des deux fichiers composant l'en-tête et le corps du
module \texttt{date}, tous deux disponibles sur SVN.

    \begin{center}\rule{0.5\linewidth}{\linethickness}\end{center}

\subparagraph{\texorpdfstring{L'en-tête
\texttt{date.h}:}{L'en-tête date.h:}}\label{len-tuxeate-date.h}

\begin{center}\rule{0.5\linewidth}{\linethickness}\end{center}

\begin{Shaded}
\begin{Highlighting}[]
\CommentTok{/**}
\CommentTok{ *  module date}
\CommentTok{ */}

\CommentTok{// Inclusion des bibliothèques nécessaires à l'interface __ET__ au corps}
\PreprocessorTok{#include }\ImportTok{<time.h>}

\CommentTok{// Declaration des types }
\KeywordTok{enum}\NormalTok{ NomJour \{DIMANCHE, LUNDI, MARDI, MERCREDI, JEUDI, VENDREDI, SAMEDI\};}
\KeywordTok{enum}\NormalTok{ Mois \{JAN, FEV, MAR, AVR, MAI, JUIN, JUIL, AOUT, SEPT, OCT, NOV, DEC \};}
\KeywordTok{typedef} \KeywordTok{enum}\NormalTok{ NomJour NomJour;}
\KeywordTok{typedef} \KeywordTok{enum}\NormalTok{ Mois Mois;}

\KeywordTok{struct}\NormalTok{ Date \{}
    \DataTypeTok{int}\NormalTok{ jour;}
\NormalTok{    NomJour nomJour;}
\NormalTok{    Mois mois;}
    \DataTypeTok{int}\NormalTok{ annee;}
    \CommentTok{// Invariant : jour>=1 && jour<=31; annee>0}
\NormalTok{\};}
\KeywordTok{typedef} \KeywordTok{struct}\NormalTok{ Date Date;}


\CommentTok{// Declaration (en avant !) des fonctions et procedures}

\CommentTok{// Initialise une date. Elle vaut alors Jeudi 01/01/1970.}
\DataTypeTok{void}\NormalTok{ initialiser(Date* date);}

\CommentTok{// Retourne la date d'aujourd'hui}
\NormalTok{Date date_aujourd_hui();}

\CommentTok{// Affiche dans stdout la date d'aujourd'hui au format d.jour/(d.mois+1)/d.annee}
\DataTypeTok{void}\NormalTok{ afficher_date(Date d);}

\CommentTok{// Convertit la date au format time_t de time.h en une date de type Date}
\DataTypeTok{void}\NormalTok{ convertir_vers_date(time_t t, Date* date); }
\end{Highlighting}
\end{Shaded}

    \begin{center}\rule{0.5\linewidth}{\linethickness}\end{center}

\subparagraph{\texorpdfstring{Le corps \texttt{date.c}
:}{Le corps date.c :}}\label{le-corps-date.c}

\begin{center}\rule{0.5\linewidth}{\linethickness}\end{center}

\begin{Shaded}
\begin{Highlighting}[]
\CommentTok{/**}
\CommentTok{ *  Module date}
\CommentTok{ */}

\CommentTok{// Inclure l'interface Date.h}
\PreprocessorTok{#include }\ImportTok{"date.h"}

\CommentTok{// Inclure les bibliothèques uniquement nécessaire à Date.c }
\PreprocessorTok{#include }\ImportTok{<stdio.h>}
\PreprocessorTok{#include }\ImportTok{<math.h>}

\DataTypeTok{void}\NormalTok{ initialiser(Date *date)\{}
\NormalTok{    date->jour = }\DecValTok{1}\NormalTok{;}
\NormalTok{    date->nomJour = JEUDI;}
\NormalTok{    date->mois = JAN;}
\NormalTok{    date->annee = }\DecValTok{1970}\NormalTok{;}
\NormalTok{\}}
\DataTypeTok{void}\NormalTok{ convertir_vers_date(time_t t, Date* date)\{}
    \KeywordTok{struct}\NormalTok{ tm now;}
\NormalTok{    localtime_r(&t, &now);}\CommentTok{//convertion fuseau horaire}
\NormalTok{    date->jour = now.tm_mday;}\CommentTok{//jour}
\NormalTok{    date->nomJour = now.tm_wday;}\CommentTok{//jour de la semaine }
\NormalTok{    date->mois = now.tm_mon;}\CommentTok{//mois}
\NormalTok{    date->annee = now.tm_year+}\DecValTok{1900}\NormalTok{;}\CommentTok{//annee (a partir de 1900) }
\NormalTok{\}}
\NormalTok{Date date_aujourd_hui()\{}
\NormalTok{    time_t t = time(}\DecValTok{0}\NormalTok{);   }\CommentTok{// date systeme avec #include <time.h>}
\NormalTok{    Date auj;}
\NormalTok{    convertir_vers_date(t, &auj);}
    \ControlFlowTok{return}\NormalTok{ auj;}
\NormalTok{\}}
\DataTypeTok{void}\NormalTok{ afficher_date(Date d)\{}
\NormalTok{    printf(}\StringTok{"Date %i/%i/%i }\SpecialCharTok{\textbackslash{}n}\StringTok{"}\NormalTok{,d.jour, (d.mois+}\DecValTok{1}\NormalTok{), d.annee);}
\NormalTok{\}}
\end{Highlighting}
\end{Shaded}

    \subsubsection{5.4 Compiler un module.}\label{compiler-un-module.}

On peut compiler un module une fois l'interface et le corps définis avec
l'option \texttt{-c}:\\
\textgreater{}\textgreater{}\texttt{c99\ -Wextra\ -pedantic\ -c\ date.c}

Dans cette étape, les commandes pré-processeur (\texttt{\#define},
\texttt{\#include}, etc) sont réalisées, puis le compilateur vérifie la
correction syntaxique du fichier et \textbf{génère un binaire
(non-exécutable)} appelé \texttt{date.o}

\begin{quote}
\textbf{Exercice}\\
- Compiler, dans le répertoire SVN, le module \texttt{date} déjà
présent.
\end{quote}

    \subsubsection{5.5 Comment utiliser un module
?}\label{comment-utiliser-un-module}

On l'inclut dans l'application voulue à l'aide de la commande
pré-processeur :\\
\texttt{\#include\ "nom\_module.h"}

Voici un exemple d'utilisation du module \texttt{date} dans une
application \texttt{ephemeride.c} qui affiche la date du jour :

\begin{Shaded}
\begin{Highlighting}[]
\PreprocessorTok{#include }\ImportTok{"date.h"}\PreprocessorTok{   }\CommentTok{//Inclure le module date}
\DataTypeTok{int}\NormalTok{ main()\{}
\NormalTok{    Date auj = date_aujourd_hui();}
\NormalTok{    afficher_date(auj);}
\NormalTok{\}}
\end{Highlighting}
\end{Shaded}

Il est possible de compiler ce fichier \texttt{ephemeride.c} sans
générer d'exécutable. Comme pour un module, on génère un fichier binaire
\texttt{ephemeride.o} avec la commande :\\
\textgreater{}\texttt{c99\ -Wextra\ -pedantic\ -c\ ephemeride.c}

Le fichier \texttt{executable.o} n'est pas un exécutable car il lui
manque la définition des sous-programmes du module \texttt{date}.

\begin{quote}
\textbf{Exercice}\\
- Compiler, dans le répertoire SVN, le programme \texttt{ephemeride.c}
déjà présent pour obtenir \texttt{ephemeride.o}.
\end{quote}

    \subsubsection{5.6 Compilation d'une application
modulaire}\label{compilation-dune-application-modulaire}

Pour générer un fichier exécutable, il faut lier les fichiers
\texttt{.o} entre eux pour créer l'exécutable final. C'est la phase
\textbf{d'édition de liens}.\\
Pour cela il faut: - Ne pas mettre l'option \texttt{-c} - Lister
l'ensemble des fichiers \texttt{.o} - Qu'il n'y ait qu'une unique
fonction \texttt{main()} dans tous les fichiers objet \texttt{.o} -
Donner le nom de l'exécutable après l'option \texttt{-o} (à défaut, un
exécutable avec un nom par défaut est créé).

La ligne de commande suivante génère l'exécutable \texttt{main}:
\textgreater{} c99 date.o ephemeride.o -o main

Ce fichier exécutables, \texttt{main}, s'exécute avec la commande
système habituelle : \textgreater{} ./main

\paragraph{Exemple de compilation
séparée}\label{exemple-de-compilation-suxe9paruxe9e}

Voici un exemple de \textbf{compilation séparée} de tous les fichiers,
avec création des fichiers objet d'abord, puis l'édition de liens à la
fin :

\begin{verbatim}
    c99 -Wextra -pedantic -c  date.c 
    c99 -Wextra -pedantic -c  ephemeride.c
    c99  date.o ephemeride.o -o main 
\end{verbatim}

\begin{quote}
\textbf{Exercice}\\
- Créer le fichier exécutable dénommé \texttt{ephemeride} dans le
répertoire SVN. L'exécuter.
\end{quote}

\paragraph{Exemple de compilation
directe}\label{exemple-de-compilation-directe}

Il est possible de réaliser toutes ces étapes en utilisant une seule
ligne de commande.\\
Dans ce cas, le compilateur réalise automatiquement les étapes de
compilation et d'édition de liens.

Pour cela, il faut lister l'ensemble des fichiers \texttt{.c} comme dans
l'exemple suivant :

\begin{verbatim}
    c99  date.c ephemeride.c -o main 
\end{verbatim}

\begin{quote}
\textbf{Exercice}\\
- Dans le répertoire SVN, supprimer les fichiers objet \texttt{date.o}
et \texttt{ephemeride.o}, ainsi que l'exécutable \texttt{ephemeride}. -
Réaliser la compilation directe pour engendrer \texttt{ephemeride} -
Observer les fichiers créés. Qu'en déduire ?
\end{quote}

    \subsubsection{5.7 Inclusion multiple d'un même
module}\label{inclusion-multiple-dun-muxeame-module}

Supposons que l'on souhaite développer une application \texttt{EDT} qui
permette de gérer un emploi du temps pour une école. Cette application
va définir un ensemble de modules pour réprésenter les ressources à
gérer : - les enseignants, - les salles, - les élèves, - les cours,
etc..

Supposons également que les modules \texttt{enseignant} et
\texttt{salle} aient besoin de manipuler des dates, et donc d'inclure le
module \texttt{date} dans leur en-tête.\\
Pour définir un module \texttt{cours}, on a besoin d'inclure les modules
\texttt{enseignant} et \texttt{salle}.\\
On se retrouve alors dans la situation suivante : -
\texttt{enseignant.h} inclut \texttt{date.h} - \texttt{salle.h} inclut
\texttt{date.h} - \texttt{cours.h} inclut \texttt{enseignant.h} et
\texttt{salle.h}.

\begin{quote}
\textbf{Inclusions multiples}\\
Les inclusions du pré-processeur ne sont que des 'copier/coller' des
fichiers en lieu et place de la commande \texttt{\#include}.\\
Le pré-processeur fournit ainsi au compilateur un fichier unique
\texttt{cours.cpp} qui comporte l'inclusion de \texttt{enseignant.h} et
\texttt{salle.h}, qui eux-même incluent \texttt{cours.h}.\\
On retrouve donc \textbf{deux inclusions de \texttt{cours.h}}.
\end{quote}

Or, \textbf{le langage C interdit la déclaration multiple de variables,
types ou sous-programmes de même nom.}\\
Ainsi, le compilateur vérifie dans \texttt{cours.cpp} qu'il n'existe pas
deux identificateurs ou plus qui soient identiques.

\begin{quote}
\textbf{!!! Problème !!! : Le compilateur ici refuse de compiler le
module \texttt{cours}}\\
car il annonce la double définition du type struct date, et des
sous-programmes de \texttt{date.h}.
\end{quote}

    \subsubsection{5.8 Solution : la garde
conditionnelle}\label{solution-la-garde-conditionnelle}

Pour résoudre ce problème, on doit rajouter à l'interface \textbf{la
garde conditionnelle} avec les commandes pré-processeur suivantes :

\begin{Shaded}
\begin{Highlighting}[]
    \PreprocessorTok{#ifndef DATE__H  }\CommentTok{// Garde conditionnelle : si la variable DATE__H n'existe pas }
    \PreprocessorTok{#define DATE__H  }\CommentTok{// La déclarer.}
    
    \CommentTok{/**}
\CommentTok{     *  module date}
\CommentTok{     */}

    \CommentTok{// Inclusion des bibliothèques nécessaires à l'interface __ET__ au corps}
    \PreprocessorTok{#include }\ImportTok{<time.h>}

    \CommentTok{// Declaration des types }
    \KeywordTok{enum}\NormalTok{ NomJour \{ DIMANCHE, LUNDI, MARDI, MERCREDI, JEUDI, VENDREDI, SAMEDI\};}
    \KeywordTok{enum}\NormalTok{ Mois \{JAN, FEV, MAR, AVR, MAI, JUIN, JUIL, AOUT, SEPT, OCT, NOV, DEC \};}
    
\NormalTok{    ..... }
    
\NormalTok{    .....}
    
    \PreprocessorTok{#endif }\CommentTok{// on clot la garde conditionne à la toute fin du fichier date.h}
\end{Highlighting}
\end{Shaded}

Ainsi: - lors de la première inclusion de \texttt{date.h}, la variable
pré-processeur \texttt{DATE\_\_H} n'existe pas, et le contenu de
\texttt{date.h} est inclut. \texttt{DATE\_\_H} est déclarée et existe
pour tous les traitements suivants du pré-processeur. - lors des
instructions d'inclusion suivantes, la variable \texttt{DATE\_\_H}
existe déjà. Avec la close \texttt{\#ifndef}, tout ce qui se trouve
avant \texttt{\#endif} n'est pas considéré pour l'inclusion si
\texttt{DATE\_H} existe.

\begin{quote}
\textbf{Règle importante, voire fondamentale !}\\
Il faut TOUJOURS ajouter une garde conditionnelle à son fichier
d'en-tête.
\end{quote}

\textbf{Note} L'identificateur de la constante pré-processeur est choisi
arbitrairement. Il est d'usage d'utiliser le nom du module pour garantir
son unicité.

    \begin{quote}
\textbf{Exercice}\\
- Dans le répertoire SVN, compiler les fichiers \texttt{enseignant.c},
\texttt{cours.c}, \texttt{salle.c}, \texttt{date.c} et \texttt{EDT.c}
afin de créer l'exécutable \texttt{EDT}. - Quels erreurs de double
inclusion observez-vous ? - Corriger les fichiers d'en-tête qui en sont
responsables.
\end{quote}

    \subsubsection{5.9 Visibilité des variables en
C}\label{visibilituxe9-des-variables-en-c}

Par défaut, tous les sous-programmes et variables globales définies dans
\textbf{le corps \texttt{module.c} sont visibles} par les modules et
programmes qui l'incluent.

\begin{quote}
Il faut \emph{explicitement} rendre une variable ou fonction locale au
module.
\end{quote}

\paragraph{\texorpdfstring{La propriété
\texttt{static}}{La propriété static}}\label{la-propriuxe9tuxe9-static}

Pour fournir une visibilité locale à une variable ou à un
sous-programme, il faut précéder sa déclaration avec le mot-clé
\texttt{static}. \textgreater{} \textbf{Attention} : on ne peut pas
rendre un type static

\paragraph{Exemple}\label{exemple}

\begin{Shaded}
\begin{Highlighting}[]
\PreprocessorTok{#ifndef EXEMPLE_STATIC__H}
\PreprocessorTok{#define EXEMPLE_STATIC__H}

\CommentTok{// Unique fonction}
\CommentTok{// visible par}
\CommentTok{// les autres modules}
\DataTypeTok{int}\NormalTok{ f();}

\PreprocessorTok{#endif}
\end{Highlighting}
\end{Shaded}

\begin{center}\rule{0.5\linewidth}{\linethickness}\end{center}

Et voici le corps \texttt{exemple\_static.c}

\begin{longtable}[]{@{}l@{}}
\toprule
\begin{minipage}[b]{0.04\columnwidth}\raggedright\strut
\texttt{c\ \#include\ "exemple\_static.h"\ //\ fonction\ locale\ au\ module\ Static\ //\ non\ visible\ des\ autres\ modules.\ static\ int\ max(int\ a,\ int\ b)\ \{\ if\ (a\ \textgreater{}\ b)\ \{\ return\ a;\ \}\ else\ \{\ return\ b;\ \}\ \}\ //\ fonction\ f()\ presente\ dans\ le\ .h,\ //\ visible\ par\ les\ autres\ modules\ int\ f()\{\ int\ val1\ =\ 2;\ int\ val2\ =\ 8;\ return\ max(val1,\ val2);\ \}}\strut
\end{minipage}\tabularnewline
\midrule
\endhead
\begin{minipage}[t]{0.04\columnwidth}\raggedright\strut
La fonction \texttt{max} est ici \texttt{static}, et donc uniquement
visible des sous-programmes du corps du module.\strut
\end{minipage}\tabularnewline
\bottomrule
\end{longtable}

    \paragraph{\texorpdfstring{Cas d'utilisation de
\texttt{static}}{Cas d'utilisation de static}}\label{cas-dutilisation-de-static}

Voici quelques cas d'utilisation (non exhaustifs) : - Pour définir une
fonction, localement au module, qui présente le même identificateur
qu'une fonction déjà présente dans un des modules inclus dans
l'interface.\\
Typiquement, si on souhaite définir sa propre fonction \texttt{max}
alors qu'on a inclut \texttt{math.h} dans l'interface. - Pour définir
une variable, à portée globale pour le module, mais que l'on ne veut pas
visible au reste de l'application.

    \begin{quote}
\textbf{Exercice}\\
- Dans le répertoire SVN, rendre \texttt{static} le sous-programme
\texttt{afficher\_date} du module \texttt{date}. Compiler à nouveau
l'application \texttt{ephemeride}.\\
Observer que ce sous-programme n'est plus visible.
\end{quote}

    \subsubsection{5.10 Externaliser une variable du
module}\label{externaliser-une-variable-du-module}

Il est possible de déclarer une variable globale au module, et de la
rendre utilisable par le reste de l'application.\\
Cette variable globale est visible et modifiable par tous les
sous-programmes appartenant aux modules et au programme principal qui
incluent ce module.

On prendra l'exemple d'un module \texttt{compteur} qui présente une
variable globale \texttt{compteur}, que l'on souhaite consulter,
incrémenter ou ré-initialiser par les modules qui incluent
\texttt{module.h}.\\
Dans l'exemple suivant, \texttt{compteur.h} offre des sous-programmes
qui permettent de maniputer ce \texttt{compteur}.

\begin{Shaded}
\begin{Highlighting}[]
\PreprocessorTok{#ifndef _COMPTEUR_H}
\PreprocessorTok{#define _COMPTEUR_H}

\CommentTok{// Specification de la procédure re-initialiser}
\DataTypeTok{void}\NormalTok{ re_initialiser();}
\CommentTok{// Specification de la procedure incrementer}
\DataTypeTok{void}\NormalTok{ incrementer();}
\CommentTok{// Specification de la function valeur}
\DataTypeTok{int}\NormalTok{ valeur();}

\PreprocessorTok{#endif }
\end{Highlighting}
\end{Shaded}

\begin{Shaded}
\begin{Highlighting}[]
\PreprocessorTok{#include }\ImportTok{<stdio.h>}

\CommentTok{// declaration de la variable globale + initialisation à 0. }
\DataTypeTok{int}\NormalTok{ compteur = }\DecValTok{0}\NormalTok{;}

\DataTypeTok{void}\NormalTok{ re_initialiser() \{}
\NormalTok{    compteur=}\DecValTok{0}\NormalTok{;}
\NormalTok{\}}
\DataTypeTok{void}\NormalTok{ incrementer() \{}
\NormalTok{    compteur++;}
\NormalTok{\}}
\DataTypeTok{int}\NormalTok{ valeur() \{}
    \ControlFlowTok{return}\NormalTok{ compteur;}
\NormalTok{\}}
\end{Highlighting}
\end{Shaded}

\begin{Shaded}
\begin{Highlighting}[]
\PreprocessorTok{#include }\ImportTok{"compteur.h"}
\PreprocessorTok{#include }\ImportTok{<stdio.h>}

\CommentTok{// acces au compteur de compteur.c}
\KeywordTok{extern} \DataTypeTok{int}\NormalTok{ compteur;}

\DataTypeTok{int}\NormalTok{ main()\{}
  \CommentTok{// initialiser}
\NormalTok{  re_initialiser();}
\NormalTok{  printf(}\StringTok{"Init Compteur c=%d}\SpecialCharTok{\textbackslash{}n}\StringTok{"}\NormalTok{, valeur());}
  \CommentTok{// incrementer}
\NormalTok{  incrementer();}
\NormalTok{  printf(}\StringTok{"Incrementer c=%d}\SpecialCharTok{\textbackslash{}n}\StringTok{"}\NormalTok{, valeur());}
  \CommentTok{// access au compteur sans appel a valeur()}
\NormalTok{  incrementer();}
\NormalTok{  incrementer();}
\NormalTok{  printf(}\StringTok{"Acces direct a compteur c=%d}\SpecialCharTok{\textbackslash{}n}\StringTok{"}\NormalTok{, compteur);}

  \ControlFlowTok{return} \DecValTok{0}\NormalTok{; }
\NormalTok{\}}
\end{Highlighting}
\end{Shaded}

    \begin{quote}
\textbf{Exercice}\\
- Dans le répertoire SVN, compiler le fichier \texttt{test\_compteur.c}.
Observer que le compteur s'incrémente bien. - Dé-commenter les deux
instructions qui se trouvent entre les commentaires
\texttt{//\ ***\ DECOMMENTER\ ***} et
\texttt{//****\ FIN\ DECOMMENTER\ ***}.\\
Qu'observez-vous ? Que peut-on en conclure sur la possibilité
d'encapsuler complètement une variable externalisée ?
\end{quote}

    \subsubsection{5.10 Exercice - UNO}\label{exercice---uno}

La correction des exercices 2 et 3 qui ont été traités dans le sujet C2
est disponible sous SVN, dans le répertoire
\texttt{c3/fichiers\_C/modules/UNO}.\\
\textbf{Sans créer de nouveaux sous-programmes}, il est demandé dans ce
travail de réorganiser le code des ces deux fichiers \texttt{.c} pour en
extraire des modules.\\
Pour cela, on vous demande de :

\begin{itemize}
\tightlist
\item
  Créer 4 modules (\texttt{carte}, \texttt{main}, \texttt{jeu} et
  \texttt{UNO}) ;
\item
  Produire les exécutables \texttt{tester\_UNO} et \texttt{jouer\_UNO}.
\end{itemize}

L'exécutable \texttt{jouer\_UNO} permet de lancer le jeu de UNO. Il
n'est pas complet car l'ensemble des sous-programmes nécessaires au jeu
n'est pas disponible. Dans les faits, il ne fera que préparer le jeu.

\emph{Attention à la garde conditionnelle !}

Pour sauvegarder ce travail, vous ajouterez vos fichiers au répertoire
UNO avec la commande \texttt{svn\ add} (ce travail n'est pas évalué).

    \begin{center}\rule{0.5\linewidth}{\linethickness}\end{center}

    \subsection{6. L'outil Make pour automatiser la
compilation}\label{loutil-make-pour-automatiser-la-compilation}

\begin{quote}
\textbf{Note :} L'ensemble des fichiers liés à cette partie se trouvent
dans le répertoire : \textbf{\texttt{c3/fichiers\_C/make}}.
\end{quote}

    \subsubsection{6.1 Rappel : modules en C}\label{rappel-modules-en-c}

Un module en C, c'est : - Une convention de nommage qui décompose un
module en deux fichiers, l'en-tête \texttt{.h} et le corps
\texttt{.c}.\\
Par exemple, le module \texttt{complexe} se compose : \textgreater{} -
d'un fichier d'interface typiquement nommé \texttt{complexe.h}
\textgreater{} - d'un fichier corps typiquement nommé
\texttt{complexe.c} - Un outillage pour compiler cette structure de
fichiers : \textgreater{} - Comme le compilateur C ne sait travailler
que sur un unique fichier qui regroupe interface et corps, il faut
\textbf{inclure l'interface \texttt{complexe.h} au début du corps
\texttt{complexe.c}} à l'aide de la commande pré-processeur
\texttt{\#include\ "complexe.h"}. \textgreater{} - \textbf{Pour utiliser
un module \texttt{complexe} dans un programme principal}
(\texttt{calculer.c} par exemple), on inclut son interface
\texttt{complexe.h} au début du fichier avec la commande\\
\texttt{\#include\ "complexe.h"}

    \begin{center}\rule{0.5\linewidth}{\linethickness}\end{center}

    \subsubsection{6.2 Rappel : compilation d'une application
modulaire}\label{rappel-compilation-dune-application-modulaire}

La compilation d'une application modulaire peut-être réalisée de deux
façon différentes : la compilation séparée, et la compilation directe.

Dans la suite de ce sujet C3, on reprend l'exemple des applications
\texttt{ephemeride} et \texttt{EDT}, qui incluent les modules
\texttt{date}, \texttt{cours}, \texttt{enseignant} et \texttt{salle}.

\paragraph{Exemple de compilation
séparée}\label{exemple-de-compilation-suxe9paruxe9e}

Voici un exemple de \textbf{compilation séparée} de tous les fichiers,
avec création des fichiers objet :

\begin{verbatim}
    c99 -Wextra -pedantic -c  date.c 
    c99 -Wextra -pedantic -c  salle.c 
    c99 -Wextra -pedantic -c  enseignant.c 
    c99 -Wextra -pedantic -c  cours.c 
    c99 -Wextra -pedantic -c  EDT.c 
    c99 -Wextra -pedantic -c  ephemeride.c
\end{verbatim}

\begin{quote}
\textbf{Note} : on peut réaliser cette étape avec la commande
\texttt{c99\ -Wextra\ -pedantic\ -c\ *.c}.\\
On obtient alors les fichiers objet de TOUS LES FICHIERS .c qui se
trouvent dans le répertoire courant.
\end{quote}

Et création des exécutables \texttt{ephemeride} et \texttt{EDT} par
l'étape d'édition des liens :

\begin{verbatim}
    c99  date.o ephemeride.o -o ephemeride 
    c99  date.o salle.o enseignant.o cours.o EDT.o -o EDT 
\end{verbatim}

\paragraph{Exemple de compilation
directe}\label{exemple-de-compilation-directe}

Il est possible de réaliser toutes ces étapes en utilisant une seule
ligne de commande.\\
Dans ce cas, le compilateur réalise automatiquement les étapes de
compilation et d'édition de liens.

Pour cela, il faut lister l'ensemble des fichiers \texttt{.c} comme dans
l'exemple suivant :

\begin{verbatim}
    c99  date.c ephemeride.c -o ephemeride
    c99  date.c salle.c enseignant.c cours.c EDT.c -o EDT     
\end{verbatim}

    \begin{center}\rule{0.5\linewidth}{\linethickness}\end{center}

    \subsubsection{6.3 Automatiser la
compilation}\label{automatiser-la-compilation}

Il est possible d'automatiser la compilation d'une application modulaire
avec l'outil \texttt{make}. \textgreater{} \textbf{Note} : L'outil
\texttt{make} n'est pas uniquement utilisé pour compiler du C.\\
Il peut automatiser des commandes qui engendrent des fichiers (ou non !)
de nature différente.

La commande \texttt{make} réalise les étapes suivantes : 1. Elle
recherche le fichier dénommé \texttt{Makefile} ou \texttt{makefile} dans
le répertoire courant 2. \textbf{Exécute la première règle} qu'elle
trouve dans le fichier.

Il est possible de lui spécifier un nom de fichier avec l'option
\texttt{-f} : \texttt{make\ -f\ mon\_makefile}

    \begin{center}\rule{0.5\linewidth}{\linethickness}\end{center}

    \subsubsection{\texorpdfstring{6.4 Le fichier
\texttt{Makefile}}{6.4 Le fichier Makefile}}\label{le-fichier-makefile}

\paragraph{Une règle}\label{une-ruxe8gle}

Le fichier présente une liste de \textbf{règles}. Chaque règle : 1. a
pour objectif de créer un fichier que l'on appelle la \textbf{cible}. 2.
indique également quelle \textbf{commande} sera lancée pour obtenir la
cible. 3. présente une \textbf{liste des dépendances}, sous la forme
d'une liste de fichiers qui doivent être présents dans le répertoire
pour engendrer la cible.

L'écriture d'une règle suit \textbf{strictement} la syntaxe suivante :

\begin{verbatim}
nom_cible: dependances_regle1
[tab]commande_regle1
\end{verbatim}

\begin{quote}
\textbf{Attention}: - \textbf{pas d'espace} entre le nom de la cible et
\texttt{:} - par contre, il y a un espace entre le caractère \texttt{:}
et le premier fichier de la liste des dépendances. - la seconde ligne de
la règle commance par une tabulation
\end{quote}

    \subsubsection{\texorpdfstring{Exemple d'utilisation pour l'application
\texttt{ephemeride}.}{Exemple d'utilisation pour l'application ephemeride.}}\label{exemple-dutilisation-pour-lapplication-ephemeride.}

Voici un fichier \texttt{Makefile} permettant de générer l'exécutable
\texttt{ephemeride}.\\
Il comporte 4 règles permettant de générer les cibles
\texttt{ephemeride}, \texttt{date.o}, \texttt{ephemeride.o} et
\texttt{clean}.

\begin{verbatim}
ephemeride: date.o ephemeride.o
    c99 date.o ephemeride.o -o ephemeride

date.o: date.c date.h
    c99 -Wextra -pedantic -c date.c

ephemeride.o: ephemeride.c
    c99 -Wextra -pedantic -c ephemeride.c
    
clean:
    rm *.o ephemeride
\end{verbatim}

\begin{quote}
\textbf{Exercice}\\
- Ce fichier Makefile se trouve dans le répertoire SVN. L'exécuter en
tapant la commande \texttt{make}. - Dans quel ordre les commandes
sont-elles exécutées ? - Quels sont les fichiers générés par cette
commande ?
\end{quote}

    \begin{center}\rule{0.5\linewidth}{\linethickness}\end{center}

    \subsubsection{\texorpdfstring{6.5 Comportement de
\texttt{make}}{6.5 Comportement de make}}\label{comportement-de-make}

La commande : - \texttt{make} exécute la \textbf{première règle}. -
\texttt{make\ date.o} exécute la seconde règle, la commande
\texttt{make\ ephemeride.o} la troisième, etc.

\begin{quote}
\textbf{Exercice}\\
- Executer la commande \texttt{make\ clean}. - Que se passe-t-il ? Quels
sont les fichiers supprimés ?
\end{quote}

La première règle du fichier :

\begin{verbatim}
ephemeride: date.o ephemeride.o
    c99 date.o ephemeride.o -o ephemeride
\end{verbatim}

\begin{itemize}
\tightlist
\item
  permet de générer la cible \texttt{ephemeride}
\item
  avec la commande \texttt{c99\ date.o\ ephemeride.o\ -o\ ephemeride}
\item
  et présente les dépendances \texttt{date.o\ ephemeride.o}.
\end{itemize}

Ainsi, pour que la cible puisse être générée avec la commande donnée, il
faut que les fichiers \texttt{date.o} et \texttt{ephemeride.o} soient
présents dans le répertoire courant. \textbf{Si un fichier n'existe pas,
\texttt{make} recherche une règle dans le \texttt{Makefile} qui lui
permet de la produire}.\\
Par exemple, si \texttt{date.o} est absent, \texttt{make} exécute la
seconde règle pour générer \texttt{date.o}.

\begin{quote}
\textbf{Exercice}\\
- Tapez à nouveau la commande \texttt{make} pour lancer la première
règle.
\end{quote}

On observe que la première commande exécutée est la commande
\texttt{c99\ -Wextra\ -pedantic\ -c\ date.c} car la dépendance
\texttt{date.o} n'existe pas. La seconde commande permet de générer
\texttt{ephemeride.o} et la dernière commande exécutée est la commande
qui produit l'exécutable \texttt{ephemeride}.

\begin{quote}
\textbf{Exercice}\\
- Tapez à nouveau la commande \texttt{make} pour lancer la première
règle. Qu'observez-vous ?
\end{quote}

La commande nous indique que les cibles sont à jour : aucune commande
n'est exécutée.

\begin{quote}
\textbf{Exercice}\\
- Modifier le fichier date.c en y ajoutant un commentaire par exemple.\\
- Taper à nouveau la commande \texttt{make} pour lancer la première
règle. Qu'observez-vous ?
\end{quote}

Seuls la création des cibles \texttt{date.o} et \texttt{ephemeride} est
lancée. La règle \texttt{ephemeride.o} n'est pas activée.\\
En effet, elle n'est pas concernée par les changements réalisés dans le
fichier \texttt{date.c}.

\begin{quote}
\begin{quote}
La commande \texttt{make} ne génère une cible que si celle-ci présente
une date de modification \textbf{antérieure} à celle de l'un des
fichiers listés dans ses dépendances.\\
En d'autres termes, si une des dépendances est plus récente que la
cible, \texttt{make} relance la règle correspondante pour que la cible
soit mise à jour.\\
La \textbf{liste des dépendances est donc très importante}.
\end{quote}
\end{quote}

\paragraph{Règles sans dépendances}\label{ruxe8gles-sans-duxe9pendances}

La règle \texttt{clean} ne présente pas de dépendances. Elle est
uniquement appliquée quand on l'appelle explicitement avec la commande
\texttt{make\ clean}.

\paragraph{Règles qui ne génèrent pas de
fichier}\label{ruxe8gles-qui-ne-guxe9nuxe8rent-pas-de-fichier}

On peut indiquer à \texttt{make} que la règle \texttt{clean} ne produit
pas un fichier dénommé \texttt{clean} en rajoutant la ligne à la fin du
\texttt{Makefile} :

\begin{verbatim}
.PHONY: clean
\end{verbatim}

Ainsi, \texttt{make} n'ira pas vérifier si le fichier \texttt{clean}
existe avant d'exécuter la règle. Il se comporte comme si les fichiers
listés après \texttt{.PHONY} n'étaient jamais à jour.

    \begin{quote}
\textbf{Exercice}\\
1. Ajouter la règle suivante \textbf{au début} du \texttt{Makefile} :\\
\texttt{all:\ ephemeride} 2. Insérer la ligne suivante \textbf{à la fin}
du \texttt{Makefile} :\\
\texttt{.PHONY:\ clean\ all} A quoi sert la règle \texttt{all} ? 3.
Ajouter les règles nécessaires à la création de l'exécutable
\texttt{EDT} dans \texttt{Makefile}. 4. S'assurer que les deux
exécutables, \texttt{ephemeride} et \texttt{EDT} soient générés quand on
tape \texttt{make\ all}.
\end{quote}

    \begin{center}\rule{0.5\linewidth}{\linethickness}\end{center}

    \subsubsection{6.6 Variables et
commentaires}\label{variables-et-commentaires}

\paragraph{Commentaires}\label{commentaires}

Il est possible d'insérer \textbf{des commentaires} dans un
\texttt{Makefile} en utilisant la balise \texttt{\#}.

\paragraph{Variables}\label{variables}

Il est également possible de définir \textbf{des variable} pour rendre
le \texttt{Makefile} plus générique.\\
L'accès \textbf{au contenu} de la variable se fait par la notation
\texttt{S(VAR)}, comme illustré ici :

\begin{verbatim}
# Ceci est un commentaire pour introdure la variable CC qui comporte le nom du compilateur utilisé :
CC=c99
# Les variables CFLAGS et LDFLAGS comportent les options de compilation et d'édition des liens :
CFLAGS=-Wextra -pedantic -c
LDFLAGS=

ephemeride: date.o ephemeride.o
    $(CC) $(LDFLAGS) date.o ephemeride.o -o ephemeride

date.o: date.c date.h
    $(CC) $(CFLAGS) date.c

ephemeride.o: ephemeride.c
    $(CC) $(CFLAGS) ephemeride.c
    
clean:
    rm *.o ephemeride
    
.PHONY: clean
\end{verbatim}

\paragraph{Variables automatiques}\label{variables-automatiques}

Des variables automatiques peuvent être utilisées pour définir une règle
de façon encore plus générique : - \texttt{\$@} : le nom de la cible de
la règle courante - \texttt{\$\textless{}} : le nom du premier fichier
dans la liste des dépendances - \texttt{\$\^{}} : le nom de tous les
fichiers (séparés par un espace) listées dans les dépendances de la
règle courante - \texttt{\$?} : le nom de tous les fichiers listés dans
les dépendances et qui sont plus récents que la cible courante

On obtient par exemple :

\begin{verbatim}
date.o: date.c date.h
    $(CC) $(CFLAGS) $<
\end{verbatim}

avec \texttt{\$\textless{}} qui représente ainsi le nom \texttt{date.c}.

\begin{quote}
\textbf{Exercice}\\
1. Editer votre fichier \texttt{Makefile} pour ajouter les variables
\texttt{CC},\texttt{CFLAGS} et \texttt{LDFLAGS}. 2. Editer votre fichier
en utilisant les variables automatiques \texttt{\$@} et
\texttt{@\textless{}}
\end{quote}

    \begin{center}\rule{0.5\linewidth}{\linethickness}\end{center}

    \subsubsection{6.7 Règles implicites}\label{ruxe8gles-implicites}

Certaines règles sont déjà connues de \texttt{make}, on dit qu'elles
sont \texttt{implicites}.\\
Par exemple, il n'est pas nécessaire d'écrire de règle pour engendrer un
fichier objet \texttt{.o}.

\begin{quote}
\textbf{Exercice}\\
1. Supprimer les règles qui permettent de générer des fichiers objet
(\texttt{date.o}, \texttt{ephemeride.o}, etc.) 2. Lancer la commande
\texttt{make\ clean} 3. Lancer la commande \texttt{make\ all}\\
Observer que la création des fichiers objets est réalisée tout de même !
\end{quote}

    \begin{center}\rule{0.5\linewidth}{\linethickness}\end{center}

    \subsubsection{\texorpdfstring{6.8 L'outil \texttt{makedepend} pour les
\texttt{.h}}{6.8 L'outil makedepend pour les .h}}\label{loutil-makedepend-pour-les-.h}

Nous avons vu qu'il est nécessaire de lister l'ensemble des fichiers
\texttt{.c} et \texttt{.h} dans les dépendances d'une règle qui compile
une application modulaire.\\
Or, la commande qui génère l'exécutable ne liste en général que les
\texttt{.o} (compilation séparée) ou que les \texttt{.c} (compilation
directe).

Pour s'affranchir de lister \emph{à la main} les fichiers d'en-tête
\texttt{.h} en plus des \texttt{.c} dans les dépendances, on peut
utiliser l'outil \texttt{makedepend}.

Pour l'utiliser, il faut lancer la commande système suivante dans le
répertoire qui contient les sources :

\begin{verbatim}
makedepend *.c -Y.
\end{verbatim}

L'outil inclut les lignes suivantes à la fin du \texttt{Makefile}

\begin{verbatim}
# DO NOT DELETE

EDT.o: cours.h salle.h date.h enseignant.h
cours.o: cours.h salle.h date.h enseignant.h
date.o: date.h
enseignant.o: enseignant.h date.h
ephemeride.o: date.h
salle.o: salle.h date.h
\end{verbatim}

Ces lignes représentent les dépendantes existant entre les fichiers
objet et les fichiers d'en-tête. Ainsi, si on utilise
\texttt{makedepend}, on peut s'affranchir d'ajouter \texttt{date.h} dans
les dépendances de la règle \texttt{date.o} par exemple.

\paragraph{\texorpdfstring{Règle \texttt{depend} dans le
\texttt{Makefile}}{Règle depend dans le Makefile}}\label{ruxe8gle-depend-dans-le-makefile}

Il est d'usage d'ajouter la règle suivante dans le \texttt{Makefile} :

\begin{verbatim}
depend: 
    makedepend *.c -Y.
\end{verbatim}

Ainsi, après avoir rédigé son \texttt{Makefile} avec cette règle, on
lance la commande \texttt{makedepend} en utilisation directement la
règle \texttt{depend} du makefile : il suffit de taper la commande
\texttt{make\ depend} dans un terminal pour mettre à jour les
dépendances à la fin du fichier \texttt{Makefile}.

\begin{quote}
\textbf{Exercice} 1. Ajouter la règle \texttt{depend} au
\texttt{Makefile} 2. Lancer la commande \texttt{make\ depend}
\end{quote}

    \begin{center}\rule{0.5\linewidth}{\linethickness}\end{center}

    \subsubsection{6.9 Exercice - UNO}\label{exercice---uno}

La correction de l'exercice 5.10 de ce sujet est disponible dans le
répertoire \texttt{c3/fichiers\_C/make/UNO}. Dans cet exercice, vous
aviez conçu l'application modulaire UNO qui se compose : - de 4 modules
(\texttt{carte}, \texttt{main}, \texttt{jeu} et \texttt{UNO}) ; - des
exécutables \texttt{test\_UNO} et \texttt{jouer\_UNO}.

Ici, il est demandé d'écrire un fichier \texttt{Makefile} qui permet de
générer les deux exécutables à partir des modules.\\
Vous ajouterez aussi une règle \texttt{clean} pour nettoyer le
répertoire et une règle \texttt{All} comme première règle pour générer
toutes les cibles automatiquement. On vous encourage à utiliser les
variables usuelles (\texttt{CC}, \texttt{CFLAGS}, \texttt{LDFLAGS}) et
les variables automatiques.

Pour sauvegarder ce travail, vous ajouterez le fichier \texttt{Makefile}
au répertoire UNO avec la commande \texttt{svn\ add}.

    \begin{center}\rule{0.5\linewidth}{\linethickness}\end{center}

    \subsection{\#\# BILAN sur les modules en C et Make (à
rendre)}\label{bilan-sur-les-modules-en-c-et-make-uxe0-rendre}

\begin{center}\rule{0.5\linewidth}{\linethickness}\end{center}

    \subsection{Définition d'une file
d'attente.}\label{duxe9finition-dune-file-dattente.}

Une file d'attente est une structure de données qui contient des
éléments de même nature. Le principe de la file d'attente que l'on
observe ici est \emph{First In First Out} (FIFO) : le premier élément
inséré dans la file est le premier à en être retiré.

Les opérations sur une telle file sont les suivantes :

\begin{itemize}
\tightlist
\item
  \textbf{initialiser} : initialiser une file (une variable de type
  File). Une file ne peut être utilisée que si elle a été initialisée ;
\item
  \textbf{detruire} : détruire un file, elle ne pourra plus être
  utilisée (sauf à être de nouveau initialisée) ;
\item
  \textbf{inserer} : insérer un nouvel élément dans la file ;
\item
  \textbf{extraire} : extraire le plus ancien élément de la file ;
\item
  \textbf{tete} : retourne la valeur de l'élément en tête de la file
  (mais ne le supprime pas) ;
\item
  \textbf{est\_vide} : savoir si une file est vide ou non ;
\item
  \textbf{longueur} : obtenir la longueur de la file ;
\end{itemize}

On choisit de représenter la file en utilisant une structure chaînée.
Chaque élément de la file est représenté par une cellule composée de
l'élément conservé et d'un pointeur sur la cellule suivante. Pour des
raisons d'efficacité, on décide de représenter le type file comme un
enregistrement composé d'un pointeur sur la première cellule de la file
et d'un pointeur sur la dernière.

    \subsection{Questions}\label{questions}

Dans le répertoire SVN \texttt{c3/fichiers\_C/bilan}, vous trouverez les
fichiers \texttt{file.h}, \texttt{file.c}, \texttt{test\_file.c},
\texttt{exemple\_file.c}, \texttt{Makefile} et \texttt{reponses1.txt}.

Le module \texttt{file} permet ici d'enregistrer des éléments de type
caractères.

    \subsubsection{\texorpdfstring{1. Module
\texttt{file}.}{1. Module file.}}\label{module-file.}

\paragraph{Question 1.1}\label{question-1.1}

\begin{quote}
Dans le fichier \texttt{reponses1.txt}, indiquer : - comment obtenir le
ou les exécutables \emph{par compilation séparée}. - les dépendances
entre les fichiers qui constituent l'application (pour l'exemple et les
tests). - le rôle de la règle \texttt{.PHONY}.
\end{quote}

\paragraph{Question 1.2}\label{question-1.2}

\begin{quote}
Compléter le fichier \texttt{Makefile} (et vérifier qu'il fonctionne).
On définira en particulier - la cible \texttt{all} pour produire les
deux exécutables ; - la cible \texttt{clean} qui supprime les fichiers
engendrés (fichiers .o et exécutables) ; - la cible \texttt{depend} qui
appelle l'outil \texttt{makedepend} qui automatise l'inclusion des
interfaces des modules dans les dépendances.\\
On utilisera : - les variables usuelles \texttt{CC}, \texttt{CFLAGS},
\texttt{LDFLAGS}. - les variables automatiques \texttt{\$@} et
\texttt{\$\^{}}
\end{quote}

\paragraph{Question 1.3}\label{question-1.3}

\begin{quote}
On s'intéresse ici à la spécification des sous-programmes qui est donnée
dans \texttt{file.h}. - Ce fichier contient une erreur dans sa
structure. L'identifier et la corriger.
\end{quote}

\paragraph{Question 1.4}\label{question-1.4}

\begin{quote}
Écrire l'implantation du module \texttt{file} (et tester le module grâce
au module de test fourni).
\end{quote}

\paragraph{Question 1.5}\label{question-1.5}

\begin{quote}
La réponse aux questions suivantes est attendue dans le fichier
\texttt{reponses1.txt}. - Quel est le sous-programme non-visible des
autres modules ? - Que se passe-t-il si on enlève le mot-clé static à la
compilation ? - Que risque-t-on a ne pas utiliser le mot-clé statique ?
\end{quote}


    % Add a bibliography block to the postdoc
    
    
    
    \end{document}
