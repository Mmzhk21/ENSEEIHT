
% Default to the notebook output style

    


% Inherit from the specified cell style.




    
\documentclass[11pt]{article}

    
    
    \usepackage[T1]{fontenc}
    % Nicer default font (+ math font) than Computer Modern for most use cases
    \usepackage{mathpazo}

    % Basic figure setup, for now with no caption control since it's done
    % automatically by Pandoc (which extracts ![](path) syntax from Markdown).
    \usepackage{graphicx}
    % We will generate all images so they have a width \maxwidth. This means
    % that they will get their normal width if they fit onto the page, but
    % are scaled down if they would overflow the margins.
    \makeatletter
    \def\maxwidth{\ifdim\Gin@nat@width>\linewidth\linewidth
    \else\Gin@nat@width\fi}
    \makeatother
    \let\Oldincludegraphics\includegraphics
    % Set max figure width to be 80% of text width, for now hardcoded.
    \renewcommand{\includegraphics}[1]{\Oldincludegraphics[width=.8\maxwidth]{#1}}
    % Ensure that by default, figures have no caption (until we provide a
    % proper Figure object with a Caption API and a way to capture that
    % in the conversion process - todo).
    \usepackage{caption}
    \DeclareCaptionLabelFormat{nolabel}{}
    \captionsetup{labelformat=nolabel}

    \usepackage{adjustbox} % Used to constrain images to a maximum size 
    \usepackage{xcolor} % Allow colors to be defined
    \usepackage{enumerate} % Needed for markdown enumerations to work
    \usepackage{geometry} % Used to adjust the document margins
    \usepackage{amsmath} % Equations
    \usepackage{amssymb} % Equations
    \usepackage{textcomp} % defines textquotesingle
    % Hack from http://tex.stackexchange.com/a/47451/13684:
    \AtBeginDocument{%
        \def\PYZsq{\textquotesingle}% Upright quotes in Pygmentized code
    }
    \usepackage{upquote} % Upright quotes for verbatim code
    \usepackage{eurosym} % defines \euro
    \usepackage[mathletters]{ucs} % Extended unicode (utf-8) support
    \usepackage[utf8x]{inputenc} % Allow utf-8 characters in the tex document
    \usepackage{fancyvrb} % verbatim replacement that allows latex
    \usepackage{grffile} % extends the file name processing of package graphics 
                         % to support a larger range 
    % The hyperref package gives us a pdf with properly built
    % internal navigation ('pdf bookmarks' for the table of contents,
    % internal cross-reference links, web links for URLs, etc.)
    \usepackage{hyperref}
    \usepackage{longtable} % longtable support required by pandoc >1.10
    \usepackage{booktabs}  % table support for pandoc > 1.12.2
    \usepackage[inline]{enumitem} % IRkernel/repr support (it uses the enumerate* environment)
    \usepackage[normalem]{ulem} % ulem is needed to support strikethroughs (\sout)
                                % normalem makes italics be italics, not underlines
    

    
    
    % Colors for the hyperref package
    \definecolor{urlcolor}{rgb}{0,.145,.698}
    \definecolor{linkcolor}{rgb}{.71,0.21,0.01}
    \definecolor{citecolor}{rgb}{.12,.54,.11}

    % ANSI colors
    \definecolor{ansi-black}{HTML}{3E424D}
    \definecolor{ansi-black-intense}{HTML}{282C36}
    \definecolor{ansi-red}{HTML}{E75C58}
    \definecolor{ansi-red-intense}{HTML}{B22B31}
    \definecolor{ansi-green}{HTML}{00A250}
    \definecolor{ansi-green-intense}{HTML}{007427}
    \definecolor{ansi-yellow}{HTML}{DDB62B}
    \definecolor{ansi-yellow-intense}{HTML}{B27D12}
    \definecolor{ansi-blue}{HTML}{208FFB}
    \definecolor{ansi-blue-intense}{HTML}{0065CA}
    \definecolor{ansi-magenta}{HTML}{D160C4}
    \definecolor{ansi-magenta-intense}{HTML}{A03196}
    \definecolor{ansi-cyan}{HTML}{60C6C8}
    \definecolor{ansi-cyan-intense}{HTML}{258F8F}
    \definecolor{ansi-white}{HTML}{C5C1B4}
    \definecolor{ansi-white-intense}{HTML}{A1A6B2}

    % commands and environments needed by pandoc snippets
    % extracted from the output of `pandoc -s`
    \providecommand{\tightlist}{%
      \setlength{\itemsep}{0pt}\setlength{\parskip}{0pt}}
    \DefineVerbatimEnvironment{Highlighting}{Verbatim}{commandchars=\\\{\}}
    % Add ',fontsize=\small' for more characters per line
    \newenvironment{Shaded}{}{}
    \newcommand{\KeywordTok}[1]{\textcolor[rgb]{0.00,0.44,0.13}{\textbf{{#1}}}}
    \newcommand{\DataTypeTok}[1]{\textcolor[rgb]{0.56,0.13,0.00}{{#1}}}
    \newcommand{\DecValTok}[1]{\textcolor[rgb]{0.25,0.63,0.44}{{#1}}}
    \newcommand{\BaseNTok}[1]{\textcolor[rgb]{0.25,0.63,0.44}{{#1}}}
    \newcommand{\FloatTok}[1]{\textcolor[rgb]{0.25,0.63,0.44}{{#1}}}
    \newcommand{\CharTok}[1]{\textcolor[rgb]{0.25,0.44,0.63}{{#1}}}
    \newcommand{\StringTok}[1]{\textcolor[rgb]{0.25,0.44,0.63}{{#1}}}
    \newcommand{\CommentTok}[1]{\textcolor[rgb]{0.38,0.63,0.69}{\textit{{#1}}}}
    \newcommand{\OtherTok}[1]{\textcolor[rgb]{0.00,0.44,0.13}{{#1}}}
    \newcommand{\AlertTok}[1]{\textcolor[rgb]{1.00,0.00,0.00}{\textbf{{#1}}}}
    \newcommand{\FunctionTok}[1]{\textcolor[rgb]{0.02,0.16,0.49}{{#1}}}
    \newcommand{\RegionMarkerTok}[1]{{#1}}
    \newcommand{\ErrorTok}[1]{\textcolor[rgb]{1.00,0.00,0.00}{\textbf{{#1}}}}
    \newcommand{\NormalTok}[1]{{#1}}
    
    % Additional commands for more recent versions of Pandoc
    \newcommand{\ConstantTok}[1]{\textcolor[rgb]{0.53,0.00,0.00}{{#1}}}
    \newcommand{\SpecialCharTok}[1]{\textcolor[rgb]{0.25,0.44,0.63}{{#1}}}
    \newcommand{\VerbatimStringTok}[1]{\textcolor[rgb]{0.25,0.44,0.63}{{#1}}}
    \newcommand{\SpecialStringTok}[1]{\textcolor[rgb]{0.73,0.40,0.53}{{#1}}}
    \newcommand{\ImportTok}[1]{{#1}}
    \newcommand{\DocumentationTok}[1]{\textcolor[rgb]{0.73,0.13,0.13}{\textit{{#1}}}}
    \newcommand{\AnnotationTok}[1]{\textcolor[rgb]{0.38,0.63,0.69}{\textbf{\textit{{#1}}}}}
    \newcommand{\CommentVarTok}[1]{\textcolor[rgb]{0.38,0.63,0.69}{\textbf{\textit{{#1}}}}}
    \newcommand{\VariableTok}[1]{\textcolor[rgb]{0.10,0.09,0.49}{{#1}}}
    \newcommand{\ControlFlowTok}[1]{\textcolor[rgb]{0.00,0.44,0.13}{\textbf{{#1}}}}
    \newcommand{\OperatorTok}[1]{\textcolor[rgb]{0.40,0.40,0.40}{{#1}}}
    \newcommand{\BuiltInTok}[1]{{#1}}
    \newcommand{\ExtensionTok}[1]{{#1}}
    \newcommand{\PreprocessorTok}[1]{\textcolor[rgb]{0.74,0.48,0.00}{{#1}}}
    \newcommand{\AttributeTok}[1]{\textcolor[rgb]{0.49,0.56,0.16}{{#1}}}
    \newcommand{\InformationTok}[1]{\textcolor[rgb]{0.38,0.63,0.69}{\textbf{\textit{{#1}}}}}
    \newcommand{\WarningTok}[1]{\textcolor[rgb]{0.38,0.63,0.69}{\textbf{\textit{{#1}}}}}
    
    
    % Define a nice break command that doesn't care if a line doesn't already
    % exist.
    \def\br{\hspace*{\fill} \\* }
    % Math Jax compatability definitions
    \def\gt{>}
    \def\lt{<}
    % Document parameters
    \title{1SN\_LangageC\_C1}
    
    
    

    % Pygments definitions
    
\makeatletter
\def\PY@reset{\let\PY@it=\relax \let\PY@bf=\relax%
    \let\PY@ul=\relax \let\PY@tc=\relax%
    \let\PY@bc=\relax \let\PY@ff=\relax}
\def\PY@tok#1{\csname PY@tok@#1\endcsname}
\def\PY@toks#1+{\ifx\relax#1\empty\else%
    \PY@tok{#1}\expandafter\PY@toks\fi}
\def\PY@do#1{\PY@bc{\PY@tc{\PY@ul{%
    \PY@it{\PY@bf{\PY@ff{#1}}}}}}}
\def\PY#1#2{\PY@reset\PY@toks#1+\relax+\PY@do{#2}}

\expandafter\def\csname PY@tok@w\endcsname{\def\PY@tc##1{\textcolor[rgb]{0.73,0.73,0.73}{##1}}}
\expandafter\def\csname PY@tok@c\endcsname{\let\PY@it=\textit\def\PY@tc##1{\textcolor[rgb]{0.25,0.50,0.50}{##1}}}
\expandafter\def\csname PY@tok@cp\endcsname{\def\PY@tc##1{\textcolor[rgb]{0.74,0.48,0.00}{##1}}}
\expandafter\def\csname PY@tok@k\endcsname{\let\PY@bf=\textbf\def\PY@tc##1{\textcolor[rgb]{0.00,0.50,0.00}{##1}}}
\expandafter\def\csname PY@tok@kp\endcsname{\def\PY@tc##1{\textcolor[rgb]{0.00,0.50,0.00}{##1}}}
\expandafter\def\csname PY@tok@kt\endcsname{\def\PY@tc##1{\textcolor[rgb]{0.69,0.00,0.25}{##1}}}
\expandafter\def\csname PY@tok@o\endcsname{\def\PY@tc##1{\textcolor[rgb]{0.40,0.40,0.40}{##1}}}
\expandafter\def\csname PY@tok@ow\endcsname{\let\PY@bf=\textbf\def\PY@tc##1{\textcolor[rgb]{0.67,0.13,1.00}{##1}}}
\expandafter\def\csname PY@tok@nb\endcsname{\def\PY@tc##1{\textcolor[rgb]{0.00,0.50,0.00}{##1}}}
\expandafter\def\csname PY@tok@nf\endcsname{\def\PY@tc##1{\textcolor[rgb]{0.00,0.00,1.00}{##1}}}
\expandafter\def\csname PY@tok@nc\endcsname{\let\PY@bf=\textbf\def\PY@tc##1{\textcolor[rgb]{0.00,0.00,1.00}{##1}}}
\expandafter\def\csname PY@tok@nn\endcsname{\let\PY@bf=\textbf\def\PY@tc##1{\textcolor[rgb]{0.00,0.00,1.00}{##1}}}
\expandafter\def\csname PY@tok@ne\endcsname{\let\PY@bf=\textbf\def\PY@tc##1{\textcolor[rgb]{0.82,0.25,0.23}{##1}}}
\expandafter\def\csname PY@tok@nv\endcsname{\def\PY@tc##1{\textcolor[rgb]{0.10,0.09,0.49}{##1}}}
\expandafter\def\csname PY@tok@no\endcsname{\def\PY@tc##1{\textcolor[rgb]{0.53,0.00,0.00}{##1}}}
\expandafter\def\csname PY@tok@nl\endcsname{\def\PY@tc##1{\textcolor[rgb]{0.63,0.63,0.00}{##1}}}
\expandafter\def\csname PY@tok@ni\endcsname{\let\PY@bf=\textbf\def\PY@tc##1{\textcolor[rgb]{0.60,0.60,0.60}{##1}}}
\expandafter\def\csname PY@tok@na\endcsname{\def\PY@tc##1{\textcolor[rgb]{0.49,0.56,0.16}{##1}}}
\expandafter\def\csname PY@tok@nt\endcsname{\let\PY@bf=\textbf\def\PY@tc##1{\textcolor[rgb]{0.00,0.50,0.00}{##1}}}
\expandafter\def\csname PY@tok@nd\endcsname{\def\PY@tc##1{\textcolor[rgb]{0.67,0.13,1.00}{##1}}}
\expandafter\def\csname PY@tok@s\endcsname{\def\PY@tc##1{\textcolor[rgb]{0.73,0.13,0.13}{##1}}}
\expandafter\def\csname PY@tok@sd\endcsname{\let\PY@it=\textit\def\PY@tc##1{\textcolor[rgb]{0.73,0.13,0.13}{##1}}}
\expandafter\def\csname PY@tok@si\endcsname{\let\PY@bf=\textbf\def\PY@tc##1{\textcolor[rgb]{0.73,0.40,0.53}{##1}}}
\expandafter\def\csname PY@tok@se\endcsname{\let\PY@bf=\textbf\def\PY@tc##1{\textcolor[rgb]{0.73,0.40,0.13}{##1}}}
\expandafter\def\csname PY@tok@sr\endcsname{\def\PY@tc##1{\textcolor[rgb]{0.73,0.40,0.53}{##1}}}
\expandafter\def\csname PY@tok@ss\endcsname{\def\PY@tc##1{\textcolor[rgb]{0.10,0.09,0.49}{##1}}}
\expandafter\def\csname PY@tok@sx\endcsname{\def\PY@tc##1{\textcolor[rgb]{0.00,0.50,0.00}{##1}}}
\expandafter\def\csname PY@tok@m\endcsname{\def\PY@tc##1{\textcolor[rgb]{0.40,0.40,0.40}{##1}}}
\expandafter\def\csname PY@tok@gh\endcsname{\let\PY@bf=\textbf\def\PY@tc##1{\textcolor[rgb]{0.00,0.00,0.50}{##1}}}
\expandafter\def\csname PY@tok@gu\endcsname{\let\PY@bf=\textbf\def\PY@tc##1{\textcolor[rgb]{0.50,0.00,0.50}{##1}}}
\expandafter\def\csname PY@tok@gd\endcsname{\def\PY@tc##1{\textcolor[rgb]{0.63,0.00,0.00}{##1}}}
\expandafter\def\csname PY@tok@gi\endcsname{\def\PY@tc##1{\textcolor[rgb]{0.00,0.63,0.00}{##1}}}
\expandafter\def\csname PY@tok@gr\endcsname{\def\PY@tc##1{\textcolor[rgb]{1.00,0.00,0.00}{##1}}}
\expandafter\def\csname PY@tok@ge\endcsname{\let\PY@it=\textit}
\expandafter\def\csname PY@tok@gs\endcsname{\let\PY@bf=\textbf}
\expandafter\def\csname PY@tok@gp\endcsname{\let\PY@bf=\textbf\def\PY@tc##1{\textcolor[rgb]{0.00,0.00,0.50}{##1}}}
\expandafter\def\csname PY@tok@go\endcsname{\def\PY@tc##1{\textcolor[rgb]{0.53,0.53,0.53}{##1}}}
\expandafter\def\csname PY@tok@gt\endcsname{\def\PY@tc##1{\textcolor[rgb]{0.00,0.27,0.87}{##1}}}
\expandafter\def\csname PY@tok@err\endcsname{\def\PY@bc##1{\setlength{\fboxsep}{0pt}\fcolorbox[rgb]{1.00,0.00,0.00}{1,1,1}{\strut ##1}}}
\expandafter\def\csname PY@tok@kc\endcsname{\let\PY@bf=\textbf\def\PY@tc##1{\textcolor[rgb]{0.00,0.50,0.00}{##1}}}
\expandafter\def\csname PY@tok@kd\endcsname{\let\PY@bf=\textbf\def\PY@tc##1{\textcolor[rgb]{0.00,0.50,0.00}{##1}}}
\expandafter\def\csname PY@tok@kn\endcsname{\let\PY@bf=\textbf\def\PY@tc##1{\textcolor[rgb]{0.00,0.50,0.00}{##1}}}
\expandafter\def\csname PY@tok@kr\endcsname{\let\PY@bf=\textbf\def\PY@tc##1{\textcolor[rgb]{0.00,0.50,0.00}{##1}}}
\expandafter\def\csname PY@tok@bp\endcsname{\def\PY@tc##1{\textcolor[rgb]{0.00,0.50,0.00}{##1}}}
\expandafter\def\csname PY@tok@fm\endcsname{\def\PY@tc##1{\textcolor[rgb]{0.00,0.00,1.00}{##1}}}
\expandafter\def\csname PY@tok@vc\endcsname{\def\PY@tc##1{\textcolor[rgb]{0.10,0.09,0.49}{##1}}}
\expandafter\def\csname PY@tok@vg\endcsname{\def\PY@tc##1{\textcolor[rgb]{0.10,0.09,0.49}{##1}}}
\expandafter\def\csname PY@tok@vi\endcsname{\def\PY@tc##1{\textcolor[rgb]{0.10,0.09,0.49}{##1}}}
\expandafter\def\csname PY@tok@vm\endcsname{\def\PY@tc##1{\textcolor[rgb]{0.10,0.09,0.49}{##1}}}
\expandafter\def\csname PY@tok@sa\endcsname{\def\PY@tc##1{\textcolor[rgb]{0.73,0.13,0.13}{##1}}}
\expandafter\def\csname PY@tok@sb\endcsname{\def\PY@tc##1{\textcolor[rgb]{0.73,0.13,0.13}{##1}}}
\expandafter\def\csname PY@tok@sc\endcsname{\def\PY@tc##1{\textcolor[rgb]{0.73,0.13,0.13}{##1}}}
\expandafter\def\csname PY@tok@dl\endcsname{\def\PY@tc##1{\textcolor[rgb]{0.73,0.13,0.13}{##1}}}
\expandafter\def\csname PY@tok@s2\endcsname{\def\PY@tc##1{\textcolor[rgb]{0.73,0.13,0.13}{##1}}}
\expandafter\def\csname PY@tok@sh\endcsname{\def\PY@tc##1{\textcolor[rgb]{0.73,0.13,0.13}{##1}}}
\expandafter\def\csname PY@tok@s1\endcsname{\def\PY@tc##1{\textcolor[rgb]{0.73,0.13,0.13}{##1}}}
\expandafter\def\csname PY@tok@mb\endcsname{\def\PY@tc##1{\textcolor[rgb]{0.40,0.40,0.40}{##1}}}
\expandafter\def\csname PY@tok@mf\endcsname{\def\PY@tc##1{\textcolor[rgb]{0.40,0.40,0.40}{##1}}}
\expandafter\def\csname PY@tok@mh\endcsname{\def\PY@tc##1{\textcolor[rgb]{0.40,0.40,0.40}{##1}}}
\expandafter\def\csname PY@tok@mi\endcsname{\def\PY@tc##1{\textcolor[rgb]{0.40,0.40,0.40}{##1}}}
\expandafter\def\csname PY@tok@il\endcsname{\def\PY@tc##1{\textcolor[rgb]{0.40,0.40,0.40}{##1}}}
\expandafter\def\csname PY@tok@mo\endcsname{\def\PY@tc##1{\textcolor[rgb]{0.40,0.40,0.40}{##1}}}
\expandafter\def\csname PY@tok@ch\endcsname{\let\PY@it=\textit\def\PY@tc##1{\textcolor[rgb]{0.25,0.50,0.50}{##1}}}
\expandafter\def\csname PY@tok@cm\endcsname{\let\PY@it=\textit\def\PY@tc##1{\textcolor[rgb]{0.25,0.50,0.50}{##1}}}
\expandafter\def\csname PY@tok@cpf\endcsname{\let\PY@it=\textit\def\PY@tc##1{\textcolor[rgb]{0.25,0.50,0.50}{##1}}}
\expandafter\def\csname PY@tok@c1\endcsname{\let\PY@it=\textit\def\PY@tc##1{\textcolor[rgb]{0.25,0.50,0.50}{##1}}}
\expandafter\def\csname PY@tok@cs\endcsname{\let\PY@it=\textit\def\PY@tc##1{\textcolor[rgb]{0.25,0.50,0.50}{##1}}}

\def\PYZbs{\char`\\}
\def\PYZus{\char`\_}
\def\PYZob{\char`\{}
\def\PYZcb{\char`\}}
\def\PYZca{\char`\^}
\def\PYZam{\char`\&}
\def\PYZlt{\char`\<}
\def\PYZgt{\char`\>}
\def\PYZsh{\char`\#}
\def\PYZpc{\char`\%}
\def\PYZdl{\char`\$}
\def\PYZhy{\char`\-}
\def\PYZsq{\char`\'}
\def\PYZdq{\char`\"}
\def\PYZti{\char`\~}
% for compatibility with earlier versions
\def\PYZat{@}
\def\PYZlb{[}
\def\PYZrb{]}
\makeatother


    % Exact colors from NB
    \definecolor{incolor}{rgb}{0.0, 0.0, 0.5}
    \definecolor{outcolor}{rgb}{0.545, 0.0, 0.0}



    
    % Prevent overflowing lines due to hard-to-break entities
    \sloppy 
    % Setup hyperref package
    \hypersetup{
      breaklinks=true,  % so long urls are correctly broken across lines
      colorlinks=true,
      urlcolor=urlcolor,
      linkcolor=linkcolor,
      citecolor=citecolor,
      }
    % Slightly bigger margins than the latex defaults
    
    \geometry{verbose,tmargin=1in,bmargin=1in,lmargin=1in,rmargin=1in}
    
    

    \begin{document}
    
    
    \maketitle
    
    

    
    \section{Langage C - Notebook C1.}\label{langage-c---notebook-c1.}

\paragraph{Katia Jaffrès-Runser, Xavier
Crégut}\label{katia-jaffruxe8s-runser-xavier-cruxe9gut}

Toulouse INP - ENSEEIHT,

1ère année, Dept. Sciences du Numérique, 2020-2021.

    \subsection{\#\#\#\#\# Déroulement du
cours}\label{duxe9roulement-du-cours}

Ce cours se déroule sur 6 séances de TP.

\begin{itemize}
\item
  Lors des trois premières séances, vous suivrez ce notebook Jupyter C1
  présentant des éléments de cours associés à un ensemble d'exercices à
  réaliser. Ce travail se poursuit hors séance, en autonomie, si besoin.
  Le notebook est archivé sous SVN mais ne sera pas noté.
\item
  A la fin de ce notebook, dans la partie Bilan 2, sont listés un
  ensemble d'exercices à rendre sur SVN. Ces rendus donneront lieu à une
  note.\\
  \textgreater{}\textbf{Attention : Ce travail est individuel}. Des
  outils de détection de recopie de code seront utilisés pour détecter
  la fraude.
\item
  Lors des trois séances suivantes, vous suivrez un d'autres notebook
  Jupyter, qui seront également validés par le rendu d'exercices de
  Bilan.
\end{itemize}

Vous aurez, en fin de cours, un QCM d'une heure. La note finale est une
moyenne des deux notes (QCM et exercices rendus).

    \subsection{Objectifs}\label{objectifs}

Ce cours, sous la forme de notebooks Jupyter et d'un ensemble
d'exercices à réaliser en TP, a pour objectif de vous présenter les
spécificités de la programmation en langage C. Il se base sur vos acquis
du cours de Programmation Impérative en algorithmique et vous détaille
les éléments du langage C nécessaires à la production d'un programme en
C.

Un support de cours PDF vous est également fournit sur Moodle :
\href{http://moodle-n7.inp-toulouse.fr/pluginfile.php/49240/mod_resource/content/5/LangageC_poly.pdf}{Cours
C}.

    \subsection{\#\# Plan du sujet C1.}\label{plan-du-sujet-c1.}

Les éléments suivants de la programmation en Langage C sont présentés
dans ce notebook au cours des 3 premières séances de TP : - La structure
d'un programme et sa compilation - Les constantes, types et variables -
Les entrées / sorties - Les structures de contrôle - Conditionnelles -
Boucles - Les types énumération, enregistrement et tableaux - Les
chaînes de caractère - Le type pointeur - Les sous-programmes en C -
Leur signature - Passage par valeur - Passage par adresse

    \subsection{Jupyter notebook}\label{jupyter-notebook}

Le support de cours que vous lisez est un notebook Jupyter. Pour
visualiser le notebook, lancer l'editeur web avec la commande\\
\textgreater{} \texttt{jupyter-notebok}

et rechercher le fichier dans l'arborescence. Le fichier est édité dans
votre navigateur Web par défaut. L'enregistrement est automatique
(\texttt{CTRL\ S} pour le forcer).

Pour fermer votre fichier, il faut fermer le navigateur et terminer le
processus serveur qui s'exécute dans le terminal (\texttt{CTRL\ C}, puis
\texttt{y}).

\begin{quote}
\textbf{Important} : - Pour faire fonctionner le kernel C de jupyter
notebook, il faut, avant votre \textbf{première utilisation} de ce
Notebook, lancer la commande suivante dans un \texttt{Terminal} : -
\texttt{install\_c\_kernel\ -\/-user}
\end{quote}

Il se compose de cellules présentant soit : - Des éléments de cours, au
format \href{https://fr.wikipedia.org/wiki/Markdown}{Markdown}. Ce
langage est interprété pour un affichage aisé quand on clique sur la
flèche \texttt{Exécuter} et que la cellule est active. - Du code en
Langage C (ou Python, ou autre..). Pour compiler et exécuter le code
écrit dans la cellule active, on clique sur la flèche \texttt{Exécuter}.
Si la compilation se déroule sans erreur ni avertissement, le programme
est exécuté et les sorties sont affichées en bas de la cellule. Si ce
n'est pas le cas, les avertissements et warnings sont affichés en bas de
la cellule.

En double-cliquant sur une cellule, on peut éditer son contenu. Vous
pouvez ainsi : - Editer une cellule markdown pour y intégrer vos propres
notes. - Modifier les programmes pour répondre aux questions et
exercices proposés.

Il est possible d'exporter votre travail en PDF, HTML, etc.

Le programme \textbf{\texttt{premier.c}} dans la cellule suivante
s'exécute sans erreur. Vous pouvez - le tester en l'exécutant. - y
introduire une erreur (suppression d'un point-virgule par exemple) pour
observer la sortie du compilateur.

    \begin{Verbatim}[commandchars=\\\{\}]
{\color{incolor}In [{\color{incolor}1}]:} \PY{c+cp}{\PYZsh{}}\PY{c+cp}{include} \PY{c+cpf}{\PYZlt{}stdlib.h\PYZgt{}}\PY{c+c1}{ }
        \PY{c+cp}{\PYZsh{}}\PY{c+cp}{include} \PY{c+cpf}{\PYZlt{}stdio.h\PYZgt{}}
        
        \PY{k+kt}{int} \PY{n+nf}{main}\PY{p}{(}\PY{p}{)}\PY{p}{\PYZob{}}
            \PY{n}{printf}\PY{p}{(}\PY{l+s}{\PYZdq{}}\PY{l+s}{******************************}\PY{l+s+se}{\PYZbs{}n}\PY{l+s}{\PYZdq{}}\PY{p}{)}\PY{p}{;}
            \PY{n}{printf}\PY{p}{(}\PY{l+s}{\PYZdq{}}\PY{l+s}{******** Langage C ***********}\PY{l+s+se}{\PYZbs{}n}\PY{l+s}{\PYZdq{}}\PY{p}{)}\PY{p}{;}
            \PY{n}{printf}\PY{p}{(}\PY{l+s}{\PYZdq{}}\PY{l+s}{******************************}\PY{l+s+se}{\PYZbs{}n}\PY{l+s}{\PYZdq{}}\PY{p}{)}\PY{p}{;}
            \PY{k}{return} \PY{n}{EXIT\PYZus{}SUCCESS}\PY{p}{;}
        \PY{p}{\PYZcb{}}
\end{Verbatim}


    \begin{Verbatim}[commandchars=\\\{\}]
******************************
******** Langage C ***********
******************************

    \end{Verbatim}

    \subsection{Un premier programme en Langage
C}\label{un-premier-programme-en-langage-c}

Le fichier \textbf{\texttt{pgcd.c}} suivant comporte un programme en
Langagce C. Exécutez-le.

    \begin{Verbatim}[commandchars=\\\{\}]
{\color{incolor}In [{\color{incolor}3}]:} \PY{c+cp}{\PYZsh{}}\PY{c+cp}{include} \PY{c+cpf}{\PYZlt{}stdio.h\PYZgt{}}
        \PY{c+cp}{\PYZsh{}}\PY{c+cp}{include} \PY{c+cpf}{\PYZlt{}stdlib.h\PYZgt{}}
        \PY{c+cp}{\PYZsh{}}\PY{c+cp}{include} \PY{c+cpf}{\PYZlt{}assert.h\PYZgt{}}
        
        \PY{c+cm}{/* Afficher le pgcd de deux entiers strictement positifs. */}
        \PY{k+kt}{int} \PY{n+nf}{main}\PY{p}{(}\PY{p}{)} \PY{p}{\PYZob{}}
            \PY{c+c1}{// Déclaration et initialisation de deux entiers}
            \PY{k+kt}{int} \PY{n}{a} \PY{o}{=} \PY{l+m+mi}{105}\PY{p}{,} \PY{n}{b} \PY{o}{=} \PY{l+m+mi}{35}\PY{p}{;}  
            
            \PY{c+c1}{// Déterminer le pgcd de a et b}
            \PY{k+kt}{int} \PY{n}{na} \PY{o}{=} \PY{n}{a}\PY{p}{,} \PY{n}{nb} \PY{o}{=} \PY{n}{b}\PY{p}{;}  \PY{c+c1}{// gain de place ! À éviter !}
            \PY{k}{while} \PY{p}{(}\PY{n}{na} \PY{o}{!}\PY{o}{=} \PY{n}{nb}\PY{p}{)} \PY{p}{\PYZob{}}	\PY{c+c1}{// na et nb différents}
                \PY{c+c1}{// Soustraire au plus grand le plus petit}
                \PY{k}{if} \PY{p}{(}\PY{n}{na} \PY{o}{\PYZgt{}} \PY{n}{nb}\PY{p}{)} \PY{p}{\PYZob{}}
                    \PY{n}{na} \PY{o}{=} \PY{n}{na} \PY{o}{\PYZhy{}} \PY{n}{nb}\PY{p}{;}
                \PY{p}{\PYZcb{}} \PY{k}{else}  \PY{p}{\PYZob{}}
                    \PY{n}{nb} \PY{o}{=} \PY{n}{nb} \PY{o}{\PYZhy{}} \PY{n}{na}\PY{p}{;}
                \PY{p}{\PYZcb{}}
            \PY{p}{\PYZcb{}}
            \PY{k+kt}{int} \PY{n}{pgcd} \PY{o}{=} \PY{n}{na}\PY{p}{;}     \PY{c+c1}{// le pgcd de a et b}
            
            \PY{c+c1}{// Afficher le pgcd}
            \PY{n}{printf}\PY{p}{(}\PY{l+s}{\PYZdq{}}\PY{l+s}{Le pgcd de \PYZpc{}d et \PYZpc{}d est \PYZpc{}d}\PY{l+s+se}{\PYZbs{}n}\PY{l+s}{\PYZdq{}}\PY{p}{,} \PY{n}{a}\PY{p}{,} \PY{n}{b} \PY{p}{,} \PY{n}{pgcd}\PY{p}{)}\PY{p}{;}
            \PY{k}{return} \PY{n}{EXIT\PYZus{}SUCCESS}\PY{p}{;}
        \PY{p}{\PYZcb{}}
\end{Verbatim}


    \begin{Verbatim}[commandchars=\\\{\}]
Le pgcd de 105 et 35 est 35

    \end{Verbatim}

    Ce programme se compose de : - Trois commandes \textbf{pré-processeur}
\texttt{\#include}. - Toutes les commandes pré-processeur commencent par
le caractère \#. - Ces deux commandes importent des librairies (i.e. des
modules). - La fonction \texttt{int\ main()}, qui correspond au
programme principal. Ses instructions sont définies entre accolades.
\textgreater{} \textbf{Règle} : L'identificateur du programme principal
est forcément \texttt{main()}. - Un ensemble d'instructions entre les
accolades. \textgreater{} \textbf{Règle} : Chaque instruction se termine
avec un \textbf{point-virgule}. - Un appel au sous-programme d'affichage
à l'écran \texttt{printf} du module \texttt{stdio}. A l'exécution, on
observe que les valeurs des variables \texttt{a}, \texttt{b} et
\texttt{pgcd} sont écrites en lieu et place des \texttt{\%d}, dans
l'ordre de leurs appels. - Le retour d'une constante
\texttt{EXIT\_SUCCESS} définie dans le module \texttt{stdlib}. Cette
constante vaut 0 et indique que l'exécution s'est terminée avec succès.
Il existe aussi \texttt{EXIT\_FAILURE} qui indique la mauvaise
terminaison du programme. \textgreater{} \textbf{Règle} : L'instruction
\texttt{return} arrête et indique le résultat de la fonction.\\
- Une boucle TantQue avec la structure de contrôle \texttt{while}. Les
instruction de corps de la boucle sont définies entre accolades. - Une
conditionelle \texttt{if\ (condition)\ then\ \{..\}\ else\ \{..\}} - Des
déclarations de variables, des opérations d'initialisation et
d'affectations. \textgreater{} \textbf{Règle} : L'opérateur
d'affectation est \texttt{=}. - Des tests. \textgreater{} \textbf{Règle}
: L'opérateur de test d'égalité est \texttt{==}, et d'inégalité est
\texttt{!=}.

    \subsection{\#\# Compilation et
pré-processeur}\label{compilation-et-pruxe9-processeur}

La compilation en C se décompose en deux étapes successives : 1.
L'exécution du pré-processeur, 2. L'exécution du compilateur C.

Les deux étapes sont réalisées par un seul appel à la suite de
compilation avec la commande : \textgreater{}
\texttt{gcc\ -Wall\ premier\_programme.c\ -o\ premier\_programme}

Les options permettent : - \texttt{Wall} : d'afficher l'ensemble des
avertissements produits par la compilation - \texttt{-o} : de choisir le
nom de l'exécutable généré. \textgreater{} \emph{Note} : dans le cas
particulier où on utiliser la bibliothèque \texttt{math.h}, il faut
rajouter \texttt{-lm} à la commande de compilation.

Le pré-processeur fournit un unique fichier au compilateur, qui le
transforme en un fichier binaire exécutable. Ce pré-processeur : -
Supprime les commentaires de ligne \texttt{//} ou de bloc
\texttt{/*\ \ \ \ */}. - Interprête les commandes pré-processeur qui
commencent par \texttt{\#} (\texttt{\#define}, \texttt{\#include}, etc.)

\begin{quote}
\textbf{Règle :} Il n'y a pas de point-virgule à la fin d'une
instruction pré-processeur.
\end{quote}

    \subsection{\#\#\#\# Exercice 1 -\/-
Compilation.}\label{exercice-1----compilation.}

\begin{center}\rule{0.5\linewidth}{\linethickness}\end{center}

\textbf{{[}1.1{]}} Compiler votre premier programme dans un terminal.
Pour se faire, créer un répertoire \texttt{Langage\_C} et y ajouter un
fichier nommé \texttt{pgcd.c}. Recopier le programme de l'exemple
précédent. Le compiler avec le compilateur \texttt{gcc}et l'exécuter
avec la commande \texttt{./pgcd}

\textbf{{[}1.2{]}} Introduire une erreur dans les instructions et
observer le retour du compilateur :

\begin{itemize}
\tightlist
\item
  Suppression d'un point-virgule en fin de ligne,
\item
  Ajout d'un point-virgule supplémentaire en fin de ligne,
\item
  Supprimer la déclaration de la variable \texttt{a}.
\item
  Supprimer l'accolade de fin de bloc de la boucle \texttt{while}.
\end{itemize}

\textbf{{[}1.3{]}} Observer l'unique fichier généré par le
pré-processeur avec l'appel à la commande
\texttt{cpp\ -P\ premier\_programme.c}. Quel est l'effet de la commande
\texttt{\#include\ \textless{}stdio.h\textgreater{}} ?

    \subsection{\texorpdfstring{\#\#\#\# Exercice 2 -\/- Comprendre la macro
\texttt{assert()}.}{\#\#\#\# Exercice 2 -\/- Comprendre la macro assert().}}\label{exercice-2----comprendre-la-macro-assert.}

\begin{center}\rule{0.5\linewidth}{\linethickness}\end{center}

Voyons comment fonctionne la macro \texttt{assert} du langage C. Nous
nous appuyons sur le programme \texttt{assert-comprendre.c}.

\textbf{{[}2.1{]}} Compiler et exécuter dans Jupyter Notebook ce
programme (fichier \textbf{\texttt{macro\_assert.c}}). Qu'observez-vous
?

    \begin{Verbatim}[commandchars=\\\{\}]
{\color{incolor}In [{\color{incolor}4}]:} \PY{c+cp}{\PYZsh{}}\PY{c+cp}{include} \PY{c+cpf}{\PYZlt{}stdlib.h\PYZgt{}}
        \PY{c+cp}{\PYZsh{}}\PY{c+cp}{include} \PY{c+cpf}{\PYZlt{}stdio.h\PYZgt{}}
        \PY{c+cp}{\PYZsh{}}\PY{c+cp}{include} \PY{c+cpf}{\PYZlt{}assert.h\PYZgt{}}
        
        \PY{k+kt}{void} \PY{n+nf}{assert\PYZus{}ok}\PY{p}{(}\PY{p}{)} \PY{p}{\PYZob{}}
            \PY{k+kt}{int} \PY{n}{n} \PY{o}{=} \PY{l+m+mi}{10}\PY{p}{;}
            \PY{n}{assert}\PY{p}{(}\PY{n}{n} \PY{o}{\PYZgt{}} \PY{l+m+mi}{0}\PY{p}{)}\PY{p}{;}
            \PY{n}{printf}\PY{p}{(}\PY{l+s}{\PYZdq{}}\PY{l+s}{(assert\PYZus{}ok) n = \PYZpc{}d}\PY{l+s+se}{\PYZbs{}n}\PY{l+s}{\PYZdq{}}\PY{p}{,} \PY{n}{n}\PY{p}{)}\PY{p}{;}
        \PY{p}{\PYZcb{}}
        
        
        \PY{k+kt}{void} \PY{n+nf}{assert\PYZus{}erreur}\PY{p}{(}\PY{p}{)} \PY{p}{\PYZob{}}
            \PY{k+kt}{int} \PY{n}{n} \PY{o}{=} \PY{l+m+mi}{10}\PY{p}{;}
            \PY{n}{assert}\PY{p}{(}\PY{n}{n} \PY{o}{\PYZlt{}}\PY{o}{=} \PY{l+m+mi}{0}\PY{p}{)}\PY{p}{;}
            \PY{n}{printf}\PY{p}{(}\PY{l+s}{\PYZdq{}}\PY{l+s}{(assert\PYZus{}erreur) n = \PYZpc{}d}\PY{l+s+se}{\PYZbs{}n}\PY{l+s}{\PYZdq{}}\PY{p}{,} \PY{n}{n}\PY{p}{)}\PY{p}{;}
        \PY{p}{\PYZcb{}}
        
        
        \PY{k+kt}{int} \PY{n+nf}{main}\PY{p}{(}\PY{k+kt}{void}\PY{p}{)} \PY{p}{\PYZob{}}
            \PY{n}{assert\PYZus{}ok}\PY{p}{(}\PY{p}{)}\PY{p}{;}
            \PY{n}{assert\PYZus{}erreur}\PY{p}{(}\PY{p}{)}\PY{p}{;}
            \PY{k}{return} \PY{n}{EXIT\PYZus{}SUCCESS}\PY{p}{;}
        \PY{p}{\PYZcb{}}
\end{Verbatim}


    \begin{Verbatim}[commandchars=\\\{\}]
(assert\_ok) n = 10

    \end{Verbatim}

    \begin{Verbatim}[commandchars=\\\{\}]
tmp5e9b452w.out: /tmp/tmptvagws03.c:14: assert\_erreur: Assertion `n <= 0' failed.
[C kernel] Executable exited with code -6
    \end{Verbatim}

    \section{\texorpdfstring{L'appel à \texttt{assert\_ok} se déroule
normalement car le paramètre effectif de \texttt{assert} s'évalue à
vrai.}{L'appel à assert\_ok se déroule normalement car le paramètre effectif de assert s'évalue à vrai.}}\label{lappel-uxe0-assert_ok-se-duxe9roule-normalement-car-le-paramuxe8tre-effectif-de-assert-suxe9value-uxe0-vrai.}

Au contraire, l'appel à \texttt{assert\_erreur} provoque l'arrêt du
programme car le paramètre effectif de assert s'évalue à faux. Un
message d'erreur indique la ligne dans le fichier source contenant
l'appel à \texttt{assert}.

\textbf{{[}2.2{]}} Compiler et exécuter le programme dans un terminal,
dans le répertoire SVN fourni, avec les commandes :

\begin{Shaded}
\begin{Highlighting}[]
\NormalTok{make assert-comprendre}
\NormalTok{./assert-comprendre}
\end{Highlighting}
\end{Shaded}

\begin{quote}
\emph{Note} : La commande \texttt{make} sera présentée à la fin du
cours. Elle permet d'automatiser la compilation. Elle est paramétrée par
le fichier \texttt{Makefile}. Vous pouvez le consulter mais sa
compréhension n'est pas l'objet de cette question.
\end{quote}

Qu'observez-vous ?

\textbf{{[}2.3{]}} L'évaluation des assert peut être désactivée en
définissant la macro \texttt{NDEBUG} (no debug). Par exemple, en début
du fichier \texttt{assert-comprendre.c} (mais avant l'inclusion de
\texttt{assert.h}), on peut ajouter la commande préprocesseur suivante
qui définit NDEBUG :

\begin{Shaded}
\begin{Highlighting}[]
\PreprocessorTok{#define NDEBUG}
\end{Highlighting}
\end{Shaded}

Modifier le fichier \texttt{assert-comprendre.c}, compiler et exécuter à
nouveau pour constater que les assert ne sont plus vérifiés. Vous pouvez
aussi le tester sur le notebook Jupyter.

\begin{quote}
\emph{Note} : En général, on positionne \texttt{NDEBUG} à la
compilation, sans l'écrire dans le fichier C, en utilisant l'option
\texttt{-D} du compilateur (\texttt{-DNDEBUG}) :
\end{quote}

\begin{verbatim}
gcc -Wall -pedantic -DNDEBUG assert-comprendre.c -o assert-comprendre
\end{verbatim}

On peut aussi ajouter \texttt{-DNDEBUG} à la définition de
\texttt{CFLAGS} dans le fichier \texttt{Makefile}.

    \begin{center}\rule{0.5\linewidth}{\linethickness}\end{center}

    \subsection{\#\# Types}\label{types}

    Plusieurs types fondamentaux sont définis en C : - Des types discrets :
\texttt{int} (entier), \texttt{bool}(bouléen), \texttt{char} (caractère)
- Des types réels : \texttt{float} et \texttt{double}, à simple et
double précision. \textgreater{} Note : \textgreater{} Il faut inclure
le module \texttt{stdbool} pour utiliser le type booléen, et ses valeur
\texttt{true} et \texttt{false}.

\paragraph{Exemples}\label{exemples}

\begin{Shaded}
\begin{Highlighting}[]
\DataTypeTok{int}\NormalTok{ entier_1 = }\DecValTok{20}\NormalTok{;}
\NormalTok{bool est_vide = false;}
\DataTypeTok{char}\NormalTok{ initiale = }\CharTok{'B'}\NormalTok{; }\CommentTok{//Caractère constant 'B' entre guillemets simples/ }
\end{Highlighting}
\end{Shaded}

\subsubsection{Modificateurs de type}\label{modificateurs-de-type}

Il existe aussi des modificateurs de type : \texttt{long},
\texttt{short}, \texttt{unsigned}. Ils sont utilisés pour modifier
certains types fondamentaux \texttt{int}, \texttt{double},
\texttt{float}.

La taille en mémoire d'une variable entière de type \texttt{short\ int}
est inférieure à la taille mémoire d'une variable de type \texttt{int},
qui elle même est inférieure à une variable de taille
\texttt{long\ int}.

Le modificateur \texttt{unsigned} définit un type à valeurs positives ou
nulles.

    \subsection{\#\#\#\# Exercice 3 -\/- Valeurs maximales et conversion
implicites}\label{exercice-3----valeurs-maximales-et-conversion-implicites}

\begin{center}\rule{0.5\linewidth}{\linethickness}\end{center}

La valeur maximale des types dépend du système d'exploitation. Elles
sont enregistrées dans les bibliothèques \texttt{limits.h} pour les
entiers et \texttt{float.h} pour les flottants.

\textbf{{[}3.1{]}} Exécuter l'exemple du fichier
\_\_\texttt{conversions.c}\_\_suivant pour les observer.

    \begin{Verbatim}[commandchars=\\\{\}]
{\color{incolor}In [{\color{incolor}17}]:} \PY{c+cp}{\PYZsh{}}\PY{c+cp}{include} \PY{c+cpf}{\PYZlt{}stdio.h\PYZgt{}}
         \PY{c+cp}{\PYZsh{}}\PY{c+cp}{include} \PY{c+cpf}{\PYZlt{}stdlib.h\PYZgt{}}
         \PY{c+c1}{// Liste les valeurs maximales des entiers pour votre système}
         \PY{c+cp}{\PYZsh{}}\PY{c+cp}{include} \PY{c+cpf}{\PYZlt{}limits.h\PYZgt{}}\PY{c+c1}{ }
         \PY{c+c1}{// Liste les valeurs maximales des flottants pour votre système}
         \PY{c+cp}{\PYZsh{}}\PY{c+cp}{include} \PY{c+cpf}{\PYZlt{}float.h\PYZgt{}}\PY{c+c1}{ }
         
         \PY{k+kt}{int} \PY{n+nf}{main}\PY{p}{(}\PY{p}{)}\PY{p}{\PYZob{}}
             \PY{n}{printf}\PY{p}{(}\PY{l+s}{\PYZdq{}}\PY{l+s}{Valeur maximale d\PYZsq{}un entier \PYZpc{}d }\PY{l+s+se}{\PYZbs{}n}\PY{l+s}{\PYZdq{}}\PY{p}{,} \PY{n}{INT\PYZus{}MAX}\PY{p}{)}\PY{p}{;} 
             \PY{k+kt}{long} \PY{k+kt}{int} \PY{n}{entier\PYZus{}long} \PY{o}{=} \PY{l+m+mi}{\PYZhy{}20000000}\PY{p}{;} \PY{c+c1}{// Déclaration d\PYZsq{}un entier long}
             \PY{n}{printf}\PY{p}{(}\PY{l+s}{\PYZdq{}}\PY{l+s}{Valeur maximale d\PYZsq{}un entier long \PYZpc{}ld \PYZgt{} \PYZpc{}ld }\PY{l+s+se}{\PYZbs{}n}\PY{l+s+se}{\PYZbs{}n}\PY{l+s}{\PYZdq{}}\PY{p}{,} \PY{n}{LONG\PYZus{}MAX}\PY{p}{,} \PY{n}{entier\PYZus{}long}\PY{p}{)}\PY{p}{;} 
             
             \PY{k+kt}{unsigned} \PY{k+kt}{long} \PY{k+kt}{int} \PY{n}{entier\PYZus{}non\PYZus{}signe} \PY{o}{=} \PY{n}{entier\PYZus{}long}\PY{p}{;} \PY{c+c1}{// Il y a conversion implicite }
             \PY{n}{printf}\PY{p}{(}\PY{l+s}{\PYZdq{}}\PY{l+s}{Valeur maximale d\PYZsq{}un entier non signé \PYZpc{}d }\PY{l+s+se}{\PYZbs{}n}\PY{l+s}{\PYZdq{}}\PY{p}{,} \PY{n}{UINT\PYZus{}MAX}\PY{p}{)}\PY{p}{;} 
             \PY{n}{printf}\PY{p}{(}\PY{l+s}{\PYZdq{}}\PY{l+s}{Valeur maximale d\PYZsq{}un entier non signé long \PYZpc{}lu \PYZgt{} \PYZpc{}lu }\PY{l+s+se}{\PYZbs{}n}\PY{l+s+se}{\PYZbs{}n}\PY{l+s}{\PYZdq{}}\PY{p}{,} \PY{n}{ULONG\PYZus{}MAX}\PY{p}{,} \PY{n}{entier\PYZus{}non\PYZus{}signe}\PY{p}{)}\PY{p}{;} 
             
             \PY{k+kt}{float} \PY{n}{flottant\PYZus{}simple} \PY{o}{=} \PY{l+m+mf}{20.13}\PY{p}{;}
             \PY{k+kt}{double} \PY{n}{flottant\PYZus{}double}\PY{p}{;} 
             \PY{k+kt}{long} \PY{k+kt}{double} \PY{n}{long\PYZus{}double} \PY{o}{=} \PY{l+m+mf}{200001102.2}\PY{p}{;}
             \PY{n}{printf}\PY{p}{(}\PY{l+s}{\PYZdq{}}\PY{l+s}{Valeur maximale d\PYZsq{}un réel simple : }\PY{l+s+se}{\PYZbs{}n}\PY{l+s}{\PYZpc{}f }\PY{l+s+se}{\PYZbs{}n}\PY{l+s}{ \PYZlt{} valeur max double : }\PY{l+s+se}{\PYZbs{}n}\PY{l+s}{\PYZpc{}lf }\PY{l+s+se}{\PYZbs{}n}\PY{l+s}{ \PYZlt{} valeur max long double : }\PY{l+s+se}{\PYZbs{}n}\PY{l+s}{\PYZpc{}Lf }\PY{l+s}{\PYZdq{}}\PY{p}{,} \PY{n}{FLT\PYZus{}MAX}\PY{p}{,} \PY{n}{DBL\PYZus{}MAX}\PY{p}{,} \PY{n}{LDBL\PYZus{}MAX}\PY{p}{)}\PY{p}{;}
             
             \PY{k}{return} \PY{n}{EXIT\PYZus{}SUCCESS}\PY{p}{;}
         \PY{p}{\PYZcb{}}
\end{Verbatim}


    \begin{Verbatim}[commandchars=\\\{\}]
Valeur maximale d'un entier 2147483647 
Valeur maximale d'un entier long 9223372036854775807 > -20000000 

Valeur maximale d'un entier non signé -1 
Valeur maximale d'un entier non signé long 18446744073709551615 > 18446744073689551616 

Valeur maximale d'un réel simple : 
340282346638528859811704183484516925440.000000 
 < valeur max double : 
179769313486231570814527423731704356798070567525844996598917476803157260780028538760589558632766878171540458953514382464234321326889464182768467546703537516986049910576551282076245490090389328944075868508455133942304583236903222948165808559332123348274797826204144723168738177180919299881250404026184124858368.000000 
 < valeur max long double : 
1189731495357231765021263853030970205169063322294624200440323733891737005522970722616410290336528882853545697807495577314427443153670288434198125573853743678673593200706973263201915918282961524365529510646791086614311790632169778838896134786560600399148753433211454911160088679845154866512852340149773037600009125479393966223151383622417838542743917838138717805889487540575168226347659235576974805113725649020884855222494791399377585026011773549180099796226026859508558883608159846900235645132346594476384939859276456284579661772930407806609229102715046085388087959327781622986827547830768080040150694942303411728957777100335714010559775242124057347007386251660110828379119623008469277200965153500208474470792443848545912886723000619085126472111951361467527633519562927597957250278002980795904193139603021470997035276467445530922022679656280991498232083329641241038509239184734786121921697210543484287048353408113042573002216421348917347174234800714880751002064390517234247656004721768096486107994943415703476320643558624207443504424380566136017608837478165389027809576975977286860071487028287955567141404632615832623602762896316173978484254486860609948270867968048078702511858930838546584223040908805996294594586201903766048446790926002225410530775901065760671347200125846406957030257138960983757998926954553052368560758683179223113639519468850880771872104705203957587480013143131444254943919940175753169339392366881856189129931729104252921236835159922322050998001677102784035360140829296398115122877768135706045789343535451696539561254048846447169786893211671087229088082778350518228857646062218739702851655083720992349483334435228984751232753726636066213902281264706234075352071724058665079518217303463782631353393706774901950197841690441824738063162828586857741432581165364040218402724913393320949219498422442730427019873044536620350262386957804682003601447291997123095530057206141866974852846856186514832715974481203121946751686379343096189615107330065552421485195201762858595091051839472502863871632494167613804996319791441870254302706758495192008837915169401581740046711477877201459644461175204059453504764721807975761111720846273639279600339670470037613374509553184150073796412605047923251661354841291884211340823015473304754067072818763503617332908005951896325207071673904547777129682265206225651439919376804400292380903112437912614776255964694221981375146967079446870358004392507659451618379811859392049544036114915310782251072691486979809240946772142727012404377187409216756613634938900451232351668146089322400697993176017805338191849981933008410985993938760292601390911414526003720284872132411955424282101831204216104467404621635336900583664606591156298764745525068145003932941404131495400677602951005962253022823003631473824681059648442441324864573137437595096416168048024129351876204668135636877532814675538798871771836512893947195335061885003267607354388673368002074387849657014576090349857571243045102038730494854256702479339322809110526041538528994849203991091946129912491633289917998094380337879522093131466946149705939664152375949285890960489916121944989986384837022486672249148924678410206183364627416969576307632480235587975245253737035433882960862753427740016333434055083537048507374544819754722228975281083020898682633020285259923084168054539687911418297629988964576482765287504562854924265165217750799516259669229114977788962356670956627138482018191348321687995863652637620978285070099337294396784639879024914514222742527006363942327998483976739987154418554201562244154926653014515504685489258620276085761837129763358761215382565129633538141663949516556000264159186554850057052611431952919918807954522394649627635630178580896692226406235382898535867595990647008385687123810329591926494846250768992258419305480763620215089022149220528069842018350840586938493815498909445461977893029113576516775406232278298314033473276603952231603422824717528181818844304880921321933550869873395861276073670866652375555675803171490108477320096424318780070008797346032906278943553743564448851907191616455141155761939399690767415156402826543664026760095087523945507341556135867933066031744720924446513532366647649735400851967040771103640538150073486891798364049570606189535005089840913826869535090066783324472578712196604415284924840041850932811908963634175739897166596000759487800619164094854338758520657116541072260996288150123144377944008749301944744330784388995701842710004808305012177123560622895076269042856800047718893158089358515593863176652948089031267747029662545110861548958395087796755464137944895960527975209874813839762578592105756284401759349324162148339565350189196811389091843795734703269406342890087805846940352453479398080674273236297887100867175802531561302356064878709259865288416350972529537091114317204887747405539054009425375424119317944175137064689643861517718849867010341532542385911089624710885385808688837777258648564145934262121086647588489260031762345960769508849149662444156604419552086811989770240.000000 
    \end{Verbatim}

    \textbf{{[}3.2{]}} Il est possible d'initialiser un entier non signé
avec un entier signé. Observer la valeur obtenue pour l'entier non
signé. D'où provient-elle ?

    \begin{quote}
\emph{Note} : pour afficher ces valeurs maximales avec \texttt{printf},
on doit modifier les lettres qui suivent le signe \texttt{\%} dans
\texttt{printf} pour adapter le format au type des variables : -
\texttt{\%d}, \texttt{\%ld} : permet d'afficher un entier, un entier
long. - \texttt{\%u}, \texttt{\%lu} : permet d'afficher un entier non
signé, un entier non signé long. - \texttt{\%f}, \texttt{\%lf},
\texttt{\%Lf} : permet d'afficher un flottant, un double et un long
double. - \texttt{\%c} : permet d'afficher un caractère.
\end{quote}

    \textbf{{[}3.3{]}} Modifier le code ci-dessus pour que : 1. l'entier non
signé soit affiché comme un entier signé par \texttt{printf}.\\
2. l'entier non signé soit affiché comme un flottant simple.

Qu'observez-vous dans les deux cas ? Que peut-on en conclure sur les
avertissements du compilateur ?

    \begin{center}\rule{0.5\linewidth}{\linethickness}\end{center}

    \subsection{\#\# Variables}\label{variables}

    \subsubsection{Déclaration de
variables}\label{duxe9claration-de-variables}

Les variables peuvent être déclarées n'importe quand. Typiquement,
\textbf{on déclare une variable au moment où on l'utilise} de la façon
suivante :

\texttt{type\ identifiant\_variable;}

Exemples :

\begin{Shaded}
\begin{Highlighting}[]
\DataTypeTok{int}\NormalTok{ valeur, produit; }\CommentTok{// déclaration de deux variables entières, }
\DataTypeTok{double}\NormalTok{ numerateur; }\CommentTok{// déclaration d'une variable réelle.}
\DataTypeTok{char}\NormalTok{ initiale; }\CommentTok{// déclaration d'une variable caractère}
\end{Highlighting}
\end{Shaded}

\subsubsection{Affectation de variables}\label{affectation-de-variables}

L'initialisation et l'affectation des variables est réalisé \textbf{avec
l'opérateur =} comme dans le fichier \textbf{\texttt{affectations.c}} :

    \begin{Verbatim}[commandchars=\\\{\}]
{\color{incolor}In [{\color{incolor}6}]:} \PY{c+cp}{\PYZsh{}}\PY{c+cp}{include} \PY{c+cpf}{\PYZlt{}stdlib.h\PYZgt{}}
        \PY{k+kt}{int} \PY{n+nf}{main}\PY{p}{(}\PY{p}{)}\PY{p}{\PYZob{}}
            \PY{k+kt}{int} \PY{n}{valeur} \PY{o}{=} \PY{l+m+mi}{10}\PY{p}{,} \PY{n}{produit} \PY{o}{=} \PY{l+m+mi}{23}\PY{p}{;} \PY{c+c1}{// déclaration et initialisation de deux entiers, }
            \PY{k+kt}{double} \PY{n}{numerateur} \PY{o}{=} \PY{l+m+mf}{10.3}\PY{p}{;}
            \PY{k+kt}{char} \PY{n}{initiale} \PY{o}{=} \PY{l+s+sc}{\PYZsq{}}\PY{l+s+sc}{A}\PY{l+s+sc}{\PYZsq{}}\PY{p}{;} 
        
            \PY{n}{produit} \PY{o}{=} \PY{n}{produit} \PY{o}{*} \PY{n}{valeur}\PY{p}{;} \PY{c+c1}{// affectation }
            \PY{n}{valeur} \PY{o}{=} \PY{n}{valeur} \PY{o}{+} \PY{l+m+mi}{1}\PY{p}{;}
            \PY{k}{return} \PY{n}{EXIT\PYZus{}SUCCESS}\PY{p}{;}
        \PY{p}{\PYZcb{}}
\end{Verbatim}


    \subsubsection{Opérateurs
arithmétiques}\label{opuxe9rateurs-arithmuxe9tiques}

Les opérateurs binaires \textbf{+, - , * et /} s'appliquent à des
variables de type entier (signés ou non), booléens, flottant ou double.

Les opérateurs binaires \textbf{/ et \%} utilisés sur \textbf{des
entiers}, fournissent respectivement le quotient de le reste de la
division entière des deux termes.

\begin{quote}
\textbf{Règle} : - Division \textbf{entière} : la division
\texttt{a\ \textbackslash{}\ b} \textbf{si \texttt{b} est entier}
fournit le quotient de la division entière. - Division \textbf{réélle} :
la division \texttt{a\ \textbackslash{}\ b} \textbf{si \texttt{b} est
réel} fournit un résultat réel.
\end{quote}

Les opérateur unaires \textbf{- et +} s'appliquent aux entiers signés et
aux types réels.

\begin{quote}
Note : Des opérateur mathématiques avancés sont disponibles dans la
bibliothèque \texttt{\textless{}math.h\textgreater{}} (puissance, log,
etc.)
\end{quote}

\textbf{Exemples} (à exécuter, cf. fichier
\textbf{\texttt{operateurs.c}}).

    \begin{Verbatim}[commandchars=\\\{\}]
{\color{incolor}In [{\color{incolor}18}]:} \PY{c+cp}{\PYZsh{}}\PY{c+cp}{include} \PY{c+cpf}{\PYZlt{}assert.h\PYZgt{}}
         \PY{c+cp}{\PYZsh{}}\PY{c+cp}{include} \PY{c+cpf}{\PYZlt{}stdio.h\PYZgt{}}
         \PY{c+cp}{\PYZsh{}}\PY{c+cp}{include} \PY{c+cpf}{\PYZlt{}stdlib.h\PYZgt{}}
         
         \PY{k+kt}{int} \PY{n+nf}{main}\PY{p}{(}\PY{p}{)}\PY{p}{\PYZob{}}
             \PY{k+kt}{int} \PY{n}{quantite} \PY{o}{=} \PY{l+m+mi}{10}\PY{p}{;} 
             \PY{k+kt}{int} \PY{n}{prix} \PY{o}{=} \PY{l+m+mi}{15}\PY{p}{;}
         
             \PY{k+kt}{int} \PY{n}{total} \PY{o}{=} \PY{n}{quantite} \PY{o}{*} \PY{n}{prix}\PY{p}{;} 
             \PY{n}{assert}\PY{p}{(}\PY{n}{total} \PY{o}{=}\PY{o}{=} \PY{l+m+mi}{150}\PY{p}{)}\PY{p}{;}
             
             \PY{k+kt}{float} \PY{n}{nb\PYZus{}personnes} \PY{o}{=} \PY{l+m+mi}{60}\PY{p}{;}
             \PY{k+kt}{float} \PY{n}{prix\PYZus{}par\PYZus{}personne} \PY{o}{=} \PY{n}{total} \PY{o}{/} \PY{n}{nb\PYZus{}personnes}\PY{p}{;}
             \PY{n}{assert}\PY{p}{(}\PY{n}{prix\PYZus{}par\PYZus{}personne} \PY{o}{=}\PY{o}{=} \PY{l+m+mf}{2.5}\PY{p}{)}\PY{p}{;}
             
             \PY{k+kt}{int} \PY{n}{nb\PYZus{}personnes\PYZus{}int} \PY{o}{=} \PY{l+m+mi}{60}\PY{p}{;}
             \PY{n}{prix\PYZus{}par\PYZus{}personne} \PY{o}{=} \PY{n}{total} \PY{o}{/} \PY{n}{nb\PYZus{}personnes\PYZus{}int}\PY{p}{;}
             \PY{n}{assert}\PY{p}{(}\PY{n}{prix\PYZus{}par\PYZus{}personne} \PY{o}{!}\PY{o}{=} \PY{l+m+mf}{2.5}\PY{p}{)}\PY{p}{;} \PY{c+c1}{// Quelle est la valeur de prix\PYZus{}par\PYZus{}personne ici ?}
             
             \PY{n}{printf}\PY{p}{(}\PY{l+s}{\PYZdq{}}\PY{l+s}{Le prix par personne est de \PYZpc{}f euros}\PY{l+s}{\PYZdq{}}\PY{p}{,} \PY{n}{prix\PYZus{}par\PYZus{}personne}\PY{p}{)}\PY{p}{;}
             \PY{k}{return} \PY{n}{EXIT\PYZus{}SUCCESS}\PY{p}{;}
         \PY{p}{\PYZcb{}}
\end{Verbatim}


    \begin{Verbatim}[commandchars=\\\{\}]
Le prix par personne est de 2.000000 euros
    \end{Verbatim}

    \subsubsection{Affectations avec
opération}\label{affectations-avec-opuxe9ration}

Les instructions de la forme \texttt{x\ =\ x\ \#\ y} avec \texttt{\#} un
opérateur arithmétique binaire, se synthétisent en C par :

\texttt{x\ \#=\ y}

Il est aussi possible de simplifier l'incrémentation et la
décrémentation avec les opérateurs \textbf{++ et -\/-}. On a : -
\texttt{i++;} équivalent à \texttt{i\ =\ i+1;} - \texttt{i-\/-;}
équivalent à \texttt{i\ =\ i-1;}.

\textbf{Exemples} (à exécuter, cf. fichier
\textbf{\texttt{plusegal.c}}):

    \begin{Verbatim}[commandchars=\\\{\}]
{\color{incolor}In [{\color{incolor}5}]:} \PY{c+cp}{\PYZsh{}}\PY{c+cp}{include} \PY{c+cpf}{\PYZlt{}assert.h\PYZgt{}}
        \PY{c+cp}{\PYZsh{}}\PY{c+cp}{include} \PY{c+cpf}{\PYZlt{}stdio.h\PYZgt{}}
        \PY{c+cp}{\PYZsh{}}\PY{c+cp}{include} \PY{c+cpf}{\PYZlt{}stdlib.h\PYZgt{}}
        
        \PY{k+kt}{int} \PY{n+nf}{main}\PY{p}{(}\PY{p}{)}\PY{p}{\PYZob{}}
            \PY{k+kt}{int} \PY{n}{valeur} \PY{o}{=} \PY{l+m+mi}{10}\PY{p}{,} \PY{n}{produit} \PY{o}{=} \PY{l+m+mi}{23}\PY{p}{;}
            \PY{n}{produit} \PY{o}{+}\PY{o}{=} \PY{n}{valeur}\PY{p}{;}  \PY{c+c1}{// On ajoute valeur à produit}
            \PY{n}{assert}\PY{p}{(}\PY{n}{produit} \PY{o}{=}\PY{o}{=} \PY{l+m+mi}{33}\PY{p}{)}\PY{p}{;}
            
            \PY{n}{produit} \PY{o}{*}\PY{o}{=} \PY{l+m+mi}{2}\PY{p}{;} \PY{c+c1}{// multiplication par 2 puis affectation}
            \PY{n}{assert}\PY{p}{(}\PY{n}{produit} \PY{o}{=}\PY{o}{=} \PY{l+m+mi}{66}\PY{p}{)}\PY{p}{;}
            
            \PY{n}{produit} \PY{o}{/}\PY{o}{=} \PY{l+m+mi}{3}\PY{p}{;} \PY{c+c1}{// division par 3 puis affectation}
            \PY{n}{assert}\PY{p}{(}\PY{n}{produit} \PY{o}{=}\PY{o}{=} \PY{l+m+mi}{22}\PY{p}{)}\PY{p}{;}
            
            \PY{n}{valeur}\PY{o}{+}\PY{o}{+}\PY{p}{;} \PY{c+c1}{// incrémentation de valeur}
            \PY{n}{produit}\PY{o}{\PYZhy{}}\PY{o}{\PYZhy{}}\PY{p}{;} \PY{c+c1}{// décrémentation de produit}
            \PY{n}{assert}\PY{p}{(}\PY{n}{valeur} \PY{o}{=}\PY{o}{=} \PY{l+m+mi}{11} \PY{o}{\PYZam{}}\PY{o}{\PYZam{}} \PY{n}{produit} \PY{o}{=}\PY{o}{=} \PY{l+m+mi}{21}\PY{p}{)}\PY{p}{;}
            
            \PY{n}{printf}\PY{p}{(}\PY{l+s}{\PYZdq{}}\PY{l+s}{\PYZpc{}s}\PY{l+s}{\PYZdq{}}\PY{p}{,} \PY{l+s}{\PYZdq{}}\PY{l+s}{Tous les tests passent.}\PY{l+s+se}{\PYZbs{}n}\PY{l+s}{\PYZdq{}}\PY{p}{)}\PY{p}{;}
            \PY{k}{return} \PY{n}{EXIT\PYZus{}SUCCESS}\PY{p}{;}
        \PY{p}{\PYZcb{}}
\end{Verbatim}


    \begin{Verbatim}[commandchars=\\\{\}]
Tous les tests passent.

    \end{Verbatim}

    \subsubsection{Opérateurs de
comparaison}\label{opuxe9rateurs-de-comparaison}

Les opérateurs \textbf{==, !=, \textless{}, \textgreater{},
\textgreater{}= et \textless{}=} permettent de comparer deux variables.
La proposition \texttt{vrai} s'évalue à \texttt{1} et proposition
\texttt{faux} à \texttt{0}.

\subsubsection{Opérateurs logiques}\label{opuxe9rateurs-logiques}

C définit les opérateurs logiques suivants : - Le EtAlors algorithmique
: \textbf{\&\&}, - Le OuSinon algorithmique :
\textbf{\textbar{}\textbar{}}, - La négation : \textbf{!}

    \subsubsection{Types caractère et entier en langage
C}\label{types-caractuxe8re-et-entier-en-langage-c}

Un caractère est représenté en mémoire comme un entier non-signé
(\texttt{unsigned\ int}) qui correspond au code ASCII de ce caractère.
Les types caractère et entier (non-signé) sont donc compatibles.

L'exemple suivant (à exécuter) présente les différentes opérations
permettant de convertir un entier en caractère, et réciproquement.

    \begin{Verbatim}[commandchars=\\\{\}]
{\color{incolor}In [{\color{incolor}9}]:} \PY{c+cp}{\PYZsh{}}\PY{c+cp}{include} \PY{c+cpf}{\PYZlt{}assert.h\PYZgt{}}
        \PY{c+cp}{\PYZsh{}}\PY{c+cp}{include} \PY{c+cpf}{\PYZlt{}stdio.h\PYZgt{}}
        \PY{c+cp}{\PYZsh{}}\PY{c+cp}{include} \PY{c+cpf}{\PYZlt{}stdlib.h\PYZgt{}}
        \PY{k+kt}{int} \PY{n+nf}{main}\PY{p}{(}\PY{p}{)}\PY{p}{\PYZob{}}
            \PY{c+c1}{// Conversion du char \PYZsq{}1\PYZsq{} en l\PYZsq{}entier 1}
            \PY{k+kt}{char} \PY{n}{c\PYZus{}char} \PY{o}{=} \PY{l+s+sc}{\PYZsq{}}\PY{l+s+sc}{1}\PY{l+s+sc}{\PYZsq{}}\PY{p}{;}
            \PY{k+kt}{int} \PY{n}{c\PYZus{}int} \PY{o}{=} \PY{n}{c\PYZus{}char} \PY{o}{\PYZhy{}} \PY{l+s+sc}{\PYZsq{}}\PY{l+s+sc}{0}\PY{l+s+sc}{\PYZsq{}}\PY{p}{;} \PY{c+c1}{//on retire le code ascii du caractère \PYZsq{}0\PYZsq{}}
            \PY{n}{assert}\PY{p}{(}\PY{n}{c\PYZus{}int} \PY{o}{=}\PY{o}{=} \PY{l+m+mi}{1}\PY{p}{)}\PY{p}{;}
            
            \PY{c+c1}{// Conversion de l\PYZsq{}entier 1 en un char qui vaut \PYZsq{}1\PYZsq{} }
            \PY{k+kt}{int} \PY{n}{new\PYZus{}int} \PY{o}{=} \PY{l+m+mi}{1}\PY{p}{;}
            \PY{k+kt}{char} \PY{n}{c\PYZus{}char2} \PY{o}{=} \PY{n}{new\PYZus{}int} \PY{o}{+} \PY{l+s+sc}{\PYZsq{}}\PY{l+s+sc}{0}\PY{l+s+sc}{\PYZsq{}}\PY{p}{;} \PY{c+c1}{//on ajoute le code ascii du caractère \PYZsq{}0\PYZsq{}}
            \PY{n}{assert}\PY{p}{(}\PY{n}{c\PYZus{}char2} \PY{o}{=}\PY{o}{=} \PY{l+s+sc}{\PYZsq{}}\PY{l+s+sc}{1}\PY{l+s+sc}{\PYZsq{}}\PY{p}{)}\PY{p}{;}  \PY{c+c1}{// c\PYZus{}char2 est bien égal au caractère }
            
            \PY{n}{printf}\PY{p}{(}\PY{l+s}{\PYZdq{}}\PY{l+s}{\PYZpc{}s}\PY{l+s}{\PYZdq{}}\PY{p}{,} \PY{l+s}{\PYZdq{}}\PY{l+s}{Tous les tests passent.}\PY{l+s+se}{\PYZbs{}n}\PY{l+s}{\PYZdq{}}\PY{p}{)}\PY{p}{;}
            \PY{k}{return} \PY{n}{EXIT\PYZus{}SUCCESS}\PY{p}{;}
        \PY{p}{\PYZcb{}}
\end{Verbatim}


    \begin{Verbatim}[commandchars=\\\{\}]
Tous les tests passent.

    \end{Verbatim}

    \subsection{\#\#\#\# Exercice 4 -\/- Comprendre les opérateurs
arithmétiques et les relations entre caractere et
entier.}\label{exercice-4----comprendre-les-opuxe9rateurs-arithmuxe9tiques-et-les-relations-entre-caractere-et-entier.}

\begin{center}\rule{0.5\linewidth}{\linethickness}\end{center}

\textbf{{[}4.1{]}} Dans cet exercice (fichier
\textbf{\texttt{exercice4.c}}), suivre la consigne présentée dans les
commentaires.

    \begin{Verbatim}[commandchars=\\\{\}]
{\color{incolor}In [{\color{incolor}12}]:} \PY{c+cp}{\PYZsh{}}\PY{c+cp}{define XXX \PYZhy{}1}
         
         \PY{c+c1}{// Consigne : dans la suite *** uniquement ***, remplacer XXX par le bon }
         \PY{c+c1}{// résultat (une constante littérale).}
         
         \PY{c+cp}{\PYZsh{}}\PY{c+cp}{include} \PY{c+cpf}{\PYZlt{}assert.h\PYZgt{}}
         \PY{c+cp}{\PYZsh{}}\PY{c+cp}{include} \PY{c+cpf}{\PYZlt{}stdlib.h\PYZgt{}}
         \PY{c+cp}{\PYZsh{}}\PY{c+cp}{include} \PY{c+cpf}{\PYZlt{}stdio.h\PYZgt{}}
         
         \PY{k+kt}{int} \PY{n+nf}{main}\PY{p}{(}\PY{k+kt}{void}\PY{p}{)} \PY{p}{\PYZob{}}
             \PY{c+c1}{// Comprendre les opérateurs arithmétique}
             \PY{n}{assert}\PY{p}{(} \PY{l+m+mi}{\PYZhy{}5} \PY{o}{=}\PY{o}{=} \PY{l+m+mi}{5} \PY{o}{\PYZhy{}} \PY{l+m+mi}{2} \PY{o}{*} \PY{l+m+mi}{5}\PY{p}{)}\PY{p}{;}
             \PY{n}{assert}\PY{p}{(}\PY{l+m+mi}{5} \PY{o}{=}\PY{o}{=} \PY{l+m+mi}{25} \PY{o}{\PYZpc{}} \PY{l+m+mi}{10}\PY{p}{)}\PY{p}{;}
             \PY{n}{assert}\PY{p}{(}\PY{l+m+mi}{2} \PY{o}{=}\PY{o}{=} \PY{l+m+mi}{25} \PY{o}{/} \PY{l+m+mi}{10}\PY{p}{)}\PY{p}{;}
             \PY{n}{assert}\PY{p}{(} \PY{l+m+mf}{2.5}\PY{o}{=}\PY{o}{=} \PY{l+m+mi}{25} \PY{o}{/} \PY{l+m+mf}{10.0}\PY{p}{)}\PY{p}{;}
         
             \PY{c+c1}{// Comprendre les relations caractères et entiers}
             \PY{n}{assert}\PY{p}{(}\PY{l+m+mi}{5} \PY{o}{=}\PY{o}{=} \PY{l+s+sc}{\PYZsq{}}\PY{l+s+sc}{5}\PY{l+s+sc}{\PYZsq{}} \PY{o}{\PYZhy{}} \PY{l+s+sc}{\PYZsq{}}\PY{l+s+sc}{0}\PY{l+s+sc}{\PYZsq{}}\PY{p}{)}\PY{p}{;}
             \PY{n}{assert}\PY{p}{(}\PY{l+s+sc}{\PYZsq{}}\PY{l+s+sc}{7}\PY{l+s+sc}{\PYZsq{}} \PY{o}{=}\PY{o}{=} \PY{l+s+sc}{\PYZsq{}}\PY{l+s+sc}{0}\PY{l+s+sc}{\PYZsq{}} \PY{o}{+} \PY{l+m+mi}{7}\PY{p}{)}\PY{p}{;}
             \PY{n}{assert}\PY{p}{(}\PY{l+m+mi}{68} \PY{o}{=}\PY{o}{=} \PY{l+s+sc}{\PYZsq{}}\PY{l+s+sc}{A}\PY{l+s+sc}{\PYZsq{}} \PY{o}{+} \PY{l+m+mi}{3}\PY{p}{)}\PY{p}{;}
         
             \PY{n}{printf}\PY{p}{(}\PY{l+s}{\PYZdq{}}\PY{l+s}{\PYZpc{}s}\PY{l+s}{\PYZdq{}}\PY{p}{,} \PY{l+s}{\PYZdq{}}\PY{l+s}{Bravo ! Tous les tests passent.}\PY{l+s+se}{\PYZbs{}n}\PY{l+s}{\PYZdq{}}\PY{p}{)}\PY{p}{;}
         
             \PY{k}{return} \PY{n}{EXIT\PYZus{}SUCCESS}\PY{p}{;}
         \PY{p}{\PYZcb{}}
\end{Verbatim}


    \begin{Verbatim}[commandchars=\\\{\}]
Bravo ! Tous les tests passent.

    \end{Verbatim}

    \subsubsection{Portée et masquage des
variables}\label{portuxe9e-et-masquage-des-variables}

    \paragraph{Portée et masquage des
variables.}\label{portuxe9e-et-masquage-des-variables.}

\textbf{Un bloc} est une série d'instructions délimitée par \textbf{une
paire d'accolades}.

\begin{itemize}
\tightlist
\item
  \textbf{Portée} : Les variables déclarées dans un bloc sont libérées
  quand l'accolade fermante est exécutée. On dit que leur portée se
  limite au bloc où elles sont déclarées.
\end{itemize}

\begin{Shaded}
\begin{Highlighting}[]
\NormalTok{\{ }\CommentTok{//debut du bloc B1}
    \DataTypeTok{int}\NormalTok{ age = }\DecValTok{20}\NormalTok{;}
\NormalTok{    \{ }\CommentTok{// debut du bloc B2}
        \DataTypeTok{int}\NormalTok{ nouvel_age = }\DecValTok{25}\NormalTok{ ; }\CommentTok{// variable locale à B2}
\NormalTok{    \} }\CommentTok{// fin du bloc B2}
    \CommentTok{// La variable nouvel_age n'existe plus. }
\NormalTok{    age = age + }\DecValTok{1}\NormalTok{;}
\NormalTok{\} }\CommentTok{//fin du bloc B1}
\end{Highlighting}
\end{Shaded}

\begin{itemize}
\tightlist
\item
  \textbf{Masquage} : Les variables déclarées dans un bloc peuvent avoir
  le même identifiant qu'une variable déclarée avant l'ouverture du
  bloc. Dans ce cas, la variable déclarées dans le bloc masque la
  variable homonyme déclarées avant : c'est elle qui est utilisée par
  les instructions du bloc.
\end{itemize}

Masquage et portée sont illustrés dans l'exemple
(\textbf{\texttt{portee\_masquage.c}} à exécuter) suivant :

    \begin{Verbatim}[commandchars=\\\{\}]
{\color{incolor}In [{\color{incolor}1}]:} \PY{c+cp}{\PYZsh{}}\PY{c+cp}{include} \PY{c+cpf}{\PYZlt{}assert.h\PYZgt{}}
        \PY{c+cp}{\PYZsh{}}\PY{c+cp}{include} \PY{c+cpf}{\PYZlt{}stdio.h\PYZgt{}}
        \PY{c+cp}{\PYZsh{}}\PY{c+cp}{include} \PY{c+cpf}{\PYZlt{}stdlib.h\PYZgt{}}
        
        \PY{k+kt}{int} \PY{n+nf}{main}\PY{p}{(}\PY{p}{)} \PY{p}{\PYZob{}} \PY{c+c1}{//debut du bloc B1}
            \PY{k+kt}{int} \PY{n}{alea} \PY{o}{=} \PY{l+m+mi}{20}\PY{p}{,} \PY{n}{diviseur} \PY{o}{=} \PY{l+m+mi}{2}\PY{p}{;}
            
            \PY{p}{\PYZob{}} \PY{c+c1}{//debut du bloc B2}
                
                \PY{k+kt}{int} \PY{n}{alea} \PY{o}{=} \PY{l+m+mi}{3}\PY{p}{;} \PY{c+c1}{// masquage de la variable entière alea par la variable alea entière}
                \PY{k+kt}{float} \PY{n}{diviseur} \PY{o}{=} \PY{l+m+mf}{2.0}\PY{p}{;} \PY{c+c1}{// idem pour le diviseur réel qui masque le diviseur de type entier.}
                \PY{k+kt}{float} \PY{n}{res\PYZus{}reel} \PY{o}{=} \PY{n}{alea} \PY{o}{/} \PY{n}{diviseur}\PY{p}{;}
                \PY{n}{assert}\PY{p}{(}\PY{n}{res\PYZus{}reel} \PY{o}{=} \PY{l+m+mf}{1.5}\PY{p}{)}\PY{p}{;}
                
            \PY{p}{\PYZcb{}} \PY{c+c1}{// du bloc B2}
            \PY{k+kt}{int} \PY{n}{res\PYZus{}int} \PY{o}{=} \PY{n}{alea} \PY{o}{/} \PY{n}{diviseur}\PY{p}{;}
            \PY{n}{assert}\PY{p}{(}\PY{n}{res\PYZus{}int} \PY{o}{=} \PY{l+m+mi}{10}\PY{p}{)}\PY{p}{;}
            
            \PY{n}{printf}\PY{p}{(}\PY{l+s}{\PYZdq{}}\PY{l+s}{\PYZpc{}s}\PY{l+s}{\PYZdq{}}\PY{p}{,} \PY{l+s}{\PYZdq{}}\PY{l+s}{Les tests passent}\PY{l+s+se}{\PYZbs{}n}\PY{l+s}{\PYZdq{}}\PY{p}{)}\PY{p}{;}
            \PY{k}{return} \PY{n}{EXIT\PYZus{}SUCCESS}\PY{p}{;}
        \PY{p}{\PYZcb{}} \PY{c+c1}{//fin du bloc B1}
\end{Verbatim}


    \begin{Verbatim}[commandchars=\\\{\}]
Les tests passent

    \end{Verbatim}

    \subsection{\#\#\#\# Exercice 5 -\/- Portée et masquage des
variables}\label{exercice-5----portuxe9e-et-masquage-des-variables}

\begin{center}\rule{0.5\linewidth}{\linethickness}\end{center}

\textbf{{[}5.1{]}} Dans cet exercice (fichier
\textbf{\texttt{exercice5.c}}), suivre la consigne présentée dans les
commentaires.

    \begin{Verbatim}[commandchars=\\\{\}]
{\color{incolor}In [{\color{incolor}17}]:} \PY{c+c1}{// Objectifs : Illustrer portée et masquage.}
         
         \PY{c+cp}{\PYZsh{}}\PY{c+cp}{define XXX \PYZhy{}1}
         
         \PY{c+c1}{// Consigne : *** dans la suite uniquement ***, remplacer XXX par le bon résultat (une}
         \PY{c+c1}{// constante littérale).  Ne compiler et exécuter que quand tous les XXX ont été traités.}
         
         \PY{c+cp}{\PYZsh{}}\PY{c+cp}{include} \PY{c+cpf}{\PYZlt{}assert.h\PYZgt{}}
         \PY{c+cp}{\PYZsh{}}\PY{c+cp}{include} \PY{c+cpf}{\PYZlt{}stdlib.h\PYZgt{}}
         \PY{c+cp}{\PYZsh{}}\PY{c+cp}{include} \PY{c+cpf}{\PYZlt{}stdio.h\PYZgt{}}
         
         \PY{k+kt}{int} \PY{n+nf}{main}\PY{p}{(}\PY{k+kt}{void}\PY{p}{)} \PY{p}{\PYZob{}}
             \PY{k+kt}{int} \PY{n}{x} \PY{o}{=} \PY{l+m+mi}{10}\PY{p}{;}
             \PY{n}{assert}\PY{p}{(}\PY{n}{x} \PY{o}{=}\PY{o}{=} \PY{l+m+mi}{10}\PY{p}{)}\PY{p}{;}
         
             \PY{p}{\PYZob{}}
             \PY{k+kt}{int} \PY{n}{y} \PY{o}{=} \PY{l+m+mi}{7}\PY{p}{;}
             \PY{n}{assert}\PY{p}{(}\PY{l+m+mi}{10} \PY{o}{=}\PY{o}{=} \PY{l+m+mi}{10}\PY{p}{)}\PY{p}{;}
             \PY{n}{assert}\PY{p}{(}\PY{n}{y} \PY{o}{=}\PY{o}{=} \PY{l+m+mi}{7}\PY{p}{)}\PY{p}{;}
         
             \PY{p}{\PYZob{}}
                 \PY{k+kt}{char} \PY{n}{x} \PY{o}{=} \PY{l+s+sc}{\PYZsq{}}\PY{l+s+sc}{?}\PY{l+s+sc}{\PYZsq{}}\PY{p}{;}
                 \PY{n}{assert}\PY{p}{(}\PY{l+s+sc}{\PYZsq{}}\PY{l+s+sc}{?}\PY{l+s+sc}{\PYZsq{}} \PY{o}{=}\PY{o}{=} \PY{n}{x}\PY{p}{)}\PY{p}{;}
                 \PY{n}{assert}\PY{p}{(}\PY{n}{y} \PY{o}{=}\PY{o}{=} \PY{n}{y}\PY{p}{)}\PY{p}{;}
                 \PY{n}{y} \PY{o}{=} \PY{l+m+mi}{7}\PY{p}{;}
             \PY{p}{\PYZcb{}}
         
             \PY{n}{assert}\PY{p}{(}\PY{l+m+mi}{10}\PY{o}{=}\PY{o}{=} \PY{l+m+mi}{10}\PY{p}{)}\PY{p}{;}
             \PY{n}{assert}\PY{p}{(}\PY{l+m+mi}{7} \PY{o}{=}\PY{o}{=} \PY{n}{y}\PY{p}{)}\PY{p}{;}
             \PY{p}{\PYZcb{}}
         
             \PY{n}{assert}\PY{p}{(}\PY{l+m+mi}{10} \PY{o}{=}\PY{o}{=} \PY{n}{x}\PY{p}{)}\PY{p}{;}
         
             \PY{n}{printf}\PY{p}{(}\PY{l+s}{\PYZdq{}}\PY{l+s}{\PYZpc{}s}\PY{l+s}{\PYZdq{}}\PY{p}{,} \PY{l+s}{\PYZdq{}}\PY{l+s}{Bravo ! Tous les tests passent.}\PY{l+s+se}{\PYZbs{}n}\PY{l+s}{\PYZdq{}}\PY{p}{)}\PY{p}{;}
             \PY{k}{return} \PY{n}{EXIT\PYZus{}SUCCESS}\PY{p}{;}
         \PY{p}{\PYZcb{}}
\end{Verbatim}


    \begin{Verbatim}[commandchars=\\\{\}]
Bravo ! Tous les tests passent.

    \end{Verbatim}

    \subsection{\#\# Constantes}\label{constantes}

    Les constantes se déclarent généralement juste après l'inclusion des
bibliothèques. Leur valeur ne peut être modifiée. Il est possible de
définir des constantes de deux façons : 1. En utilisant le mot-clé
\texttt{const} pour obtenir une constante typée

\begin{Shaded}
\begin{Highlighting}[]
    \DataTypeTok{const} \DataTypeTok{int}\NormalTok{ MAJORITE_EU = }\DecValTok{18}\NormalTok{; }\CommentTok{// déclaration d'une constante typée MAJORITE_EU }
    \DataTypeTok{const} \DataTypeTok{int}\NormalTok{ MAJORITE_US = }\DecValTok{21}\NormalTok{; }\CommentTok{// déclaration d'une constante typée MAJORITE_US }
    \DataTypeTok{const} \DataTypeTok{char}\NormalTok{ CM = }\CharTok{'c'}\NormalTok{; }\CommentTok{// déclaration d'une constante caractère. }
\end{Highlighting}
\end{Shaded}

\begin{quote}
\textbf{Règle} : Un caractère se distingue par l'utilisation de
guillements simples (apostrophe) : 'A', 'c', 'D', '\n', '\t'.
\end{quote}

\begin{enumerate}
\def\labelenumi{\arabic{enumi}.}
\setcounter{enumi}{1}
\tightlist
\item
  En définissant une constante pré-processeur :
\end{enumerate}

\begin{Shaded}
\begin{Highlighting}[]
    \PreprocessorTok{#define MAJORITE_EU 18 }\CommentTok{// déclaration d'une constante pré-processeur 18, }
\end{Highlighting}
\end{Shaded}

Le pré-processeur remplace les occurrences de \texttt{MAJORITE\_EU} par
la valeur 18.

\begin{quote}
\textbf{Règle} : Pas de point-virgule à la fin d'une instruction
pré-processeur.
\end{quote}

\textbf{Constantes littérales}

Ce sont les valeurs numériques écrites directement dans les instructions
:

\begin{Shaded}
\begin{Highlighting}[]
    \DataTypeTok{int}\NormalTok{ age = }\DecValTok{20}\NormalTok{; }\CommentTok{// 20 est une constante littérale}
    \DataTypeTok{char}\NormalTok{ initiale_nom = }\CharTok{'M'} \CommentTok{// Le caractère 'M' est une constante littérale.}
\end{Highlighting}
\end{Shaded}

    \begin{center}\rule{0.5\linewidth}{\linethickness}\end{center}

    \subsection{\#\# Expressions et compatibilité entre
types}\label{expressions-et-compatibilituxe9-entre-types}

    \subsubsection{Définition d'une
expression}\label{duxe9finition-dune-expression}

Une expression est une instruction qui est caractérisée par une valeur
de retour. Voici quelques exemples : - Une variable initialisée

\begin{Shaded}
\begin{Highlighting}[]
     \DataTypeTok{int}\NormalTok{ val = }\DecValTok{20}\NormalTok{;}
\NormalTok{     val; }\CommentTok{// La variable `val` vaut 20 dans cette instruction.}
\end{Highlighting}
\end{Shaded}

\begin{itemize}
\item
  Une comparaison : \texttt{(b\ \textgreater{}\ 20)}. Cette expression
  vaudra \texttt{true} ou \texttt{false}.
\item
  L'utilisation d'opérateur arithmétiques :

\begin{Shaded}
\begin{Highlighting}[]
\DataTypeTok{int}\NormalTok{ x = }\DecValTok{3}\NormalTok{;}
\NormalTok{x + }\DecValTok{3}\NormalTok{; }\CommentTok{// Cette expression vaut 6}
\NormalTok{(x * }\DecValTok{2}\NormalTok{) / }\DecValTok{3}\NormalTok{; }\CommentTok{// Cette expression vaut 2}
\end{Highlighting}
\end{Shaded}
\end{itemize}

\begin{quote}
\emph{Note} : Une affectation est aussi une expression :
\texttt{val\ =\ 40} est une expression qui vaut 40. L'utilisateur de
l'affectation comme expression est à éviter en C.
\end{quote}

    \subsubsection{Priorité des
opérateurs}\label{priorituxe9-des-opuxe9rateurs}

En C, la priorité des opérateurs évalués dans une même expression est la
suivante :

\begin{longtable}[]{@{}cc@{}}
\toprule
Priorité & Opérateurs\tabularnewline
\midrule
\endhead
1 & +, -, \textbf{!} (unaires)\tabularnewline
2 & *, /(entier), /(flottant), \%, \textbf{\&\&}\tabularnewline
3 & +, -, \textbf{\textbar{}\textbar{}}\tabularnewline
4 & \textbf{\textless{}, \textgreater{}, \textless{}=, \textgreater{}=,
==, !=}\tabularnewline
\bottomrule
\end{longtable}

La priorité 1 est la plus forte. Les opérateurs booléens sont présentés
en gras.

    \subsubsection{Compatibilité entre
types}\label{compatibilituxe9-entre-types}

En C, \textbf{une expression peut être composée d'expressions de types
différents si ces types sont compatibles}. Voici quelques exemples
(fichier \textbf{\texttt{compatibilite.c}}:

    \begin{Verbatim}[commandchars=\\\{\}]
{\color{incolor}In [{\color{incolor}18}]:} \PY{c+cp}{\PYZsh{}}\PY{c+cp}{include} \PY{c+cpf}{\PYZlt{}assert.h\PYZgt{}}
         \PY{c+cp}{\PYZsh{}}\PY{c+cp}{include} \PY{c+cpf}{\PYZlt{}stdlib.h\PYZgt{}}
         \PY{k+kt}{int} \PY{n+nf}{main}\PY{p}{(}\PY{p}{)}\PY{p}{\PYZob{}}
             \PY{k+kt}{int} \PY{n}{quantite} \PY{o}{=} \PY{l+m+mi}{5}\PY{p}{;}
             \PY{k+kt}{float} \PY{n}{prix} \PY{o}{=} \PY{l+m+mf}{12.3}\PY{p}{;}
         
             \PY{k+kt}{float} \PY{n}{total} \PY{o}{=} \PY{n}{prix} \PY{o}{*} \PY{n}{quantite}\PY{p}{;} \PY{c+c1}{//l\PYZsq{}entier quantité est compatible avec les flottants    }
             \PY{k+kt}{float} \PY{n}{recette} \PY{o}{=} \PY{l+m+mi}{12}\PY{p}{;} \PY{c+c1}{// l\PYZsq{}entier 12 est compatible avec le flottant recette.}
             \PY{n}{assert}\PY{p}{(}\PY{n}{total} \PY{o}{=}\PY{o}{=} \PY{l+m+mf}{12.3}\PY{o}{*}\PY{l+m+mf}{5.0} \PY{o}{\PYZam{}}\PY{o}{\PYZam{}} \PY{n}{recette} \PY{o}{=}\PY{o}{=} \PY{l+m+mf}{12.0}\PY{p}{)}\PY{p}{;}
         
             \PY{n}{quantite} \PY{o}{=} \PY{n}{total} \PY{o}{/} \PY{n}{recette}\PY{p}{;}
             \PY{n}{assert}\PY{p}{(}\PY{n}{quantite} \PY{o}{!}\PY{o}{=} \PY{n}{total} \PY{o}{/} \PY{n}{recette}\PY{p}{)}\PY{p}{;}
             \PY{c+c1}{// le réel obtenue par la division de total et recette }
             \PY{c+c1}{// n\PYZsq{}est pas compatible avec l\PYZsq{}entier quantite.}
             
             \PY{k}{return} \PY{n}{EXIT\PYZus{}SUCCESS}\PY{p}{;}
         \PY{p}{\PYZcb{}}
\end{Verbatim}


    Si un type A est compatible avec un type B, on peut interchanger une
expression de type B par une expression de type A sans changer la valeur
de l'expression.

Typiquement, un type A est compatible avec un type B si le passage de
l'un à l'autre n'engendre pas de perte de donnée : - Un entier est
compatible avec un réel (12 devient 12.0) - Un réel n'est pas compatible
avec un entier (le passage de 1.3 à 1 introduit une perte
d'information).

    \subsubsection{Conversion explicite}\label{conversion-explicite}

Il est possible de convertir explicitement une expression pour qu'elle
soit évaluée avec un autre type. Pour cela, on utilise la notation :
\textgreater{} \texttt{(type)\ expression}

\textbf{Exemple} Voici l'exemple illustrant la division entière et
réelle présenté précédement (cf fichier
\textbf{\texttt{conversion\_explicite.c}}). Il a été modifié pour
déclarer le nombre de personnes avec un entier, et dériver tout de même
un prix par personnes avec une division réelle grâce à une conversion
explicite.

    \begin{Verbatim}[commandchars=\\\{\}]
{\color{incolor}In [{\color{incolor}38}]:} \PY{c+cp}{\PYZsh{}}\PY{c+cp}{include} \PY{c+cpf}{\PYZlt{}assert.h\PYZgt{}}
         \PY{c+cp}{\PYZsh{}}\PY{c+cp}{include} \PY{c+cpf}{\PYZlt{}stdio.h\PYZgt{}}
         
         \PY{k+kt}{int} \PY{n+nf}{main}\PY{p}{(}\PY{p}{)}\PY{p}{\PYZob{}}
             \PY{k+kt}{int} \PY{n}{quantite} \PY{o}{=} \PY{l+m+mi}{10}\PY{p}{;} 
             \PY{k+kt}{int} \PY{n}{prix} \PY{o}{=} \PY{l+m+mi}{15}\PY{p}{;}
         
             \PY{k+kt}{int} \PY{n}{total} \PY{o}{=} \PY{n}{quantite} \PY{o}{*} \PY{n}{prix}\PY{p}{;} 
             \PY{n}{assert}\PY{p}{(}\PY{n}{total} \PY{o}{=}\PY{o}{=} \PY{l+m+mi}{150}\PY{p}{)}\PY{p}{;}
                 
             \PY{k+kt}{int} \PY{n}{nb\PYZus{}personnes\PYZus{}int} \PY{o}{=} \PY{l+m+mi}{60}\PY{p}{;}
             \PY{k+kt}{float} \PY{n}{prix\PYZus{}par\PYZus{}personne} \PY{o}{=} \PY{n}{total} \PY{o}{/}\PY{p}{(}\PY{k+kt}{float}\PY{p}{)} \PY{n}{nb\PYZus{}personnes\PYZus{}int}\PY{p}{;}
             \PY{n}{assert}\PY{p}{(}\PY{n}{prix\PYZus{}par\PYZus{}personne} \PY{o}{=}\PY{o}{=} \PY{l+m+mf}{2.5}\PY{p}{)}\PY{p}{;} \PY{c+c1}{// Maintenant on effectue bien une division réelle }
             
             \PY{n}{printf}\PY{p}{(}\PY{l+s}{\PYZdq{}}\PY{l+s}{Le prix par personne est de \PYZpc{}1.2f euros}\PY{l+s}{\PYZdq{}}\PY{p}{,} \PY{n}{prix\PYZus{}par\PYZus{}personne}\PY{p}{)}\PY{p}{;}
             \PY{k}{return} \PY{l+m+mi}{0}\PY{p}{;}
         \PY{p}{\PYZcb{}}
\end{Verbatim}


    \begin{Verbatim}[commandchars=\\\{\}]
Le prix par personne est de 2.50 euros
    \end{Verbatim}

    \begin{quote}
\emph{Note} : On observe que le descripteur de format \texttt{\%f} a été
étendu à \texttt{\%1.2f} pour limiter le nombre de décimales à 2.
\end{quote}

    \subsection{\#\#\# Exercice 6}\label{exercice-6}

\begin{center}\rule{0.5\linewidth}{\linethickness}\end{center}

    \textbf{{[}6.1{]}} Ecrire un programme (fichier
\textbf{\texttt{exercice6.c}}) qui calcule le périmètre et l'aire d'un
cercle, étant donné un rayon qui vaut \texttt{15}. Le rayon est une
variable entière. Les éventuelles constantes seront déclarées comme des
constantes pré-processeur.

\textbf{{[}6.2{]}} Ecrire les deux résultats réels à l'écran.

    \begin{Verbatim}[commandchars=\\\{\}]
{\color{incolor}In [{\color{incolor}8}]:} \PY{c+cp}{\PYZsh{}}\PY{c+cp}{include} \PY{c+cpf}{\PYZlt{}stdlib.h\PYZgt{}}\PY{c+c1}{ }
        \PY{c+cp}{\PYZsh{}}\PY{c+cp}{include} \PY{c+cpf}{\PYZlt{}stdio.h\PYZgt{}}
        \PY{c+cp}{\PYZsh{}}\PY{c+cp}{define PI 3.14}
        \PY{c+cp}{\PYZsh{}}\PY{c+cp}{define R 15}
        
        \PY{k+kt}{int} \PY{n+nf}{main}\PY{p}{(}\PY{p}{)}\PY{p}{\PYZob{}}
            \PY{k+kt}{float} \PY{n}{perimetre} \PY{o}{=} \PY{l+m+mi}{2}\PY{o}{*}\PY{n}{PI}\PY{o}{*}\PY{n}{R}\PY{p}{;}
            \PY{k+kt}{float} \PY{n}{air} \PY{o}{=} \PY{n}{PI}\PY{o}{*}\PY{n}{R}\PY{o}{*}\PY{n}{R}\PY{p}{;}
            \PY{n}{printf}\PY{p}{(}\PY{l+s}{\PYZdq{}}\PY{l+s}{le périmetre du cecle est \PYZpc{}f }\PY{l+s}{\PYZdq{}}\PY{p}{,} \PY{n}{perimetre}\PY{p}{)}\PY{p}{;}
            
            \PY{n}{printf}\PY{p}{(}\PY{l+s}{\PYZdq{}}\PY{l+s}{l\PYZsq{}air du cercle est \PYZpc{}f}\PY{l+s}{\PYZdq{}} \PY{p}{,} \PY{n}{air}\PY{p}{)}\PY{p}{;}
            \PY{k}{return} \PY{n}{EXIT\PYZus{}SUCCESS}\PY{p}{;}
        \PY{p}{\PYZcb{}}
\end{Verbatim}


    \begin{Verbatim}[commandchars=\\\{\}]
le périmetre du cecle est 94.199997 l'air du cercle est 706.500000
    \end{Verbatim}

    \subsection{\#\#\# Exercice 7}\label{exercice-7}

\begin{center}\rule{0.5\linewidth}{\linethickness}\end{center}

\textbf{{[}7.1{]}} Compléter et corriger le corps des fonctions du
fichier \textbf{\texttt{exercice7.c}} ci-dessous (voir TODO)

    \begin{Verbatim}[commandchars=\\\{\}]
{\color{incolor}In [{\color{incolor}24}]:} \PY{c+cp}{\PYZsh{}}\PY{c+cp}{include} \PY{c+cpf}{\PYZlt{}stdlib.h\PYZgt{}}
         \PY{c+cp}{\PYZsh{}}\PY{c+cp}{include} \PY{c+cpf}{\PYZlt{}stdio.h\PYZgt{}}
         \PY{c+cp}{\PYZsh{}}\PY{c+cp}{include} \PY{c+cpf}{\PYZlt{}stdbool.h\PYZgt{}}
         \PY{c+cp}{\PYZsh{}}\PY{c+cp}{include} \PY{c+cpf}{\PYZlt{}assert.h\PYZgt{}}
         
         \PY{c+cm}{/**}
         \PY{c+cm}{ * \PYZbs{}brief obtenir le chiffre des unités d\PYZsq{}un entier naturel.}
         \PY{c+cm}{ * \PYZbs{}param[in] nombre le nombre dont on veut obtenir le chiffre des unités}
         \PY{c+cm}{ * \PYZbs{}return le chiffre des unités de nombre}
         \PY{c+cm}{ * \PYZbs{}pre nombre positif : nombre \PYZgt{}= 0}
         \PY{c+cm}{ */}
         \PY{k+kt}{int} \PY{n+nf}{chiffre\PYZus{}unites}\PY{p}{(}\PY{k+kt}{int} \PY{n}{nombre}\PY{p}{)}
         \PY{p}{\PYZob{}}
             \PY{n}{assert}\PY{p}{(}\PY{n}{nombre} \PY{o}{\PYZgt{}}\PY{o}{=} \PY{l+m+mi}{0}\PY{p}{)}\PY{p}{;}
         
             \PY{k+kt}{int} \PY{n}{r} \PY{o}{=} \PY{n}{nombre}\PY{o}{\PYZpc{}}\PY{l+m+mi}{10}\PY{p}{;}
             \PY{k}{return} \PY{n}{r}\PY{p}{;}
         \PY{p}{\PYZcb{}}
         
         \PY{c+cm}{/**}
         \PY{c+cm}{ * \PYZbs{}brief obtenir le chiffre des dizaines d\PYZsq{}un entier naturel.}
         \PY{c+cm}{ * \PYZbs{}param[in] nombre le nombre dont on veut obtenir le chiffre des dizaines}
         \PY{c+cm}{ * \PYZbs{}return le chiffre des dizaines de nombre}
         \PY{c+cm}{ * \PYZbs{}pre nombre positif : nombre \PYZgt{}= 0}
         \PY{c+cm}{ */}
         \PY{k+kt}{int} \PY{n+nf}{chiffre\PYZus{}dizaines}\PY{p}{(}\PY{k+kt}{int} \PY{n}{nombre}\PY{p}{)}
         \PY{p}{\PYZob{}}
             \PY{n}{assert}\PY{p}{(}\PY{n}{nombre} \PY{o}{\PYZgt{}}\PY{o}{=} \PY{l+m+mi}{0}\PY{p}{)}\PY{p}{;}
         
             \PY{k+kt}{int} \PY{n}{a} \PY{o}{=} \PY{n}{nombre}\PY{o}{/}\PY{l+m+mi}{10}\PY{p}{;}
             \PY{k+kt}{int} \PY{n}{r} \PY{o}{=} \PY{n}{a}\PY{o}{\PYZpc{}}\PY{l+m+mi}{10}\PY{p}{;}
             \PY{k}{return} \PY{n}{r}\PY{p}{;}
         \PY{p}{\PYZcb{}}
         
         \PY{c+cm}{/**}
         \PY{c+cm}{ * \PYZbs{}brief Indiquer si une année est bissextile.}
         \PY{c+cm}{ * \PYZbs{}param[in] annee l\PYZsq{}année à considérer}
         \PY{c+cm}{ * \PYZbs{}return vrai si l\PYZsq{}année est bissextile}
         \PY{c+cm}{ * \PYZbs{}pre année positive : annee \PYZgt{} 0}
         \PY{c+cm}{ */}
         \PY{k+kt}{bool} \PY{n+nf}{est\PYZus{}bissextile}\PY{p}{(}\PY{k+kt}{int} \PY{n}{annee}\PY{p}{)} \PY{p}{\PYZob{}}
             \PY{n}{annee}\PY{o}{\PYZpc{}}\PY{l+m+mi}{400} \PY{o}{=}\PY{o}{=} \PY{l+m+mi}{0}\PY{p}{;}
             \PY{c+c1}{// Attention : on n\PYZsq{}utilisera pas de conditionnelle,}
             \PY{c+c1}{// seulement les opérateurs logiques.}
             \PY{k}{return} \PY{n}{annee}\PY{o}{\PYZpc{}}\PY{l+m+mi}{400} \PY{o}{=}\PY{o}{=} \PY{l+m+mi}{0}\PY{p}{;}
         \PY{p}{\PYZcb{}}
         
         
         \PY{c+c1}{////////////////////////////////////////////////////////////////////////////////}
         \PY{c+c1}{//                                                                            //}
         \PY{c+c1}{//                    NE PAS MODIFIER CE QUI SUIT...                          //}
         \PY{c+c1}{//                                                                            //}
         \PY{c+c1}{////////////////////////////////////////////////////////////////////////////////}
         
         
         \PY{k+kt}{void} \PY{n+nf}{test\PYZus{}chiffre\PYZus{}unites}\PY{p}{(}\PY{k+kt}{void}\PY{p}{)} \PY{p}{\PYZob{}}
             \PY{n}{assert}\PY{p}{(}\PY{l+m+mi}{5} \PY{o}{=}\PY{o}{=} \PY{n}{chiffre\PYZus{}unites}\PY{p}{(}\PY{l+m+mi}{1515}\PY{p}{)}\PY{p}{)}\PY{p}{;}
             \PY{n}{assert}\PY{p}{(}\PY{l+m+mi}{2} \PY{o}{=}\PY{o}{=} \PY{n}{chiffre\PYZus{}unites}\PY{p}{(}\PY{l+m+mi}{142}\PY{p}{)}\PY{p}{)}\PY{p}{;}
             \PY{n}{assert}\PY{p}{(}\PY{l+m+mi}{0} \PY{o}{=}\PY{o}{=} \PY{n}{chiffre\PYZus{}unites}\PY{p}{(}\PY{l+m+mi}{0}\PY{p}{)}\PY{p}{)}\PY{p}{;}
             \PY{n}{printf}\PY{p}{(}\PY{l+s}{\PYZdq{}}\PY{l+s}{\PYZpc{}s}\PY{l+s}{\PYZdq{}}\PY{p}{,} \PY{l+s}{\PYZdq{}}\PY{l+s}{chiffre\PYZus{}unites... ok}\PY{l+s+se}{\PYZbs{}n}\PY{l+s}{\PYZdq{}}\PY{p}{)}\PY{p}{;}
         \PY{p}{\PYZcb{}}
         
         \PY{k+kt}{void} \PY{n+nf}{test\PYZus{}chiffre\PYZus{}dizaines}\PY{p}{(}\PY{k+kt}{void}\PY{p}{)} \PY{p}{\PYZob{}}
             \PY{n}{assert}\PY{p}{(}\PY{l+m+mi}{1} \PY{o}{=}\PY{o}{=} \PY{n}{chiffre\PYZus{}dizaines}\PY{p}{(}\PY{l+m+mi}{1515}\PY{p}{)}\PY{p}{)}\PY{p}{;}
             \PY{n}{assert}\PY{p}{(}\PY{l+m+mi}{4} \PY{o}{=}\PY{o}{=} \PY{n}{chiffre\PYZus{}dizaines}\PY{p}{(}\PY{l+m+mi}{142}\PY{p}{)}\PY{p}{)}\PY{p}{;}
             \PY{n}{assert}\PY{p}{(}\PY{l+m+mi}{9} \PY{o}{=}\PY{o}{=} \PY{n}{chiffre\PYZus{}dizaines}\PY{p}{(}\PY{l+m+mi}{91}\PY{p}{)}\PY{p}{)}\PY{p}{;}
             \PY{n}{assert}\PY{p}{(}\PY{l+m+mi}{8} \PY{o}{=}\PY{o}{=} \PY{n}{chiffre\PYZus{}dizaines}\PY{p}{(}\PY{l+m+mi}{80}\PY{p}{)}\PY{p}{)}\PY{p}{;}
             \PY{n}{assert}\PY{p}{(}\PY{l+m+mi}{0} \PY{o}{=}\PY{o}{=} \PY{n}{chiffre\PYZus{}dizaines}\PY{p}{(}\PY{l+m+mi}{7}\PY{p}{)}\PY{p}{)}\PY{p}{;}
             \PY{n}{assert}\PY{p}{(}\PY{l+m+mi}{0} \PY{o}{=}\PY{o}{=} \PY{n}{chiffre\PYZus{}dizaines}\PY{p}{(}\PY{l+m+mi}{0}\PY{p}{)}\PY{p}{)}\PY{p}{;}
             \PY{n}{printf}\PY{p}{(}\PY{l+s}{\PYZdq{}}\PY{l+s}{\PYZpc{}s}\PY{l+s}{\PYZdq{}}\PY{p}{,} \PY{l+s}{\PYZdq{}}\PY{l+s}{chiffre\PYZus{}dizaines... ok}\PY{l+s+se}{\PYZbs{}n}\PY{l+s}{\PYZdq{}}\PY{p}{)}\PY{p}{;}
         \PY{p}{\PYZcb{}}
         
         
         \PY{k+kt}{void} \PY{n+nf}{test\PYZus{}annee\PYZus{}bissextile}\PY{p}{(}\PY{k+kt}{void}\PY{p}{)} \PY{p}{\PYZob{}}
             \PY{c+c1}{// cas simples}
             \PY{n}{assert}\PY{p}{(}\PY{o}{!} \PY{n}{est\PYZus{}bissextile}\PY{p}{(}\PY{l+m+mi}{2019}\PY{p}{)}\PY{p}{)}\PY{p}{;}
             \PY{n}{assert}\PY{p}{(}\PY{n}{est\PYZus{}bissextile}\PY{p}{(}\PY{l+m+mi}{2020}\PY{p}{)}\PY{p}{)}\PY{p}{;}
             \PY{n}{assert}\PY{p}{(}\PY{n}{est\PYZus{}bissextile}\PY{p}{(}\PY{l+m+mi}{2016}\PY{p}{)}\PY{p}{)}\PY{p}{;}
         
             \PY{c+c1}{// multiples de 100}
             \PY{n}{assert}\PY{p}{(}\PY{o}{!} \PY{n}{est\PYZus{}bissextile}\PY{p}{(}\PY{l+m+mi}{1900}\PY{p}{)}\PY{p}{)}\PY{p}{;}
             \PY{n}{assert}\PY{p}{(}\PY{o}{!} \PY{n}{est\PYZus{}bissextile}\PY{p}{(}\PY{l+m+mi}{2100}\PY{p}{)}\PY{p}{)}\PY{p}{;}
         
             \PY{c+c1}{// multiples de 400}
             \PY{n}{assert}\PY{p}{(}\PY{n}{est\PYZus{}bissextile}\PY{p}{(}\PY{l+m+mi}{1600}\PY{p}{)}\PY{p}{)}\PY{p}{;}
             \PY{n}{assert}\PY{p}{(}\PY{n}{est\PYZus{}bissextile}\PY{p}{(}\PY{l+m+mi}{2000}\PY{p}{)}\PY{p}{)}\PY{p}{;}
             \PY{n}{assert}\PY{p}{(}\PY{n}{est\PYZus{}bissextile}\PY{p}{(}\PY{l+m+mi}{2400}\PY{p}{)}\PY{p}{)}\PY{p}{;}
         
             \PY{n}{printf}\PY{p}{(}\PY{l+s}{\PYZdq{}}\PY{l+s}{\PYZpc{}s}\PY{l+s}{\PYZdq{}}\PY{p}{,} \PY{l+s}{\PYZdq{}}\PY{l+s}{annee\PYZus{}bissextile... ok}\PY{l+s+se}{\PYZbs{}n}\PY{l+s}{\PYZdq{}}\PY{p}{)}\PY{p}{;}
         \PY{p}{\PYZcb{}}
         
         
         \PY{k+kt}{int} \PY{n+nf}{main}\PY{p}{(}\PY{k+kt}{void}\PY{p}{)} \PY{p}{\PYZob{}}
             \PY{n}{test\PYZus{}chiffre\PYZus{}unites}\PY{p}{(}\PY{p}{)}\PY{p}{;}
             \PY{n}{test\PYZus{}chiffre\PYZus{}dizaines}\PY{p}{(}\PY{p}{)}\PY{p}{;}
             \PY{n}{test\PYZus{}annee\PYZus{}bissextile}\PY{p}{(}\PY{p}{)}\PY{p}{;}
             \PY{n}{printf}\PY{p}{(}\PY{l+s}{\PYZdq{}}\PY{l+s}{\PYZpc{}s}\PY{l+s}{\PYZdq{}}\PY{p}{,} \PY{l+s}{\PYZdq{}}\PY{l+s}{Bravo ! Tous les tests passent.}\PY{l+s+se}{\PYZbs{}n}\PY{l+s}{\PYZdq{}}\PY{p}{)}\PY{p}{;}
             \PY{k}{return} \PY{n}{EXIT\PYZus{}SUCCESS}\PY{p}{;}
         \PY{p}{\PYZcb{}}
\end{Verbatim}


    \begin{Verbatim}[commandchars=\\\{\}]
chiffre\_unites{\ldots} ok
chiffre\_dizaines{\ldots} ok

    \end{Verbatim}

    \begin{Verbatim}[commandchars=\\\{\}]
tmp5e9b452w.out: /tmp/tmpi1dsmdgn.c:77: test\_annee\_bissextile: Assertion `est\_bissextile(2020)' failed.
[C kernel] Executable exited with code -6
    \end{Verbatim}

    \begin{center}\rule{0.5\linewidth}{\linethickness}\end{center}

    \subsection{\#\# Entrées et sorties en Langage
C}\label{entruxe9es-et-sorties-en-langage-c}

    \subsubsection{Définition}\label{duxe9finition}

\begin{itemize}
\tightlist
\item
  Les entrées sont des instructions qui permettent de lire des données
  provenant de l'environnement d'exécution (clavier, souris, capteur,
  fichier, réseau, etc.).\\
\item
  Les sorties sont des instructions qui permettent de transférer des
  données à l'environnement d'exécution (moniteur, actuateur, fichier,
  réseau, etc.).
\end{itemize}

Cette partie présente l'utilisation des \textbf{entrées clavier ou des
sorties moniteur}. Les autres types de périphériques seront abordés dans
d'autres enseignements.

\begin{quote}
Attention : \textbf{la gestion des entrées/sorties en C n'est pas
triviale !}
\end{quote}

    \subsubsection{Flux de données}\label{flux-de-donnuxe9es}

La notion de flux de données est fondamentale. Les flux permettent
d'interagir avec les périphériques pour échanger des données. Un flux
est une file d'attente de type FIFO (\emph{fist in first out}) : - la
données la plus ancienne peut être lue. Elle est alors consommée
(supprimée de la file), - la donnée la plus récente est insérée en fin
de file. On dit qu'elle est écrite dans la file.

En C, on peut manipuler des flux qui enregistrent des données de deux
types : - \textbf{texte} : on y enregistre une suite de caractères,
séparés par des retour-chariots, - \textbf{binaire} : on y enregistre
une suite d'octets.

Il existe des files sont définies par défaut : - \texttt{FILE*\ STDOUT}
: flux de sortie vers le moniteur - \texttt{FILE*\ STDIN}: flux d'entree
depuis le clavier Elles sont toutes de type \texttt{FILE*}

\begin{quote}
Note : - Ces files sont définies dans le module \texttt{stdlib} - Les
sous-programmes mentionnés par la suite sont définis dans le module
\texttt{stdio}
\end{quote}

\begin{quote}
Remarque : - En 1SN nous ne traiterons que les entrées sorties en mode
\textbf{texte}
\end{quote}

    \subsubsection{Ecrire les sorties}\label{ecrire-les-sorties}

    \paragraph{Sorties formattés}\label{sorties-formattuxe9s}

L'objectif du sous-programme \texttt{printf()} est d'écrire des données
typées à l'écran. L'objectif est d'afficher en une seule instruction le
contenu de variables de type entier, flottant, chaine de caractères,
voire une combinaison hétérogène de variables.

Le sous-programme \texttt{printf} du module \texttt{stdio} est définit
comme suit :

\begin{quote}
\texttt{int\ printf("format",\ param1,\ param2,\ etc.);}
\end{quote}

La chaîne "format" est une chaîne de caractères, parsemée de
\textbf{spécificateurs de format}.

\textbf{Un spécificateur de format commence par le caractère \%}. Il y
autant de spécificateurs de format que de paramètres. A l'exécution, -
le 1er spécificateur est remplacé par la valeur du 1er paramètre, - le
2er spécificateur est remplacé par la valeur du 2er paramètre, - etc.

Le spécificateur indique comment afficher la variable qui lui correspond
: - \%d ou \%i : indique à printf que l'on souhaite afficher le
paramètre comme un entier signé - \%u : indique à printf que l'on
souhaite afficher le paramètre comme un entier \_\_non\_\_signé - \%f,
\%lf, \%Lf : indique à printf que l'on souhaite afficher le paramètre
comme un float, double ou long double. Il est possible de limiter le
nombre de décimales : \%1.3f limite le nombre de décimales à 3. - \%c :
indique à printf que l'on souhaite afficher le paramètre comme un
caractère - \%s : indique à printf que l'on souhaite afficher le
paramètre comme une chaîne de caractères - \%p : indique à printf que
l'on souhaite afficher le paramètre comme une adresse - etc.

\begin{quote}
\textbf{ATTENTION} : le compilateur ne vérifie pas forcément la
cohérence entre le spécificateur et le type du paramètre correspondant !
Des warnings sont généralement observés.
\end{quote}

\textbf{Exemples} (fichier \textbf{\texttt{exemple\_ES.c}}):

    \begin{Verbatim}[commandchars=\\\{\}]
{\color{incolor}In [{\color{incolor}25}]:} \PY{c+cp}{\PYZsh{}}\PY{c+cp}{include} \PY{c+cpf}{\PYZlt{}stdio.h\PYZgt{}}
         \PY{c+cp}{\PYZsh{}}\PY{c+cp}{include} \PY{c+cpf}{\PYZlt{}stdlib.h\PYZgt{}}
         
         \PY{k+kt}{int} \PY{n+nf}{main}\PY{p}{(}\PY{p}{)}\PY{p}{\PYZob{}}
             \PY{k+kt}{float} \PY{n}{cote} \PY{o}{=} \PY{l+m+mf}{2.0}\PY{p}{;}     \PY{c+c1}{//longueur du côté}
             \PY{k+kt}{char} \PY{n}{unite} \PY{o}{=} \PY{l+s+sc}{\PYZsq{}}\PY{l+s+sc}{m}\PY{l+s+sc}{\PYZsq{}}\PY{p}{;}
             \PY{n}{printf}\PY{p}{(}\PY{l+s}{\PYZdq{}}\PY{l+s}{Le périmètre du carré de côté \PYZpc{}1.0f\PYZpc{}c est : }\PY{l+s}{\PYZdq{}}\PY{p}{,} \PY{n}{cote}\PY{p}{,} \PY{n}{unite}\PY{p}{)}\PY{p}{;} 
             \PY{c+c1}{//affichage du flottant avec 0 chiffres après la virgule}
             
             \PY{c+c1}{//calcul et affichage du perimètre}
             \PY{k+kt}{float} \PY{n}{perimetre} \PY{o}{=} \PY{l+m+mi}{4} \PY{o}{*} \PY{n}{cote}\PY{p}{;} 
             \PY{n}{printf}\PY{p}{(}\PY{l+s}{\PYZdq{}}\PY{l+s}{\PYZpc{}1.2f\PYZpc{}c}\PY{l+s+se}{\PYZbs{}n}\PY{l+s}{\PYZdq{}}\PY{p}{,} \PY{n}{perimetre}\PY{p}{,} \PY{n}{unite}\PY{p}{)}\PY{p}{;} 
             \PY{c+c1}{//affichage du flottant avec 2 chiffres après la virgule}
             
             \PY{k}{return} \PY{n}{EXIT\PYZus{}SUCCESS}\PY{p}{;}
         \PY{p}{\PYZcb{}}
\end{Verbatim}


    \begin{Verbatim}[commandchars=\\\{\}]
Le périmètre du carré de côté 2m est : 8.00m

    \end{Verbatim}

    \begin{quote}
\emph{Note} : Il existe d'autres sous-programmes d'écriture qui ne
seront pas présentés ici : \texttt{putchar()}, \texttt{fputc()},
\texttt{sprintf()}, \texttt{fprintf()}.
\end{quote}

    \subsubsection{Lire les entrées}\label{lire-les-entruxe9es}

    \paragraph{Entrées formattés}\label{entruxe9es-formattuxe9s}

L'objectif du sous-programme \texttt{scanf()} est de lire des données
typées depuis le clavier. Le sous-programme \texttt{scanf} du module
\texttt{stdio} est définit comme suit :

\begin{quote}
\texttt{int\ scanf("format",\ \&param1,\ \&param2,\ etc.);}
\end{quote}

La chaîne "format" ne comporte \textbf{principalement des spécificateurs
de formats}. Chaque format fait référence à un des paramètres, dans
l'ordre d'apparition. Le spécificateur indique comment lire la variable
qui lui correspond : - \%d ou \%i : indique à \texttt{scanf} qu'il doit
lire un entier - \%f : indique à scanf qu'il doit lire un float - etc.

La donnée lue est écrite à \textbf{l'adresse} de \texttt{param1},
\texttt{param2}. etc.

\begin{quote}
Si dans le "format" on insère un espace entre deux spécificateurs, tous
les caractères 'blancs' (espace, tabulation) sont consommés mais non
interprétés.
\end{quote}

L'entier retourné par \texttt{scanf} représente le nombre de paramètres
lus avec succès.

\textbf{Exemples} :

\begin{Shaded}
\begin{Highlighting}[]
\CommentTok{// Lire un entier}
\DataTypeTok{int}\NormalTok{ monentier;}
\NormalTok{scanf(}\StringTok{"%i"}\NormalTok{, &monentier);}
\CommentTok{// Lire un flottant avec 2 décimales maximum.}
\DataTypeTok{float}\NormalTok{ monfloat;}
\NormalTok{scanf(}\StringTok{"%1.2f"}\NormalTok{, &monfloat); }
\CommentTok{// Lire deux caracteres d'affilée non blancs}
\DataTypeTok{char}\NormalTok{ c1, c2;}
\NormalTok{scanf(}\StringTok{"%c %c"}\NormalTok{, &c1, &c2);}
\end{Highlighting}
\end{Shaded}

\begin{quote}
\emph{Note} : Il existe d'autres sous-programmes de lecture des entrées
qui ne seront pas présentés ici : \texttt{getchar()}, \texttt{fgetc()},
\texttt{sscanf()}, \texttt{fscanf()}.
\end{quote}

    \begin{center}\rule{0.5\linewidth}{\linethickness}\end{center}

    \subsection{\#\# Les structures de
contrôle}\label{les-structures-de-contruxf4le}

    Elles permettent de contrôle l'ordre d'exécution des instructions. En C,
il existe - La séquence - Les structures conditionnelles :\\
- \texttt{if\ ...\ then\ ...\ else} - \texttt{switch\ ...\ case\ ...} -
Les boucles : - Répéter : \texttt{do\ ...\ while} - TantQue :
\texttt{while\ ...} - Pour : \texttt{for\ ...}

    \begin{center}\rule{0.5\linewidth}{\linethickness}\end{center}

    \subsection{\#\# Les conditionelles}\label{les-conditionelles}

    \textbf{1. La conditionnelle simple :}

\begin{Shaded}
\begin{Highlighting}[]
\ControlFlowTok{if}\NormalTok{ (cond) \{}
\NormalTok{    sequence1}
\NormalTok{\} }\ControlFlowTok{else}\NormalTok{ \{}
\NormalTok{    sequence2}
\NormalTok{\}}
\end{Highlighting}
\end{Shaded}

Si la condition \texttt{cond} est vraie, \texttt{séquenc1} est exécutée,
sinon, \texttt{sequence2} est exécutée.

La clause \texttt{SinonSi} n'existe pas, on imbrique les conditionnelles
pour introduire une étape de sélection supplémentaire :

\begin{Shaded}
\begin{Highlighting}[]
\ControlFlowTok{if}\NormalTok{ (cond1) \{}
\NormalTok{    sequence1}
\NormalTok{\} }\ControlFlowTok{else} \ControlFlowTok{if}\NormalTok{ (cond2) \{}
\NormalTok{    sequence2}
\NormalTok{\} }\ControlFlowTok{else}\NormalTok{ \{}
\NormalTok{    sequence3}
\NormalTok{\}}
\end{Highlighting}
\end{Shaded}

\texttt{sequence3} est exécuté si \texttt{cond1} et \texttt{cond2} sont
fausses.

    \subsection{\#\#\# Exercice 8 - Ecrire des
conditionnelles.}\label{exercice-8---ecrire-des-conditionnelles.}

\begin{center}\rule{0.5\linewidth}{\linethickness}\end{center}

\textbf{{[}8.1{]}} Compléter et corriger le corps des fonctions
ci-dessous (voir TODO).

    \begin{Verbatim}[commandchars=\\\{\}]
{\color{incolor}In [{\color{incolor}26}]:} \PY{c+cp}{\PYZsh{}}\PY{c+cp}{include} \PY{c+cpf}{\PYZlt{}stdlib.h\PYZgt{}}
         \PY{c+cp}{\PYZsh{}}\PY{c+cp}{include} \PY{c+cpf}{\PYZlt{}stdio.h\PYZgt{}}
         \PY{c+cp}{\PYZsh{}}\PY{c+cp}{include} \PY{c+cpf}{\PYZlt{}assert.h\PYZgt{}}
         
         \PY{c+cm}{/**}
         \PY{c+cm}{ * \PYZbs{}brief Retourner \PYZsq{}\PYZlt{}\PYZsq{}, \PYZsq{}\PYZgt{}\PYZsq{} ou \PYZsq{}=\PYZsq{} pour indiquer si n est strictement négatif,}
         \PY{c+cm}{ * strictement positif ou nul.}
         \PY{c+cm}{ * \PYZbs{}param[in] nombre le nombre dont on veut évaluer le signe}
         \PY{c+cm}{ * \PYZbs{}return un caractère donnant le signe d\PYZsq{}un nombre}
         \PY{c+cm}{ */}
         \PY{k+kt}{char} \PY{n+nf}{signe}\PY{p}{(}\PY{k+kt}{int} \PY{n}{nombre}\PY{p}{)}
         \PY{p}{\PYZob{}}
             \PY{k}{if} \PY{p}{(}\PY{n}{nombre} \PY{o}{\PYZgt{}} \PY{l+m+mi}{0}\PY{p}{)} \PY{p}{\PYZob{}}
                 \PY{k}{return} \PY{l+s+sc}{\PYZsq{}}\PY{l+s+sc}{\PYZgt{}}\PY{l+s+sc}{\PYZsq{}}\PY{p}{;}
             \PY{p}{\PYZcb{}} \PY{k}{else} \PY{k}{if} \PY{p}{(}\PY{n}{nombre} \PY{o}{\PYZlt{}}\PY{l+m+mi}{0}\PY{p}{)} \PY{p}{\PYZob{}}
                 \PY{k}{return} \PY{l+s+sc}{\PYZsq{}}\PY{l+s+sc}{\PYZlt{}}\PY{l+s+sc}{\PYZsq{}}\PY{p}{;}
             \PY{p}{\PYZcb{}} \PY{k}{else} \PY{p}{\PYZob{}}
                 \PY{k}{return} \PY{l+s+sc}{\PYZsq{}}\PY{l+s+sc}{=}\PY{l+s+sc}{\PYZsq{}}\PY{p}{;}
             \PY{p}{\PYZcb{}}
              
         \PY{p}{\PYZcb{}}
         
         
         \PY{c+c1}{////////////////////////////////////////////////////////////////////////////////}
         \PY{c+c1}{//                                                                            //}
         \PY{c+c1}{//                    NE PAS MODIFIER CE QUI SUIT...                          //}
         \PY{c+c1}{//                                                                            //}
         \PY{c+c1}{////////////////////////////////////////////////////////////////////////////////}
         
         \PY{k+kt}{void} \PY{n+nf}{test\PYZus{}signe}\PY{p}{(}\PY{p}{)} \PY{p}{\PYZob{}}
             \PY{n}{assert}\PY{p}{(}\PY{l+s+sc}{\PYZsq{}}\PY{l+s+sc}{\PYZlt{}}\PY{l+s+sc}{\PYZsq{}} \PY{o}{=}\PY{o}{=} \PY{n}{signe}\PY{p}{(}\PY{l+m+mi}{\PYZhy{}821}\PY{p}{)}\PY{p}{)}\PY{p}{;}
             \PY{n}{assert}\PY{p}{(}\PY{l+s+sc}{\PYZsq{}}\PY{l+s+sc}{\PYZlt{}}\PY{l+s+sc}{\PYZsq{}} \PY{o}{=}\PY{o}{=} \PY{n}{signe}\PY{p}{(}\PY{l+m+mi}{\PYZhy{}1}\PY{p}{)}\PY{p}{)}\PY{p}{;}
             \PY{n}{assert}\PY{p}{(}\PY{l+s+sc}{\PYZsq{}}\PY{l+s+sc}{=}\PY{l+s+sc}{\PYZsq{}} \PY{o}{=}\PY{o}{=} \PY{n}{signe}\PY{p}{(}\PY{l+m+mi}{0}\PY{p}{)}\PY{p}{)}\PY{p}{;}
             \PY{n}{assert}\PY{p}{(}\PY{l+s+sc}{\PYZsq{}}\PY{l+s+sc}{\PYZgt{}}\PY{l+s+sc}{\PYZsq{}} \PY{o}{=}\PY{o}{=} \PY{n}{signe}\PY{p}{(}\PY{l+m+mi}{125}\PY{p}{)}\PY{p}{)}\PY{p}{;}
             \PY{n}{assert}\PY{p}{(}\PY{l+s+sc}{\PYZsq{}}\PY{l+s+sc}{\PYZgt{}}\PY{l+s+sc}{\PYZsq{}} \PY{o}{=}\PY{o}{=} \PY{n}{signe}\PY{p}{(}\PY{l+m+mi}{1}\PY{p}{)}\PY{p}{)}\PY{p}{;}
             \PY{n}{printf}\PY{p}{(}\PY{l+s}{\PYZdq{}}\PY{l+s}{\PYZpc{}s}\PY{l+s}{\PYZdq{}}\PY{p}{,} \PY{l+s}{\PYZdq{}}\PY{l+s}{signe... ok}\PY{l+s+se}{\PYZbs{}n}\PY{l+s}{\PYZdq{}}\PY{p}{)}\PY{p}{;}
         \PY{p}{\PYZcb{}}
         
         
         \PY{k+kt}{int} \PY{n+nf}{main}\PY{p}{(}\PY{k+kt}{void}\PY{p}{)} \PY{p}{\PYZob{}}
             \PY{n}{test\PYZus{}signe}\PY{p}{(}\PY{p}{)}\PY{p}{;}
             \PY{n}{printf}\PY{p}{(}\PY{l+s}{\PYZdq{}}\PY{l+s}{\PYZpc{}s}\PY{l+s}{\PYZdq{}}\PY{p}{,} \PY{l+s}{\PYZdq{}}\PY{l+s}{Bravo ! Tous les tests passent.}\PY{l+s+se}{\PYZbs{}n}\PY{l+s}{\PYZdq{}}\PY{p}{)}\PY{p}{;}
             \PY{k}{return} \PY{n}{EXIT\PYZus{}SUCCESS}\PY{p}{;}
         \PY{p}{\PYZcb{}}
\end{Verbatim}


    \begin{Verbatim}[commandchars=\\\{\}]
signe{\ldots} ok
Bravo ! Tous les tests passent.

    \end{Verbatim}

    \textbf{2. La conditionelle multiple :} Elle s'exprime avec la structure
de contrôle \texttt{switch\ ..\ case}. Elle suit la syntaxe suivante :

\begin{Shaded}
\begin{Highlighting}[]
\ControlFlowTok{switch}\NormalTok{ (expr) \{}
    \ControlFlowTok{case}\NormalTok{ choix1 : }
\NormalTok{        sequence1;}
        \ControlFlowTok{break}\NormalTok{;}
    \ControlFlowTok{case}\NormalTok{ choix2 : }
\NormalTok{        sequence2;}
        \ControlFlowTok{break}\NormalTok{;}
    \ControlFlowTok{case} \ControlFlowTok{default}\NormalTok{ : }
\NormalTok{        sequence_def;}
\NormalTok{\}}
\end{Highlighting}
\end{Shaded}

A l'exécution:

\begin{enumerate}
\def\labelenumi{\arabic{enumi}.}
\tightlist
\item
  \texttt{(expr)} est évalué. \texttt{(expr)} est une expression de type
  \textbf{discret} (entier, booléen ou caractère)
\item
  L'exécution se poursuit au niveau du \texttt{case} qui correspond a la
  valeur de \texttt{(expr)} ou au niveau du \texttt{default} si aucune
  correspondance n'est trouvée.
\end{enumerate}

Autrement dit, si . - Si \texttt{expr\ ==\ choix1}, toutes les
instructions sont exécutées \textbf{à partir de sequence1}. - Si
\texttt{expr\ ==\ choix2}, toutes les instructions sont exécutées
\textbf{à partir de sequence2}. - Si
\texttt{expr\ !=\ choix1\ \&\&\ expr\ !=\ choix2}, sequence\_def est
exécuté.

Si l'instruction \textbf{break;} est rencontrée, les instructions
suivantes du bloc \texttt{switch} ne sont jamais exécutées.

\begin{quote}
Note : Il est important d'utiliser l'instruction \textbf{break;} pour
n'exécuter qu'une séquence par choix possible pour retrouver le
comportement algorithmique d'un \texttt{Selon\ ..\ Dans}.
\end{quote}

    \subsection{\texorpdfstring{\#\#\# Exercice 9 - Comprendre le
\texttt{switch\ ...\ case}}{\#\#\# Exercice 9 - Comprendre le switch ... case}}\label{exercice-9---comprendre-le-switch-...-case}

\begin{center}\rule{0.5\linewidth}{\linethickness}\end{center}

    \textbf{{[}9.1{]}} Dans la fonction \texttt{test\_f} (cf fichier
\textbf{\texttt{exercice9.c}}) du programme suivant, remplacer XXX par
la valeur qui sera retournée par l'appel correspondant à la fonction
\texttt{f}.

    \begin{Verbatim}[commandchars=\\\{\}]
{\color{incolor}In [{\color{incolor}27}]:} \PY{c+cp}{\PYZsh{}}\PY{c+cp}{include} \PY{c+cpf}{\PYZlt{}stdlib.h\PYZgt{}}
         \PY{c+cp}{\PYZsh{}}\PY{c+cp}{include} \PY{c+cpf}{\PYZlt{}stdio.h\PYZgt{}}
         \PY{c+cp}{\PYZsh{}}\PY{c+cp}{include} \PY{c+cpf}{\PYZlt{}assert.h\PYZgt{}}
         
         \PY{c+cp}{\PYZsh{}}\PY{c+cp}{define XXX \PYZhy{}1}
         
         \PY{c+c1}{// Une fonction f qui retourne un entier en fonction du paramètre n fourni.}
         \PY{k+kt}{int} \PY{n+nf}{f}\PY{p}{(}\PY{k+kt}{int} \PY{n}{n}\PY{p}{)} \PY{p}{\PYZob{}}
             \PY{k+kt}{int} \PY{n}{r} \PY{o}{=} \PY{l+m+mi}{0}\PY{p}{;}
         
             \PY{c+c1}{// modifier r}
             \PY{k}{switch} \PY{p}{(}\PY{n}{n}\PY{p}{)} \PY{p}{\PYZob{}}
             \PY{k}{case} \PY{l+m+mi}{1}\PY{o}{:}
                 \PY{n}{r} \PY{o}{+}\PY{o}{=} \PY{l+m+mi}{1}\PY{p}{;}
                 \PY{k}{break}\PY{p}{;}
             \PY{k}{case} \PY{l+m+mi}{2}\PY{o}{:}
             \PY{k}{case} \PY{l+m+mi}{3}\PY{o}{:}
                 \PY{n}{r} \PY{o}{+}\PY{o}{=} \PY{l+m+mi}{8}\PY{p}{;}
                 \PY{k}{break}\PY{p}{;}
             \PY{k}{case} \PY{l+m+mi}{4}\PY{o}{:}
             \PY{k}{case} \PY{l+m+mi}{5}\PY{o}{:}
             \PY{k}{case} \PY{l+m+mi}{7}\PY{o}{:}
                 \PY{n}{r} \PY{o}{+}\PY{o}{=} \PY{l+m+mi}{10}\PY{p}{;}
             \PY{k}{case} \PY{l+m+mi}{10}\PY{o}{:}
             \PY{k}{case} \PY{l+m+mi}{11}\PY{o}{:}
                 \PY{n}{r} \PY{o}{+}\PY{o}{=} \PY{l+m+mi}{5}\PY{p}{;}
                 \PY{k}{break}\PY{p}{;}
             \PY{k}{case} \PY{l+m+mi}{12}\PY{o}{:}
                 \PY{n}{r} \PY{o}{+}\PY{o}{=} \PY{l+m+mi}{50}\PY{p}{;}
                 \PY{k}{break}\PY{p}{;}
             \PY{k}{case} \PY{l+m+mi}{13}\PY{o}{:}
                 \PY{n}{r} \PY{o}{+}\PY{o}{=} \PY{l+m+mi}{100}\PY{p}{;}
             \PY{k}{default}\PY{o}{:}
                 \PY{n}{r} \PY{o}{\PYZhy{}}\PY{o}{=} \PY{l+m+mi}{1}\PY{p}{;}
             \PY{p}{\PYZcb{}}
         
             \PY{k}{return} \PY{n}{r}\PY{p}{;}
         \PY{p}{\PYZcb{}}
         
         \PY{k+kt}{void} \PY{n+nf}{test\PYZus{}f}\PY{p}{(}\PY{k+kt}{void}\PY{p}{)}
         \PY{p}{\PYZob{}}
             \PY{n}{assert}\PY{p}{(}\PY{l+m+mi}{8} \PY{o}{=}\PY{o}{=} \PY{n}{f}\PY{p}{(}\PY{l+m+mi}{3}\PY{p}{)}\PY{p}{)}\PY{p}{;}
             \PY{n}{assert}\PY{p}{(}\PY{l+m+mi}{\PYZhy{}1} \PY{o}{=}\PY{o}{=} \PY{n}{f}\PY{p}{(}\PY{l+m+mi}{\PYZhy{}5}\PY{p}{)}\PY{p}{)}\PY{p}{;}
             \PY{n}{assert}\PY{p}{(}\PY{l+m+mi}{\PYZhy{}1} \PY{o}{=}\PY{o}{=} \PY{n}{f}\PY{p}{(}\PY{l+m+mi}{0}\PY{p}{)}\PY{p}{)}\PY{p}{;}
             \PY{n}{assert}\PY{p}{(}\PY{l+m+mi}{50} \PY{o}{=}\PY{o}{=} \PY{n}{f}\PY{p}{(}\PY{l+m+mi}{12}\PY{p}{)}\PY{p}{)}\PY{p}{;}
             \PY{n}{assert}\PY{p}{(}\PY{l+m+mi}{99} \PY{o}{=}\PY{o}{=} \PY{n}{f}\PY{p}{(}\PY{l+m+mi}{13}\PY{p}{)}\PY{p}{)}\PY{p}{;}
             \PY{n}{assert}\PY{p}{(}\PY{l+m+mi}{8} \PY{o}{=}\PY{o}{=} \PY{n}{f}\PY{p}{(}\PY{l+m+mi}{2}\PY{p}{)}\PY{p}{)}\PY{p}{;}
             \PY{n}{assert}\PY{p}{(}\PY{l+m+mi}{5} \PY{o}{=}\PY{o}{=} \PY{n}{f}\PY{p}{(}\PY{l+m+mi}{10}\PY{p}{)}\PY{p}{)}\PY{p}{;}
             \PY{n}{assert}\PY{p}{(}\PY{l+m+mi}{15} \PY{o}{=}\PY{o}{=} \PY{n}{f}\PY{p}{(}\PY{l+m+mi}{5}\PY{p}{)}\PY{p}{)}\PY{p}{;}
         \PY{p}{\PYZcb{}}
         
         \PY{k+kt}{int} \PY{n+nf}{main}\PY{p}{(}\PY{k+kt}{void}\PY{p}{)} \PY{p}{\PYZob{}}
             \PY{n}{test\PYZus{}f}\PY{p}{(}\PY{p}{)}\PY{p}{;}
             \PY{n}{printf}\PY{p}{(}\PY{l+s}{\PYZdq{}}\PY{l+s}{\PYZpc{}s}\PY{l+s}{\PYZdq{}}\PY{p}{,} \PY{l+s}{\PYZdq{}}\PY{l+s}{Bravo !  Pas d\PYZsq{}erreur détectée.}\PY{l+s+se}{\PYZbs{}n}\PY{l+s}{\PYZdq{}}\PY{p}{)}\PY{p}{;}
             \PY{k}{return} \PY{n}{EXIT\PYZus{}SUCCESS}\PY{p}{;}
         \PY{p}{\PYZcb{}}
\end{Verbatim}


    \begin{Verbatim}[commandchars=\\\{\}]
Bravo !  Pas d'erreur détectée.

    \end{Verbatim}

    \subsection{\texorpdfstring{\#\#\# Exercice 10 - Ecrire un
\texttt{switch\ ..\ case}}{\#\#\# Exercice 10 - Ecrire un switch .. case}}\label{exercice-10---ecrire-un-switch-..-case}

\begin{center}\rule{0.5\linewidth}{\linethickness}\end{center}

    \textbf{{[}10.1{]}} Compléter et corriger le corps de la fonction
\texttt{nb\_jours\_mois} (cf fichier \textbf{\texttt{exercice10.c}})
ci-dessous (voir TODO).

    \begin{Verbatim}[commandchars=\\\{\}]
{\color{incolor}In [{\color{incolor}28}]:} \PY{c+cp}{\PYZsh{}}\PY{c+cp}{include} \PY{c+cpf}{\PYZlt{}stdlib.h\PYZgt{}}
         \PY{c+cp}{\PYZsh{}}\PY{c+cp}{include} \PY{c+cpf}{\PYZlt{}stdio.h\PYZgt{}}
         \PY{c+cp}{\PYZsh{}}\PY{c+cp}{include} \PY{c+cpf}{\PYZlt{}assert.h\PYZgt{}}
         
         \PY{c+cm}{/**}
         \PY{c+cm}{ * \PYZbs{}brief Obtenir le nombres de jour d\PYZsq{}un mois d\PYZsq{}une année non bissextile.}
         \PY{c+cm}{ * \PYZbs{}param[in] mois le mois considéré (de 1, janvier, à 12, décembre)}
         \PY{c+cm}{ * \PYZbs{}return le nombre de jours du mois considéré}
         \PY{c+cm}{ */}
         \PY{k+kt}{char} \PY{n+nf}{nb\PYZus{}jours\PYZus{}mois}\PY{p}{(}\PY{k+kt}{int} \PY{n}{mois}\PY{p}{)}
         \PY{p}{\PYZob{}}
             \PY{c+c1}{// Contraintes :}
             \PY{c+c1}{//   1. On utilisera un Selon et aucune autre structure de contrôle.}
             \PY{c+c1}{//   2. On fera un seul « return » à la fin de la fonction.}
             \PY{c+c1}{//   3. On n\PYZsq{}utilisera pas de tableau !}
             \PY{k}{switch} \PY{p}{(}\PY{n}{mois}\PY{p}{)} \PY{p}{\PYZob{}}
                 \PY{k}{case} \PY{l+m+mi}{1}\PY{o}{:}
                     \PY{k}{return} \PY{l+m+mi}{31}\PY{p}{;}
                     \PY{k}{break}\PY{p}{;}
                 \PY{k}{case} \PY{l+m+mi}{2}\PY{o}{:}
                     \PY{k}{return} \PY{l+m+mi}{28}\PY{p}{;}
                     \PY{k}{break}\PY{p}{;}
                 \PY{k}{case} \PY{l+m+mi}{3}\PY{o}{:}
                     \PY{k}{return} \PY{l+m+mi}{31}\PY{p}{;}
                     \PY{k}{break}\PY{p}{;}
                 \PY{k}{case} \PY{l+m+mi}{4}\PY{o}{:}
                     \PY{k}{return} \PY{l+m+mi}{30}\PY{p}{;}
                     \PY{k}{break}\PY{p}{;}
                 \PY{k}{case} \PY{l+m+mi}{5}\PY{o}{:}
                     \PY{k}{return} \PY{l+m+mi}{31}\PY{p}{;}
                     \PY{k}{break}\PY{p}{;}
                 \PY{k}{case} \PY{l+m+mi}{6}\PY{o}{:}
                     \PY{k}{return} \PY{l+m+mi}{30}\PY{p}{;}
                     \PY{k}{break}\PY{p}{;}
                 \PY{k}{case} \PY{l+m+mi}{7}\PY{o}{:}
                     \PY{k}{return} \PY{l+m+mi}{31}\PY{p}{;}
                     \PY{k}{break}\PY{p}{;}
                 \PY{k}{case} \PY{l+m+mi}{8}\PY{o}{:}
                     \PY{k}{return} \PY{l+m+mi}{31}\PY{p}{;}
                     \PY{k}{break}\PY{p}{;}
                 \PY{k}{case} \PY{l+m+mi}{9}\PY{o}{:}
                     \PY{k}{return} \PY{l+m+mi}{30}\PY{p}{;}
                     \PY{k}{break}\PY{p}{;}
                 \PY{k}{case} \PY{l+m+mi}{10}\PY{o}{:}
                     \PY{k}{return} \PY{l+m+mi}{31}\PY{p}{;}
                     \PY{k}{break}\PY{p}{;}
                 \PY{k}{case} \PY{l+m+mi}{11}\PY{o}{:}
                     \PY{k}{return} \PY{l+m+mi}{30}\PY{p}{;}
                     \PY{k}{break}\PY{p}{;}
                 \PY{k}{case} \PY{l+m+mi}{12}\PY{o}{:}
                     \PY{k}{return} \PY{l+m+mi}{31}\PY{p}{;}
                     \PY{k}{break}\PY{p}{;}
             \PY{p}{\PYZcb{}}
                 
            
         \PY{p}{\PYZcb{}}
         
         
         \PY{c+c1}{////////////////////////////////////////////////////////////////////////////////}
         \PY{c+c1}{//                                                                            //}
         \PY{c+c1}{//                    NE PAS MODIFIER CE QUI SUIT...                          //}
         \PY{c+c1}{//                                                                            //}
         \PY{c+c1}{////////////////////////////////////////////////////////////////////////////////}
         
         \PY{k+kt}{void} \PY{n+nf}{test\PYZus{}nb\PYZus{}jours\PYZus{}mois}\PY{p}{(}\PY{p}{)} \PY{p}{\PYZob{}}
             \PY{n}{assert}\PY{p}{(}\PY{l+m+mi}{31} \PY{o}{=}\PY{o}{=} \PY{n}{nb\PYZus{}jours\PYZus{}mois}\PY{p}{(}\PY{l+m+mi}{1}\PY{p}{)}\PY{p}{)}\PY{p}{;}
             \PY{n}{assert}\PY{p}{(}\PY{l+m+mi}{28} \PY{o}{=}\PY{o}{=} \PY{n}{nb\PYZus{}jours\PYZus{}mois}\PY{p}{(}\PY{l+m+mi}{2}\PY{p}{)}\PY{p}{)}\PY{p}{;}
             \PY{n}{assert}\PY{p}{(}\PY{l+m+mi}{31} \PY{o}{=}\PY{o}{=} \PY{n}{nb\PYZus{}jours\PYZus{}mois}\PY{p}{(}\PY{l+m+mi}{3}\PY{p}{)}\PY{p}{)}\PY{p}{;}
             \PY{n}{assert}\PY{p}{(}\PY{l+m+mi}{30} \PY{o}{=}\PY{o}{=} \PY{n}{nb\PYZus{}jours\PYZus{}mois}\PY{p}{(}\PY{l+m+mi}{4}\PY{p}{)}\PY{p}{)}\PY{p}{;}
             \PY{n}{assert}\PY{p}{(}\PY{l+m+mi}{31} \PY{o}{=}\PY{o}{=} \PY{n}{nb\PYZus{}jours\PYZus{}mois}\PY{p}{(}\PY{l+m+mi}{5}\PY{p}{)}\PY{p}{)}\PY{p}{;}
             \PY{n}{assert}\PY{p}{(}\PY{l+m+mi}{30} \PY{o}{=}\PY{o}{=} \PY{n}{nb\PYZus{}jours\PYZus{}mois}\PY{p}{(}\PY{l+m+mi}{6}\PY{p}{)}\PY{p}{)}\PY{p}{;}
             \PY{n}{assert}\PY{p}{(}\PY{l+m+mi}{31} \PY{o}{=}\PY{o}{=} \PY{n}{nb\PYZus{}jours\PYZus{}mois}\PY{p}{(}\PY{l+m+mi}{7}\PY{p}{)}\PY{p}{)}\PY{p}{;}
             \PY{n}{assert}\PY{p}{(}\PY{l+m+mi}{31} \PY{o}{=}\PY{o}{=} \PY{n}{nb\PYZus{}jours\PYZus{}mois}\PY{p}{(}\PY{l+m+mi}{8}\PY{p}{)}\PY{p}{)}\PY{p}{;}
             \PY{n}{assert}\PY{p}{(}\PY{l+m+mi}{30} \PY{o}{=}\PY{o}{=} \PY{n}{nb\PYZus{}jours\PYZus{}mois}\PY{p}{(}\PY{l+m+mi}{9}\PY{p}{)}\PY{p}{)}\PY{p}{;}
             \PY{n}{assert}\PY{p}{(}\PY{l+m+mi}{31} \PY{o}{=}\PY{o}{=} \PY{n}{nb\PYZus{}jours\PYZus{}mois}\PY{p}{(}\PY{l+m+mi}{10}\PY{p}{)}\PY{p}{)}\PY{p}{;}
             \PY{n}{assert}\PY{p}{(}\PY{l+m+mi}{30} \PY{o}{=}\PY{o}{=} \PY{n}{nb\PYZus{}jours\PYZus{}mois}\PY{p}{(}\PY{l+m+mi}{11}\PY{p}{)}\PY{p}{)}\PY{p}{;}
             \PY{n}{assert}\PY{p}{(}\PY{l+m+mi}{31} \PY{o}{=}\PY{o}{=} \PY{n}{nb\PYZus{}jours\PYZus{}mois}\PY{p}{(}\PY{l+m+mi}{12}\PY{p}{)}\PY{p}{)}\PY{p}{;}
             \PY{n}{printf}\PY{p}{(}\PY{l+s}{\PYZdq{}}\PY{l+s}{\PYZpc{}s}\PY{l+s}{\PYZdq{}}\PY{p}{,} \PY{l+s}{\PYZdq{}}\PY{l+s}{nb\PYZus{}jours\PYZus{}mois... ok}\PY{l+s+se}{\PYZbs{}n}\PY{l+s}{\PYZdq{}}\PY{p}{)}\PY{p}{;}
         \PY{p}{\PYZcb{}}
         
         
         \PY{k+kt}{int} \PY{n+nf}{main}\PY{p}{(}\PY{k+kt}{void}\PY{p}{)} \PY{p}{\PYZob{}}
             \PY{n}{test\PYZus{}nb\PYZus{}jours\PYZus{}mois}\PY{p}{(}\PY{p}{)}\PY{p}{;}
             \PY{n}{printf}\PY{p}{(}\PY{l+s}{\PYZdq{}}\PY{l+s}{\PYZpc{}s}\PY{l+s}{\PYZdq{}}\PY{p}{,} \PY{l+s}{\PYZdq{}}\PY{l+s}{Bravo ! Tous les tests passent.}\PY{l+s+se}{\PYZbs{}n}\PY{l+s}{\PYZdq{}}\PY{p}{)}\PY{p}{;}
             \PY{k}{return} \PY{n}{EXIT\PYZus{}SUCCESS}\PY{p}{;}
         \PY{p}{\PYZcb{}}
\end{Verbatim}


    \begin{Verbatim}[commandchars=\\\{\}]
nb\_jours\_mois{\ldots} ok
Bravo ! Tous les tests passent.

    \end{Verbatim}

    \begin{center}\rule{0.5\linewidth}{\linethickness}\end{center}

    \subsection{\#\# Les boucles /
répétitions}\label{les-boucles-ruxe9puxe9titions}

    \subsubsection{\texorpdfstring{La répétition \texttt{do} ...
\texttt{while}}{La répétition do ... while}}\label{la-ruxe9puxe9tition-do-...-while}

On répète au moins une fois une séquence. La condition d'arrêt est
testée une fois la séquence exécutée.

\begin{Shaded}
\begin{Highlighting}[]
    \ControlFlowTok{do}\NormalTok{ \{}
\NormalTok{        sequence;}
\NormalTok{    \}}
    \ControlFlowTok{while}\NormalTok{ (cond);}
\end{Highlighting}
\end{Shaded}

\textbf{Exemple} (cf fichier \textbf{\texttt{alea\_borne.c}}):

    \begin{Verbatim}[commandchars=\\\{\}]
{\color{incolor}In [{\color{incolor}18}]:} \PY{c+cp}{\PYZsh{}}\PY{c+cp}{include} \PY{c+cpf}{\PYZlt{}stdlib.h\PYZgt{}}
         \PY{c+cp}{\PYZsh{}}\PY{c+cp}{include} \PY{c+cpf}{\PYZlt{}stdio.h\PYZgt{}}
         \PY{c+cp}{\PYZsh{}}\PY{c+cp}{include} \PY{c+cpf}{\PYZlt{}time.h\PYZgt{}}
         \PY{c+cp}{\PYZsh{}}\PY{c+cp}{include} \PY{c+cpf}{\PYZlt{}assert.h\PYZgt{}}
         \PY{c+cm}{/*}
         \PY{c+cm}{ * \PYZbs{}brief Obtenir une valeur aléatoire entre min et max, inclus.}
         \PY{c+cm}{ * \PYZbs{}param[in] min borne minimale, }
         \PY{c+cm}{ * \PYZbs{}param[in] max borne maximale, }
         \PY{c+cm}{ * \PYZbs{}return valeur aleatoire entre min et max}
         \PY{c+cm}{ * \PYZbs{}pre min \PYZgt{}= 0, max \PYZlt{}= RAND\PYZus{}MAX}
         \PY{c+cm}{ */} 
         \PY{k+kt}{int} \PY{n+nf}{alea\PYZus{}borne}\PY{p}{(}\PY{k+kt}{int} \PY{n}{min}\PY{p}{,} \PY{k+kt}{int} \PY{n}{max}\PY{p}{)}\PY{p}{\PYZob{}}
             \PY{n}{assert}\PY{p}{(}\PY{n}{min} \PY{o}{\PYZgt{}}\PY{o}{=} \PY{l+m+mi}{0}\PY{p}{)}\PY{p}{;}
             \PY{n}{assert}\PY{p}{(}\PY{n}{max} \PY{o}{\PYZlt{}}\PY{o}{=} \PY{n}{RAND\PYZus{}MAX}\PY{p}{)}\PY{p}{;}
             
             \PY{c+c1}{// Initialisation du générateur de nombres aléatoires avec la date courante}
             \PY{n}{srand}\PY{p}{(}\PY{n}{time}\PY{p}{(}\PY{n+nb}{NULL}\PY{p}{)}\PY{p}{)}\PY{p}{;}
             \PY{k+kt}{int} \PY{n}{alea}\PY{p}{;}
             \PY{k}{do} \PY{p}{\PYZob{}}
                 \PY{n}{alea} \PY{o}{=} \PY{n}{rand}\PY{p}{(}\PY{p}{)}\PY{p}{;} \PY{c+c1}{// valeur aléatoire entre 0 et RAND\PYZus{}MAX}
             \PY{p}{\PYZcb{}}
             \PY{k}{while} \PY{p}{(}\PY{n}{alea} \PY{o}{\PYZlt{}} \PY{n}{min} \PY{o}{|}\PY{o}{|} \PY{n}{alea} \PY{o}{\PYZgt{}} \PY{n}{max}\PY{p}{)}\PY{p}{;}
             \PY{k}{return} \PY{n}{alea}\PY{p}{;}
         \PY{p}{\PYZcb{}}
         
         \PY{k+kt}{int} \PY{n+nf}{main}\PY{p}{(}\PY{k+kt}{void}\PY{p}{)} \PY{p}{\PYZob{}}
             \PY{k+kt}{int} \PY{n}{val} \PY{o}{=} \PY{n}{alea\PYZus{}borne}\PY{p}{(}\PY{l+m+mi}{4}\PY{p}{,} \PY{l+m+mi}{10}\PY{p}{)}\PY{p}{;}
             \PY{n}{assert}\PY{p}{(}\PY{n}{val} \PY{o}{\PYZgt{}}\PY{o}{=} \PY{l+m+mi}{4} \PY{o}{\PYZam{}}\PY{o}{\PYZam{}} \PY{n}{val} \PY{o}{\PYZlt{}}\PY{o}{=} \PY{l+m+mi}{10}\PY{p}{)}\PY{p}{;}
             \PY{n}{printf}\PY{p}{(}\PY{l+s}{\PYZdq{}}\PY{l+s}{val = \PYZpc{}d}\PY{l+s+se}{\PYZbs{}n}\PY{l+s}{\PYZdq{}}\PY{p}{,} \PY{n}{val}\PY{p}{)}\PY{p}{;}
             
             \PY{n}{val} \PY{o}{=} \PY{n}{alea\PYZus{}borne}\PY{p}{(}\PY{l+m+mi}{2}\PY{p}{,} \PY{l+m+mi}{25}\PY{p}{)}\PY{p}{;}
             \PY{n}{assert}\PY{p}{(}\PY{n}{val} \PY{o}{\PYZgt{}}\PY{o}{=} \PY{l+m+mi}{2} \PY{o}{\PYZam{}}\PY{o}{\PYZam{}} \PY{n}{val} \PY{o}{\PYZlt{}}\PY{o}{=} \PY{l+m+mi}{25}\PY{p}{)}\PY{p}{;}
             \PY{n}{printf}\PY{p}{(}\PY{l+s}{\PYZdq{}}\PY{l+s}{val = \PYZpc{}d}\PY{l+s+se}{\PYZbs{}n}\PY{l+s}{\PYZdq{}}\PY{p}{,} \PY{n}{val}\PY{p}{)}\PY{p}{;}
             \PY{k}{return} \PY{n}{EXIT\PYZus{}SUCCESS}\PY{p}{;}
         \PY{p}{\PYZcb{}}
\end{Verbatim}


    \begin{Verbatim}[commandchars=\\\{\}]
val = 6
val = 13

    \end{Verbatim}

    \subsubsection{\texorpdfstring{La répétition
\texttt{while}}{La répétition while}}\label{la-ruxe9puxe9tition-while}

On répète une séquence qui peut ne jamais être exécutée. On sort de la
boucle quand la condition est fausse :

\begin{Shaded}
\begin{Highlighting}[]
    \ControlFlowTok{while}\NormalTok{ (cond) \{}
\NormalTok{        sequence;}
\NormalTok{    \}}
\end{Highlighting}
\end{Shaded}

\textbf{Exemple} (cf fichier \textbf{\texttt{while.c}}):

    \begin{Verbatim}[commandchars=\\\{\}]
{\color{incolor}In [{\color{incolor}29}]:} \PY{c+cp}{\PYZsh{}}\PY{c+cp}{include} \PY{c+cpf}{\PYZlt{}stdlib.h\PYZgt{}}
         \PY{c+cp}{\PYZsh{}}\PY{c+cp}{include} \PY{c+cpf}{\PYZlt{}stdio.h\PYZgt{}}
         
         \PY{c+cp}{\PYZsh{}}\PY{c+cp}{define LIMITE 300}
         
         \PY{k+kt}{int} \PY{n+nf}{main}\PY{p}{(}\PY{k+kt}{void}\PY{p}{)} \PY{p}{\PYZob{}}
             \PY{k+kt}{int} \PY{n}{prec} \PY{o}{=} \PY{l+m+mi}{1}\PY{p}{,} \PY{n}{un} \PY{o}{=} \PY{l+m+mi}{2}\PY{p}{;}
             \PY{k+kt}{int} \PY{n}{rang} \PY{o}{=} \PY{l+m+mi}{2}\PY{p}{;}
             \PY{k+kt}{int} \PY{n}{nouveau}\PY{p}{;}
         
             \PY{k}{while} \PY{p}{(}\PY{n}{un} \PY{o}{\PYZlt{}} \PY{n}{LIMITE}\PY{p}{)} \PY{p}{\PYZob{}}
                 \PY{c+c1}{// Determiner le nouveau terme de la suite de Fibonacci}
                 \PY{n}{nouveau} \PY{o}{=} \PY{n}{un} \PY{o}{+} \PY{n}{prec}\PY{p}{;}
                 \PY{c+c1}{// Enregistrer les termes un et prec}
                 \PY{n}{prec} \PY{o}{=} \PY{n}{un}\PY{p}{;}
                 \PY{n}{un} \PY{o}{=} \PY{n}{nouveau}\PY{p}{;}
                 \PY{c+c1}{// Calculer le rang}
                 \PY{n}{rang} \PY{o}{+}\PY{o}{+}\PY{p}{;}
             \PY{p}{\PYZcb{}} 
             \PY{n}{printf}\PY{p}{(}\PY{l+s}{\PYZdq{}}\PY{l+s}{La valeur de la suite de fibonacci \PYZgt{}= \PYZpc{}d est \PYZpc{}d. Elle est de rang \PYZpc{}d}\PY{l+s+se}{\PYZbs{}n}\PY{l+s}{\PYZdq{}}\PY{p}{,} \PY{n}{LIMITE}\PY{p}{,} \PY{n}{un}\PY{p}{,} \PY{n}{rang}\PY{p}{)}\PY{p}{;}
             
             \PY{k}{return} \PY{n}{EXIT\PYZus{}SUCCESS}\PY{p}{;}
         \PY{p}{\PYZcb{}}
\end{Verbatim}


    \begin{Verbatim}[commandchars=\\\{\}]
La valeur de la suite de fibonacci >= 300 est 377. Elle est de rang 13

    \end{Verbatim}

    \subsubsection{\texorpdfstring{La répétition
\texttt{for}}{La répétition for}}\label{la-ruxe9puxe9tition-for}

Si on connait le nombre d'itérations, on utilise une boucle Pour :

\begin{Shaded}
\begin{Highlighting}[]
    \ControlFlowTok{for}\NormalTok{ (instruction_init_compteur; condition_boucle; instruction_incr_compteur) \{}
\NormalTok{        sequence;}
\NormalTok{    \}}
\end{Highlighting}
\end{Shaded}

On a ici : - \texttt{instruction\_init\_increment} : une instruction qui
initialise (voire déclare) le compteur, - \texttt{condition\_boucle} :
une condition qui, \textbf{si fausse}, arrête la répétition. -
\texttt{instruction\_incr\_compteur} : une instruction qui précise
comment le compteur varie à chaque répétition.

\textbf{Exemple} (cf fichier \textbf{\texttt{for.c}}):

    \begin{Verbatim}[commandchars=\\\{\}]
{\color{incolor}In [{\color{incolor}30}]:} \PY{c+cp}{\PYZsh{}}\PY{c+cp}{include} \PY{c+cpf}{\PYZlt{}stdlib.h\PYZgt{}}
         \PY{c+cp}{\PYZsh{}}\PY{c+cp}{include} \PY{c+cpf}{\PYZlt{}stdio.h\PYZgt{}}
         \PY{c+cp}{\PYZsh{}}\PY{c+cp}{define LIMITE 30}
         
         \PY{k+kt}{int} \PY{n+nf}{main}\PY{p}{(}\PY{k+kt}{void}\PY{p}{)} \PY{p}{\PYZob{}}
             \PY{c+c1}{//calcul de la moyenne des LIMITE premiers entiers}
             \PY{k+kt}{int} \PY{n}{somme} \PY{o}{=} \PY{l+m+mi}{0}\PY{p}{;}
             
             \PY{c+c1}{// Déclaration du compteur i et initialisation à 1}
             \PY{c+c1}{// Répétition si i \PYZlt{}= LIMITE}
             \PY{c+c1}{// Incrémentation i = i + 1 à chaque répétition.}
             \PY{k}{for} \PY{p}{(}\PY{k+kt}{int} \PY{n}{i} \PY{o}{=} \PY{l+m+mi}{1}\PY{p}{;} \PY{n}{i} \PY{o}{\PYZlt{}}\PY{o}{=} \PY{n}{LIMITE}\PY{p}{;} \PY{n}{i}\PY{o}{+}\PY{o}{+}\PY{p}{)} \PY{p}{\PYZob{}}
                 \PY{n}{somme} \PY{o}{+}\PY{o}{=} \PY{n}{i}\PY{p}{;}
             \PY{p}{\PYZcb{}}
             \PY{k+kt}{float} \PY{n}{moyenne} \PY{o}{=} \PY{n}{somme} \PY{o}{/} \PY{p}{(}\PY{k+kt}{float}\PY{p}{)} \PY{n}{LIMITE}\PY{p}{;} 
             \PY{n}{printf}\PY{p}{(}\PY{l+s}{\PYZdq{}}\PY{l+s}{La moyenne des \PYZpc{}d premiers entiers est \PYZpc{}1.2f}\PY{l+s+se}{\PYZbs{}n}\PY{l+s}{\PYZdq{}}\PY{p}{,} \PY{n}{LIMITE}\PY{p}{,} \PY{n}{moyenne}\PY{p}{)}\PY{p}{;}
             \PY{k}{return} \PY{n}{EXIT\PYZus{}SUCCESS}\PY{p}{;}
         \PY{p}{\PYZcb{}}
\end{Verbatim}


    \begin{Verbatim}[commandchars=\\\{\}]
La moyenne des 30 premiers entiers est 15.50

    \end{Verbatim}

    \subsection{\texorpdfstring{\#\#\# Exercice 11 - Ecrire un
\texttt{TantQue}}{\#\#\# Exercice 11 - Ecrire un TantQue}}\label{exercice-11---ecrire-un-tantque}

\begin{center}\rule{0.5\linewidth}{\linethickness}\end{center}

\textbf{{[}11.1{]}} Compléter et corriger le corps de la function
\texttt{sommes\_cubes\_inférieurs\_a} (cf. fichier
\textbf{\texttt{exercice11.c}}) (voir TODO).

    \begin{Verbatim}[commandchars=\\\{\}]
{\color{incolor}In [{\color{incolor} }]:} \PY{c+cp}{\PYZsh{}}\PY{c+cp}{include} \PY{c+cpf}{\PYZlt{}stdlib.h\PYZgt{}}
        \PY{c+cp}{\PYZsh{}}\PY{c+cp}{include} \PY{c+cpf}{\PYZlt{}stdio.h\PYZgt{}}
        \PY{c+cp}{\PYZsh{}}\PY{c+cp}{include} \PY{c+cpf}{\PYZlt{}stdbool.h\PYZgt{}}
        \PY{c+cp}{\PYZsh{}}\PY{c+cp}{include} \PY{c+cpf}{\PYZlt{}assert.h\PYZgt{}}
        
        \PY{c+cm}{/**}
        \PY{c+cm}{ * \PYZbs{}brief Calculer la somme des cubes des entiers naturels dont le cube est inférieur}
        \PY{c+cm}{ * ou égal à limite.}
        \PY{c+cm}{ * \PYZbs{}param[in] limite la limite à ne pas dépasser pour les cubes}
        \PY{c+cm}{ * \PYZbs{}return la sommes des cubes}
        \PY{c+cm}{ * \PYZbs{}pre limite positive : limite \PYZgt{} 0}
        \PY{c+cm}{ */}
        \PY{k+kt}{int} \PY{n+nf}{sommes\PYZus{}cubes\PYZus{}inferieurs\PYZus{}a}\PY{p}{(}\PY{k+kt}{int} \PY{n}{limite}\PY{p}{)}
        \PY{p}{\PYZob{}}
            \PY{n}{assert}\PY{p}{(}\PY{n}{limite} \PY{o}{\PYZgt{}}\PY{o}{=} \PY{l+m+mi}{0}\PY{p}{)}\PY{p}{;}
        
            \PY{c+c1}{// Consigne :}
            \PY{c+c1}{//   1. On n\PYZsq{}utilisera pas l\PYZsq{}exponentielle.}
            \PY{c+c1}{//   2. On utilisera seulement un TantQue}
            \PY{k+kt}{int} \PY{n}{s}\PY{o}{=} \PY{l+m+mi}{0}\PY{p}{;}
            \PY{k+kt}{int} \PY{n}{i} \PY{o}{=}\PY{l+m+mi}{1}\PY{p}{;}
            \PY{k}{while} \PY{p}{(} \PY{n}{i} \PY{o}{\PYZlt{}}\PY{o}{=} \PY{n}{limite}\PY{p}{)} \PY{p}{\PYZob{}}
               \PY{n}{i}\PY{o}{\PYZhy{}}\PY{o}{=}\PY{n}{i}\PY{o}{*}\PY{n}{i}\PY{o}{*}\PY{n}{i}\PY{p}{;}
               \PY{n}{s}\PY{o}{=}\PY{n}{s}\PY{o}{+}\PY{l+m+mi}{1}\PY{p}{;}
            \PY{p}{\PYZcb{}}
            \PY{k}{return} \PY{n}{s}\PY{p}{;}
        \PY{p}{\PYZcb{}}
        
        
        \PY{c+c1}{////////////////////////////////////////////////////////////////////////////////}
        \PY{c+c1}{//                                                                            //}
        \PY{c+c1}{//                    NE PAS MODIFIER CE QUI SUIT...                          //}
        \PY{c+c1}{//                                                                            //}
        \PY{c+c1}{////////////////////////////////////////////////////////////////////////////////}
        
        
        \PY{k+kt}{void} \PY{n+nf}{test\PYZus{}sommes\PYZus{}cubes\PYZus{}inferieurs\PYZus{}a}\PY{p}{(}\PY{k+kt}{void}\PY{p}{)} \PY{p}{\PYZob{}}
            \PY{n}{assert}\PY{p}{(}\PY{l+m+mi}{1} \PY{o}{=}\PY{o}{=} \PY{n}{sommes\PYZus{}cubes\PYZus{}inferieurs\PYZus{}a}\PY{p}{(}\PY{l+m+mi}{5}\PY{p}{)}\PY{p}{)}\PY{p}{;}
            \PY{n}{assert}\PY{p}{(}\PY{l+m+mi}{1} \PY{o}{=}\PY{o}{=} \PY{n}{sommes\PYZus{}cubes\PYZus{}inferieurs\PYZus{}a}\PY{p}{(}\PY{l+m+mi}{1}\PY{p}{)}\PY{p}{)}\PY{p}{;}
            \PY{n}{assert}\PY{p}{(}\PY{l+m+mi}{1} \PY{o}{=}\PY{o}{=} \PY{n}{sommes\PYZus{}cubes\PYZus{}inferieurs\PYZus{}a}\PY{p}{(}\PY{l+m+mi}{7}\PY{p}{)}\PY{p}{)}\PY{p}{;}
            \PY{n}{assert}\PY{p}{(}\PY{l+m+mi}{9} \PY{o}{=}\PY{o}{=} \PY{n}{sommes\PYZus{}cubes\PYZus{}inferieurs\PYZus{}a}\PY{p}{(}\PY{l+m+mi}{8}\PY{p}{)}\PY{p}{)}\PY{p}{;}
            \PY{n}{assert}\PY{p}{(}\PY{l+m+mi}{9} \PY{o}{=}\PY{o}{=} \PY{n}{sommes\PYZus{}cubes\PYZus{}inferieurs\PYZus{}a}\PY{p}{(}\PY{l+m+mi}{26}\PY{p}{)}\PY{p}{)}\PY{p}{;}
            \PY{n}{assert}\PY{p}{(}\PY{l+m+mi}{36} \PY{o}{=}\PY{o}{=} \PY{n}{sommes\PYZus{}cubes\PYZus{}inferieurs\PYZus{}a}\PY{p}{(}\PY{l+m+mi}{27}\PY{p}{)}\PY{p}{)}\PY{p}{;}
            \PY{n}{assert}\PY{p}{(}\PY{l+m+mi}{36} \PY{o}{=}\PY{o}{=} \PY{n}{sommes\PYZus{}cubes\PYZus{}inferieurs\PYZus{}a}\PY{p}{(}\PY{l+m+mi}{63}\PY{p}{)}\PY{p}{)}\PY{p}{;}
            \PY{n}{assert}\PY{p}{(}\PY{l+m+mi}{100} \PY{o}{=}\PY{o}{=} \PY{n}{sommes\PYZus{}cubes\PYZus{}inferieurs\PYZus{}a}\PY{p}{(}\PY{l+m+mi}{64}\PY{p}{)}\PY{p}{)}\PY{p}{;}
            \PY{n}{assert}\PY{p}{(}\PY{l+m+mi}{100} \PY{o}{=}\PY{o}{=} \PY{n}{sommes\PYZus{}cubes\PYZus{}inferieurs\PYZus{}a}\PY{p}{(}\PY{l+m+mi}{124}\PY{p}{)}\PY{p}{)}\PY{p}{;}
            \PY{n}{assert}\PY{p}{(}\PY{l+m+mi}{225} \PY{o}{=}\PY{o}{=} \PY{n}{sommes\PYZus{}cubes\PYZus{}inferieurs\PYZus{}a}\PY{p}{(}\PY{l+m+mi}{125}\PY{p}{)}\PY{p}{)}\PY{p}{;}
            \PY{n}{printf}\PY{p}{(}\PY{l+s}{\PYZdq{}}\PY{l+s}{\PYZpc{}s}\PY{l+s}{\PYZdq{}}\PY{p}{,} \PY{l+s}{\PYZdq{}}\PY{l+s}{sommes\PYZus{}cubes\PYZus{}inferieurs\PYZus{}a... ok}\PY{l+s+se}{\PYZbs{}n}\PY{l+s}{\PYZdq{}}\PY{p}{)}\PY{p}{;}
        \PY{p}{\PYZcb{}}
        
        
        \PY{k+kt}{int} \PY{n+nf}{main}\PY{p}{(}\PY{k+kt}{void}\PY{p}{)} \PY{p}{\PYZob{}}
            \PY{n}{test\PYZus{}sommes\PYZus{}cubes\PYZus{}inferieurs\PYZus{}a}\PY{p}{(}\PY{p}{)}\PY{p}{;}
            \PY{n}{printf}\PY{p}{(}\PY{l+s}{\PYZdq{}}\PY{l+s}{\PYZpc{}s}\PY{l+s}{\PYZdq{}}\PY{p}{,} \PY{l+s}{\PYZdq{}}\PY{l+s}{Bravo ! Tous les tests passent.}\PY{l+s+se}{\PYZbs{}n}\PY{l+s}{\PYZdq{}}\PY{p}{)}\PY{p}{;}
            \PY{k}{return} \PY{n}{EXIT\PYZus{}SUCCESS}\PY{p}{;}
        \PY{p}{\PYZcb{}}
\end{Verbatim}


    \subsection{\texorpdfstring{\#\#\# Exercice 12 - Ecrire un
\texttt{Répéter\ ...\ Jusqu\textquotesingle{}à}}{\#\#\# Exercice 12 - Ecrire un Répéter ... Jusqu'à}}\label{exercice-12---ecrire-un-ruxe9puxe9ter-...-jusquuxe0}

\begin{center}\rule{0.5\linewidth}{\linethickness}\end{center}

\textbf{{[}12.1{]}} Compléter et corriger le corps de la function
\texttt{frequence} (cf. fichier \textbf{\texttt{exercice12.c}})(voir
TODO).

    \begin{Verbatim}[commandchars=\\\{\}]
{\color{incolor}In [{\color{incolor} }]:} \PY{c+cp}{\PYZsh{}}\PY{c+cp}{include} \PY{c+cpf}{\PYZlt{}stdlib.h\PYZgt{}}
        \PY{c+cp}{\PYZsh{}}\PY{c+cp}{include} \PY{c+cpf}{\PYZlt{}stdio.h\PYZgt{}}
        \PY{c+cp}{\PYZsh{}}\PY{c+cp}{include} \PY{c+cpf}{\PYZlt{}stdbool.h\PYZgt{}}
        \PY{c+cp}{\PYZsh{}}\PY{c+cp}{include} \PY{c+cpf}{\PYZlt{}assert.h\PYZgt{}}
        
        \PY{c+cm}{/**}
        \PY{c+cm}{ * \PYZbs{}brief Obtenir la fréquence d\PYZsq{}un chiffre dans un nombre.}
        \PY{c+cm}{ * Exemples : la fréquence de 5 dans 1515 est 2. La fréquence de 3 dans 123 est 1.}
        \PY{c+cm}{ * La fréquence de 0 dans 412 est 0.}
        \PY{c+cm}{ * \PYZbs{}param[in] chiffre dont ont veut calculer la fréquence}
        \PY{c+cm}{ * \PYZbs{}param[in] nombre pour lequel on veut calculer la fréquence de chiffre}
        \PY{c+cm}{ * \PYZbs{}return la fréquence de chiffre dans nombre}
        \PY{c+cm}{ * \PYZbs{}pre chiffre est un vrai chiffre : 0 \PYZlt{}= chiffre \PYZlt{}= 9}
        \PY{c+cm}{ */}
        \PY{k+kt}{int} \PY{n+nf}{frequence}\PY{p}{(}\PY{k+kt}{int} \PY{n}{nombre}\PY{p}{,} \PY{k+kt}{int} \PY{n}{chiffre}\PY{p}{)}
        \PY{p}{\PYZob{}}
            \PY{n}{assert}\PY{p}{(}\PY{n}{chiffre} \PY{o}{\PYZgt{}}\PY{o}{=} \PY{l+m+mi}{0}\PY{p}{)}\PY{p}{;}
            \PY{n}{assert}\PY{p}{(}\PY{n}{chiffre} \PY{o}{\PYZlt{}}\PY{o}{=} \PY{l+m+mi}{9}\PY{p}{)}\PY{p}{;}
            \PY{k+kt}{int} \PY{n}{s} \PY{o}{=} \PY{l+m+mi}{0}\PY{p}{;}
            \PY{k}{do}\PY{p}{\PYZob{}}
                \PY{n}{s}\PY{o}{+}\PY{o}{+}\PY{p}{;}
            \PY{p}{\PYZcb{}}
            \PY{k}{while} \PY{n}{nombre}\PY{p}{[}\PY{n}{s}\PY{p}{]}\PY{o}{=} \PY{n}{chiffre}\PY{p}{;}
            \PY{k}{return} \PY{l+m+mi}{\PYZhy{}1}\PY{p}{;}
        \PY{p}{\PYZcb{}}
        
        
        \PY{c+c1}{////////////////////////////////////////////////////////////////////////////////}
        \PY{c+c1}{//                                                                            //}
        \PY{c+c1}{//                    NE PAS MODIFIER CE QUI SUIT...                          //}
        \PY{c+c1}{//                                                                            //}
        \PY{c+c1}{////////////////////////////////////////////////////////////////////////////////}
        
        
        \PY{k+kt}{void} \PY{n+nf}{test\PYZus{}frequence}\PY{p}{(}\PY{k+kt}{void}\PY{p}{)} \PY{p}{\PYZob{}}
            \PY{n}{assert}\PY{p}{(}\PY{l+m+mi}{2} \PY{o}{=}\PY{o}{=} \PY{n}{frequence}\PY{p}{(}\PY{l+m+mi}{1515}\PY{p}{,} \PY{l+m+mi}{5}\PY{p}{)}\PY{p}{)}\PY{p}{;}
            \PY{n}{assert}\PY{p}{(}\PY{l+m+mi}{1} \PY{o}{=}\PY{o}{=} \PY{n}{frequence}\PY{p}{(}\PY{l+m+mi}{123}\PY{p}{,} \PY{l+m+mi}{3}\PY{p}{)}\PY{p}{)}\PY{p}{;}
            \PY{n}{assert}\PY{p}{(}\PY{l+m+mi}{0} \PY{o}{=}\PY{o}{=} \PY{n}{frequence}\PY{p}{(}\PY{l+m+mi}{421}\PY{p}{,} \PY{l+m+mi}{0}\PY{p}{)}\PY{p}{)}\PY{p}{;}
            \PY{n}{assert}\PY{p}{(}\PY{l+m+mi}{3} \PY{o}{=}\PY{o}{=} \PY{n}{frequence}\PY{p}{(}\PY{l+m+mi}{444}\PY{p}{,} \PY{l+m+mi}{4}\PY{p}{)}\PY{p}{)}\PY{p}{;}
            \PY{n}{assert}\PY{p}{(}\PY{l+m+mi}{1} \PY{o}{=}\PY{o}{=} \PY{n}{frequence}\PY{p}{(}\PY{l+m+mi}{0}\PY{p}{,} \PY{l+m+mi}{0}\PY{p}{)}\PY{p}{)}\PY{p}{;}
            \PY{n}{printf}\PY{p}{(}\PY{l+s}{\PYZdq{}}\PY{l+s}{\PYZpc{}s}\PY{l+s}{\PYZdq{}}\PY{p}{,} \PY{l+s}{\PYZdq{}}\PY{l+s}{frequence... ok}\PY{l+s+se}{\PYZbs{}n}\PY{l+s}{\PYZdq{}}\PY{p}{)}\PY{p}{;}
        \PY{p}{\PYZcb{}}
        
        
        \PY{k+kt}{int} \PY{n+nf}{main}\PY{p}{(}\PY{k+kt}{void}\PY{p}{)} \PY{p}{\PYZob{}}
            \PY{n}{test\PYZus{}frequence}\PY{p}{(}\PY{p}{)}\PY{p}{;}
            \PY{n}{printf}\PY{p}{(}\PY{l+s}{\PYZdq{}}\PY{l+s}{\PYZpc{}s}\PY{l+s}{\PYZdq{}}\PY{p}{,} \PY{l+s}{\PYZdq{}}\PY{l+s}{Bravo ! Tous les tests passent.}\PY{l+s+se}{\PYZbs{}n}\PY{l+s}{\PYZdq{}}\PY{p}{)}\PY{p}{;}
            \PY{k}{return} \PY{n}{EXIT\PYZus{}SUCCESS}\PY{p}{;}
        \PY{p}{\PYZcb{}}
\end{Verbatim}


    \subsection{\texorpdfstring{\#\#\# Exercice 13 - Ecrire un
\texttt{Pour}}{\#\#\# Exercice 13 - Ecrire un Pour}}\label{exercice-13---ecrire-un-pour}

\begin{center}\rule{0.5\linewidth}{\linethickness}\end{center}

\textbf{{[}13.1{]}} Compléter et corriger le corps de la function
\texttt{frequence} (cf. fichier \textbf{\texttt{exercice13.c}})(voir
TODO).

    \begin{Verbatim}[commandchars=\\\{\}]
{\color{incolor}In [{\color{incolor} }]:} \PY{c+c1}{// Consigne : compléter et corriger le corps des fonctions ci\PYZhy{}dessous (voir TODO).}
        
        \PY{c+cp}{\PYZsh{}}\PY{c+cp}{include} \PY{c+cpf}{\PYZlt{}stdlib.h\PYZgt{}}
        \PY{c+cp}{\PYZsh{}}\PY{c+cp}{include} \PY{c+cpf}{\PYZlt{}stdio.h\PYZgt{}}
        \PY{c+cp}{\PYZsh{}}\PY{c+cp}{include} \PY{c+cpf}{\PYZlt{}stdbool.h\PYZgt{}}
        \PY{c+cp}{\PYZsh{}}\PY{c+cp}{include} \PY{c+cpf}{\PYZlt{}assert.h\PYZgt{}}
        
        \PY{c+cm}{/**}
        \PY{c+cm}{ * \PYZbs{}brief Calculer la somme des cubes des entiers naturels de 1 à max.}
        \PY{c+cm}{ * \PYZbs{}param[in] max un entier naturel}
        \PY{c+cm}{ * \PYZbs{}return la sommes des cubes de 1 à max}
        \PY{c+cm}{ * \PYZbs{}pre max positif : max \PYZgt{}= 0}
        \PY{c+cm}{ */}
        \PY{k+kt}{int} \PY{n+nf}{sommes\PYZus{}cubes}\PY{p}{(}\PY{k+kt}{int} \PY{n}{max}\PY{p}{)}
        \PY{p}{\PYZob{}}
            \PY{n}{assert}\PY{p}{(}\PY{n}{max} \PY{o}{\PYZgt{}}\PY{o}{=} \PY{l+m+mi}{0}\PY{p}{)}\PY{p}{;}
        
            \PY{c+c1}{// Consigne :}
            \PY{c+c1}{//   1. On n\PYZsq{}utilisera pas l\PYZsq{}exponentielle.}
            \PY{c+c1}{//   2. On utilisera seulement un Pour}
            \PY{k+kt}{int} \PY{n}{s}\PY{o}{=}\PY{l+m+mi}{0}\PY{p}{;}
            \PY{k}{for}\PY{p}{(}\PY{k+kt}{int} \PY{n}{i}\PY{o}{=}\PY{l+m+mi}{1}\PY{p}{;}\PY{n}{i}\PY{o}{\PYZlt{}}\PY{n}{max}\PY{p}{;}\PY{n}{i}\PY{o}{+}\PY{o}{+}\PY{p}{)}\PY{p}{\PYZob{}}
                \PY{n}{s}\PY{o}{+}\PY{o}{=}\PY{n}{i}\PY{o}{*}\PY{o}{*}\PY{l+m+mi}{3}\PY{p}{;}
                
            \PY{p}{\PYZcb{}}
            \PY{k}{return} \PY{n}{s}\PY{p}{;}
        \PY{p}{\PYZcb{}}
        
        
        \PY{c+c1}{////////////////////////////////////////////////////////////////////////////////}
        \PY{c+c1}{//                                                                            //}
        \PY{c+c1}{//                    NE PAS MODIFIER CE QUI SUIT...                          //}
        \PY{c+c1}{//                                                                            //}
        \PY{c+c1}{////////////////////////////////////////////////////////////////////////////////}
        
        
        \PY{k+kt}{void} \PY{n+nf}{test\PYZus{}sommes\PYZus{}cubes}\PY{p}{(}\PY{k+kt}{void}\PY{p}{)} \PY{p}{\PYZob{}}
            \PY{n}{assert}\PY{p}{(}\PY{l+m+mi}{1} \PY{o}{=}\PY{o}{=} \PY{n}{sommes\PYZus{}cubes}\PY{p}{(}\PY{l+m+mi}{1}\PY{p}{)}\PY{p}{)}\PY{p}{;}
            \PY{n}{assert}\PY{p}{(}\PY{l+m+mi}{9} \PY{o}{=}\PY{o}{=} \PY{n}{sommes\PYZus{}cubes}\PY{p}{(}\PY{l+m+mi}{2}\PY{p}{)}\PY{p}{)}\PY{p}{;}
            \PY{n}{assert}\PY{p}{(}\PY{l+m+mi}{36} \PY{o}{=}\PY{o}{=} \PY{n}{sommes\PYZus{}cubes}\PY{p}{(}\PY{l+m+mi}{3}\PY{p}{)}\PY{p}{)}\PY{p}{;}
            \PY{n}{assert}\PY{p}{(}\PY{l+m+mi}{100} \PY{o}{=}\PY{o}{=} \PY{n}{sommes\PYZus{}cubes}\PY{p}{(}\PY{l+m+mi}{4}\PY{p}{)}\PY{p}{)}\PY{p}{;}
            \PY{n}{assert}\PY{p}{(}\PY{l+m+mi}{225} \PY{o}{=}\PY{o}{=} \PY{n}{sommes\PYZus{}cubes}\PY{p}{(}\PY{l+m+mi}{5}\PY{p}{)}\PY{p}{)}\PY{p}{;}
            \PY{n}{assert}\PY{p}{(}\PY{l+m+mi}{0} \PY{o}{=}\PY{o}{=} \PY{n}{sommes\PYZus{}cubes}\PY{p}{(}\PY{l+m+mi}{0}\PY{p}{)}\PY{p}{)}\PY{p}{;}
            \PY{n}{printf}\PY{p}{(}\PY{l+s}{\PYZdq{}}\PY{l+s}{\PYZpc{}s}\PY{l+s}{\PYZdq{}}\PY{p}{,} \PY{l+s}{\PYZdq{}}\PY{l+s}{sommes\PYZus{}cubes... ok}\PY{l+s+se}{\PYZbs{}n}\PY{l+s}{\PYZdq{}}\PY{p}{)}\PY{p}{;}
        \PY{p}{\PYZcb{}}
        
        
        \PY{k+kt}{int} \PY{n+nf}{main}\PY{p}{(}\PY{k+kt}{void}\PY{p}{)} \PY{p}{\PYZob{}}
            \PY{n}{test\PYZus{}sommes\PYZus{}cubes}\PY{p}{(}\PY{p}{)}\PY{p}{;}
            \PY{n}{printf}\PY{p}{(}\PY{l+s}{\PYZdq{}}\PY{l+s}{\PYZpc{}s}\PY{l+s}{\PYZdq{}}\PY{p}{,} \PY{l+s}{\PYZdq{}}\PY{l+s}{Bravo ! Tous les tests passent.}\PY{l+s+se}{\PYZbs{}n}\PY{l+s}{\PYZdq{}}\PY{p}{)}\PY{p}{;}
            \PY{k}{return} \PY{n}{EXIT\PYZus{}SUCCESS}\PY{p}{;}
        \PY{p}{\PYZcb{}}
\end{Verbatim}


    \subsection{\#\# Exercice BILAN 1 : Conversion
pouce/centimètres}\label{exercice-bilan-1-conversion-poucecentimuxe8tres}

\begin{center}\rule{0.5\linewidth}{\linethickness}\end{center}

Cet exercice de Bilan 1 n'est pas à rendre. Le squelette du programme
est disponible dans votre répertoire SVN, et est à éditer avec un
éditeur de fichier standard, et à compiler / exécuter en ligne de
commande.

\textbf{{[}B.1{]}} Traduire l'algorithme du listing en Langage C. Il
permet de convertir en pouces et en centimètres une longueur saisie en
pouces, centimètres ou mètres.

\textbf{{[}B.2{]}} Modifier le programme pour que l'utilisateur puisse
mettre des espaces (des blancs) entre la valeur et l'unité de la
longueur.

\textbf{{[}B.3{]}} Ajouter la possibilité de recommencer.
\includegraphics{https://www.irit.fr/~Katia.Jaffres/Cours/pouce2cm.png}

    \begin{Verbatim}[commandchars=\\\{\}]
{\color{incolor}In [{\color{incolor}2}]:} \PY{c+cm}{/** Squelette du programme **/}
        \PY{c+cm}{/*********************************************************************}
        \PY{c+cm}{ *  Auteur  : }
        \PY{c+cm}{ *  Version : }
        \PY{c+cm}{ *  Objectif : Conversion pouces/centimètres}
        \PY{c+cm}{ ********************************************************************/}
        
        \PY{c+cp}{\PYZsh{}}\PY{c+cp}{include} \PY{c+cpf}{\PYZlt{}stdio.h\PYZgt{}}
        \PY{c+cp}{\PYZsh{}}\PY{c+cp}{include} \PY{c+cpf}{\PYZlt{}stdlib.h\PYZgt{}}
        
        \PY{k+kt}{int} \PY{n+nf}{main}\PY{p}{(}\PY{p}{)}
        \PY{p}{\PYZob{}}
        
            \PY{c+cm}{/* Saisir la longueur */}
        
            \PY{c+cm}{/* Calculer la longueur en pouces et en centimètres */}
        
            \PY{c+cm}{/* Afficher la longueur en pouces et en centimètres */}
        
            \PY{k}{return} \PY{n}{EXIT\PYZus{}SUCCESS}\PY{p}{;}
        \PY{p}{\PYZcb{}}
\end{Verbatim}


    \begin{center}\rule{0.5\linewidth}{\linethickness}\end{center}

    \subsection{\#\# Les types utilisateurs}\label{les-types-utilisateurs}

    Les types utilisateurs permettent au programmeur de définir des types
plus évolués. Les 3 types en C sont : - Les types énumérés. - Les
enregistrements. - Les tableaux.

Ces types se définissent au début d'un programme, avant la signature du
programme principal \texttt{int\ main()}

    \subsubsection{Les types énumérés}\label{les-types-uxe9numuxe9ruxe9s}

Un type énuméré permet de définir un ensemble discret de valeurs
possibles. L'exemple suivant déclare un type énuméré \texttt{enum\ Jour}
: \textgreater{}
\texttt{enum\ Jour\ \{\ LUNDI,\ MARDI,\ MERCREDI,\ JEUDI,\ VENDREDI,\ SAMEDI,\ DIMANCHE\};}

Les constantes \texttt{LUNDI}, \texttt{MARDI}, sont des constantes
entières qui vallent respectivemet 0, 1, 2, etc. On peut donc les
comparer.

Une variable de type \texttt{enum\ Jour} ne peut prendre que ces
valeurs. \textgreater{}
\texttt{enum\ Jour\ mon\_jour\ =\ LUNDI;\ \ //declaration\ d\textquotesingle{}une\ variable\ initialsée\ à\ LUNDI}

\textbf{Il est conseillé} de créer un alias au type \texttt{enum\ Jour}
à l'aide de l'instruction \texttt{typedef} \textgreater{}
\texttt{typedef\ enum\ Jour\ Jour} Ici on a créé l'alias (le synonyme)
\texttt{Jour}.

    \subsubsection{Les types enregistrement}\label{les-types-enregistrement}

Un type enregistrement permet de déclarer une variable qui regroupe
plusieurs données hétérogènes (i.e. de type différent). En C, on le
définit de la sorte :

\begin{Shaded}
\begin{Highlighting}[]
\KeywordTok{struct}\NormalTok{ Date \{}
    \DataTypeTok{int}\NormalTok{ jour;}
\NormalTok{    Mois mois;}
    \DataTypeTok{int}\NormalTok{ annee;}
\NormalTok{\};}
\end{Highlighting}
\end{Shaded}

\begin{quote}
Note : ne pas oublier le ; après la dernière parenthèse.
\end{quote}

Le type \texttt{struct\ Date} est un 3-uplet qui regroupe un jour, un
mois et une année.

\begin{Shaded}
\begin{Highlighting}[]
    \KeywordTok{struct}\NormalTok{ Date d1, d2;  }\CommentTok{// déclaration de deux dates}
    \CommentTok{// initialisation champ par champ}
\NormalTok{    d1.jour = }\DecValTok{30}\NormalTok{;}
\NormalTok{    d1.mois = AVRIL; }\CommentTok{//ici mois est un type énuméré}
\NormalTok{    d1.annee = }\DecValTok{1997}\NormalTok{;}
    \CommentTok{// initialisation directe des 3 champs}
\NormalTok{    d2 = \{}\DecValTok{31}\NormalTok{, DECEMBRE, }\DecValTok{2012}\NormalTok{\};}
\end{Highlighting}
\end{Shaded}

Il est aussi possible de créer un alias au type \texttt{struct\ Date} à
l'aide de l'instruction \texttt{typedef} \textgreater{}
\texttt{typedef\ struct\ Date\ Date} Ici on a créé l'alias (le synonyme)
\texttt{Date}.

    \subsection{\#\#\# Exercice 14 : definir et utiliser un type
enregistrement}\label{exercice-14-definir-et-utiliser-un-type-enregistrement}

\begin{center}\rule{0.5\linewidth}{\linethickness}\end{center}

(cf fichier \textbf{\texttt{exercice14.c}})

\textbf{{[}14.1{]}} Définir un type Point qui regroupe deux coordonnées
entières, X et Y.

\textbf{{[}14.2{]}} Ecrire un programme principal qui génère deux points
ptA et ptB au coordonnées (0,0) et (10,10) respectives. Il calcule la
distance entre ptA et ptB en norme Euclidienne.

    \begin{Verbatim}[commandchars=\\\{\}]
{\color{incolor}In [{\color{incolor} }]:} \PY{c+cp}{\PYZsh{}}\PY{c+cp}{include} \PY{c+cpf}{\PYZlt{}stdlib.h\PYZgt{}}\PY{c+c1}{ }
        \PY{c+cp}{\PYZsh{}}\PY{c+cp}{include} \PY{c+cpf}{\PYZlt{}stdio.h\PYZgt{}}
        \PY{c+cp}{\PYZsh{}}\PY{c+cp}{include} \PY{c+cpf}{\PYZlt{}math.h\PYZgt{}}
        \PY{c+cp}{\PYZsh{}}\PY{c+cp}{include} \PY{c+cpf}{\PYZlt{}assert.h\PYZgt{}}
        
        \PY{c+c1}{// Definition du type Point }
        \PY{k}{struct} \PY{n+nc}{Point}\PY{p}{\PYZob{}}
            \PY{k+kt}{int} \PY{n}{x}\PY{p}{;}
            \PY{k+kt}{int} \PY{n}{Y}\PY{p}{;}
        \PY{p}{\PYZcb{}}\PY{p}{;}
        \PY{k+kt}{int} \PY{n+nf}{main}\PY{p}{(}\PY{p}{)}\PY{p}{\PYZob{}}
            \PY{c+c1}{// Déclarer deux variables ptA et ptB de types Point}
            \PY{k}{struct} \PY{n+nc}{Point} \PY{n}{ptA} \PY{n}{ptB}\PY{p}{;}
            \PY{c+c1}{// Initialiser ptA à (0,0)}
               \PY{n}{ptA} \PY{o}{=} \PY{p}{\PYZob{}}\PY{l+m+mf}{0.0}\PY{p}{\PYZcb{}}\PY{p}{;}
            \PY{c+c1}{// Initialiser ptB à (10,10)}
             \PY{n}{ptB} \PY{o}{=} \PY{p}{\PYZob{}}\PY{l+m+mf}{0.0}\PY{p}{\PYZcb{}}\PY{p}{;}
            \PY{c+c1}{// Calculer la distance entre ptA et ptB.}
            \PY{k+kt}{float} \PY{n}{distance} \PY{o}{=} \PY{l+m+mi}{0}\PY{p}{;}
            
            \PY{n}{assert}\PY{p}{(}\PY{n}{distance} \PY{o}{=}\PY{o}{=} \PY{n}{sqrt}\PY{p}{(}\PY{l+m+mi}{200}\PY{p}{)}\PY{p}{)}\PY{p}{;}
            
            \PY{k}{return} \PY{n}{EXIT\PYZus{}SUCCESS}\PY{p}{;}
        \PY{p}{\PYZcb{}}
\end{Verbatim}


    \subsubsection{Tableaux}\label{tableaux}

Les tableaux permettent d'enregistrer \textbf{un nombre fini de données
de même type}.

\paragraph{Déclarer une variable
tableau}\label{duxe9clarer-une-variable-tableau}

On pourra par exemple definir une \textbf{variable tableau} capable
d'enregistrer NB entiers. \textgreater{} Attention : En C, \textbf{les
indices varient entre 0 et NB-1}.

\begin{Shaded}
\begin{Highlighting}[]
    \PreprocessorTok{#define NB 4}
    \CommentTok{// déclaration d'une variable tableau de NB entiers}
    \DataTypeTok{int}\NormalTok{ tab[NB]; }
    \CommentTok{// Initialisation de la 2e case : }
\NormalTok{    tab[}\DecValTok{1}\NormalTok{] = }\DecValTok{20}\NormalTok{;}

    \CommentTok{// Si on initialise à la déclaration, on n'a pas besoin de donner la taille}
    \DataTypeTok{int}\NormalTok{ tab_2[] = \{}\DecValTok{1}\NormalTok{, }\DecValTok{4}\NormalTok{, }\DecValTok{-1}\NormalTok{, }\DecValTok{4}\NormalTok{\};}
\end{Highlighting}
\end{Shaded}

\paragraph{\texorpdfstring{Déclarer un \textbf{type
tableau}}{Déclarer un type tableau}}\label{duxe9clarer-un-type-tableau}

La déclaration d'un type tableau est réalisé avec \texttt{typedef} :

\begin{Shaded}
\begin{Highlighting}[]
    \CommentTok{// declaration du type t_tab }
    \KeywordTok{typedef} \DataTypeTok{int}\NormalTok{ t_tab[NB];}
    \CommentTok{// declaration de variables tableau de type t_tab}
\NormalTok{    t_tab tab1, tab2;}
    \CommentTok{// l'accès aux données de tab1 et tab2 se fait de ma même façon : }
\NormalTok{    tab1[}\DecValTok{0}\NormalTok{] = }\DecValTok{20}\NormalTok{;}
\end{Highlighting}
\end{Shaded}

\begin{quote}
\textbf{Attention !} \texttt{tab1\ =\ tab2} \textbf{est interdit}.
\end{quote}

    \subsection{\#\#\# Exercice 15}\label{exercice-15}

\begin{center}\rule{0.5\linewidth}{\linethickness}\end{center}

(cf fichier \textbf{\texttt{exercice15.c}})

\textbf{{[}15.1{]}} Definir un type \texttt{t\_tableau} de réels de
capacité 20.

\textbf{{[}15.2{]}} Completer et corriger la fonction
\texttt{initialiser} qui permet d'initialiser chaque élément d'un
tableau de type \texttt{t\_tableau} à \texttt{0.0}.

\textbf{{[}15.3{]}} Completer et corriger la fonction \texttt{est\_vide}
qui vérifie que tous les éléments sont bien initialisés à \texttt{0.0}.

    \begin{Verbatim}[commandchars=\\\{\}]
{\color{incolor}In [{\color{incolor} }]:} \PY{c+cp}{\PYZsh{}}\PY{c+cp}{include} \PY{c+cpf}{\PYZlt{}stdlib.h\PYZgt{}}\PY{c+c1}{ }
        \PY{c+cp}{\PYZsh{}}\PY{c+cp}{include} \PY{c+cpf}{\PYZlt{}stdio.h\PYZgt{}}
        \PY{c+cp}{\PYZsh{}}\PY{c+cp}{include} \PY{c+cpf}{\PYZlt{}assert.h\PYZgt{}}
        \PY{c+cp}{\PYZsh{}}\PY{c+cp}{include} \PY{c+cpf}{\PYZlt{}stdbool.h\PYZgt{}}
        
        \PY{c+cp}{\PYZsh{}}\PY{c+cp}{define CAPACITE 20}
        \PY{c+c1}{// Definition du type tableau}
        \PY{k}{typedef} \PY{k+kt}{float} \PY{n}{t\PYZus{}tableau}\PY{p}{[}\PY{l+m+mi}{20}\PY{p}{]}\PY{p}{;}
        
        \PY{c+cm}{/**}
        \PY{c+cm}{ * \PYZbs{}brief Initialiser les éléments d\PYZsq{}un tableau de réels avec 0.0}
        \PY{c+cm}{ * \PYZbs{}param[out] tab tableau à initialiser}
        \PY{c+cm}{ * \PYZbs{}param[in] taille nombre d\PYZsq{}éléments du tableau}
        \PY{c+cm}{ * \PYZbs{}pre taille \PYZlt{}= CAPACITE}
        \PY{c+cm}{ */} 
        \PY{k+kt}{void} \PY{n+nf}{initialiser}\PY{p}{(}\PY{n}{t\PYZus{}tableau} \PY{n}{tab}\PY{p}{,} \PY{k+kt}{int} \PY{n}{taille}\PY{p}{)}\PY{p}{\PYZob{}}
            \PY{n}{assert}\PY{p}{(}\PY{n}{taille} \PY{o}{\PYZlt{}}\PY{o}{=} \PY{n}{CAPACITE}\PY{p}{)}\PY{p}{;}
            \PY{c+c1}{// TODO}
        \PY{p}{\PYZcb{}}
        
        \PY{c+cm}{/**}
        \PY{c+cm}{ * \PYZbs{}brief le tableau est\PYZhy{}il vide ?}
        \PY{c+cm}{ * \PYZbs{}param[in out] tab tableau à tester}
        \PY{c+cm}{ * \PYZbs{}param[in] taille nombre d\PYZsq{}éléments du tableau}
        \PY{c+cm}{ * \PYZbs{}pre taille \PYZlt{}= CAPACITE}
        \PY{c+cm}{ */} 
        \PY{k+kt}{bool} \PY{n+nf}{est\PYZus{}vide}\PY{p}{(}\PY{n}{t\PYZus{}tableau} \PY{n}{tab}\PY{p}{,} \PY{k+kt}{int} \PY{n}{taille}\PY{p}{)}\PY{p}{\PYZob{}}
            \PY{n}{assert}\PY{p}{(}\PY{n}{taille} \PY{o}{\PYZlt{}}\PY{o}{=} \PY{n}{CAPACITE}\PY{p}{)}\PY{p}{;}
            \PY{k+kt}{bool} \PY{n}{vide} \PY{o}{=} \PY{n+nb}{false}\PY{p}{;}
            \PY{c+c1}{// TODO}
            \PY{k}{return} \PY{n}{vide}\PY{p}{;}
        \PY{p}{\PYZcb{}}
        
        \PY{k+kt}{int} \PY{n+nf}{main}\PY{p}{(}\PY{k+kt}{void}\PY{p}{)}\PY{p}{\PYZob{}}
            \PY{n}{t\PYZus{}tableau} \PY{n}{T}\PY{p}{;}
            \PY{c+c1}{//Initialiser les éléments d\PYZsq{}une variable tableau à 0.0}
            \PY{n}{initialiser}\PY{p}{(}\PY{n}{T}\PY{p}{)}\PY{p}{;}
            \PY{c+c1}{//Vérifier avec assert que tous les éléments vallent bien 0.0}
            \PY{n}{assert}\PY{p}{(}\PY{n}{est\PYZus{}vide}\PY{p}{(}\PY{n}{T}\PY{p}{)}\PY{p}{)}\PY{p}{;}
            
            \PY{k}{return} \PY{n}{EXIT\PYZus{}SUCCESS}\PY{p}{;}
        \PY{p}{\PYZcb{}}
\end{Verbatim}


    \subsubsection{Chaines de caractères en
C}\label{chaines-de-caractuxe8res-en-c}

Les chaines de caractères sont des \textbf{tableaux de caractères} :

\begin{Shaded}
\begin{Highlighting}[]
    \DataTypeTok{char}\NormalTok{[}\DecValTok{10}\NormalTok{] ma_chaine; }\CommentTok{// un tableau de caractères de taille 10.}
    \DataTypeTok{char}\NormalTok{[] mon_nom = }\StringTok{"Jaffres-Runser"}\NormalTok{; }\CommentTok{// initialise le tableau avec une chaine constante}
\end{Highlighting}
\end{Shaded}

La bibliothèque \texttt{string.h} permet de manipuler les chaines de
caractères : - \texttt{strlen(s)} retourne le nombre de caractères de la
chaine - \texttt{strcpy(s1,\ s2)} recopie le contenu de \texttt{s1} dans
\texttt{s2}. Attention, il faut que \texttt{s2} ait une capacité
suffisante ! - \texttt{strcat} concatène deux chaines, etc.

\textbf{Exécuter} l'exemple suivant (cf fichier
\textbf{\texttt{chaine.c}}) :

    \begin{Verbatim}[commandchars=\\\{\}]
{\color{incolor}In [{\color{incolor}28}]:} \PY{c+cp}{\PYZsh{}}\PY{c+cp}{include} \PY{c+cpf}{\PYZlt{}string.h\PYZgt{}}
         \PY{c+cp}{\PYZsh{}}\PY{c+cp}{include} \PY{c+cpf}{\PYZlt{}stdio.h\PYZgt{}}
         \PY{c+cp}{\PYZsh{}}\PY{c+cp}{include} \PY{c+cpf}{\PYZlt{}stdlib.h\PYZgt{}}
         
         \PY{k+kt}{int} \PY{n+nf}{main}\PY{p}{(}\PY{k+kt}{void}\PY{p}{)}\PY{p}{\PYZob{}}
             \PY{k+kt}{char} \PY{n}{mon\PYZus{}nom}\PY{p}{[}\PY{p}{]} \PY{o}{=} \PY{l+s}{\PYZdq{}}\PY{l+s}{Jaffres\PYZhy{}Runser}\PY{l+s}{\PYZdq{}}\PY{p}{;} 
             \PY{n}{printf}\PY{p}{(}\PY{l+s}{\PYZdq{}}\PY{l+s}{Longueur de \PYZsq{}\PYZpc{}s\PYZsq{} : \PYZpc{}lu caractères}\PY{l+s+se}{\PYZbs{}n}\PY{l+s}{\PYZdq{}}\PY{p}{,} \PY{n}{mon\PYZus{}nom}\PY{p}{,} \PY{n}{strlen}\PY{p}{(}\PY{n}{mon\PYZus{}nom}\PY{p}{)}\PY{p}{)}\PY{p}{;}
             \PY{n}{printf}\PY{p}{(}\PY{l+s}{\PYZdq{}}\PY{l+s}{Taille du tableau : \PYZpc{}lu éléments }\PY{l+s+se}{\PYZbs{}n}\PY{l+s}{\PYZdq{}}\PY{p}{,} \PY{k}{sizeof}\PY{p}{(}\PY{n}{mon\PYZus{}nom}\PY{p}{)}\PY{p}{)}\PY{p}{;}
             \PY{n}{printf}\PY{p}{(}\PY{l+s}{\PYZdq{}}\PY{l+s}{dernier élément : \PYZsq{}\PYZpc{}i\PYZsq{} }\PY{l+s+se}{\PYZbs{}n}\PY{l+s}{\PYZdq{}}\PY{p}{,} \PY{n}{mon\PYZus{}nom}\PY{p}{[}\PY{k}{sizeof}\PY{p}{(}\PY{n}{mon\PYZus{}nom}\PY{p}{)}\PY{l+m+mi}{\PYZhy{}1}\PY{p}{]}\PY{p}{)}\PY{p}{;}
             \PY{k}{return} \PY{n}{EXIT\PYZus{}SUCCESS}\PY{p}{;}
         \PY{p}{\PYZcb{}}
\end{Verbatim}


    \begin{Verbatim}[commandchars=\\\{\}]
Longueur de 'Jaffres-Runser' : 14 caractères
Taille du tableau : 15 éléments 
dernier élément : '0' 

    \end{Verbatim}

    Le dernier caractère d'une chaine est le caractère
\texttt{\textbackslash{}0} de code ascii 0. \textgreater{}
\textbf{Règle} : une chaine de caractère se termine toujours par le
caractère \texttt{\textbackslash{}0}.

    \begin{center}\rule{0.5\linewidth}{\linethickness}\end{center}

    \subsection{\#\# Type pointeur et adresse
mémoire}\label{type-pointeur-et-adresse-muxe9moire}

    \subsubsection{Adresse mémoire}\label{adresse-muxe9moire}

En C, il est possible de connaitre l'adresse mémoire à laquelle est
stockée une variable \texttt{var1} avec l'opérateur unaire \texttt{\&}
(éperluette) :

\begin{quote}
\&var1
\end{quote}

    \subsubsection{Déclaration d'un
pointeur}\label{duxe9claration-dun-pointeur}

Il est possible d'enregistrer cette adresse dans une variable de type
\textbf{pointeur}. Une variable de type pointeur est aussi communément
appelée pointeur. Elle enregistre la référence (i.e. l'adresse) d'une
variable ou d'une donnée.

Pour déclarer un pointeur en C, il faut connaitre le type de la donnée
qui sera enregistrée à cette adresse. Par exemple, on peut déclarer un
pointeur sur une variable de type \texttt{entier} ou un pointeur sur une
variable de type \texttt{double}. Pour déclarer un pointeur, on met *
devant le nom de la variable :

\begin{quote}
\texttt{type\_pointé*\ pointeur;}
\end{quote}

Quelques exemples :

\begin{Shaded}
\begin{Highlighting}[]
    \DataTypeTok{int}\NormalTok{* ptr_int; }\CommentTok{// déclaration du pointeur ptr_int sur un entier}
    \DataTypeTok{double}\NormalTok{* ptr_dbl; }\CommentTok{// déclaration du pointeur ptr_dbl sur un double }
    \DataTypeTok{char}\NormalTok{* ptr_char; }\CommentTok{// un pointeur sur un caractère}
    \DataTypeTok{int}\NormalTok{** ptr_ptr_int; }\CommentTok{// un pointeur sur un pointeur, qui pointe sur un entier}
\end{Highlighting}
\end{Shaded}

\begin{quote}
Note : il peut y avoir un espace avant l'opérateur *.
\end{quote}

    \subsubsection{Initialisation d'un
pointeur}\label{initialisation-dun-pointeur}

Comme pour n'importe quelle déclaration de variable en C, le pointeur
n'est pas initialisé à une valeur par défaut à sa déclaration. En
d'autres termes, l'adresse enregistrée n'a aucun sens, elle est
aléatoire.

On doit initialiser un pointeur soit avec : - Le pointeur \texttt{NULL}
(élément neutre des adresses possibles) :

\begin{Shaded}
\begin{Highlighting}[]
    \DataTypeTok{double}\NormalTok{* ptr_d = NULL;}
\end{Highlighting}
\end{Shaded}

\begin{itemize}
\item
  Soit avec l'adresse mémoire d'une variable du bon type:

\begin{Shaded}
\begin{Highlighting}[]
\DataTypeTok{int}\NormalTok{ var1 = }\DecValTok{10}\NormalTok{;}
\DataTypeTok{int}\NormalTok{* ptr_int = &var1; }\CommentTok{// initialisation avec l'adresse de la variable var1}
\end{Highlighting}
\end{Shaded}
\item
  Ou avec la valeur d'un pointeur de même type

\begin{Shaded}
\begin{Highlighting}[]
\DataTypeTok{int}\NormalTok{* ptr_int_2 = ptr_int; }
\end{Highlighting}
\end{Shaded}
\end{itemize}

\begin{quote}
\textbf{Si l'adresse est n'est pas connue au moment de la déclaration,
il faut toujours initialiser le pointeur à \texttt{NULL}}
\end{quote}

    \subsubsection{Accès à la donnée
pointée}\label{accuxe8s-uxe0-la-donnuxe9e-pointuxe9e}

Pour accéder à la variable pointée, on utilise aussi l'opérateur * placé
avant l'identificateur. \textgreater{} \texttt{*ptr\_int\ =\ 25}

Quelques exemples :

\begin{Shaded}
\begin{Highlighting}[]
    \DataTypeTok{int}\NormalTok{ var1 = }\DecValTok{10}\NormalTok{;}
    \DataTypeTok{int}\NormalTok{* ptr_int = &var1;}
\NormalTok{    *ptr_int = }\DecValTok{20} \CommentTok{// On modifie ici la variable var1, qui vaudra 20 par la suite.}
    \CommentTok{// Déclaration et initialisation d'un nouvel entier var2 avec la donnée référencée par le pointeur ptr_int}
    \DataTypeTok{int}\NormalTok{ var2 = *ptr_int ; }
\NormalTok{    assert(var1 == }\DecValTok{20}\NormalTok{);}
\end{Highlighting}
\end{Shaded}

    \subsubsection{Affectation de pointeurs}\label{affectation-de-pointeurs}

Affecter un pointeur \texttt{p1} à un pointeur \texttt{p2}, comme pour
toute affectation, recopie l'adresse \texttt{p1} dans \texttt{p2}. Les
deux pointeurs référencent alors la même zone mémoire.

Quelques exemples :

\begin{Shaded}
\begin{Highlighting}[]
    \DataTypeTok{int}\NormalTok{ var1 = }\DecValTok{10}\NormalTok{;}
    \DataTypeTok{int}\NormalTok{* p1 = &var1;}
    \DataTypeTok{int}\NormalTok{* p2 = p1;}
    \CommentTok{// A cet instant, on peut modifier le contenu de var1 en passant par p1 ou par p2.}
\NormalTok{    *p1 = }\DecValTok{100}\NormalTok{; }\CommentTok{//var1 vaut 100}
\NormalTok{    *p2 = }\DecValTok{1000}\NormalTok{; }\CommentTok{//var1 vaut 1000 maintenant !}
\end{Highlighting}
\end{Shaded}

    \subsection{\#\#\# Exercice 16 : Manipulation de
pointeurs}\label{exercice-16-manipulation-de-pointeurs}

\begin{center}\rule{0.5\linewidth}{\linethickness}\end{center}

    \textbf{{[}16.1{]}} Compiler le programme suivant (cf fichier
\textbf{\texttt{pointeur1.c}}).

\begin{itemize}
\tightlist
\item
  Qu'observez-vous pour le premier affichage ? Le résultat dépend du
  compilateur, du système, etc.
\item
  Qu'observez-vous pour le second affichage ?
\end{itemize}

    \begin{Verbatim}[commandchars=\\\{\}]
{\color{incolor}In [{\color{incolor} }]:} \PY{c+cp}{\PYZsh{}}\PY{c+cp}{include} \PY{c+cpf}{\PYZlt{}stdlib.h\PYZgt{}}\PY{c+c1}{ }
        \PY{c+cp}{\PYZsh{}}\PY{c+cp}{include} \PY{c+cpf}{\PYZlt{}stdio.h\PYZgt{}}
        
        \PY{k+kt}{int} \PY{n+nf}{main}\PY{p}{(}\PY{p}{)}\PY{p}{\PYZob{}}
            \PY{k+kt}{int} \PY{n}{d1} \PY{o}{=} \PY{l+m+mi}{1}\PY{p}{;}
            \PY{k+kt}{int} \PY{n}{d2} \PY{o}{=} \PY{l+m+mi}{4}\PY{p}{;} 
            \PY{k+kt}{int}\PY{o}{*} \PY{n}{p\PYZus{}1} \PY{p}{;}
            \PY{k+kt}{int}\PY{o}{*} \PY{n}{p\PYZus{}2} \PY{p}{;}
            \PY{n}{printf}\PY{p}{(}\PY{l+s}{\PYZdq{}}\PY{l+s}{*p\PYZus{}1 = \PYZpc{}d, *p\PYZus{}2 = \PYZpc{}d}\PY{l+s+se}{\PYZbs{}n}\PY{l+s}{\PYZdq{}}\PY{p}{,} \PY{o}{*}\PY{n}{p\PYZus{}1}\PY{p}{,} \PY{o}{*}\PY{n}{p\PYZus{}2}\PY{p}{)}\PY{p}{;}
            \PY{n}{printf}\PY{p}{(}\PY{l+s}{\PYZdq{}}\PY{l+s}{p\PYZus{}1 = \PYZpc{}p, p\PYZus{}2 = \PYZpc{}p}\PY{l+s}{\PYZdq{}}\PY{p}{,} \PY{n}{p\PYZus{}1}\PY{p}{,} \PY{n}{p\PYZus{}2}\PY{p}{)}\PY{p}{;}
            \PY{k}{return} \PY{n}{EXIT\PYZus{}SUCCESS}\PY{p}{;}
        \PY{p}{\PYZcb{}}
\end{Verbatim}


    \textbf{{[}16.2{]}} Modifier ce programme pour que \texttt{p\_1} et
\texttt{p\_2}pointent respectivement sur \texttt{d1} et \texttt{d2} (cf
fichier \textbf{\texttt{pointeur2.c}}).

    \begin{Verbatim}[commandchars=\\\{\}]
{\color{incolor}In [{\color{incolor} }]:} \PY{c+cp}{\PYZsh{}}\PY{c+cp}{include} \PY{c+cpf}{\PYZlt{}stdlib.h\PYZgt{}}\PY{c+c1}{ }
        \PY{c+cp}{\PYZsh{}}\PY{c+cp}{include} \PY{c+cpf}{\PYZlt{}stdio.h\PYZgt{}}
        
        \PY{k+kt}{int} \PY{n+nf}{main}\PY{p}{(}\PY{p}{)}\PY{p}{\PYZob{}}
            \PY{k+kt}{int} \PY{n}{d1} \PY{o}{=} \PY{l+m+mi}{1}\PY{p}{;}
            \PY{k+kt}{int} \PY{n}{d2} \PY{o}{=} \PY{l+m+mi}{4}\PY{p}{;} 
            \PY{k+kt}{int}\PY{o}{*} \PY{n}{p\PYZus{}1} \PY{p}{;}
            \PY{k+kt}{int}\PY{o}{*} \PY{n}{p\PYZus{}2} \PY{p}{;}
            \PY{n}{printf}\PY{p}{(}\PY{l+s}{\PYZdq{}}\PY{l+s}{*p\PYZus{}1 = \PYZpc{}d, *p\PYZus{}2 = \PYZpc{}d}\PY{l+s}{\PYZdq{}}\PY{p}{,} \PY{o}{*}\PY{n}{d1}\PY{p}{,} \PY{o}{*}\PY{n}{d2}\PY{p}{)}\PY{p}{;}
            \PY{k}{return} \PY{n}{EXIT\PYZus{}SUCCESS}\PY{p}{;}
        \PY{p}{\PYZcb{}}
\end{Verbatim}


    \textbf{{[}16.3{]}} Compléter le programme pour échanger les entiers
pointés par \texttt{p1} et \texttt{p2}. Après initialisation des
pointeurs, on n'accèdera aux entiers qu'à travers des pointeurs (cf
fichier \textbf{\texttt{pointeur3.c}}).

    \begin{Verbatim}[commandchars=\\\{\}]
{\color{incolor}In [{\color{incolor} }]:} \PY{c+cp}{\PYZsh{}}\PY{c+cp}{include} \PY{c+cpf}{\PYZlt{}stdlib.h\PYZgt{}}\PY{c+c1}{ }
        \PY{c+cp}{\PYZsh{}}\PY{c+cp}{include} \PY{c+cpf}{\PYZlt{}stdio.h\PYZgt{}}
        
        \PY{k+kt}{int} \PY{n+nf}{main}\PY{p}{(}\PY{p}{)}\PY{p}{\PYZob{}}
            \PY{k+kt}{int} \PY{n}{d1} \PY{o}{=} \PY{l+m+mi}{1}\PY{p}{;}
            \PY{k+kt}{int} \PY{n}{d2} \PY{o}{=} \PY{l+m+mi}{4}\PY{p}{;} 
            \PY{k+kt}{int}\PY{o}{*} \PY{n}{p\PYZus{}1} \PY{p}{;}
            \PY{k+kt}{int}\PY{o}{*} \PY{n}{p\PYZus{}2} \PY{p}{;}
            \PY{n}{printf}\PY{p}{(}\PY{l+s}{\PYZdq{}}\PY{l+s}{Avant échange : *p\PYZus{}1 = \PYZpc{}d, *p\PYZus{}2 = \PYZpc{}d}\PY{l+s}{\PYZdq{}}\PY{p}{,} \PY{o}{*}\PY{n}{p\PYZus{}1}\PY{p}{,} \PY{o}{*}\PY{n}{p\PYZus{}2}\PY{p}{)}\PY{p}{;}
            
            \PY{c+c1}{// TODO : echanger les pointeurs}
            
            \PY{n}{printf}\PY{p}{(}\PY{l+s}{\PYZdq{}}\PY{l+s}{Après échange : *p\PYZus{}1 = \PYZpc{}d, *p\PYZus{}2 = \PYZpc{}d}\PY{l+s}{\PYZdq{}}\PY{p}{,} \PY{o}{*}\PY{n}{p\PYZus{}1}\PY{p}{,} \PY{o}{*}\PY{n}{p\PYZus{}2}\PY{p}{)}\PY{p}{;}
            \PY{k}{return} \PY{n}{EXIT\PYZus{}SUCCESS}\PY{p}{;}
        \PY{p}{\PYZcb{}}
\end{Verbatim}


    \textbf{{[}16.4{]}} Qu'en est-il des données enregistrées dans
\texttt{d1} et \texttt{d2} ? Ont-elles changé ?

    \begin{center}\rule{0.5\linewidth}{\linethickness}\end{center}

\subsubsection{Enregistrement et
pointeurs}\label{enregistrement-et-pointeurs}

On suppose le type enregistrement \texttt{point} suivant :

\begin{Shaded}
\begin{Highlighting}[]
    \KeywordTok{struct}\NormalTok{ point \{}
        \DataTypeTok{int}\NormalTok{ x;}
        \DataTypeTok{int}\NormalTok{ y;}
\NormalTok{    \};}
    \KeywordTok{typedef} \KeywordTok{struct}\NormalTok{ point point;}
\end{Highlighting}
\end{Shaded}

Un pointeur sur un enregistrement permet d'accéder au contenu de
l'enregistrement de deux manières :

\begin{enumerate}
\def\labelenumi{\arabic{enumi}.}
\item
  Avec les opérateurs \textbf{* et .}

\begin{Shaded}
\begin{Highlighting}[]
\NormalTok{point pt1;}
\KeywordTok{struct}\NormalTok{ point * ptr_point = &pt1; }
\NormalTok{(*ptr_point).x = }\DecValTok{12}\NormalTok{; }
\NormalTok{(*ptr_point).y = }\DecValTok{0}\NormalTok{; }
\end{Highlighting}
\end{Shaded}
\item
  Avec l'opérateur \textbf{-\textgreater{}}

\begin{Shaded}
\begin{Highlighting}[]
\NormalTok{point pt1;}
\KeywordTok{struct}\NormalTok{ point * ptr_point = &pt1; }
\NormalTok{ptr_point->x = }\DecValTok{12}\NormalTok{; }
\NormalTok{ptr_point->y = }\DecValTok{0}\NormalTok{; }
\end{Highlighting}
\end{Shaded}

  \begin{quote}
  \textbf{Règle} : \textbf{Il faut utiliser la notation -\textgreater{}}
  \end{quote}
\end{enumerate}

    \subsubsection{Tableau et pointeurs}\label{tableau-et-pointeurs}

En C, le nom de la variable tableau est \textbf{l'identifiant d'un
pointeur sur la première case} du tableau. On peut donc accéder au
contenu de la première case par ce pointeur. Par exemple :

\begin{Shaded}
\begin{Highlighting}[]
    \DataTypeTok{int}\NormalTok{ tab[] = \{}\DecValTok{1}\NormalTok{, }\DecValTok{4}\NormalTok{, }\DecValTok{8}\NormalTok{, }\DecValTok{16}\NormalTok{\};}
    \CommentTok{// tab est un pointeur sur la case 0}
\NormalTok{    *tab = }\DecValTok{20}\NormalTok{; }\CommentTok{// équivalent à tab[0] = 20}
\end{Highlighting}
\end{Shaded}

Il est possible d'accéder à la case suivante \textbf{en incrémentant de
1 le pointeur} (arithmétique des pointeurs) :

\begin{Shaded}
\begin{Highlighting}[]
\NormalTok{    *(tab+}\DecValTok{1}\NormalTok{) = }\DecValTok{40}\NormalTok{; }\CommentTok{// équivalent à tab[1] = 40}
    \CommentTok{// déclaration d'un pointeur sur la 4e case du tableau}
    \DataTypeTok{int}\NormalTok{* ptr = tab+}\DecValTok{3}\NormalTok{;}
\NormalTok{    assert(*ptr == }\DecValTok{16}\NormalTok{);}
\end{Highlighting}
\end{Shaded}

L'opérateur \textbf{-} permet de se déplacer vers la gauche dans le
tableau :

\begin{Shaded}
\begin{Highlighting}[]
\NormalTok{    ptr = ptr}\DecValTok{-2}\NormalTok{;}
    \CommentTok{//ptr pointe sur la 2e case du tableau}
\NormalTok{    assert(*ptr == }\DecValTok{40}\NormalTok{);}
\end{Highlighting}
\end{Shaded}

    \subsection{\#\#\# Exercice 17}\label{exercice-17}

\begin{center}\rule{0.5\linewidth}{\linethickness}\end{center}

\textbf{{[}17.1{]}} Ré-écrire la fonction \texttt{initialiser} de
l'exercice 15 avec la notation pointeur du tableau et l'arithmétique
associée. (cf fichier \textbf{\texttt{exercice17.c}}).

    \begin{Verbatim}[commandchars=\\\{\}]
{\color{incolor}In [{\color{incolor} }]:} \PY{c+cp}{\PYZsh{}}\PY{c+cp}{include} \PY{c+cpf}{\PYZlt{}stdlib.h\PYZgt{}}\PY{c+c1}{ }
        \PY{c+cp}{\PYZsh{}}\PY{c+cp}{include} \PY{c+cpf}{\PYZlt{}stdio.h\PYZgt{}}
        \PY{c+cp}{\PYZsh{}}\PY{c+cp}{include} \PY{c+cpf}{\PYZlt{}assert.h\PYZgt{}}
        \PY{c+cp}{\PYZsh{}}\PY{c+cp}{include} \PY{c+cpf}{\PYZlt{}stdbool.h\PYZgt{}}
        
        \PY{c+cp}{\PYZsh{}}\PY{c+cp}{define CAPACITE 20}
        \PY{c+c1}{// Definition du type tableau}
        \PY{c+c1}{// TODO }
        
        \PY{c+cm}{/**}
        \PY{c+cm}{ * \PYZbs{}brief Initialiser les éléments d\PYZsq{}un tableau de réels avec 0.0}
        \PY{c+cm}{ * \PYZbs{}param[out] tab tableau à initialiser}
        \PY{c+cm}{ * \PYZbs{}param[in] taille nombre d\PYZsq{}éléments du tableau}
        \PY{c+cm}{ * \PYZbs{}pre taille \PYZlt{}= CAPACITE}
        \PY{c+cm}{ */} 
        \PY{k+kt}{void} \PY{n+nf}{initialiser}\PY{p}{(}\PY{n}{t\PYZus{}tableau} \PY{n}{tab}\PY{p}{,} \PY{k+kt}{int} \PY{n}{taille}\PY{p}{)}\PY{p}{\PYZob{}}
            \PY{n}{assert}\PY{p}{(}\PY{n}{taille} \PY{o}{\PYZlt{}}\PY{o}{=} \PY{n}{CAPACITE}\PY{p}{)}\PY{p}{;}
            \PY{c+c1}{// TODO}
        \PY{p}{\PYZcb{}}
        
        \PY{k+kt}{int} \PY{n+nf}{main}\PY{p}{(}\PY{k+kt}{void}\PY{p}{)}\PY{p}{\PYZob{}}
            \PY{n}{t\PYZus{}tableau} \PY{n}{T}\PY{p}{;}
            \PY{c+c1}{//Initialiser les éléments d\PYZsq{}une variable tableau à 0.0}
            \PY{n}{initialiser}\PY{p}{(}\PY{n}{T}\PY{p}{)}\PY{p}{;}
            
            \PY{k}{return} \PY{n}{EXIT\PYZus{}SUCCESS}\PY{p}{;}
        \PY{p}{\PYZcb{}}
\end{Verbatim}


    \begin{center}\rule{0.5\linewidth}{\linethickness}\end{center}

    \subsection{\#\# Les sous-programmes}\label{les-sous-programmes}

    Le langage C ne permet pas de différencier les fonctions des procédures
algorithmiques. Le seul sous-programme utilisable est la fonction :

\begin{quote}
\texttt{type\_retour\ identificateur\_fonction\ (\ type\_param1\ id\_param1,\ type\_param2\ id\_param2,\ ...)}
\end{quote}

Chaque paramètre formel est typé et \textbf{passé par valeur} (mode IN
algorithmique). Ainsi, l'appel à une fonction sur des paramètres réels
variables ne modifie pas la donnée. Par contre, les instructions de la
fonction connaissent la donnée (valeur) et peuvent la manipuler pour
fournir l'unique résulat retourné via le type retour.

    \subsubsection{Illustration du passage par
valeur}\label{illustration-du-passage-par-valeur}

Exécuter le programme suivant (cf fichier \textbf{\texttt{valeur.c}}):

    \begin{Verbatim}[commandchars=\\\{\}]
{\color{incolor}In [{\color{incolor} }]:} \PY{c+cp}{\PYZsh{}}\PY{c+cp}{include} \PY{c+cpf}{\PYZlt{}stdio.h\PYZgt{}}
        
        \PY{c+c1}{// Definition d\PYZsq{}une fonction f1}
        \PY{k+kt}{int} \PY{n+nf}{f1}\PY{p}{(}\PY{k+kt}{int} \PY{n}{valeur}\PY{p}{)} \PY{p}{\PYZob{}} 
            \PY{n}{printf}\PY{p}{(}\PY{l+s}{\PYZdq{}}\PY{l+s}{   valeur au début de f1 : \PYZpc{}i }\PY{l+s+se}{\PYZbs{}n}\PY{l+s}{\PYZdq{}}\PY{p}{,} \PY{n}{valeur}\PY{p}{)}\PY{p}{;}
            \PY{n}{valeur} \PY{o}{=} \PY{l+m+mi}{0}\PY{p}{;}
            \PY{n}{printf}\PY{p}{(}\PY{l+s}{\PYZdq{}}\PY{l+s}{   valeur à la fin de f1 : \PYZpc{}i }\PY{l+s+se}{\PYZbs{}n}\PY{l+s}{\PYZdq{}}\PY{p}{,} \PY{n}{valeur}\PY{p}{)}\PY{p}{;}
            \PY{k}{return} \PY{n}{valeur}\PY{p}{;}
        \PY{p}{\PYZcb{}}
        
        \PY{k+kt}{int} \PY{n+nf}{main}\PY{p}{(}\PY{p}{)}\PY{p}{\PYZob{}}
            \PY{k+kt}{int} \PY{n}{donnee} \PY{o}{=} \PY{l+m+mi}{20}\PY{p}{;}
            \PY{n}{printf}\PY{p}{(}\PY{l+s}{\PYZdq{}}\PY{l+s}{donnee dans main() avant f1 : \PYZpc{}i  }\PY{l+s+se}{\PYZbs{}n}\PY{l+s}{\PYZdq{}}\PY{p}{,} \PY{n}{donnee}\PY{p}{)}\PY{p}{;}
            \PY{k+kt}{int} \PY{n}{donnee\PYZus{}retournee} \PY{o}{=} \PY{n}{f1}\PY{p}{(}\PY{n}{donnee}\PY{p}{)}\PY{p}{;} \PY{c+c1}{// la fonction utilise la valeur de donnee}
            \PY{n}{printf}\PY{p}{(}\PY{l+s}{\PYZdq{}}\PY{l+s}{donnee dans main() après f1 : \PYZpc{}i  }\PY{l+s+se}{\PYZbs{}n}\PY{l+s}{\PYZdq{}}\PY{p}{,} \PY{n}{donnee}\PY{p}{)}\PY{p}{;}
            \PY{n}{printf}\PY{p}{(}\PY{l+s}{\PYZdq{}}\PY{l+s}{donnee\PYZus{}retournee dans main() : \PYZpc{}i }\PY{l+s+se}{\PYZbs{}n}\PY{l+s}{\PYZdq{}}\PY{p}{,} \PY{n}{donnee\PYZus{}retournee}\PY{p}{)}\PY{p}{;} 
        \PY{p}{\PYZcb{}}
\end{Verbatim}


    \subsubsection{Passage par adresse}\label{passage-par-adresse}

Pour pouvoir modifier le contenu d'une variable passée en paramètre
d'une fonction (mode OUT ou IN OUT algorithmique), on fournit à la
fontion \textbf{l'adresse de la variable}. Connaissant l'adresse, la
fonction pourra alors modifier sa valeur.

On passe l'adresse d'une variable à une fonction à l'aide d'un pointeur.
Pour se faire, il faut déclarer la fonction avec des paramètres formels
qui sont des pointeurs.

Petite illustration du \textbf{passage de paramètres par adresse} (cf
fichier \textbf{\texttt{adresse.c}}):

    \begin{Verbatim}[commandchars=\\\{\}]
{\color{incolor}In [{\color{incolor}31}]:} \PY{c+cp}{\PYZsh{}}\PY{c+cp}{include} \PY{c+cpf}{\PYZlt{}stdio.h\PYZgt{}}
         
         \PY{c+c1}{// Definition d\PYZsq{}une fonction f1 avec un paramètre pointeur qui }
         \PY{c+c1}{// peut enregistrer d\PYZsq{}adresse d\PYZsq{}un entier}
         \PY{k+kt}{int} \PY{n+nf}{f1}\PY{p}{(}\PY{k+kt}{int}\PY{o}{*} \PY{n}{valeur}\PY{p}{)} \PY{p}{\PYZob{}} 
             \PY{o}{*}\PY{n}{valeur} \PY{o}{+}\PY{o}{=} \PY{o}{*}\PY{n}{valeur}\PY{p}{;} \PY{c+c1}{// Accès à la variable au travers du pointeur. }
             \PY{n}{printf}\PY{p}{(}\PY{l+s}{\PYZdq{}}\PY{l+s}{valeur dans f1 après incrémentation : \PYZpc{}i }\PY{l+s+se}{\PYZbs{}n}\PY{l+s}{\PYZdq{}}\PY{p}{,} \PY{o}{*}\PY{n}{valeur}\PY{p}{)}\PY{p}{;}
             \PY{k}{return} \PY{o}{*}\PY{n}{valeur}\PY{p}{;}
         \PY{p}{\PYZcb{}}
         
         \PY{k+kt}{int} \PY{n+nf}{main}\PY{p}{(}\PY{p}{)}\PY{p}{\PYZob{}}
             \PY{k+kt}{int} \PY{n}{donnee} \PY{o}{=} \PY{l+m+mi}{20}\PY{p}{;}
             \PY{c+c1}{// Pour utiliser la fonction, on donne l\PYZsq{}adresse de la variable  }
             \PY{k+kt}{int} \PY{n}{nouvelle\PYZus{}donnee} \PY{o}{=} \PY{n}{f1}\PY{p}{(}\PY{o}{\PYZam{}}\PY{n}{donnee}\PY{p}{)}\PY{p}{;} 
             \PY{n}{printf}\PY{p}{(}\PY{l+s}{\PYZdq{}}\PY{l+s}{donnee dans main() après incrémentation : \PYZpc{}i  }\PY{l+s+se}{\PYZbs{}n}\PY{l+s}{\PYZdq{}}\PY{p}{,} \PY{n}{donnee}\PY{p}{)}\PY{p}{;}
             \PY{n}{printf}\PY{p}{(}\PY{l+s}{\PYZdq{}}\PY{l+s}{nouvelle\PYZus{}donne dans main() : \PYZpc{}i }\PY{l+s+se}{\PYZbs{}n}\PY{l+s}{\PYZdq{}}\PY{p}{,} \PY{n}{nouvelle\PYZus{}donnee}\PY{p}{)}\PY{p}{;} 
         \PY{p}{\PYZcb{}}
\end{Verbatim}


    \begin{Verbatim}[commandchars=\\\{\}]
valeur dans f1 après incrémentation : 40 
donnee dans main() après incrémentation : 40  
nouvelle\_donne dans main() : 40 

    \end{Verbatim}

    Pour ce passage par adresse, il faut : - utiliser des pointeurs pour
définir les paramètres formels - dans les instructions de la fonction,
accéder à la donnée pointée avec l'opérateur * - lors de l'appel du
sous-programme, fournir une adresse valide d'une variable à modifier
avec l'opérateur \&.

    \subsubsection{Passage d'un paramètre de type tableau en
C}\label{passage-dun-paramuxe8tre-de-type-tableau-en-c}

Un tableau étant un pointeur, le passage par valeur d'un paramètre
tableau offre naturellement un passage en mode \texttt{in\ out}. Ainsi :

\begin{itemize}
\tightlist
\item
  Il n'est pas nécessaire de passer un tableau par adresse si on
  souhaite in mode \texttt{in\ out}.
\item
  Si on veut définir un mode \texttt{in}, il faut empêche la
  modification en utilisant \texttt{const} :
\end{itemize}

\begin{Shaded}
\begin{Highlighting}[]
\CommentTok{/*}
\CommentTok{ * \textbackslash{}brief Affiche un tableau de taille éléments}
\CommentTok{ * \textbackslash{}param[in] tab tableau à afficher}
\CommentTok{ * \textbackslash{}param[in] taille nombre d'éléments du tableau}
\CommentTok{ * \textbackslash{}pre taille <= CAPACITE}
\CommentTok{ */} 
\DataTypeTok{void}\NormalTok{ afficher_tab (}\DataTypeTok{const} \DataTypeTok{int}\NormalTok{[] tab, }\DataTypeTok{int}\NormalTok{ taille)}
\end{Highlighting}
\end{Shaded}

    \subsection{\#\#\# Exercice 18 : passage par
adresse}\label{exercice-18-passage-par-adresse}

\begin{center}\rule{0.5\linewidth}{\linethickness}\end{center}

Compléter le programme suivant en répondant aux questions suivantes (cf
fichier \textbf{\texttt{exercice18.c}}):

\textbf{{[}18.1{]}} Définir le type \texttt{t\_note}, caractérisé par sa
valeur et son coefficient. Par exemple, la note de 14 a été obtenue pour
le BE d'algorithmique et programmation qui compte coefficient 1/4.

\textbf{{[}18.2{]}} Définir le type \texttt{t\_tab\_notes} qui permet
d'enregistrer 5 notes.

\textbf{{[}18.3{]}} Compléter et corriger la fonction qui initialise une
note à partir de sa valeur et de son coefficient.

\textbf{{[}18.4{]}} Compléter et corriger la fonction qui calcule la
moyenne des notes d'un tableau de notes.
\textgreater{}\textbf{Attention} il faut respecter le mode \texttt{in}
du paramètre tableau.

    \begin{Verbatim}[commandchars=\\\{\}]
{\color{incolor}In [{\color{incolor} }]:} \PY{c+cp}{\PYZsh{}}\PY{c+cp}{include} \PY{c+cpf}{\PYZlt{}stdlib.h\PYZgt{}}\PY{c+c1}{ }
        \PY{c+cp}{\PYZsh{}}\PY{c+cp}{include} \PY{c+cpf}{\PYZlt{}stdio.h\PYZgt{}}
        \PY{c+cp}{\PYZsh{}}\PY{c+cp}{include} \PY{c+cpf}{\PYZlt{}assert.h\PYZgt{}}
        \PY{c+cp}{\PYZsh{}}\PY{c+cp}{include} \PY{c+cpf}{\PYZlt{}stdbool.h\PYZgt{}}
        
        \PY{c+c1}{// Definition du type t\PYZus{}note}
        \PY{c+c1}{// TODO }
        
        \PY{c+c1}{// Definition d\PYZsq{}un tableau de notes t\PYZus{}tab\PYZus{}notes de 5 éléments.}
        \PY{c+c1}{// TODO}
        
        \PY{c+cm}{/**}
        \PY{c+cm}{ * \PYZbs{}brief Initialiser une note}
        \PY{c+cm}{ * \PYZbs{}param[out] note note à initialiser}
        \PY{c+cm}{ * \PYZbs{}param[in] valeur nombre de points}
        \PY{c+cm}{ * \PYZbs{}param[in] coef coefficient}
        \PY{c+cm}{ * \PYZbs{}pre valeur \PYZlt{}= 20 \PYZam{}\PYZam{} valeur \PYZgt{}= 0}
        \PY{c+cm}{ * \PYZbs{}pre coef \PYZlt{}= 1 \PYZam{}\PYZam{} coef \PYZgt{}= 0}
        \PY{c+cm}{ */} 
        \PY{k+kt}{void} \PY{n+nf}{initialiser\PYZus{}note}\PY{p}{(}\PY{n}{t\PYZus{}note} \PY{n}{note}\PY{p}{,} \PY{k+kt}{float} \PY{n}{valeur}\PY{p}{,} \PY{k+kt}{float} \PY{n}{coef}\PY{p}{)}\PY{p}{\PYZob{}}
            \PY{n}{assert}\PY{p}{(}\PY{n}{valeur} \PY{o}{\PYZlt{}}\PY{o}{=} \PY{l+m+mi}{20} \PY{o}{\PYZam{}}\PY{o}{\PYZam{}} \PY{n}{valeur} \PY{o}{\PYZgt{}}\PY{o}{=} \PY{l+m+mi}{0}\PY{p}{)}\PY{p}{;}
            \PY{n}{assert}\PY{p}{(}\PY{n}{coef} \PY{o}{\PYZlt{}}\PY{o}{=} \PY{l+m+mi}{20} \PY{o}{\PYZam{}}\PY{o}{\PYZam{}} \PY{n}{coef} \PY{o}{\PYZgt{}}\PY{o}{=} \PY{l+m+mi}{0}\PY{p}{)}\PY{p}{;}
            \PY{c+c1}{// TODO}
        \PY{p}{\PYZcb{}}
        
        
        \PY{c+cm}{/**}
        \PY{c+cm}{ * \PYZbs{}brief Calculer la moyenne des notes du tableau }
        \PY{c+cm}{ * \PYZbs{}param[in] tab\PYZus{}notes tableau de nodes}
        \PY{c+cm}{ * \PYZbs{}param[in] nb\PYZus{}notes nombre de notes}
        \PY{c+cm}{ */} 
        \PY{k+kt}{float} \PY{n+nf}{moyenne}\PY{p}{(}\PY{n}{t\PYZus{}tab\PYZus{}notes} \PY{n}{tab\PYZus{}notes}\PY{p}{,} \PY{k+kt}{int} \PY{n}{nb\PYZus{}notes}\PY{p}{)}\PY{p}{\PYZob{}}
            \PY{c+c1}{// TODO}
            \PY{k}{return} \PY{l+m+mi}{0}\PY{p}{;}
        \PY{p}{\PYZcb{}}
        
        
        \PY{k+kt}{int} \PY{n+nf}{main}\PY{p}{(}\PY{k+kt}{void}\PY{p}{)}\PY{p}{\PYZob{}}
            \PY{n}{t\PYZus{}tab\PYZus{}note} \PY{n}{notes}\PY{p}{;}
            
            \PY{c+c1}{//Initialiser les éléments d\PYZsq{}une variable tableau à 0.0}
            \PY{n}{initialiser}\PY{p}{(}\PY{n}{notes}\PY{p}{[}\PY{l+m+mi}{0}\PY{p}{]}\PY{p}{,} \PY{l+m+mi}{10}\PY{p}{,} \PY{l+m+mf}{0.2}\PY{p}{)}\PY{p}{;}
            \PY{n}{initialiser}\PY{p}{(}\PY{n}{notes}\PY{p}{[}\PY{l+m+mi}{1}\PY{p}{]}\PY{p}{,} \PY{l+m+mi}{1}\PY{p}{,} \PY{l+m+mf}{0.3}\PY{p}{)}\PY{p}{;}
            \PY{n}{initialiser}\PY{p}{(}\PY{n}{notes}\PY{p}{[}\PY{l+m+mi}{2}\PY{p}{]}\PY{p}{,} \PY{l+m+mi}{12}\PY{p}{,} \PY{l+m+mf}{0.5}\PY{p}{)}\PY{p}{;}
            
            \PY{c+c1}{//Calculer la moyenne des 3 notes}
            \PY{k+kt}{float} \PY{n}{moy} \PY{o}{=} \PY{n}{moyenne}\PY{p}{(}\PY{n}{notes}\PY{p}{,} \PY{l+m+mi}{3}\PY{p}{)}\PY{p}{;}
            \PY{n}{assert}\PY{p}{(}\PY{n}{moy} \PY{o}{=}\PY{o}{=} \PY{l+m+mi}{10}\PY{o}{*}\PY{l+m+mf}{0.2} \PY{o}{+} \PY{l+m+mi}{1}\PY{o}{*}\PY{l+m+mf}{0.3} \PY{o}{+} \PY{l+m+mi}{12}\PY{o}{*}\PY{l+m+mf}{0.5}\PY{p}{)}\PY{p}{;}
            \PY{k}{return} \PY{n}{EXIT\PYZus{}SUCCESS}\PY{p}{;}
        \PY{p}{\PYZcb{}}
\end{Verbatim}


    \begin{center}\rule{0.5\linewidth}{\linethickness}\end{center}

    \subsection{\#\# Arguments de la ligne de
commande}\label{arguments-de-la-ligne-de-commande}

Il est possible de fournir des arguments pour paramétrer l'exécution
d'un programme. On pourra par exemple personnaliser le message affiché à
l'utilisateur dans le \texttt{premier\_programme} en exécutant :

\texttt{./premier\_programme\ Michel}

pour qu'il présente l'affichage suivant
*************************************
********* Bienvenue Michel **********
*************************************
    Pour se faire, il faut déclarer la signature du programme principal avec
les paramètres \texttt{argc} et \texttt{argv{[}{]}} :

    \begin{Verbatim}[commandchars=\\\{\}]
{\color{incolor}In [{\color{incolor} }]:} \PY{c+cp}{\PYZsh{}}\PY{c+cp}{include} \PY{c+cpf}{\PYZlt{}stdlib.h\PYZgt{}}
        
        \PY{k+kt}{int} \PY{n+nf}{main}\PY{p}{(}\PY{k+kt}{int} \PY{n}{argc}\PY{p}{,} \PY{k+kt}{char}\PY{o}{*} \PY{n}{argv}\PY{p}{[}\PY{p}{]}\PY{p}{)}\PY{p}{\PYZob{}}
            \PY{k}{return} \PY{n}{EXIT\PYZus{}SUCCESS}\PY{p}{;}
        \PY{p}{\PYZcb{}}
\end{Verbatim}


    \begin{quote}
\texttt{int\ argc} : nombre d'arguments.
\end{quote}

\begin{quote}
\texttt{char*\ argv{[}{]}} : tableau de chaines de caractères
\end{quote}

La chaîne à l'indice 0 existe toujours et contient le nom de
l'exécutable. Les autres éventuelles chaines listent les arguments dans
l'ordre où ils sont présentés.

    \subsection{\#\#\# Exercice 19 : Lister les arguments de la ligne de
commande}\label{exercice-19-lister-les-arguments-de-la-ligne-de-commande}

\begin{center}\rule{0.5\linewidth}{\linethickness}\end{center}

\textbf{{[}19.1{]}} Ecrire un programme qui permet d'afficher les
arguments de la ligne de commande (cf. fichier
\textbf{\texttt{exercice19.c}}).

    \begin{Verbatim}[commandchars=\\\{\}]
{\color{incolor}In [{\color{incolor} }]:} \PY{c+cp}{\PYZsh{}}\PY{c+cp}{include} \PY{c+cpf}{\PYZlt{}stdlib.h\PYZgt{}}\PY{c+c1}{ }
        \PY{c+cp}{\PYZsh{}}\PY{c+cp}{include} \PY{c+cpf}{\PYZlt{}stdio.h\PYZgt{}}
        
        \PY{k+kt}{int} \PY{n+nf}{main}\PY{p}{(}\PY{k+kt}{int} \PY{n}{argc}\PY{p}{,} \PY{k+kt}{char}\PY{o}{*} \PY{n}{argv}\PY{p}{[}\PY{p}{]}\PY{p}{)}\PY{p}{\PYZob{}}
            \PY{n}{printf}\PY{p}{(}\PY{l+s}{\PYZdq{}}\PY{l+s}{Les arguments sont : }\PY{l+s+se}{\PYZbs{}n}\PY{l+s}{\PYZdq{}}\PY{p}{)}\PY{p}{;}
            \PY{c+c1}{// Afficher ici les argc arguments}
            
            \PY{k}{return} \PY{n}{EXIT\PYZus{}SUCCESS}\PY{p}{;}
        \PY{p}{\PYZcb{}}
\end{Verbatim}


    \subsection{\#\# Exercices BILAN 2}\label{exercices-bilan-2}

\begin{center}\rule{0.5\linewidth}{\linethickness}\end{center}

\begin{quote}
\textbf{ATTENTION} - Les deux exercices bilan suivants \textbf{sont à
rendre à votre intervenant} de TP sous SVN.
\end{quote}

    \subsubsection{Exercice 1 : Portée et masquage des
variables.}\label{exercice-1-portuxe9e-et-masquage-des-variables.}

Le programme fourni suivant compile sans erreur, même avec l'option
-Wall. On répondra aux questions suivantes dans \textbf{un fichier texte
\texttt{Bilan2\_Exercice1.txt} prévu à cet effet sous SVN}, sans
compiler ni exécuter le programme.

\begin{figure}
\centering
\includegraphics{https://www.irit.fr/~Katia.Jaffres/Cours/portee.png}
\caption{algo portee masquage}
\end{figure}

    \textbf{{[}B1.1{]}} Quelle est la portée de chaque variable déclarée ?
Pour chaque variable, on donnera le numéro de ligne où commence et se
termine sa portée.

\textbf{{[}B1.2{]}} Y a-t-il un exemple de masquage de variable dans ce
programme ?

\textbf{{[}B1.3{]}} Peut-on savoir ce que devrait afficher l'exécution
de ce programme ?

\textbf{{[}B1.4{]}} Même s'il compile sans erreur, ce programme est
faux. Pourquoi ?

\textbf{{[}B1.5{]}} La valeur de \texttt{p} change-t-elle après
l'initialisation de la ligne 14 ?

\textbf{{[}B1.6{]}} Que se passerait-il si on modifiait \texttt{*p}
après la ligne 19 ?

    \subsubsection{Exercice 2 : Définition d'une
monnaie.}\label{exercice-2-duxe9finition-dune-monnaie.}

Dans cet exercice nous nous intéressons à la notion de monnaie. Une
monnaie est caractérisée par sa valeur et sa devise. Nous considérerons
que la valeur est réelle et que la devise est représentée par un
caractère. La valeur d'une monnaie doit toujours être positive. Par
exemple, la monnaie «cinq euros» sera représentée par la valeur 5 et le
caractère « e », « dix dollars » par la valeur 10 et le caractère « \$
».

\textbf{Vos réponses sont attendues dans un fichier \texttt{monnaie.c}
sous SVN} prévu à cet effet.

    \textbf{{[}B2.1{]}} Définir le type \texttt{monnaie}.

\textbf{{[}B2.2{]}} Écrire un sous-programme qui initialise une monnaie
à partir d'une valeur et d'une devise. La valeur doit être strictement
positive. On utilisera la programmation par contrat pour le spécifier.

\textbf{{[}B2.3{]}} Écrire un sous-programme qui permet d'ajouter à une
monnaie la valeur d'une autre monnaie. Les deux monnaies doivent avoir
même devise pour que l'opération soit possible. Par exemple, si on
ajoute une monnaie \texttt{m1} qui vaut 5 euros à une monnaie
\texttt{m2} qui vaut 7 euros alors \texttt{m2} vaut 12 euros après
l'opération et \texttt{m1} est inchangée. Si les deux monnaies n'ont pas
la même devise, l'opération n'aura pas lieu.

On utilisera la programmation défensive et un code d'erreur, ici un
booléen (valeur retournée) indiquera si l'opération a été réalisée ou
non.

\textbf{{[}B2.4{]}} Écrire des sous-programmes de test des
sous-programmes définis sur le type \texttt{monnaie}.

\textbf{{[}B2.5{]}} Écrire un programme principal qui : 1. déclare un
tableau de 5 monnaies appelé porte\_monnaie (5 doit être une constante
préprocesseur), 2. initialise chaque élément du tableau en demandant la
valeur et la devise d'une monnaie à l'utilisateur, 3. affiche la somme
de toutes les monnaies qui sont dans une devise demandée à
l'utilisateur.

Il est bien entendu possible de créer des sous-programmes issus d'un
raffinement du programme principal.


    % Add a bibliography block to the postdoc
    
    
    
    \end{document}
